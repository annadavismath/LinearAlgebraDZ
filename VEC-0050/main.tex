\documentclass{ximera}
%% You can put user macros here
%% However, you cannot make new environments

\listfiles

\graphicspath{{./}{firstExample/}{secondExample/}}

\usepackage{tikz}
\usepackage{tkz-euclide}
\usepackage{tikz-3dplot}
\usepackage{tikz-cd}
\usetikzlibrary{shapes.geometric}
\usetikzlibrary{arrows}
%\usetkzobj{all}
\pgfplotsset{compat=1.13} % prevents compile error.

%\renewcommand{\vec}[1]{\mathbf{#1}}
\renewcommand{\vec}{\mathbf}
\newcommand{\RR}{\mathbb{R}}
\newcommand{\dfn}{\textit}
\newcommand{\dotp}{\cdot}
\newcommand{\id}{\text{id}}
\newcommand\norm[1]{\left\lVert#1\right\rVert}
 
\newtheorem{general}{Generalization}
\newtheorem{initprob}{Exploration Problem}

\tikzstyle geometryDiagrams=[ultra thick,color=blue!50!black]

%\DefineVerbatimEnvironment{octave}{Verbatim}{numbers=left,frame=lines,label=Octave,labelposition=topline}



\usepackage{mathtools}

\title{Dot Product and its Properties} \license{CC BY-NC-SA 4.0}

\begin{document}
\begin{abstract}
 \end{abstract}
\maketitle

\begin{onlineOnly}
\section*{Dot Product and its Properties}
\end{onlineOnly}

\begin{definition}\label{def:dotproduct}
  Let $\vec{u}$ and $\vec{v}$ be vectors in $\RR^n$.  The \dfn{dot
    product} of $\vec{u}$ and $\vec{v}$, denoted by
  $\vec{u}\dotp \vec{v}$, is given by
  \begin{align*}
    \vec{u}\dotp\vec{v}=\begin{bmatrix}u_1\\u_2\\\vdots\\u_n\end{bmatrix}\dotp\begin{bmatrix}v_1\\v_2\\\vdots\\v_n\end{bmatrix}=u_1v_1+u_2v_2+\ldots+u_nv_n.
  \end{align*}
\end{definition}

\begin{example}\label{ex:dotex}
  Find $\vec{u}\dotp \vec{v}$ if
  $\vec{u}=\begin{bmatrix}-2\\0\\1\end{bmatrix}$ and
  $\vec{v}=\begin{bmatrix}3\\2\\-4\end{bmatrix}$.

  \begin{explanation}
    $$\vec{u}\dotp\vec{v}=\begin{bmatrix}-2\\0\\1\end{bmatrix}\dotp\begin{bmatrix}3\\2\\-4\end{bmatrix}=(-2)(3)+(0)(2)+(1)(-4)=-6-4=-10$$
  \end{explanation}
\end{example}

Note that the dot product of two vectors is a scalar.  For this reason, the dot product is sometimes called a \dfn{scalar product}.

\subsection*{Properties of the Dot Product}

A quick examination of Example \ref{ex:dotex} will convince you that the dot product is \dfn{commutative}. In other words, $\vec{u}\dotp\vec{v}=\vec{v}\dotp\vec{u}$.  This and other properties of the dot product are stated below.

\begin{theorem}\label{th:dotproductproperties} The following properties hold for
  vectors $\vec{u}$, $\vec{v}$ and $\vec{w}$ in $\RR^n$ and scalar
  $k$ in $\RR$.
  \begin{enumerate}
  \item\label{item:commutative}
    $\vec{u}\dotp\vec{v}=\vec{v}\dotp\vec{u}$
   
  \item\label{item:distributive} $(\vec{u}+\vec{v})\dotp \vec{w}=\vec{u}\dotp \vec{w}+\vec{v}\dotp \vec{w}$
   
  \item\label{item:distributive-again} $\vec{u}\dotp (\vec{v}+\vec{w})=\vec{u}\dotp\vec{v}+\vec{u}\dotp \vec{w}$
   
  \item\label{item:scalar} $(k\vec{u})\dotp \vec{v}=k(\vec{u}\dotp\vec{v})=\vec{u}\dotp (k\vec{v})$
   
  \item \label{item:positive} $\vec{u}\dotp\vec{u}\geq 0$, and $\vec{u}\dotp\vec{u}=0$ if and only if $\vec{u}={\bf 0}$.
   
  \item \label{item:norm}
    $\norm{\vec{u}}^2=\vec{u}\dotp\vec{u}$
  \end{enumerate}
\end{theorem}

We will prove Property ~\ref{item:distributive}.  The remaining properties are left as exercises.

\begin{proof}[Proof of Property~\ref{item:distributive}:]

\begin{align*}
\left(\vec{u}+\vec{v}\right)\dotp \vec{w}&=\left(\begin{bmatrix} u_1\\ u_2\\ \vdots\\ u_n \end{bmatrix}+\begin{bmatrix} v_1\\ v_2\\ \vdots\\ v_n \end{bmatrix}\right)\dotp \begin{bmatrix}w_1\\w_2\\\vdots\\w_n\end{bmatrix}=\begin{bmatrix}
u_1+v_1\\
u_2+v_2\\
\vdots\\
u_n+v_n
\end{bmatrix}\dotp \begin{bmatrix}w_1\\w_2\\\vdots\\w_n\end{bmatrix}\\
&=(u_1+v_1)w_1+
(u_2+v_2)w_2+
\ldots+
(u_n+v_n)w_n\\
&=u_1w_1+v_1w_1+
u_2w_2+v_2w_2+
\ldots+
u_nw_n+v_nw_n\\
&=(u_1w_1+
u_2w_2\ldots+u_nw_n)+(v_1w_1+v_2w_2+
\ldots
+v_nw_n)\\
&=\begin{bmatrix}
u_1\\
u_2\\
\vdots\\
u_n
\end{bmatrix}\dotp\begin{bmatrix}w_1\\w_2\\\vdots\\w_n\end{bmatrix}+\begin{bmatrix}
v_1\\
v_2\\
\vdots\\
v_n
\end{bmatrix}\dotp \begin{bmatrix}w_1\\w_2\\\vdots\\w_n\end{bmatrix}
=\vec{u}\dotp\vec{w}+\vec{v}\dotp\vec{w}
\end{align*}
\end{proof}

We will illustrate Property~\ref{item:norm} with an example.

\begin{example}\label{ex:exprop6}
  Let $\vec{u}=\begin{bmatrix}-2\\3\end{bmatrix}$.  Use $\vec{u}$ to illustrate Property~\ref{item:norm} of Theorem~\ref{th:dotproductproperties}.
  \begin{explanation}
  
  $$\norm{\vec{u}}^2=(-2)^2+3^2=(-2)(-2)+(3)(3)=\begin{bmatrix}-2\\3\end{bmatrix}\dotp\begin{bmatrix}-2\\3\end{bmatrix}=\vec{u}\dotp\vec{u}$$
  \end{explanation}
\end{example}

If we take the square root of both sides of the equation in Property~\ref{item:norm}, we get an alternative way to think of the \href{https://ximera.osu.edu/linearalgebradzv3/LinearAlgebraInteractiveIntro/VEC-0020/main}{length of a vector}.

\begin{corollary}[Length of a Vector]\label{cor:length_via_dotprod}
    Let $\vec{v}$ be a vector in $\RR^n$, then the \dfn{length}, or the \dfn{magnitude}, of $\vec{v}$ is given by
\begin{equation*} \label{eq:norm_dotp} 
\norm{\vec{v}}=\sqrt{\vec{v} \dotp \vec{v}}
\end{equation*}
\end{corollary}

\subsection*{Practice Problems}

\begin{problem}\label{prob:dotproduct1}
Find the dot product of $\vec{u}$ and $\vec{v}$ if
  $$\vec{u}=\begin{bmatrix}-1\\-2\\5\\4\end{bmatrix},\quad \vec{v}=\begin{bmatrix}2\\-2\\-3\\1\end{bmatrix}$$
  Answer:
  $$\vec{u} \dotp \vec{v} = \answer{-9}$$
\end{problem}

\begin{problem}\label{prob:dotproduct2}
Find the dot product of $\vec{u}$ and $\vec{v}$ if 
  $$\vec{u}=\begin{bmatrix}1\\1/2\end{bmatrix},\quad \vec{v}=\begin{bmatrix}-2\\4\end{bmatrix}$$
  Answer:
  
  $$\vec{u} \dotp \vec{v} = \answer{0}$$
\end{problem}

\begin{problem}\label{prob:dotproductprop6}
  Use vector $\vec{u}=\begin{bmatrix}2\\5\\-7\end{bmatrix}$ to
  illustrate Property~\ref{item:norm} of Theorem~\ref{th:dotproductproperties}.
\end{problem}

\begin{problem}\label{prob:th:dotprductproperties}
  Prove Properties~\ref{item:commutative}, \ref{item:distributive-again}, \ref{item:scalar}, \ref{item:positive} and \ref{item:norm} of Theorem~\ref{th:dotproductproperties}.
\end{problem}

\begin{problem}\label{prob:perpvectors1}
From the given list of vector pairs, identify ALL pairs of vectors that lie on perpendicular lines.
\begin{hint}
You may want to draw a picture and think about what you know about slopes of perpendicular lines.
\end{hint}
\begin{selectAll}
  \choice[correct]{$\vec{u}=\begin{bmatrix}1\\\frac{1}{2}\end{bmatrix}$, $\vec{v}=\begin{bmatrix}-2\\4\end{bmatrix}$}
  \choice{$\vec{u}=\begin{bmatrix}-1\\\frac{1}{2}\end{bmatrix}$, $\vec{v}=\begin{bmatrix}-2\\4\end{bmatrix}$}
  \choice[correct]{$\vec{u}=\begin{bmatrix}1\\\frac{1}{2}\end{bmatrix}$, $\vec{v}=\begin{bmatrix}1\\-2\end{bmatrix}$}
  \choice[correct]{$\vec{u}=\begin{bmatrix}-1\\-\frac{1}{2}\end{bmatrix}$, $\vec{v}=\begin{bmatrix}-2\\4\end{bmatrix}$}
\end{selectAll}
Compute $\vec{u}\dotp\vec{v}$ for each pair.  What do you observe? 
\end{problem}

\emph{Problems \ref{prob:perpvectors2a}-\ref{prob:perpvectors2c}}
 
For each problem below
 \begin{enumerate}
 \item 
 Find the value of $x$ that will make vectors $\vec{u}$ and $\vec{v}$ perpendicular. 
  \begin{hint} Think of the $x$-component as the ``run" and the $y$-component as the ``rise", then use what you know about slopes of perpendicular lines.
  \end{hint}
  \item Find $\vec{u}\dotp\vec{v}$.
  \end{enumerate}
  
 
  \begin{problem}\label{prob:perpvectors2a}
    $$\vec{u} = \begin{bmatrix}1\\2\end{bmatrix},\quad \vec{v}=\begin{bmatrix}2\\x\end{bmatrix}$$
    
    Answer:  
    $$x = \answer{-1}$$
    $$\vec{u}\dotp\vec{v}=\answer{0}$$
  \end{problem}

  \begin{problem}\label{prob:perpvectors2b}
    $$\vec{u} = \begin{bmatrix}5\\2\end{bmatrix},\quad \vec{v}=\begin{bmatrix}x\\-4\end{bmatrix}$$
    Answer:
    $$x = \answer{8/5}$$
    $$\vec{u}\dotp\vec{v}=\answer{0}$$
  \end{problem}

  \begin{problem}\label{prob:perpvectors2c}
    $$\vec{u} = \begin{bmatrix} 4\\-3\end{bmatrix},\quad \vec{v} =\begin{bmatrix}6\\x\end{bmatrix}$$ 
    Answer:
    $$x = \answer{8}$$
    $$\vec{u}\dotp\vec{v}=\answer{0}$$
  \end{problem}
\end{problem}

\begin{problem}\label{prob:perpvectors3}
  \begin{enumerate}
    \item Vector $\vec{u}$ that lies on the line $y=mx$, has the form $\vec{u}=k\begin{bmatrix}1\\m\end{bmatrix}$.  Assuming that $m\neq 0$, find the general form for a vector $\vec{v}$ that lies on a line perpendicular to $y=mx$.
      \begin{hint}
        What do you know about the slopes of perpendicular
        lines?
      \end{hint}
    \item Find $\vec{u}\dotp \vec{v}$.
    \item Formulate a conjecture about the dot product of perpendicular vectors.
  \end{enumerate}
\end{problem}


\end{document}

