\documentclass{ximera}
%% You can put user macros here
%% However, you cannot make new environments

\listfiles

\graphicspath{{./}{firstExample/}{secondExample/}}

\usepackage{tikz}
\usepackage{tkz-euclide}
\usepackage{tikz-3dplot}
\usepackage{tikz-cd}
\usetikzlibrary{shapes.geometric}
\usetikzlibrary{arrows}
%\usetkzobj{all}
\pgfplotsset{compat=1.13} % prevents compile error.

%\renewcommand{\vec}[1]{\mathbf{#1}}
\renewcommand{\vec}{\mathbf}
\newcommand{\RR}{\mathbb{R}}
\newcommand{\dfn}{\textit}
\newcommand{\dotp}{\cdot}
\newcommand{\id}{\text{id}}
\newcommand\norm[1]{\left\lVert#1\right\rVert}
 
\newtheorem{general}{Generalization}
\newtheorem{initprob}{Exploration Problem}

\tikzstyle geometryDiagrams=[ultra thick,color=blue!50!black]

%\DefineVerbatimEnvironment{octave}{Verbatim}{numbers=left,frame=lines,label=Octave,labelposition=topline}



\usepackage{mathtools}


\title{Solved Problems} \license{CC BY-NC-SA 4.0}

\begin{document}

\begin{abstract}
\end{abstract}
\maketitle

\section*{Solved Problems for Chapter 6}

\begin{problem}\label{prb:7.2} Let $A = \left[ \begin{array}{rrr}
1 & 2 & 4 \\
0 & 1 & 3 \\
-2 & 5 & 1
\end{array} \right]$. When doing cofactor expansion along the top row, we encounter three minor matrices.  What are they?

Click the arrow to see answer.
\begin{expandable}
$$\begin{bmatrix}1&3\\5&1\end{bmatrix},\quad\begin{bmatrix}0&3\\-2&1\end{bmatrix},\quad\begin{bmatrix}0&1\\-2&5\end{bmatrix}$$
\end{expandable}
\end{problem}

\begin{problem}\label{prb:7.4} Find the following determinant by (a) expanding along the first row, (b)
second column.
$$\begin{vmatrix}
1 & 2 & 1 \\
2 & 1 & 3 \\
2 & 1 & 1
\end{vmatrix}$$

Click the arrow to see answer.
\begin{expandable}

$$\begin{vmatrix}
1 & 2 & 1 \\
2 & 1 & 3 \\
2 & 1 & 1
\end{vmatrix}=1\begin{vmatrix}1 & 3 \\1 & 1\end{vmatrix}-2\begin{vmatrix}2 & 3\\2 & 1\end{vmatrix}+1\begin{vmatrix}2 & 1\\2 & 1\end{vmatrix}=6$$

$$\begin{vmatrix}
1 & 2 & 1 \\
2 & 1 & 3 \\
2 & 1 & 1
\end{vmatrix}=-2\begin{vmatrix}2 & 3 \\2 & 1\end{vmatrix}+1\begin{vmatrix}1 & 1\\2 & 1\end{vmatrix}-1\begin{vmatrix}1 & 1\\2 & 3\end{vmatrix}=6$$

\end{expandable}
\end{problem}

\begin{problem}\label{prb:7.7} Compute the determinant by co-factor expansion. Pick the easiest row or
column to use.
$$\begin{vmatrix}
1 & 0 & 0 & 1 \\
2 & 1 & 1 & 0 \\
0 & 0 & 0 & 2 \\
2 & 1 & 3 & 1
\end{vmatrix}$$

Click the arrow to see answer.
\begin{expandable}
Expand along the third row.
$$\begin{vmatrix}
1 & 0 & 0 & 1 \\
2 & 1 & 1 & 0 \\
0 & 0 & 0 & 2 \\
2 & 1 & 3 & 1
\end{vmatrix}=-2\begin{vmatrix}
    1 & 0 & 0\\2 & 1 & 1\\2 & 1 & 3
\end{vmatrix}=-2\left(1\begin{vmatrix}
    1 & 1\\1 & 3
\end{vmatrix}\right)=-2(2)=-4$$
\end{expandable}
\end{problem}

\begin{problem}\label{prb:7.9} An operation is done to get from the first matrix to the second.
Identify what was done and tell how it will affect the value of the
determinant.
\begin{equation*}
\left[
\begin{array}{cc}
a & b \\
c & d
\end{array}
\right]  \rightarrow \cdots \rightarrow \left[
\begin{array}{cc}
a & c \\
b & d
\end{array}
\right]
\end{equation*}

Click the arrow to see answer.
\begin{expandable}
It does not change the determinant. This was just taking the transpose.
\end{expandable}
\end{problem}

\begin{problem}\label{prb:7.10} An operation is done to get from the first matrix to the second.
Identify what was done and tell how it will affect the value of the
determinant.
\begin{equation*}
\left[
\begin{array}{cc}
a & b \\
c & d
\end{array}
\right] \rightarrow \cdots \rightarrow \left[
\begin{array}{cc}
c & d \\
a & b
\end{array}
\right]
\end{equation*}

Click the arrow to see answer.
\begin{expandable}
In this case two rows were switched and so the resulting determinant is $-1$
times the first.
\end{expandable}
\end{problem}


\begin{problem}\label{prb:7.11} An operation is done to get from the first matrix to the second.
Identify what was done and tell how it will affect the value of the
determinant.
\begin{equation*}
\left[
\begin{array}{cc}
a & b \\
c & d
\end{array}
\right] \rightarrow \cdots \rightarrow \left[
\begin{array}{cc}
a & b \\
a+c & b+d
\end{array}
\right]
\end{equation*}

Click the arrow to see answer.
\begin{expandable}
The determinant is unchanged. It was just the first row added to the second.
\end{expandable}
\end{problem}


\begin{problem}\label{prb:7.12} An operation is done to get from the first matrix to the second.
Identify what was done and tell how it will affect the value of the
determinant.
\begin{equation*}
\left[
\begin{array}{cc}
a & b \\
c & d
\end{array}
\right] \rightarrow \cdots \rightarrow \left[
\begin{array}{cc}
a & b \\
2c & 2d
\end{array}
\right]
\end{equation*}

Click the arrow to see answer.
\begin{expandable}
The second row was multiplied by 2 so the determinant of the result is 2
times the original determinant.
\end{expandable}
\end{problem}

\begin{problem}\label{prb:7.13} An operation is done to get from the first matrix to the second.
Identify what was done and tell how it will affect the value of the
determinant.
\begin{equation*}
\left[
\begin{array}{cc}
a & b \\
c & d
\end{array}
\right] \rightarrow \cdots \rightarrow \left[
\begin{array}{cc}
b & a \\
d & c
\end{array}
\right]
\end{equation*}

Click the arrow to see answer.
\begin{expandable}
In this case the two columns were switched so the determinant of the second
is $-1$ times the determinant of the first.
\end{expandable}
\end{problem}

\begin{problem}\label{prb:7.14} Let $A$ be an $n\times n$ matrix and suppose there are $n-1$ rows
such that all rows are linear combinations of these $n-1$
rows. Show $\det \left( A\right) =0$.

Click the arrow to see answer.
\begin{expandable}
If the determinant is nonzero, then it will remain nonzero with row operations applied to the matrix.
In this case, you can obtain a row of zeros by doing row
operations. Thus the determinant must be zero.
\end{expandable}
\end{problem}

\begin{problem}\label{prb:7.16} Construct $2\times 2$ matrices $A$ and $B$ to illustrate the property that
$\det A \det B = \det (AB)$.

Click the arrow to see answer.
\begin{expandable}
\[
\det
\left( \left[
\begin{array}{cc}
1 & 2 \\
3 & 4
\end{array}
\right] \left[
\begin{array}{rr}
-1 & 2 \\
-5 & 6
\end{array}
\right] \right) = -8
\]
\[
\det \left[
\begin{array}{cc}
1 & 2 \\
3 & 4
\end{array}
\right] \det \left[
\begin{array}{rr}
-1 & 2 \\
-5 & 6
\end{array}
\right] = -2 \times 4 = -8
\]
\end{expandable}
\end{problem}

\begin{problem}\label{prb:7.18} An $n\times n$ matrix is called \dfn{nilpotent}
if for some positive integer, $k$ it follows $A^{k}=O.$ If
$A$ is a nilpotent matrix and $k$ is the smallest possible integer such that
$A^{k}=O,$ what are the possible values of $\det \left( A\right)$?

Click the arrow to see answer.
\begin{expandable}
$\det{A}=0$ because $0=\det \left( 0\right) =\det \left( A^{k}\right) =\left( \det
\left( A\right) \right) ^{k}.$
\end{expandable}
\end{problem}

\begin{problem}\label{prb:7.19}A matrix is said to be \dfn{orthogonal} if
$A^{T}A=I.$ Thus the inverse of an orthogonal matrix is just its transpose.
What are the possible values of $\det \left( A\right) $ if $A$ is an
orthogonal matrix?

Click the arrow to see answer.
\begin{expandable}
You would need $\det \left( AA^{T}\right) =\det
\left( A\right) \det \left( A^{T}\right) =\det \left( A\right) ^{2}=1$ and
so $\det \left( A\right) =1,$ or $-1$.
\end{expandable}
\end{problem}

\begin{problem}\label{prb:7.20} Let $A$ and $B$ be two $n\times n$ matrices. We say that $A$ is \textbf{similar} to $B$ and write $A\sim B$
provided that there exists an invertible matrix $P$
such that $A=P^{-1}BP.$ Show that if $A\sim B,$ then
$\det \left( A\right) =\det \left( B\right)$.

Click the arrow to see answer.
\begin{expandable}
$\det \left( A\right) =\det
\left( S^{-1}BS\right) =\det \left( S^{-1}\right) \det \left( B\right) \det
\left( S\right) =\det \left( B\right) \det \left( S^{-1}S\right) =\det
\left( B\right) $.
\end{expandable}
\end{problem}

\begin{problem}\label{prb:7.21} Determine whether each statement is true or false. If true, provide a proof. If false, provide a counter example.
\begin{enumerate}

\item If any two columns of a square matrix are equal, then the determinant
of the matrix equals zero.

\item If $A^{-1}$ exists then $\det \left( A^{-1}\right) =\det \left(
A\right) ^{-1}.$

\item If $A$ is a real $n\times n$ matrix, then $\det \left( A^{T}A\right)
\geq 0.$

\item If $AX=0$ for some $X \neq 0,$ then $\det \left(
A\right) =0.$
\end{enumerate}

Click the arrow to see answer.
\begin{expandable}
All of the statements are true.
\begin{enumerate}
\item $\det(A)=\det(A^T)$.  If two columns of $A$ are equal, then two rows of $A^T$ are equal.  Applying row operations to $A^T$ will produce a matrix with a row of zeros.
\item Suppose $A$ is invertible.  Then 
$$1=\det(I)=\det(AA^{-1})=\det(A)\det(A^{-1})$$  Divide both right and left side by $\det(A)$ to obtain the result.
\item This follows from the fact that $\det(A)=\det(A^T)$.
\item If $AX=0$ for some $X \neq \vec{0}$ then there is a non-trivial linear combination of the columns of $A$ that is equal to $\vec{0}$.  This means that the columns of $A$ are linearly dependent.  This implies that the rows are also linearly dependent (why?).  Applying elementary row operations will lead us to a row of zeros.
\end{enumerate}
\end{expandable}
\end{problem}

\begin{problem}\label{prb:7.30} Let
\begin{equation*}
A =
\left[
\begin{array}{rrr}
1 & 0 & 3 \\
1 & 0 & 1 \\
3 & 1 & 0
\end{array}
\right]
\end{equation*}
Determine whether the matrix $A$ has an inverse by finding whether the
determinant is non-zero. 

Click the arrow to see answer.
\begin{expandable}
\[
\det \left[
\begin{array}{rrr}
1 & 0 & 3 \\
1 & 0 & 1 \\
3 & 1 & 0
\end{array}
\right] = 2
\]
and so it has an inverse. The inverse turns out to equal
\[
\left[
\begin{array}{rrr}
-\frac{1}{2} & \frac{3}{2} & 0 \\
\frac{3}{2} & -\frac{9}{2} & 1 \\
\frac{1}{2} & -\frac{1}{2} & 0
\end{array}
\right]
\]
\end{expandable}
\end{problem}



\section*{Bibliography}
Some of the problems come from the end of Chapter 3 of Ken Kuttler's \href{https://open.umn.edu/opentextbooks/textbooks/a-first-course-in-linear-algebra-2017}{\it A First Course in Linear Algebra}. (CC-BY)

Ken Kuttler, {\it  A First Course in Linear Algebra}, Lyryx 2017, Open Edition, pp. 272--315.  

\end{document}