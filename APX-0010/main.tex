\documentclass{ximera}
%% You can put user macros here
%% However, you cannot make new environments

\listfiles

\graphicspath{{./}{firstExample/}{secondExample/}}

\usepackage{tikz}
\usepackage{tkz-euclide}
\usepackage{tikz-3dplot}
\usepackage{tikz-cd}
\usetikzlibrary{shapes.geometric}
\usetikzlibrary{arrows}
%\usetkzobj{all}
\pgfplotsset{compat=1.13} % prevents compile error.

%\renewcommand{\vec}[1]{\mathbf{#1}}
\renewcommand{\vec}{\mathbf}
\newcommand{\RR}{\mathbb{R}}
\newcommand{\dfn}{\textit}
\newcommand{\dotp}{\cdot}
\newcommand{\id}{\text{id}}
\newcommand\norm[1]{\left\lVert#1\right\rVert}
 
\newtheorem{general}{Generalization}
\newtheorem{initprob}{Exploration Problem}

\tikzstyle geometryDiagrams=[ultra thick,color=blue!50!black]

%\DefineVerbatimEnvironment{octave}{Verbatim}{numbers=left,frame=lines,label=Octave,labelposition=topline}



\usepackage{mathtools}


\title{Triangle Inequality} \license{CC BY-NC-SA 4.0}

\begin{document}

\begin{abstract}
 
\end{abstract}
\maketitle

\section*{Triangle Inequality}

The \dfn{Triangle Inequality} is a simple, yet powerful result used widely in analysis and topology as well as other branches of mathematics.  The triangle inequality has its roots in geometry. It initially appeared as a proposition in the \dfn{Elements} - a treatise comprised of thirteen books covering plane and solid geometry, and number theory - written by Euclid of Alexandria around 300 B.C.

The geometric version of the triangle inequality states that the sum of the lengths of any two sides of a triangle is greater than the length the third side.  

Here we are interested in the vector version of this result.  Given vectors $\vec{v}$ and $\vec{w}$, we have
$$\norm{\vec{v}+\vec{w}}\leq\norm{\vec{v}}+\norm{\vec{w}}$$

\begin{center}
    \begin{tikzpicture}
\draw[line width=1pt,-stealth,red](3,3)--(6,2);
\draw[line width=1pt,-stealth, blue](-1,1)--(3,3);
\draw[line width=1pt,-stealth](-1,1)--(6,2);

 \node[blue] at (1, 2.3)   (a) {$\vec{v}$};
 \node[red] at (4.5, 2.8)   (a) {$\vec{w}$};
  \node[black] at (2.5, 1.2)   (a) {$\vec{v}+\vec{w}$};
 \end{tikzpicture}
\end{center}

Intuitively, we observe that equality occurs when either $\vec{v}$ or $\vec{w}$ (or both) are zero, or when non-zero vectors $\vec{v}$ and $\vec{w}$ point in the same direction, otherwise the inequality is strict.

Proving the triangle inequality requires some preliminary results.

\begin{lemma}\label{lem:triLem}
    For any vectors $\vec{v}$ and $\vec{w}$ in $\RR^n$,
    \begin{equation}
\norm{\vec{v}+\vec{w}}^2=\norm{\vec{v}}^2+2(\vec{v}\dotp\vec{w})+\norm{\vec{w}}^2
    \end{equation}
\end{lemma}
\begin{proof}
    By Theorem \ref{th:dotproductproperties} we have,
    \begin{eqnarray*}
        \norm{\vec{v}+\vec{w}}^2 & = & (\vec{v}+\vec{w})\dotp (\vec{v}+\vec{w})\\
        &=& \vec{v}\dotp \vec{v}+\vec{v}\dotp\vec{w}+\vec{w}\dotp\vec{v}+\vec{w}\dotp\vec{w}\\
        &=&\norm{\vec{v}}^2+ 2(\vec{v}\dotp\vec{w})+\norm{\vec{w}}^2
    \end{eqnarray*}
\end{proof}

\begin{theorem}[Cauchy-Schwarz Inequality]\label{th:CS}
    For any vectors $\vec{v}$ and $\vec{w}$ in $\RR^n$,
    \begin{equation}
        |\vec{v}\dotp\vec{w}|\leq\norm{\vec{v}}\norm{\vec{w}}
    \end{equation}
\end{theorem}
\begin{proof}
    Recall that by Theorem \ref{th:dotproductcosine}, $\vec{v}\dotp\vec{w}=\norm{\vec{v}}\norm{\vec{w}}\cos\theta$, where $\theta$ is the included angle.  Our result follows from the fact that $|\cos\theta |\leq 1$.
\end{proof}

\begin{theorem}[Triangle Inequality]\label{th:TriIneq}
    For any vectors $\vec{v}$ and $\vec{w}$ in $\RR^n$,
    \begin{equation}
        \norm{\vec{v}+\vec{w}}\leq\norm{\vec{v}}+\norm{\vec{w}}
    \end{equation}
\end{theorem}
\begin{proof}
    We will use Lemma \ref{lem:triLem} and Theorem \ref{th:CS}.
    \begin{eqnarray*}
        \norm{\vec{v}+\vec{w}}^2 &=& \norm{\vec{v}}^2+2(\vec{v}\dotp\vec{w})+\norm{\vec{w}}^2\\
        &\leq& \norm{\vec{v}}^2+2\norm{\vec{v}}\norm{\vec{w}}+\norm{\vec{w}}^2\\
        &=&\left(\norm{\vec{v}}+\norm{\vec{w}}\right)^2
    \end{eqnarray*}
    Taking the square root of both sides yields the desired result.
    
\end{proof}

\end{document}