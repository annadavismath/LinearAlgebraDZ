\documentclass{ximera}
%% You can put user macros here
%% However, you cannot make new environments

\listfiles

\graphicspath{{./}{firstExample/}{secondExample/}}

\usepackage{tikz}
\usepackage{tkz-euclide}
\usepackage{tikz-3dplot}
\usepackage{tikz-cd}
\usetikzlibrary{shapes.geometric}
\usetikzlibrary{arrows}
%\usetkzobj{all}
\pgfplotsset{compat=1.13} % prevents compile error.

%\renewcommand{\vec}[1]{\mathbf{#1}}
\renewcommand{\vec}{\mathbf}
\newcommand{\RR}{\mathbb{R}}
\newcommand{\dfn}{\textit}
\newcommand{\dotp}{\cdot}
\newcommand{\id}{\text{id}}
\newcommand\norm[1]{\left\lVert#1\right\rVert}
 
\newtheorem{general}{Generalization}
\newtheorem{initprob}{Exploration Problem}

\tikzstyle geometryDiagrams=[ultra thick,color=blue!50!black]

%\DefineVerbatimEnvironment{octave}{Verbatim}{numbers=left,frame=lines,label=Octave,labelposition=topline}



\usepackage{mathtools}


\title{Octave for Chapter 8} \license{CC BY-NC-SA 4.0}

\begin{document}

\begin{abstract}
\end{abstract}
\maketitle

\section*{Octave for Chapter 8}

Examples in this section provide sample Octave code for computing eigenvalues. You can access our code through the link at the bottom of each template.  Feel free to modify the code and experiment to learn more!  

You can write your own code using Octave software or online Octave cells.  To access Octave cells online, go to the \href{https://sagecell.sagemath.org/}{Sage Math Cell Webpage}, select OCTAVE as the language, enter your code, and press EVALUATE.  

To ''save" or share your online code, click on the \emph{Share} button, select \emph{Permalink}, then copy the address directly from the browser window.  You can store this link to access your work later or share this link with others.  You will need to get a new Permalink every time you modify the code.

\subsection*{Octave Tutorial}
\begin{template}\label{temp:eigen}
    Find eigenvalues and the corresponding eigenvectors of the given matrix $A$.
    
    \begin{verbatim}
    % Define A
    A=[5 0 0; 
    1 2 -1; 
    1 3 -2];
    
    % Find eigenvalues of A
    eigen_A=eig(A)
    
    % Find eigenvalues and eigenvectors of A
    [V,D]=eig(A)
    % D is a diagonal matrix whose diagonal entries are the eigenvalues of A
    % Columns of V are the eigenvectors of A 
    \end{verbatim}
    
    \href{https://sagecell.sagemath.org/?z=eJx1jsEKgkAYhO8L-w5zWShIUKOTeFiSHsGLSCz5qwvrLqxaPX5qZBF1-meG-X5GIKNaW4LkTKbFASHCBJxFiBFET7VHEJcJZ5wJnLStQLohe1VmpB6untElOct0uhu5_VNVq6fL4PyLLfJdVq6kQAY9dVFp1TirDDo1eH3HrXU9vVOyUzo_9YShpR-bBI7OjJ1dfP5V_FjwAJh4TrQ=&lang=octave&interacts=eJyLjgUAARUAuQ==}{Link to code}.
    \end{template}

    \begin{template}\label{temp:charPoly}
        We can find the characteristic polynomial of the given matrix $A$ as follows.
    
        \begin{verbatim}
    % Define A
    A=[5 0 0; 
        1 2 -1; 
        1 3 -2];
    
    % Find the coefficients of the char poly of A
    char_poly = poly(A)
        \end{verbatim}
    
    The output looks like this:
    
    \begin{verbatim}
    char_poly =
    
     1 -5 -1 5
    \end{verbatim}
    
    \href{https://sagecell.sagemath.org/?z=eJxTVXBJTcvMS1Vw5OVytI02VTBQMLBW4OVSAAJDBSMFXUMEz1hB1yjWmpeLl0tVwS0zL0WhJCNVITk_NS0tMzkzNa-kWCE_DSKWkVikUJCfUwkSABoM4seD-bZgYQ1HTQCF1h-X&lang=octave&interacts=eJyLjgUAARUAuQ==}{Link to code}.
    
    The output is a vector of coefficients, starting with the leading coefficient.  In this case, we see that the characteristic polynomial is $\lambda^3-5\lambda^2-\lambda+5$.
    
    \begin{warning}
        In this text, we find the characteristic polynomial by computing $\det{\left(A-\lambda I\right)}$.  Octave, as well as some other textbooks, finds the characteristic polynomial based on $\det{\left(\lambda I-A\right)}$.  Note that setting these expressions equal to zero to find eigenvalues produces the same result. However, the coefficients of the characteristic polynomials from the two approaches are negatives of each other.  If you were to do this problem by hand, using the method in this text, you would get $-\lambda^3+5\lambda^2+\lambda-5$.
    \end{warning}
    \end{template}

    \begin{example}\label{ex:complexEig}
        In Example \ref{ex:eigsrotation}, you found that the rotation matrix $M=\begin{bmatrix}
        \frac{\sqrt{2}}{2} & -\frac{\sqrt{2}}{2}\\
        \frac{\sqrt{2}}{2} & \frac{\sqrt{2}}{2}
        \end{bmatrix}$ has complex eigenvalues.  Use the \emph{eig} command in Octave to compute the eigenvalues and eigenvectors of $M$.  
        
        \begin{explanation}
        Here is the input code:
        
        \begin{verbatim}
        % Define A
        A=[sqrt(2)/2  -sqrt(2)/2;
        sqrt(2)/2  sqrt(2)/2];
        
        % Find eigenvalues of A
        eigen=eig(A)
        
        % Find eigenvalues and eigenvectors of A
        [V,D]=eig(A)
        % D is a diagonal matrix whose diagonal entries are the eigenvalues of A
        % Columns of V are the eigenvectors of A
        \end{verbatim}
        
        This is what we get for the output:
        \begin{verbatim}
        eigen =
        
         (0.707107,0.707107)
         (0.707107,-0.707107)
        
        V =
        
         (0,0.707107) (0,-0.707107)
         (0.707107,0) (0.707107,-0)
        
        D =
        
         (0.707107,0.707107) (0,0)
         (0,0) (0.707107,-0.707107)
        \end{verbatim}
        
        \href{https://sagecell.sagemath.org/?z=eJx1jkEKgzAURPeB3GE2AYWWgtviQio9ghtxEepXAzHBGNsev1FoKqXdfOY_ZoYRKKlThlBwVuT1PDmfZOkpA45Rnznb8SibwDkTuCrTglRP5i71QjNst7ZtJA83KdI_Rhl_unnr3sm6OpRNTAqUUMGLVsneGqkxSu_UE4_BzvShZAJdSx3BD_RjkcDF6mU02199GXcLXu8oVg4=&lang=octave&interacts=eJyLjgUAARUAuQ==}{Link to code}.
        
        Each complex number in the output is presented using its real and imaginary parts as $x$ and $y$ coordinates of a point.  For example, the second eigenvalue, written as $(0.707107,-0.707107)$, is the complex number $\frac{\sqrt{2}}{2}-\frac{\sqrt{2}}{2}i$.  Compare this to the eigenvalue in Example \ref{ex:eigsrotation}.  The corresponding eigenvector is the second column of the matrix $V$: $\begin{bmatrix}(0,-0.707107)\\(0.707107,-0)\end{bmatrix}$.  This should be interpreted as $\begin{bmatrix}(-\sqrt{2}/2)i\\\sqrt{2}/2\end{bmatrix}$.  Compare this to the eigenvector in Example \ref{ex:eigsrotation}.  There is a discrepancy!  Did we make a mistake?  How do we reconcile the two eigenvectors?
        \begin{hint}
            Recall that a scalar multiple of an eigenvector is also an eigenvector.  What happens if you multiply our current answer by $i$?
        \end{hint}
        \end{explanation}
        \end{example}

\subsection*{Octave Exercises}   
\begin{problem}\label{prob_oct_eig0}
    Let $A=\begin{bmatrix}-148 & 16 & -528\\
        71 & -28 & 292\\
        39 & -4 & 140\end{bmatrix}$.  
        \begin{enumerate}
        \item Use Octave to find the eigenvalues of $A$.  
        \item Find the eigenvalues of $4A$.  How do they compare?
        \item Formulate and prove a conjecture about the relationship between the eigenvalues of $A$ and the eigenvalues of $kA$, where $k$ is a constant.
        \item How are the eigenvectors of $A$ compare to the eigenvectors of $kA$?  Prove your claim.
        \end{enumerate}
        
\end{problem}

\begin{problem}\label{prob_oct_eig1}
Let $$A=\begin{bmatrix}2 & -20 & 36\\-1 & 8 & -14\\0 & 5 & -7\end{bmatrix}$$
\begin{enumerate}
    \item
Find the eigenvalues of $A$ using Octave.  List the eigenvalues below in increasing order.
$$\lambda_1=\answer{-2},\quad\lambda_2=\answer{2},\quad\lambda_3=\answer{3}$$
\item By hand, find the corresponding eigenvectors of $A$.  List them below.
$$\vec{x}_{\lambda_1}=\begin{bmatrix}\answer{-4}\\\answer{1}\\1\end{bmatrix},\quad\vec{x}_{\lambda_2}=\begin{bmatrix}\answer{-16}\\\answer{9}\\5\end{bmatrix},\quad\vec{x}_{\lambda_3}=\begin{bmatrix}\answer{-4}\\\answer{2}\\1\end{bmatrix}$$
\item Use Octave to find the eigenvectors of $A$.  Reconcile your answers from the previous part with the answers you got from Octave.
\end{enumerate}
\end{problem}   

\begin{problem}\label{prob_oct_eig2}
    Use Octave to find the eigenvalues of $A=\begin{bmatrix}
    0 & 1\\-1 & 0
    \end{bmatrix}$.  Interpret the numbers you are seeing.  Interpret your results geometrically.
\end{problem}

\begin{problem}\label{prob_oct_eig3}
    Let $$A=\begin{bmatrix}1 & 2 & 3\\4 & 5& 6\\7 & 8 & 9\end{bmatrix}$$
    When we use Octave to find the eigenvalues of $A$, we get the following printout.
    \begin{verbatim}
        ans =

        16.1168
        -1.11684
        -1.30368e-15
    \end{verbatim}

Should we interpret the last eigenvalue as $0$, or as a very small non-zero number?  How can you be sure of your answer in this particular case?
\end{problem}

\begin{problem}\label{prob_oct_eig4}
    Use the Power Method described in \href{}{Power Method} to approximate the dominant eigenvalue of 
    $A=\begin{bmatrix}-4 & 64\\ 16 & 44\end{bmatrix}$.  Experiment with the number of iterations.  Compare your answer to the answer you get when using the \emph{eig} function.
\end{problem}
 

\end{document}