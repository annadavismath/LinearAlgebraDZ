\documentclass{ximera}

%% You can put user macros here
%% However, you cannot make new environments

\listfiles

\graphicspath{{./}{firstExample/}{secondExample/}}

\usepackage{tikz}
\usepackage{tkz-euclide}
\usepackage{tikz-3dplot}
\usepackage{tikz-cd}
\usetikzlibrary{shapes.geometric}
\usetikzlibrary{arrows}
%\usetkzobj{all}
\pgfplotsset{compat=1.13} % prevents compile error.

%\renewcommand{\vec}[1]{\mathbf{#1}}
\renewcommand{\vec}{\mathbf}
\newcommand{\RR}{\mathbb{R}}
\newcommand{\dfn}{\textit}
\newcommand{\dotp}{\cdot}
\newcommand{\id}{\text{id}}
\newcommand\norm[1]{\left\lVert#1\right\rVert}
 
\newtheorem{general}{Generalization}
\newtheorem{initprob}{Exploration Problem}

\tikzstyle geometryDiagrams=[ultra thick,color=blue!50!black]

%\DefineVerbatimEnvironment{octave}{Verbatim}{numbers=left,frame=lines,label=Octave,labelposition=topline}



\usepackage{mathtools}


\title{Standard Unit Vectors in $\RR^n$} \license{CC BY-NC-SA 4.0}

\begin{document}

\begin{abstract}
  
\end{abstract}
\maketitle

\begin{onlineOnly}
\section*{Standard Unit Vectors in $\RR^n$}
\end{onlineOnly}

\subsection*{Standard Unit Vectors in $\RR^2$ and $\RR^3$} 
A \dfn{unit vector} is a vector of length 1.  A unit vector in the positive direction of a coordinate axis is called a \dfn{standard unit vector}.  There are two standard unit vectors in $\RR^2$.  The vector $\vec{i}=\begin{bmatrix}
1\\
0
\end{bmatrix}$ is parallel the $x$-axis, and the vector $\vec{j}=\begin{bmatrix}
0\\
1
\end{bmatrix}$ is parallel the $y$-axis.  

\begin{center}
\begin{tikzpicture}[scale=1.5]
\draw[thin,gray!40] (-1,-1) grid (2,2);
  \draw[<->] (-1,0)--(2,0);
  \draw[<->] (0,-1)--(0,2);
  \draw[line width=1pt,-stealth, red](0,0)--(1,0) node[below right]{$\vec{i}=\begin{bmatrix}1\\0\end{bmatrix}$};
  
  \draw[line width=1pt,-stealth, blue](0,0)--(0,1) node[above left]{$\vec{j}=\begin{bmatrix}0\\1\end{bmatrix}$};
 \end{tikzpicture}
\end{center}

Vector names $\vec{i}$ and $\vec{j}$ are reserved for standard unit vectors in the direction of $x$ and $y$ axes, respectively.  We chose to express $\vec{i}$ and $\vec{j}$ as column vectors, instead of row vectors, because the context in which we will encounter them in the future will require them to be column vectors.  You may see them presented as row vectors in a different course.


There are three standard unit vectors in $\RR^3$: $$\vec{i}=\begin{bmatrix}
1\\
0\\
0
\end{bmatrix},\quad \vec{j}=\begin{bmatrix}
0\\
1\\
0
\end{bmatrix},\quad\vec{k}=\begin{bmatrix}
0\\
0\\
1
\end{bmatrix}$$

\begin{center}
\tdplotsetmaincoords{70}{130}
\begin{tikzpicture}
	\draw[->](-1,0,0)--(4,0,0) node[below left]{$y$};
    \draw[->](0,-1,0)--(0,4,0) node[below left]{$z$};
    \draw[->](0,0,-1)--(0,0,4) node[below left]{$x$};
        
    \draw[->, line width=1pt,blue, -stealth](0,0,0)--(3,0,0);
    
    \draw[->, line width=1pt,red, -stealth](0,0,0)--(0,0,3);
    \node[blue] at (3, 0.7)   (b) {$\vec{j}=\begin{bmatrix}0\\1\\0\end{bmatrix}$};
    \node[red] at (-1.9, -0.5)   (b) {$\vec{i}=\begin{bmatrix}1\\0\\0\end{bmatrix}$};
    \draw[->, line width=1pt,orange, -stealth](0,0,0)--(0,3,0)node[below, right]{$\vec{k}=\begin{bmatrix}0\\0\\1\end{bmatrix}$};
   
    \end{tikzpicture}
\end{center}

\subsection*{A Vector as a Linear Combination of Standard Unit Vectors} 
Every vector in $\RR^2$ and $\RR^3$ can be written as a sum of scalar multiples of $\vec{i}$, $\vec{j}$ and $\vec{k}$.  For example, if $\vec{v}=\begin{bmatrix}
3\\
-2\\
7
\end{bmatrix}$, then
$$\vec{v}=\begin{bmatrix}
3\\
-2\\
7
\end{bmatrix}=\begin{bmatrix}
3\\
0\\
0
\end{bmatrix}+\begin{bmatrix}
0\\
-2\\
0
\end{bmatrix}+\begin{bmatrix}
0\\
0\\
7
\end{bmatrix}=3\begin{bmatrix}
1\\
0\\
0
\end{bmatrix}+(-2)\begin{bmatrix}
0\\
1\\
0
\end{bmatrix}+7\begin{bmatrix}
0\\
0\\
1
\end{bmatrix}=3\vec{i}-2\vec{j}+7\vec{k}$$

The expression $3\vec{i}-2\vec{j}+7\vec{k}$ is called a \dfn{linear combination} of $\vec{i}$, $\vec{j}$ and $\vec{k}$. 

\subsection*{Standard Unit Vectors in $\RR^n$}
When working with vectors in $\RR^n$, we often use a different notation to denote the standard unit vectors.


  \begin{definition}\label{def:standardunitvecrn} 
  Let $\vec{e}_i$ denote a vector that has $1$ as the $i^{th}$ component and zeros elsewhere.  In other words, $$\vec{e}_i=\begin{bmatrix}
0\\
0\\
\vdots\\
1\\
\vdots\\
0
\end{bmatrix}$$ 
  where $1$ is in the $i^{th}$ position.  We say that  $\vec{e}_i$ is a \dfn{standard unit vector of $\RR^n$}.
\end{definition}


\section*{Practice Problems}
\emph{Problems \ref{prob:lincombijk1a}-\ref{prob:lincombijk1c}}

Express each of the vectors as a linear combination of the appropriate standard unit vectors.
 
  \begin{problem}\label{prob:lincombijk1a}
  $$\vec{u}=\begin{bmatrix}
0\\
4\\
-3
\end{bmatrix}$$
Answer:
$$\vec{u}=\answer{0}\vec{i}+\answer{4}\vec{j}+\answer{-3}\vec{k}$$
\end{problem}
\begin{problem}\label{prob:lincombijk1b}
$$\vec{v}=\begin{bmatrix}
-1\\
1
\end{bmatrix}$$
Answer:
$$\vec{v}=\answer{-1}\vec{i}+\answer{1}\vec{j}$$
\end{problem}

\begin{problem}\label{prob:lincombijk1c}
$$\vec{w}=\begin{bmatrix}
5\\
-3\\
1\\
7
\end{bmatrix}$$
Answer:
$$\vec{w}=\answer{5}\vec{e}_1+\answer{-3}\vec{e}_2+\answer{1}\vec{e}_3+\answer{7}\vec{e}_4$$
  \end{problem}


\emph{Problems \ref{prob:lincombijk2a}-\ref{prob:lincombijk2c}}

Express each given vector in component form.

  \begin{problem}\label{prob:lincombijk2a}
  $\vec{u}=\vec{i}+3\vec{j}$ is a vector in $\RR^2$.
  
  Answer:
  $$\vec{u}=\begin{bmatrix}\answer{1}\\\answer{3}\end{bmatrix}$$
  \end{problem}
  
  \begin{problem}\label{prob:lincombijk2b}
  $\vec{v}=-\vec{j}+5\vec{k}$ is a vector in $\RR^3$.
  
  Answer:
  $$\vec{v}=\begin{bmatrix}\answer{0}\\\answer{-1}\\\answer{5}\end{bmatrix}$$
  \end{problem}
  
  \begin{problem}\label{prob:lincombijk2c}
  $\vec{w}=\vec{e}_1-2\vec{e}_3+4\vec{e}_4$ is a vector in $\RR^4$.
  
  Answer:
  $$\vec{w}=\begin{bmatrix}\answer{1}\\\answer{0}\\\answer{-2}\\\answer{4}\end{bmatrix}$$
  \end{problem}

  
  \begin{problem}\label{prob:lincombijk3}
  Is it possible to express $\vec{u}=\begin{bmatrix}
-6\\
1\\
4
\end{bmatrix}$ as a linear combination of $\vec{i}$ and $\vec{j}$ alone, where $\vec{i}$ and $\vec{j}$ are in $\RR^3$?  Explain your reasoning.
\end{problem}



\end{document} 