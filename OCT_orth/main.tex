\documentclass{ximera}
%% You can put user macros here
%% However, you cannot make new environments

\listfiles

\graphicspath{{./}{firstExample/}{secondExample/}}

\usepackage{tikz}
\usepackage{tkz-euclide}
\usepackage{tikz-3dplot}
\usepackage{tikz-cd}
\usetikzlibrary{shapes.geometric}
\usetikzlibrary{arrows}
%\usetkzobj{all}
\pgfplotsset{compat=1.13} % prevents compile error.

%\renewcommand{\vec}[1]{\mathbf{#1}}
\renewcommand{\vec}{\mathbf}
\newcommand{\RR}{\mathbb{R}}
\newcommand{\dfn}{\textit}
\newcommand{\dotp}{\cdot}
\newcommand{\id}{\text{id}}
\newcommand\norm[1]{\left\lVert#1\right\rVert}
 
\newtheorem{general}{Generalization}
\newtheorem{initprob}{Exploration Problem}

\tikzstyle geometryDiagrams=[ultra thick,color=blue!50!black]

%\DefineVerbatimEnvironment{octave}{Verbatim}{numbers=left,frame=lines,label=Octave,labelposition=topline}



\usepackage{mathtools}


\title{Octave for Chapter 9} \license{CC BY-NC-SA 4.0}

\begin{document}

\begin{abstract}
\end{abstract}
\maketitle

\section*{Octave for Chapter 9}

Examples in this section provide sample Octave code for finding determinants and illustrating their uses and properties. You can access our code through the link at the bottom of each template.  Feel free to modify the code and experiment to learn more!  

You can write your own code using Octave software or online Octave cells.  To access Octave cells online, go to the \href{https://sagecell.sagemath.org/}{Sage Math Cell Webpage}, select OCTAVE as the language, enter your code, and press EVALUATE.  

To ''save" or share your online code, click on the \emph{Share} button, select \emph{Permalink}, then copy the address directly from the browser window.  You can store this link to access your work later or share this link with others.  You will need to get a new Permalink every time you modify the code.

\subsection*{Octave Tutorial}
\subsubsection*{Least Squares}
The following theorem establishes one way to implement least squares.

\begin{theorem}[\ref{th:bestApprox}]
    Let $A$ be an $m\times n$ matrix, let $\vec{b}$ be a column vector in $\RR^m$.  Consider the matrix equation
    $$A\vec{x}=\vec{b}$$
    \begin{enumerate}
        \item Any solution $\vec{z}$ to the normal equations
        $$\left(A^TA\right)\vec{z}=A^T\vec{b}$$
        is a best approximation to a solution to $A\vec{x}=\vec{b}$ in the sense that $\norm{\vec{b}-A\vec{z}}$ is minimized.
        \item If the columns of $A$ are linearly independent, then $A^TA$ is invertible and $\vec{z}$ is given uniquely by
        $$\vec{z}=\left(A^TA\right)^{-1}A^T\vec{b}$$
    \end{enumerate}
    \end{theorem}

We can implement this process as follows.

\begin{template}\label{temp:leastSq}
    We show how to find a least squares solution to $A\vec{x}=\vec{b}$.

    \begin{verbatim}
% Define matrix A
A=[9 -3 3;
    1 -1 1;
    4 0 1;
    1 1 1; 
    9 -1 2];
        
% Define vector b
b=[3;1;1;2;4];

% Now we solve for z
z=inv(transpose(A)*A)*transpose(A)*b
    \end{verbatim}

\href{https://sagecell.sagemath.org/?z=eJxTVXBJTcvMS1XITSwpyqxQcOTlcrSNtlTQNVYwtublUgACQwVdQwVDKMdEwQDONgRBawUIxxKkyigWKgUCvFyqMMPLUpNL8osUkni5kmyjja0NgdDI2gSkGKTIL79coTxVoTg_pyxVIQ2oroqXq8o2M69Mo6QoMa-4IL84VcNRUwuIUPhJAN_XLTM=&lang=octave&interacts=eJyLjgUAARUAuQ==}{Link to code}.    
\end{template}

\begin{warning}
    Finding matrix inverses is both computationally expensive and unstable.  
\end{warning}

To do least squares in Octave, we use the backslash operator ($\setminus$). Depending on the type of matrix, backslash uses different techniques to compute the least squares solution more efficiently.  For more information about \emph{inv} and the backslash operator ($\setminus$) see \href{https://www.mathworks.com/help/matlab/ref/inv.html}{Reference}.

The following two examples illustrate the two implementations of least squares side-by-side.

\begin{example}\label{ex:LeastSquaresComp1}
  We will compare the results of the two implementations of least squares for a $7\times 3$ coefficient matrix.
  
      \begin{verbatim}
        % Define matrix A
        A=[4 -1 0;
            2 -1 1;
            6 3 1;
            -1 1 2; 
            3 1 -2;
            -2 8 7;
            1 1 -5];
                
        % Define vector b
        b=[-2;4;1;2;-5;1;-3];
        
        % Implement least squares using our original method
        z1=inv(transpose(A)*A)*transpose(A)*b
        
        % Implement least squares using the backslash operator
        z2=A\b        
      \end{verbatim}
  
  \href{https://sagecell.sagemath.org/?z=eJyNjstqwzAQRfcC_cPdBNKAIFZeBeGFoZt-Q5qFlExiUVtyJDmUfn0l4hSyyzAwc2buPGb4oLN1hF6nYH_QcNbU-zVEhaXiDNlkgWqCLVb_eSlDKtwp1yHkoyXxjt0ERSY2h4mKcTZ73L3RMfkAw5mp93l-rSolldjkIFZlpmg_-6GjnlxCRzomxOuoA0WM0boL_Bjgg71Ypzv0lFp_4uy3qq27zVPQLg4-0rx5W2R_YvPK9tQSjD5-x07HFn6goPPD-YCsmy_zB_2HVXs=&lang=octave&interacts=eJyLjgUAARUAuQ==}{Link to code}.   
  
  \end{example}
  

\begin{example}\label{ex:LeastSquaresComp2}
We will compare the performance of two the implementations of least squares for a $2000\times 20$ coefficient matrix.

    \begin{verbatim}
      % Define matrix A
      A=rand(2000,20);
              
      % Define vector b
      b=rand(2000,1);
      
      % Start the timer
      tic
      
      % Implement least squares using our original method
      z1=inv(transpose(A)*A)*transpose(A)*b;
      
      % Stop the timer
      timeINV=toc
      
      % Start the timer
      tic
      
      % Implement least squares using the backslash operator
      z2=A\b;
      
      % Stop the timer
      timeBSlash=toc
    \end{verbatim}

\href{https://sagecell.sagemath.org/?z=eJylkD0LwkAMhveD-w9ZhCoOtat0qLi4uBScXNIa7WHvrubSIv56e_iBLi6GLCFPeF4ygTUdjSOwKGyuUGhV5IzukGRpms6zdLrUCp6l1eSFD1SLZ6i0qj7wRaQjVQqygDQEYiyxVmLqx2Zju5YsOYGWMAiES49MAfpg3Al8z-DZnIzDFixJ4w9a3Ra5cUMioyd0PlBSTGdjf83VW-y7b6-lzXaXi6__SxYvKqzPocXQgO-IcXzAGC7Li_0v-6qMFzHAHTlDb7g=&lang=octave&interacts=eJyLjgUAARUAuQ==}{Link to code}.   

Each time you run the algorithm, you will get slightly different run times for each implementation.  Most of the time, you will see that the backslash operator outperforms our initial method. 
\end{example}

The 
backslash operator ($\setminus$) can be used to solve systems of equations when a solution exists.  Here is an example.

\begin{example}\label{temp:systems1}
Consider the system
\begin{equation}
\begin{array}{ccccccccc}
      x &- &y&&&&&= &0 \\
	 2x& -&2y&+&z&+&2w&=&4\\
     & &y&&&+&w&=&0\\
     & &&&2z&+&w&=&5
    \end{array}
    \end{equation}

 (You may remember this system from \href{https://ximera.osu.edu/linearalgebrav3/LinearAlgebraInteractiveIntro/SYS-0020/main}{Augmented Matrix Notation and Elementary Row Operations}.)  We will solve this system using the backslash operator ($\setminus$).   
\begin{verbatim}
% Define the coefficient matrix A
A = [1 -1 0 0;
     2 -2 1 2;
     0 1 0 1;
     0 0 2 1];

% Define vector b
b = [0;4;0;5];

% Solve the system 
x = A \ b

% Compare the solution above to the solution obtained
% using rref
A_b=[A b];
rref(A_b)
\end{verbatim}

\href{https://sagecell.sagemath.org/?z=eJxVj8EOgjAQRO9N-g9zIdEDSUv0RDgQ_QOPaAzFok2ANaUQ_Hu7UWLc20zf7GwTHG3rBovwsGjItq1rnB0C-jp4t6CUokSBSiPVUFC5FODJkGbQyFatwM_6J1VE9CVqKZK1Y7ZNIA8jheGdKt_lKt-v0Im6-XPH-BqD7SHFErESZ04wcaD-WfsvQ90UHA2oDXGM_l0yoY6VN45Noxvu8N628TNXU1QlDJeys4nG9g1f4kXS&lang=octave&interacts=eJyLjgUAARUAuQ==}{Link to code}.

\end{example}

\begin{warning}
  While the backslash ($\setminus$) operator can be used to solve feasible systems, we urge you to be very cautious because
\begin{itemize}
  \item If a system has infinitely many solutions, the backslash ($\setminus$) operator will find one particular solution.
  \item If a system has no solutions, the backslash ($\setminus$) operator will find an approximate solution using least-squares.  
\end{itemize}
The way the answer is presented, you cannot tell the difference!
\end{warning}

\subsubsection*{$QR$-factorization}

Recall the definition of $QR$-factorization.
\begin{definition}[\ref{def:QR-factorization}]
Let $A$ be an $m \times n$ matrix with independent columns. A \dfn{QR-factorization} of $A$ expresses it as $A = QR$ where $Q$ is $m \times n$ with orthonormal columns and $R$ is an invertible and upper triangular matrix with positive diagonal entries.
\end{definition}

\begin{example}\label{ex:qrOrthonormalQ}
Find the $QR$-factorization of 
$$A=\begin{bmatrix}2 & -1 & 4 & 0\\
    3 & 1 & 1 & -2\\
    1 & 5 & -1 & 6\\
    -3 & 4 & 0 & 7\end{bmatrix}$$
and demonstrate that matrix $Q$ is orthogonal (columns are orthonormal).     
\begin{explanation}
    \begin{verbatim}
      % Define A
      A=[2 -1 4 0;
          3 1 1 -2;
          1 5 -1 6;
          -3 4 0 7];
      
      % Find the QR-factorization of A
      [Q,R]=qr(A)
      
      % Verify that the columns of Q are unit vectors
      for i=1:4
          normOfCol=norm(Q(:,i))
      end    
      
      % Verify that the columns of Q are pairwise orthogonal
      for i=1:3
          for j=i+1:4;
             dotProduct=dot(Q(:,i),Q(:,j))
          end
      end
    \end{verbatim}
  
\href{https://sagecell.sagemath.org/?z=eJyNTz1PwzAQ3SPlP9xSKRWJ1DQFpCIPEYgV0oGl6mAlNr0q9ZWLA4Jfz10TiZXnwffkex9ewJPzGBzUaVKb_RqKEjawekgTEFRQyinWMy3hVt_vZlpUugr3B-FpsoBnDB3Eo4NmV3jbRmL8sREpAHn13zf57mA-OKuXk-DNMfpvkdh41bXUj-cw6HoDlh2MASN8OrUa0sQTA5pyu5niA_H5xT9Sb3TKmmyb41KcnbQQ_DPiYpG_cHBAHI_0TsH2f0nVlKT0ZPBGouevCzqKr0zd2EYj4xyf63XSFroiTa51fgHN8l_0&lang=octave&interacts=eJyLjgUAARUAuQ==}{Link to code}.
\end{explanation}  

\end{example}

Recall the following algorithm for using $QR$-factorization to approximate eigenvalues.

\begin{algorithm}[\ref{alg:qrEig}]
Let $A$ be an invertible matrix.

\emph{Step 1}: Define $A_{1} = A$ and factor it as $A_{1} = Q_{1}R_{1}$.

\emph{Step 2}: Define $A_{2} = R_{1}Q_{1}$ and factor it as $A_{2} = Q_{2}R_{2}$.

\emph{Step 3}: Define $A_{3} = R_{2}Q_{2}$ and factor it as $A_{3} = Q_{3}R_{3}$.

 $\vdots$
 
\emph{Step $k+1$}: Define $A_{k + 1} = R_{k}Q_{k}$ (where $A_{k} = Q_{k}R_{k}$). 

Note that $A_{k + 1}$ is similar to $A_{k}$ (in fact, $A_{k+1} = R_{k}Q_{k} = (Q_{k}^{-1}A_{k})Q_{k}$), and hence each $A_{k}$ has the same eigenvalues as $A$. If the eigenvalues of $A$ are real and have distinct absolute values, the sequence of matrices $A_{1}, A_{2}, A_{3}, \dots$ converges to an upper triangular matrix with these eigenvalues on the main diagonal. 
\end{algorithm}

Compare the following example to Example \ref{QR-algortihm-2x2-025425}.

\begin{example}\label{ex:qrEig}
Use the $QR$ algorithm to approximate the eigenvalues of $A=\begin{bmatrix}1 & 1\\2 & 0\end{bmatrix}$.  

\begin{explanation}
    Here is a very simple implementation of the algorithm.  

    \begin{verbatim}
% Define A
A=[1 1;
    2 0];

% Set the desired number of iterations
iter=20;  

% Implement the algorithm. Note that we are over-writing the previous A with every iteration.
for i=1:iter
    [Q,R]=qr(A);
    A=R*Q
end    
    \end{verbatim}

\href{https://sagecell.sagemath.org/?z=eJwtjbEKwkAQRPuF_YdpAioWuZSGKw78gcQypFBuD7fwDjfn_5tDp3oMvJkOV0maBYEp-MXBjUzYM6Bfd2TqcJOK-hRE2dQkIn9eDzGUBK1i96olb0yN_dCPQLNSMah3l9b-BpfpPK_-bYdw_F8EP58mJsnxCzFCIso=&lang=octave&interacts=eJyLjgUAARUAuQ==}{Link to code}.   

Observe that the main diagonal entries are approaching the eigenvalues of $A$.
\end{explanation}
    
\end{example}

\subsection*{Octave Exercises}
\begin{problem}\label{prob_oct_1}
    Attempt to solve each of the following systems of equations using the backslash ($\setminus$) operator, and the \emph{rref} function.  Interpret your results.

    \begin{enumerate}
        \item 
        \begin{equation}
\begin{array}{ccccccccc}
      3x &- &y&+&4z&= &2 \\
	 &&y&+&z&=&1\\
     -2x&&&+&3z&=&1
    \end{array}
    \end{equation}
  
    
    What is true about this system?
    
    \begin{multipleChoice}
    \choice[correct]{The system is consistent.  Using the backslash operator produces the same result as using the \emph{rref} function.}
    \choice{The system is inconsistent.  The backslash operator gives the least squares approximation.}
    \choice{The system is consistent but has infinitely many solutions.  The backslash operator gives one particular solution.}
    \end{multipleChoice}

       \item 
    \begin{equation}
\begin{array}{ccccccccc}
      3x &- &y&+&4z&= &2 \\
	 x&-&y&+&2z&=&4\\
     -2x&+&y&-&3z&=&-3
    \end{array}
    \end{equation}

    What is true about this system?
    \begin{multipleChoice}
    \choice{The system is consistent.  Using the backslash operator produces the same result as using the \emph{rref} function.}
    \choice{The system is inconsistent.  The backslash operator gives the least squares approximation.}
    \choice[correct]{The system is consistent but has infinitely many solutions.  The backslash operator gives one particular solution.}
    \end{multipleChoice}

    \item 
    \begin{equation}
\begin{array}{ccccccccc}
      3x &- &y&+&4z&= &2 \\
	 x&-&y&+&2z&=&1\\
     -2x&+&y&-&3z&=&1
    \end{array}
    \end{equation}

    What is true about this system?
    \begin{multipleChoice}
    \choice{The system is consistent.  Using the backslash operator produces the same result as using the \emph{rref} function.}
    \choice[correct]{The system is inconsistent.  The backslash operator gives the least squares approximation.}
    \choice{The system is consistent but has infinitely many solutions.  The backslash operator gives one particular solution.}
    \end{multipleChoice}

         \end{enumerate}
\end{problem}

\begin{problem}\label{prob_oct_qr1}
Modify the code in Example \ref{ex:qrEig} to approximate the eigenvalues of $A=\begin{bmatrix} 9 & 2 & 8\\ 2 & -6 & -2\\ -8 & 2 & -5\end{bmatrix}$.  Compare your answers to the answers you got for Problem \ref{prob:3x3fromKuttler1}. 
\end{problem}

\begin{problem}\label{prob_oct_qr2}
  Modify the code in Example \ref{ex:qrEig} for the matrix $A=\begin{bmatrix} 0 & -1\\ 1 & 0 \end{bmatrix}$.  What are you observing?  Are we getting close to finding the eigenvalues of $A$?  Explain what is happening and why.
  \end{problem}

\end{document}