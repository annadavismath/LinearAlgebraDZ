\documentclass{ximera}
%% You can put user macros here
%% However, you cannot make new environments

\listfiles

\graphicspath{{./}{firstExample/}{secondExample/}}

\usepackage{tikz}
\usepackage{tkz-euclide}
\usepackage{tikz-3dplot}
\usepackage{tikz-cd}
\usetikzlibrary{shapes.geometric}
\usetikzlibrary{arrows}
%\usetkzobj{all}
\pgfplotsset{compat=1.13} % prevents compile error.

%\renewcommand{\vec}[1]{\mathbf{#1}}
\renewcommand{\vec}{\mathbf}
\newcommand{\RR}{\mathbb{R}}
\newcommand{\dfn}{\textit}
\newcommand{\dotp}{\cdot}
\newcommand{\id}{\text{id}}
\newcommand\norm[1]{\left\lVert#1\right\rVert}
 
\newtheorem{general}{Generalization}
\newtheorem{initprob}{Exploration Problem}

\tikzstyle geometryDiagrams=[ultra thick,color=blue!50!black]

%\DefineVerbatimEnvironment{octave}{Verbatim}{numbers=left,frame=lines,label=Octave,labelposition=topline}



\usepackage{mathtools}


\title{Transpose of a Matrix} \license{CC BY-NC-SA 4.0}

\begin{document}

\begin{abstract}
  \end{abstract}
\maketitle


\begin{onlineOnly}
\section*{Transpose of a Matrix}
\end{onlineOnly}

%{\color{red} This section was adopted from Kuttler matricesMatrixArithmeticTranspose.tex}

Another important operation on matrices is that of taking the \dfn{transpose}. For a matrix $A$, we denote the
\dfn{transpose of $A$} by $A^T$. Before formally defining the transpose, we explore this
operation on the following matrix.
\begin{equation*}
\begin{bmatrix}
1 & 4 \\
3 & 1 \\
2 & 6
\end{bmatrix}^{T}=
\begin{bmatrix}
1 & 3 & 2 \\
4 & 1 & 6
\end{bmatrix}
\end{equation*}

What happened? The first column became the first row and the second column
became the second row. Thus the $3\times 2$ matrix became a $2\times 3$
matrix. The number $4$ was in the first row and the second column and it
ended up in the second row and first column. 

The definition of the transpose is as follows.

\begin{definition}[The Transpose of a Matrix]\label{def:matrixtranspose}
Let $A=\begin{bmatrix} a _{ij}\end{bmatrix}$ be an $m\times n$ matrix. Then the \dfn{transpose of $A$}, denoted by $A^{T}$, is the $n\times m$
matrix given by 
\begin{equation*}
A^{T} = \begin{bmatrix} a _{ij}\end{bmatrix}^{T}= \begin{bmatrix} a_{ji} \end{bmatrix}
\end{equation*}
\end{definition}

The $( i, j)$-entry of $A$ becomes the 
$( j,i)$-entry of $A^T$. 



\begin{example}\label{ex:transposematrix}
Calculate $A^T$ for the following matrix
\begin{equation*}
A =
\begin{bmatrix}
1 & 2 & -6 \\
3 & 5 & 4
\end{bmatrix}
\end{equation*}
\begin{explanation}

\begin{equation*}
A^T = 
\begin{bmatrix}
1 & 3 \\
2 & 5 \\
-6 & 4
\end{bmatrix}
\end{equation*}

Note that $A$ is a $2 \times 3$ matrix, while $A^T$ is a $3 \times 2$ matrix. The columns of $A$ are the rows of $A^T$, and the rows of $A$ are the columns of $A^T$.
\end{explanation}
\end{example}

\begin{theorem}[Properties of the Transpose of a Matrix]\label{th:transposeproperties}
Let $A$ be an $m\times n$ matrix, $B$ an $n\times p$ matrix, and $k$ a scalar. Then
\begin{enumerate}
\item\label{item:transoftrans}
$\left(A^{T}\right)^{T} = A$
\item\label{item:matrixtranspose1}
$\left( AB\right) ^{T}=B^{T}A^{T} $ (Shoes and Socks rule)
\item\label{item:matrixtranspose2}
$\left( A+ B\right) ^{T}=A^{T}+ B^{T}$  
\item\label{item:matrixtranspose3}
$\left(kA\right)^T=kA^T$
\end{enumerate}
\end{theorem}
We will prove Property \ref{item:matrixtranspose1}.  The remaining properties are left as exercises.
\begin{proof}[Proof of Property~\ref{item:matrixtranspose1}:]
Note that $A$ and $B$ have compatible dimensions, so that $AB$ is defined and has dimensions $m\times p$.  Thus, $(AB)^T$ has dimensions $p\times m$.  On the right side of the equality, $A^T$ has dimensions $n\times m$, and $B^T$ has dimensions $p\times n$.  Therefore $B^TA^T$ is defined and has dimensions $p\times m$.  Now we know that $(AB)^T$ and $B^TA^T$ have the same dimensions.

To show that $(AB)^T=B^TA^T$ we need to show that their corresponding entries are equal. 
Recall that the $(i,j)$-entry of $AB$ is given by
the dot product of the $i^{th}$ row of $A$ and the $j^{th}$ column of $B$.  
The same dot product is also the $(j,i)$-entry of $(AB)^T$.  

The $(j,i)$-entry of $B^TA^T$ is given by the dot product of the $j^{th}$ row of $B^T$ and the $i^{th}$ column of $A^T$.  But the $j^{th}$ row of $B^T$ is has the same entries as the $j^{th}$ column of $B$, and   the $i^{th}$ column of $A^T$ has the same entries as the $i^{th}$ row of $A$.  Therefore the $(j,i)$-entry of $B^TA^T$ is also equal to the $(i,j)$-entry of $AB$.

Thus, the corresponding components of $(AB)^T$ are equal and we conclude that $(AB)^T=B^TA^T$.
\end{proof}

The transpose of a matrix is related to other important topics. Consider the following definition.  

\begin{definition}[Symmetric and Skew Symmetric Matrices]\label{def:symmetricandskewsymmetric}
An $n\times n$ matrix $A$ is said to be
\dfn{symmetric} if $A=A^{T}.$ It is said to be
\dfn{skew symmetric} if $A=-A^{T}.$
\end{definition}

We will explore these definitions in the following examples.

\begin{example}\label{ex:symmetricmatrix}
Let
\begin{equation*}
A=
\begin{bmatrix}
2 & 1 & 3 \\
1 & 5 & -3 \\
3 & -3 & 7
\end{bmatrix}
\end{equation*}
Show that $A$ is symmetric. 
\begin{explanation}
\begin{equation*}
A^{T} =
\begin{bmatrix}
2 & 1 & 3 \\
1 & 5 & -3 \\
3 & -3 & 7
\end{bmatrix}
\end{equation*}
Hence, $A = A^{T}$, so $A$ is symmetric.
\end{explanation}
\end{example}

\begin{example}\label{ex:skewsymmetricmatrix}
Let
\begin{equation*}
A=
\begin{bmatrix}
0 & 1 & 3 \\
-1 & 0 & 2 \\
-3 & -2 & 0
\end{bmatrix}
\end{equation*}
Show that $A$ is skew symmetric.
\begin{explanation} 
\begin{equation*}
A^{T} = 
\begin{bmatrix}
0 & -1 & -3\\
1 &  0 & -2\\
3 &  2 &  0
\end{bmatrix}
\end{equation*}

Each entry of $A^T$ is equal to $-1$ times the same entry of $A$. 
Hence, $A^{T} = - A$ and so by Definition \ref{def:symmetricandskewsymmetric}, $A$ is skew symmetric. 
\end{explanation}
\end{example}

A special case of a symmetric matrix is a \dfn{diagonal matrix}.  A diagonal matrix is a square matrix whose entries outside of the main diagonal are all zero.  The identity matrix $I$ is a diagonal matrix.  Here is another example.
$$\begin{bmatrix}2&0&0&0\\0&-3&0&0\\0&0&1&0\\0&0&0&4\end{bmatrix}$$

\section*{Practice Problems}


\begin{problem}\label{prob:transpropsproofs} Prove Properties \ref{item:transoftrans}, \ref{item:matrixtranspose2} and \ref{item:matrixtranspose3} of Theorem \ref{th:transposeproperties}.
\end{problem}

\begin{problem}\label{prob:ATtimesAdimensions} Let $A$ be an arbitrary matrix.  What can you say about the dimensions of the product $A^TA$?
\end{problem}

\emph{Problems \ref{prob:symmetricclassification1}-\ref{prob:symmetricclassification3}}

Classify each matrix as symmetric, skew symmetric, or neither.

\begin{problem}\label{prob:symmetricclassification1}
$$\begin{bmatrix}
0 & -1 & -3\\
1 & -1 & -2\\
3 &  2 &  0
\end{bmatrix}$$
\begin{multipleChoice}
 \choice{Symmetric}
 \choice{Skew symmetric}
     \choice[correct]{Neither}
   \end{multipleChoice}
\end{problem}

\begin{problem}\label{prob:symmetricclassification2}
$$\begin{bmatrix}
0 & 1 & -3\\
-1 & 0 & -2\\
3 &  2 &  0
\end{bmatrix}$$
\begin{multipleChoice}
 \choice{Symmetric}
 \choice[correct]{Skew symmetric}
     \choice{Neither}
   \end{multipleChoice}
\end{problem}

\begin{problem}\label{prob:symmetricclassification3}
$$\begin{bmatrix}
1 & 1 & 3\\
1 & -1 & 2\\
3 &  2 &  4
\end{bmatrix}$$
\begin{multipleChoice}
 \choice[correct]{Symmetric}
 \choice{Skew symmetric}
     \choice{Neither}
   \end{multipleChoice}
\end{problem}


\begin{problem}\label{prob:4by4symmetricex}
Give your own example of a $4\times 4$ skew symmetric matrix.
\end{problem}

\begin{problem} \label{prob:maindiagskewsymm}
Make a conjecture about the main diagonal entries of a skew symmetric matrix.  Prove your conjecture.
\end{problem}

\section*{Text Source}
The text in this section is an adaptation of Section 2.1 of Ken Kuttler's \href{https://open.umn.edu/opentextbooks/textbooks/a-first-course-in-linear-algebra-2017}{\it A First Course in Linear Algebra}. (CC-BY)

Ken Kuttler, {\it  A First Course in Linear Algebra}, Lyryx 2017, Open Edition, p. 68-70.

\end{document} 