\documentclass{ximera}
%% You can put user macros here
%% However, you cannot make new environments

\listfiles

\graphicspath{{./}{firstExample/}{secondExample/}}

\usepackage{tikz}
\usepackage{tkz-euclide}
\usepackage{tikz-3dplot}
\usepackage{tikz-cd}
\usetikzlibrary{shapes.geometric}
\usetikzlibrary{arrows}
%\usetkzobj{all}
\pgfplotsset{compat=1.13} % prevents compile error.

%\renewcommand{\vec}[1]{\mathbf{#1}}
\renewcommand{\vec}{\mathbf}
\newcommand{\RR}{\mathbb{R}}
\newcommand{\dfn}{\textit}
\newcommand{\dotp}{\cdot}
\newcommand{\id}{\text{id}}
\newcommand\norm[1]{\left\lVert#1\right\rVert}
 
\newtheorem{general}{Generalization}
\newtheorem{initprob}{Exploration Problem}

\tikzstyle geometryDiagrams=[ultra thick,color=blue!50!black]

%\DefineVerbatimEnvironment{octave}{Verbatim}{numbers=left,frame=lines,label=Octave,labelposition=topline}



\usepackage{mathtools}


\title{Essential Vocabulary} \license{CC BY-NC-SA 4.0}



\begin{document}
\begin{abstract}
\end{abstract}
\maketitle


\begin{onlineOnly}
\section*{Essential Vocabulary}
Here is a  \href{https://quizlet.com/880920639/chapter-5-vocabulary-flash-cards/?i=y06sd&x=1jqt}{link to a list of these terms on Quizlet}
\end{onlineOnly}

\begin{tikzpicture}[scale=1]
   \filldraw[teal, opacity=0.3](0,0)--(20,0)--(20,0.1)--(0,0.1)--cycle;
 \end{tikzpicture}

Basis
\begin{expandable}
A set $\mathcal{S}$ of vectors is called a \dfn{basis} of $\RR^n$ (or a basis of a subspace $V$ of $\RR^n$) provided that 
\begin{enumerate}
\item 
$\mbox{span}(\mathcal{S})=\RR^n$ (or $V$)
\item 
$\mathcal{S}$ is linearly independent.
\end{enumerate}
\end{expandable}

\begin{tikzpicture}[scale=1]
   \filldraw[teal, opacity=0.3](0,0)--(20,0)--(20,0.1)--(0,0.1)--cycle;
 \end{tikzpicture}

Closed under addition
\begin{expandable}
    A set $V$ is said to be \dfn{closed under addition} if for each element $\vec{u} \in V$ and $\vec{v} \in V$ the sum $\vec{u}+\vec{v}$ is also in $V$.
\end{expandable}

\begin{tikzpicture}[scale=1]
   \filldraw[teal, opacity=0.3](0,0)--(20,0)--(20,0.1)--(0,0.1)--cycle;
 \end{tikzpicture}

 Closed under scalar multiplication
\begin{expandable}
    A set $V$ is said to be \dfn{closed under scalar multiplication} if for each element $\vec{v} \in V$  and for each scalar $k \in \RR$ the product $k\vec{v}$ is also in $V$.
\end{expandable}

\begin{tikzpicture}[scale=1]
   \filldraw[teal, opacity=0.3](0,0)--(20,0)--(20,0.1)--(0,0.1)--cycle;
 \end{tikzpicture}

Column space of a matrix
\begin{expandable}
    Let $A$ be an $m\times n$ matrix.  The \dfn{column space} of $A$, denoted by $\mbox{col}(A)$, is the subspace of $\RR^m$ spanned by the columns of $A$.
\end{expandable}

\begin{tikzpicture}[scale=1]
   \filldraw[teal, opacity=0.3](0,0)--(20,0)--(20,0.1)--(0,0.1)--cycle;
 \end{tikzpicture}

Coordinate vector with respect to a basis
\begin{expandable}
    Let $\mathcal{B} = \{\vec{v}_1, \vec{v}_2,\ldots ,\vec{v}_k\}$ be an \underline{ordered} basis.  Then the \dfn{coordinate vector} $\vec{v}$ is the  column vector $\begin{bmatrix}c_1\\ c_2\\ \vdots \\c_k\end{bmatrix}$ such that $\vec{v} = c_1\vec{v}_1+c_2\vec{v}_2+\ldots +c_p\vec{v}_k$.
\end{expandable}

\begin{tikzpicture}[scale=1]
   \filldraw[teal, opacity=0.3](0,0)--(20,0)--(20,0.1)--(0,0.1)--cycle;
 \end{tikzpicture}

Dimension
\begin{expandable}
    Let $V$ be a subspace of $\RR^n$.  The \dfn{dimension} of $V$ is the number, $m$, of elements in any basis of $V$.  We write
$$\mbox{dim}(V)=m$$
\end{expandable}

\begin{tikzpicture}[scale=1]
   \filldraw[teal, opacity=0.3](0,0)--(20,0)--(20,0.1)--(0,0.1)--cycle;
 \end{tikzpicture}

Null space of a matrix
\begin{expandable}
    Let $A$ be an $m\times n$ matrix.  The \dfn{null space} of $A$, denoted by $\mbox{null}(A)$, is the set of all vectors $\vec{x}$ in $\RR^n$ such that $A\vec{x}=\vec{0}$.   
    It is a subspace of $\RR^n$.
\end{expandable}

\begin{tikzpicture}[scale=1]
   \filldraw[teal, opacity=0.3](0,0)--(20,0)--(20,0.1)--(0,0.1)--cycle;
 \end{tikzpicture}

Nullity of a matrix
\begin{expandable}
    Let $A$ be a matrix.  The dimension of the null space of $A$ is called the \dfn{nullity} of $A$.
$$\mbox{dim}\Big(\mbox{null}(A)\Big)=\mbox{nullity}(A)$$
\end{expandable}

\begin{tikzpicture}[scale=1]
   \filldraw[teal, opacity=0.3](0,0)--(20,0)--(20,0.1)--(0,0.1)--cycle;
 \end{tikzpicture}

Ordered basis
\begin{expandable}
    A basis in which the elements appear in a specific fixed order.  Establishing an order is necessary because a coordinate vector with respect to a given basis relies on the order in which the basis elements appear.
\end{expandable}

\begin{tikzpicture}[scale=1]
   \filldraw[teal, opacity=0.3](0,0)--(20,0)--(20,0.1)--(0,0.1)--cycle;
 \end{tikzpicture}

Rank of a matrix
\begin{expandable}
    Let $A$ be a matrix.  The dimension of the row space of $A$ is called the \dfn{rank} of $A$.
$$\mbox{dim}\Big(\mbox{row}(A)\Big)=\mbox{rank}(A)$$
\end{expandable}

\begin{tikzpicture}[scale=1]
   \filldraw[teal, opacity=0.3](0,0)--(20,0)--(20,0.1)--(0,0.1)--cycle;
 \end{tikzpicture}

Rank-Nullity theorem
\begin{expandable}
    Let $A$ be an $m\times n$ matrix.  Then 
$$\mbox{rank}(A)+\mbox{nullity}(A)=n$$
\end{expandable}

\begin{tikzpicture}[scale=1]
   \filldraw[teal, opacity=0.3](0,0)--(20,0)--(20,0.1)--(0,0.1)--cycle;
 \end{tikzpicture}

Row space of a matrix
\begin{expandable}
    Let $A$ be an $m\times n$ matrix.  The \dfn{row space} of $A$, denoted by $\mbox{row}(A)$, is the subspace of $\RR^n$ spanned by the rows of $A$.
\end{expandable}

\begin{tikzpicture}[scale=1]
   \filldraw[teal, opacity=0.3](0,0)--(20,0)--(20,0.1)--(0,0.1)--cycle;
 \end{tikzpicture}

Subspace
\begin{expandable}
    Suppose that $V$ is a nonempty subset of $\RR^n$ that is closed under addition and closed under scalar multiplication.  Then $V$ is a \dfn{subspace} of $\RR^n$.
\end{expandable}
\begin{tikzpicture}[scale=1]
   \filldraw[teal, opacity=0.3](0,0)--(20,0)--(20,0.1)--(0,0.1)--cycle;
 \end{tikzpicture}

\end{document}
