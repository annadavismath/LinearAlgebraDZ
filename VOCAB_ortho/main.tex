\documentclass{ximera}
%% You can put user macros here
%% However, you cannot make new environments

\listfiles

\graphicspath{{./}{firstExample/}{secondExample/}}

\usepackage{tikz}
\usepackage{tkz-euclide}
\usepackage{tikz-3dplot}
\usepackage{tikz-cd}
\usetikzlibrary{shapes.geometric}
\usetikzlibrary{arrows}
%\usetkzobj{all}
\pgfplotsset{compat=1.13} % prevents compile error.

%\renewcommand{\vec}[1]{\mathbf{#1}}
\renewcommand{\vec}{\mathbf}
\newcommand{\RR}{\mathbb{R}}
\newcommand{\dfn}{\textit}
\newcommand{\dotp}{\cdot}
\newcommand{\id}{\text{id}}
\newcommand\norm[1]{\left\lVert#1\right\rVert}
 
\newtheorem{general}{Generalization}
\newtheorem{initprob}{Exploration Problem}

\tikzstyle geometryDiagrams=[ultra thick,color=blue!50!black]

%\DefineVerbatimEnvironment{octave}{Verbatim}{numbers=left,frame=lines,label=Octave,labelposition=topline}



\usepackage{mathtools}


\title{Essential Vocabulary} \license{CC BY-NC-SA 4.0}



\begin{document}
\begin{abstract}
\end{abstract}
\maketitle


\begin{onlineOnly}
\section*{Essential Vocabulary}
Here is a  \href{https://quizlet.com/906041657/chapter-9-vocabulary-flash-cards/?i=y06sd&x=1jqt}{link to a list of these terms on Quizlet}
\end{onlineOnly}

\begin{tikzpicture}[scale=1]
   \filldraw[teal, opacity=0.3](0,0)--(20,0)--(20,0.1)--(0,0.1)--cycle;
 \end{tikzpicture}

Fundamental subspaces of a matrix
\begin{expandable}
    $\mbox{null}(A)$ is the orthogonal complement of $\mbox{row}(A)$, and $\mbox{null}(A^T)$ is the orthogonal complement of $\mbox{col}(A)$.
\end{expandable}

\begin{tikzpicture}[scale=1]
   \filldraw[teal, opacity=0.3](0,0)--(20,0)--(20,0.1)--(0,0.1)--cycle;
 \end{tikzpicture}

Gram-Schmidt process
\begin{expandable}
    An iterative process which constructs an orthogonal basis for a subspace. The idea is to build the orthogonal set one vector at a time, by taking a vector not in the span of the vectors in the current iteration of the set, and subtracting its orthogonal projection onto each of those vectors.
\end{expandable}

\begin{tikzpicture}[scale=1]
   \filldraw[teal, opacity=0.3](0,0)--(20,0)--(20,0.1)--(0,0.1)--cycle;
 \end{tikzpicture}

Orthogonal Basis
\begin{expandable}
    A set of orthogonal vectors that spans a subspace. (Any orthogonal set of vectors must be linearly independent.)
\end{expandable}

\begin{tikzpicture}[scale=1]
   \filldraw[teal, opacity=0.3](0,0)--(20,0)--(20,0.1)--(0,0.1)--cycle;
 \end{tikzpicture}

Orthogonal complement of a subspace
\begin{expandable}
    If $W$ is a subspace, we define the \dfn{orthogonal complement} $W^\perp$ as the set of all vectors orthogonal to every vector in $W$, i.e.,
$$
W^\perp = \{\vec{x} \in\RR^n \mid \vec{x} \dotp \vec{y} = 0 \mbox{ for all } \vec{y} \in W\}
$$
\end{expandable}

\begin{tikzpicture}[scale=1]
   \filldraw[teal, opacity=0.3](0,0)--(20,0)--(20,0.1)--(0,0.1)--cycle;
 \end{tikzpicture}

Orthogonal Decomposition Theorem 
\begin{expandable}
    Let $W$ be a subspace of $\RR^n$ and let $\vec{x} \in \RR^n$.  Then there exist unique vectors $\vec{w} \in W$ and $\vec{w}^\perp \in W^\perp$ such that $\vec{x} = \vec{w} + \vec{w}^\perp$.
\end{expandable}

\begin{tikzpicture}[scale=1]
   \filldraw[teal, opacity=0.3](0,0)--(20,0)--(20,0.1)--(0,0.1)--cycle;
 \end{tikzpicture}

Orthogonal matrices
\begin{expandable}
    An $n \times n$ matrix $Q$ is called an \dfn{orthogonal matrix} if its columns form an orthonormal set.  This will happen if and only if its rows form an orthonormal set.  Note also that $Q$ is an orthogonal matrix if and only if it is an invertible matrix such that $Q^{-1}=Q^{T}$.
\end{expandable}

\begin{tikzpicture}[scale=1]
   \filldraw[teal, opacity=0.3](0,0)--(20,0)--(20,0.1)--(0,0.1)--cycle;
 \end{tikzpicture}

Orthogonal projection onto a subspace
\begin{expandable}
    Let $W$ be a subspace of $\RR^n$ with orthogonal basis $\{\vec{f}_{1}, \vec{f}_{2}, \dots, \vec{f}_{m}\}$. If $\vec{x}$ is in $\RR^n$, the vector
$$
\vec{w}=\mbox{proj}_W(\vec{x}) = \mbox{proj}_{\vec{f}_1}\vec{x} + \mbox{proj}_{\vec{f}_2}\vec{x} + \dots + \mbox{proj}_{\vec{f}_m}\vec{x}
$$
is called the \dfn{orthogonal projection} of $\vec{x}$ onto $W$. 
\end{expandable}

\begin{tikzpicture}[scale=1]
   \filldraw[teal, opacity=0.3](0,0)--(20,0)--(20,0.1)--(0,0.1)--cycle;
 \end{tikzpicture}

Orthogonal set of vectors
\begin{expandable}
    Let $\{ \vec{v}_1, \vec{v}_2, \cdots, \vec{v}_k \}$ be a set of nonzero
vectors in $\RR^n$. Then this set is called an
\dfn{orthogonal set} if 
$\vec{v}_i \dotp \vec{v}_j = 0$ for all $i \neq j$.
Moreover, if $\norm{\vec{v}_i}=1$ for $i=1,\ldots,m$ (i.e. each vector in the set is a unit vector), we say the set of vectors is an \dfn{orthonormal set}.
\end{expandable}

\begin{tikzpicture}[scale=1]
   \filldraw[teal, opacity=0.3](0,0)--(20,0)--(20,0.1)--(0,0.1)--cycle;
 \end{tikzpicture}

Orthogonally diagonalizable matrix
\begin{expandable}
    An $n \times n$ matrix $A$ is said to be \dfn{orthogonally diagonalizable} if an orthogonal matrix $Q$ can be found such that  $Q^{-1}AQ = Q^{T}AQ$ is diagonal.
\end{expandable}

\begin{tikzpicture}[scale=1]
   \filldraw[teal, opacity=0.3](0,0)--(20,0)--(20,0.1)--(0,0.1)--cycle;
 \end{tikzpicture}

Orthonormal basis
\begin{expandable}
    A set of orthonormal vectors that spans a subspace. (Any orthogonal set of vectors must be linearly independent by Theorem \ref{orthbasis}.)
\end{expandable}

\begin{tikzpicture}[scale=1]
   \filldraw[teal, opacity=0.3](0,0)--(20,0)--(20,0.1)--(0,0.1)--cycle;
 \end{tikzpicture}

Orthonormal set of vectors
\begin{expandable}
    Let $\{ \vec{v}_1, \vec{v}_2, \cdots, \vec{v}_k \}$ be a set of nonzero
vectors in $\RR^n$. Then this set is called an
\dfn{orthogonal set} if 
$\vec{v}_i \dotp \vec{v}_j = 0$ for all $i \neq j$.
Moreover, if $\norm{\vec{v}_i}=1$ for $i=1,\ldots,m$ (i.e. each vector in the set is a unit vector), we say the set of vectors is an \dfn{orthonormal set}.
\end{expandable}

\begin{tikzpicture}[scale=1]
   \filldraw[teal, opacity=0.3](0,0)--(20,0)--(20,0.1)--(0,0.1)--cycle;
 \end{tikzpicture}

Properties of orthogonal matrices
\begin{expandable}
If $Q$ is an orthogonal matrix, then...
\begin{enumerate}
    \item $Q^{-1} = Q^T$ is orthogonal,

    \item $\mbox{det} Q = \pm 1$,

    \item if $\lambda$ is an eigenvalue of $Q$, then $|\lambda|=1$,

    \item the product of $Q$ with any other orthogonal matrix will be an orthogonal matrix (i.e. orthogonal matrices are closed under matrix multiplication), 

    and

    \item $Q$ is a length-preserving and angle-preserving linear transformation.

\end{enumerate}
\end{expandable}

\begin{tikzpicture}[scale=1]
   \filldraw[teal, opacity=0.3](0,0)--(20,0)--(20,0.1)--(0,0.1)--cycle;
 \end{tikzpicture}

QR factorization
\begin{expandable}
    Let $A$ be an $m \times n$ matrix with independent columns. A \dfn{QR-factorization} of $A$ expresses it as $A = QR$ where $Q$ is $m \times n$ with orthonormal columns and $R$ is an invertible and upper triangular matrix with positive diagonal entries.
\end{expandable}

\begin{tikzpicture}[scale=1]
   \filldraw[teal, opacity=0.3](0,0)--(20,0)--(20,0.1)--(0,0.1)--cycle;
 \end{tikzpicture}

Rank-Nullity Theorem for matrices (\ref{th:matrixranknullity})
\begin{expandable}
    Let $A$ be an $m\times n$ matrix.  Then 
$$\mbox{rank}(A)+\mbox{nullity}(A)=n.$$
This implies that the dimension of a subspace of $\RR^n$ plus the dimension of its orthogonal complement equals $n$.
\end{expandable}

\begin{tikzpicture}[scale=1]
   \filldraw[teal, opacity=0.3](0,0)--(20,0)--(20,0.1)--(0,0.1)--cycle;
 \end{tikzpicture}

Spectral Theorem
\begin{expandable}
    If $A$ is a real $n \times n$ matrix, then $A$ is symmetric if an only if $A$ is orthogonally diagonalizable.
\end{expandable}

\begin{tikzpicture}[scale=1]
   \filldraw[teal, opacity=0.3](0,0)--(20,0)--(20,0.1)--(0,0.1)--cycle;
 \end{tikzpicture}
 
\end{document}


