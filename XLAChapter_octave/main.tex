\documentclass{ximera}
%% You can put user macros here
%% However, you cannot make new environments

\listfiles

\graphicspath{{./}{firstExample/}{secondExample/}}

\usepackage{tikz}
\usepackage{tkz-euclide}
\usepackage{tikz-3dplot}
\usepackage{tikz-cd}
\usetikzlibrary{shapes.geometric}
\usetikzlibrary{arrows}
%\usetkzobj{all}
\pgfplotsset{compat=1.13} % prevents compile error.

%\renewcommand{\vec}[1]{\mathbf{#1}}
\renewcommand{\vec}{\mathbf}
\newcommand{\RR}{\mathbb{R}}
\newcommand{\dfn}{\textit}
\newcommand{\dotp}{\cdot}
\newcommand{\id}{\text{id}}
\newcommand\norm[1]{\left\lVert#1\right\rVert}
 
\newtheorem{general}{Generalization}
\newtheorem{initprob}{Exploration Problem}

\tikzstyle geometryDiagrams=[ultra thick,color=blue!50!black]

%\DefineVerbatimEnvironment{octave}{Verbatim}{numbers=left,frame=lines,label=Octave,labelposition=topline}



\usepackage{mathtools}


\title{Octave Tutorial} \license{CC BY-NC-SA 4.0}
\begin{document}
\begin{abstract}
\end{abstract}
\maketitle
\section*{Why Octave?}

In writing \textit{Linear Algebra, An Interactive Introduction}, we drew upon many years of experience teaching elementary linear algebra.  We have found that when students experiment with scientific computing software such as \href{https://www.mathworks.com/products/matlab.html}{MATLAB}, the subject comes alive!  Students can quickly recognize patterns, form conjectures, and test their theories.  

We recognize that not every learning institution has the funding to purchase software for student use.  For this reason we chose to use Octave as our computational tool.  \href{https://en.wikipedia.org/wiki/GNU_Octave}{GNU Octave} is a programming language compatible with MATLAB.  \href{https://www.octave.org/}{GNU Octave} can be downloaded for free at \href{https://www.octave.org/download}{https://www.octave.org/download} or executed in \href{https://sagecell.sagemath.org/}{SageCells} online.  Online SageCells and a desktop version of Octave can be equally effective in enhancing the experience in linear algebra.

The purpose of this Octave tutorial is to gather commands and examples useful for this linear algebra course into one place.  Students and instructors may find this to be a useful reference as they proceed through the course. Please refer to the \href{https://octave.org/support}{Octave manual} for additional support.

\section*{How to Use SageCells}
We present code in two different ways.  First, the code appears as text within the narrative, like this:

\begin{verbatim}
    disp('Hello, world!');
\end{verbatim}

This code can be copied and pasted into MATLAB or Octave.  Second, we provide permalinks to our code in SageCells, like this:

\href{https://sagecell.sagemath.org/?z=eJxLySwu0FD3SM3JyddRKM8vyklRVNe0BgBW7AcU&lang=octave&interacts=eJyLjgUAARUAuQ==}{Link to code}.

Users should feel free to modify our code.  To save their work, users should generate their own permalink.  Permalinks should be stored to access the code at a later time and to share the code with others.  Note that every time the code is modified a new permalink must be generated to reflect the changes.

When writing Octave code in SageCells from scratch, make sure to choose \emph{Octave} as your language from the drop down menu below the cell.

This 3:00 video explains how to use permalinks.  The key step is just past the 2:15 mark.

\youtube{NhujjT-p2D0}



\end{document}