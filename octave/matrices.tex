\documentclass{ximera}
%% You can put user macros here
%% However, you cannot make new environments

\listfiles

\graphicspath{{./}{firstExample/}{secondExample/}}

\usepackage{tikz}
\usepackage{tkz-euclide}
\usepackage{tikz-3dplot}
\usepackage{tikz-cd}
\usetikzlibrary{shapes.geometric}
\usetikzlibrary{arrows}
%\usetkzobj{all}
\pgfplotsset{compat=1.13} % prevents compile error.

%\renewcommand{\vec}[1]{\mathbf{#1}}
\renewcommand{\vec}{\mathbf}
\newcommand{\RR}{\mathbb{R}}
\newcommand{\dfn}{\textit}
\newcommand{\dotp}{\cdot}
\newcommand{\id}{\text{id}}
\newcommand\norm[1]{\left\lVert#1\right\rVert}
 
\newtheorem{general}{Generalization}
\newtheorem{initprob}{Exploration Problem}

\tikzstyle geometryDiagrams=[ultra thick,color=blue!50!black]

%\DefineVerbatimEnvironment{octave}{Verbatim}{numbers=left,frame=lines,label=Octave,labelposition=topline}



\usepackage{mathtools}


\title{Matrices} \license{CC BY-NC-SA 4.0}
\begin{document}
\begin{abstract}
\end{abstract}
\maketitle

The templates in this section provide sample Octave code for basic matrix operations.  You can access our code through the link at the bottom of each template.  Feel free to modify the code and experiment to learn more!  Alternatively, go to the \href{https://sagecell.sagemath.org/}{Sage Math Cell Webpage}, copy the code below into the cell, select OCTAVE as the language, and press EVALUATE. 

\begin{template}\label{temp:matrixOps}

\begin{verbatim}
% Define matrices A, B, and C
A = [1 -1 0 0;
     2 -2 1 2;
     0 1 0 1];
     
B = [2 3 -1 4;
     1 -1 2 -2;
     1 -2 0 3];   
     
C = [1 -2 1 4;
     2 -2 0 1;
     -2 1 1 0;
     1 3 -2 -1];     
     
% Find A+B
sum=A+B

% Find the transpose of A
% Copying and pasting A'  may not work because the prime comes through
% in a different font.
A_trans1=A'
% we can also use the transpose function
A_trans2=transpose(A)

% Find AC
product=A*C

% Find the inverse of C
inv_C=inv(C)

% Find the determinant of C
det_C=det(C)
\end{verbatim}

\href{https://sagecell.sagemath.org/?z=eJxdUEFuwyAQvFvyH-YSNWkTyZDcIg6Yqp-oogjZWEGqIcK4_X4XartOOKyY2ZmdhQ3eTWedQa9jsI0ZIPeo99CuhSoLCYFPhgNDhepcFkiH48DBwGdcIbXZZcZlUScbxzEZTzOdxyTviuBkPF7O2TR51RSZIk4PkZQx49xl_yuxlEUilmetpm3wYekp8q0ui2HsRb4sdLwZxKDdcPeDge8g6cnXzDAhX5Lux6DRDvpr8BhJ9GjpRtdE691i42JpbuVuFSXpN-_Bt2MThXxVT0tY923C3wrUInRVgupW7Z6ErYkm9NZpFycxMSSmSuJfcXxtIA==&lang=octave&interacts=eJyLjgUAARUAuQ==}{Link to code}.
\end{template}

\begin{template}\label{temp:id_matrix}
Sometimes it is useful to generate a commonly used matrix or a vector (e.g. identity matrix, zero vector). 
\begin{verbatim}
% 5 by 5 identity matrix
I_5=eye(5)

% 5 by 5 zero matrix
O_5=zeros(5)

% The zero vector in R^5
O_col=zeros(5,1)

% Row of zeros
O_row=zeros(1,5)

% Non-square zero matrix
O_block=zeros(2,3)

% Matrix filled with 1s
One_block=ones(2,3)

% How would you generate a matrix filled with 3s?
\end{verbatim}

\href{https://sagecell.sagemath.org/?z=eJxdzj0PgjAQBuCdhP9wC4kmOCBhNK46qAlx1vBxSGPtxbZY66-XYiHqcsPdc-9dBBmUti-sRqGZtnArtGTPMNiesxVanGXzMAiDaHQvlDSZQ29cQ03q2OKHPLDSJIEJyE-ZkxXx0caJ1zkZoGZYUM5IMt4k8Zi4J7FQ966Q-H-75FRdvV_Gqfe7YQ4N4xxrMEy3kLhwgX6BBH77Tf-DoY7XYKmDCwqUhUYo_KGfoFSt342kXHo=&lang=octave&interacts=eJyLjgUAARUAuQ==}{Link to code}.
\end{template}

\begin{template}\label{temp:randMat}
    When doing computational experiments you might need to create large matrices.  If the type of matrix is irrelevant, you may want to use random matrices.

    \begin{verbatim}
% Generate a random 3 by 3 matrix A with real-number values between 0 and 1
A=rand(3,3)

% Generate a random 4 by 5 matrix B with integer values between -10 and 25
B=randi([-10,25],4,5)
    \end{verbatim}

\href{https://sagecell.sagemath.org/?z=eJxtzLEKwjAUheE9kHc4S6GFFGzTjA7t0ocQhxu8aKCJEFNb397U4qTLGX44X4GRA0dKDEKkcLl7aNhXHk8puhU9FpduiExTHWZvOeJJ08wPWE4Lc8AB-YdGiv64CaVWupJCiuKP3W22-drDbruQ-Prr1s0ut0aK4UO78pSjas1ZdcpUb3XOOqE=&lang=octave&interacts=eJyLjgUAARUAuQ==}{Link to code}.    
\end{template}

\end{document}