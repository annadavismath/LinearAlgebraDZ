\documentclass{ximera}
%% You can put user macros here
%% However, you cannot make new environments

\listfiles

\graphicspath{{./}{firstExample/}{secondExample/}}

\usepackage{tikz}
\usepackage{tkz-euclide}
\usepackage{tikz-3dplot}
\usepackage{tikz-cd}
\usetikzlibrary{shapes.geometric}
\usetikzlibrary{arrows}
%\usetkzobj{all}
\pgfplotsset{compat=1.13} % prevents compile error.

%\renewcommand{\vec}[1]{\mathbf{#1}}
\renewcommand{\vec}{\mathbf}
\newcommand{\RR}{\mathbb{R}}
\newcommand{\dfn}{\textit}
\newcommand{\dotp}{\cdot}
\newcommand{\id}{\text{id}}
\newcommand\norm[1]{\left\lVert#1\right\rVert}
 
\newtheorem{general}{Generalization}
\newtheorem{initprob}{Exploration Problem}

\tikzstyle geometryDiagrams=[ultra thick,color=blue!50!black]

%\DefineVerbatimEnvironment{octave}{Verbatim}{numbers=left,frame=lines,label=Octave,labelposition=topline}



\usepackage{mathtools}


\title{Least Squares} \license{CC BY-NC-SA 4.0}
\begin{document}
\begin{abstract}
\end{abstract}
\maketitle
\section*{Least Squares}

The templates in this section provide code samples for implementing the least squares method. You can access our Octave code through the link at the bottom of each template.  Feel free to modify the code and experiment to learn more!  Alternatively, go to the \href{https://sagecell.sagemath.org/}{Sage Math Cell Webpage}, copy the code below into the cell, select OCTAVE as the language, and press EVALUATE.  

%When implementing least squares by hand, you probably utilize the process established by the following theorem.

The following theorem establishes one way to implement least squares.

\begin{theorem}[\ref{th:bestApprox}]
    Let $A$ be an $m\times n$ matrix, let $\vec{b}$ be a column vector in $\RR^m$.  Consider the matrix equation
    $$A\vec{x}=\vec{b}$$
    \begin{enumerate}
        \item Any solution $\vec{z}$ to the normal equations
        $$\left(A^TA\right)\vec{z}=A^T\vec{b}$$
        is a best approximation to a solution to $A\vec{x}=\vec{b}$ in the sense that $\norm{\vec{b}-A\vec{z}}$ is minimized.
        \item If the columns of $A$ are linearly independent, then $A^TA$ is invertible and $\vec{z}$ is given uniquely by
        $$\vec{z}=\left(A^TA\right)^{-1}A^T\vec{b}$$
    \end{enumerate}
    \end{theorem}

We can implement this process as follows.

\begin{template}\label{temp:leastSq}
    We show how to find a least squares solution to 

    \begin{verbatim}
% Define matrix A
A=[9 -3 3;
    1 -1 1;
    4 0 1;
    1 1 1; 
    9 -1 2];
        
% Define vector b
b=[3;1;1;2;4];

% Now we solve for z
z=inv(transpose(A)*A)*transpose(A)*b
    \end{verbatim}

\href{https://sagecell.sagemath.org/?z=eJxTVXBJTcvMS1XITSwpyqxQcOTlcrSNtlTQNVYwtublUgACQwVdQwVDKMdEwQDONgRBawUIxxKkyigWKgUCvFyqMMPLUpNL8osUkni5kmyjja0NgdDI2gSkGKTIL79coTxVoTg_pyxVIQ2oroqXq8o2M69Mo6QoMa-4IL84VcNRUwuIUPhJAN_XLTM=&lang=octave&interacts=eJyLjgUAARUAuQ==}{Link to code}.    
\end{template}

\begin{warning}
    Finding matrix inverses is both computationally expensive and unstable.  For this reason, various factoring/decomposition techniques such as $LU$ and $QR$ factorizations are used.  
\end{warning}

The 
backslash operator ($\setminus$) can be used to implement least squares in Octave.  Depending on the type of matrix, backslash uses different techniques to compute the least squares solution more efficiently.  The following template will allow you to compare answers and computing times for the two implementations of least squares.

\begin{template}\label{temp:LeastSquaresComp}
We will compare the performance of two implementations of least squares for a $7\times 3$ coefficient matrix.

    \begin{verbatim}
% Define matrix A
A=[9 -3 3;
    1 -1 1;
    4 0 1;
    1 1 1; 
    9 -1 2;
    -2 8 7;
    1 1 -5];
        
% Define vector b
b=[3;1;1;2;4;1;-3];

% Start the timer
tic

% Implement least squares using our original method
z1=inv(transpose(A)*A)*transpose(A)*b

% Stop the timer
toc

% Start the timer
tic

z2=A\b

% Stop the timer
toc
    \end{verbatim}

\href{https://sagecell.sagemath.org/?z=eJx9TkFqwzAQvAv0h7kE0oKgslPaYHww9NJzjkkPsrtNBJbkSusQ8vpYxCnNod1Zlh12htkF3ujLeoIzHO0JjRRNvV1DlSgrKTCVhtLQM1nh6WfXGRWuZJ1VxXxRBV7x8kumnj9mlkuKxS32SB2HiFaKtt6WlZ5QVKtpqjI7snLDJjL4QGDrKErBtrte3t3QkyPP6MkkRvoeTaSEMVm_RxgjQrR7600PR3wIn1KcdW39ccnR-DSERMvm4XHqO97ecsNwFxu6_x86F3Wz-9t8Af-UXbs=&lang=octave&interacts=eJyLjgUAARUAuQ==}{Link to code}.   

Each time you run the algorithm, you will get slightly different run times for each implementation.  Most of the time, you will see that the backslash operator outperforms our initial method. 
\end{template}



The 
backslash operator ($\setminus$) can also be used to solve systems of equations when a solution exists.  Here is an example.

\begin{example}\label{temp:systems1}
Consider the system
\begin{equation}
\begin{array}{ccccccccc}
      x &- &y&&&&&= &0 \\
	 2x& -&2y&+&z&+&2w&=&4\\
     & &y&&&+&w&=&0\\
     & &&&2z&+&w&=&5
    \end{array}
    \end{equation}

 (You may remember this system from \href{https://ximera.osu.edu/linearalgebrav3/LinearAlgebraInteractiveIntro/SYS-0020/main}{Augmented Matrix Notation and Elementary Row Operations}.)  We will solve this system with Octave.   
\begin{verbatim}
% Define the coefficient matrix A
A = [1 -1 0 0;
     2 -2 1 2;
     0 1 0 1;
     0 0 2 1];

% Define vector b
b = [0;4;0;5];

% Solve the system 
x = A \ b
\end{verbatim}

\href{https://sagecell.sagemath.org/?z=eJxFjsEKwjAQRO-B_Ye59FjYBD0FDwH_wGP1YMIGA7aBNpT69yZIcG-PeczsgKvEtAjKSxCyxJhCkqVgfpY1HXCkHC6YNEYNBltSaGcwGmiYzowW6z9yVfSjMqmhb-wSSl7hSfnWyfZk2Z67dMvv_ffH9tmKzCB1VM3hDv8FLh8nkA==&lang=octave&interacts=eJyLjgUAARUAuQ==}{Link to code}.

\begin{warning}
    The system above has a unique solution.  Try manipulating the system so that it has infinitely many solutions or no solutions.  Run the code to see what happens.  You will observe the following:

\begin{itemize}
    \item If a system has infinitely many solutions, Octave will find one particular solution.
    \item If a system has no solutions, Octave will find an approximate solution using least-squares.  %We will cover this method in \href{https://ximera.osu.edu/linearalgebrav3/LinearAlgebraInteractiveIntro/RTH-0030/main}{Least-Squares Approximation}.
\end{itemize}
The way the answer is presented, you cannot tell the difference!
\end{warning}

\end{example}

%Due to how the backslash operator ($\setminus$) functions (see the warning above), we recommend that you solve systems by using the reduced row echelon form.  

%As remarked in Template \ref{temp:systems1},






\end{document}