\documentclass{ximera}
%% You can put user macros here
%% However, you cannot make new environments

\listfiles

\graphicspath{{./}{firstExample/}{secondExample/}}

\usepackage{tikz}
\usepackage{tkz-euclide}
\usepackage{tikz-3dplot}
\usepackage{tikz-cd}
\usetikzlibrary{shapes.geometric}
\usetikzlibrary{arrows}
%\usetkzobj{all}
\pgfplotsset{compat=1.13} % prevents compile error.

%\renewcommand{\vec}[1]{\mathbf{#1}}
\renewcommand{\vec}{\mathbf}
\newcommand{\RR}{\mathbb{R}}
\newcommand{\dfn}{\textit}
\newcommand{\dotp}{\cdot}
\newcommand{\id}{\text{id}}
\newcommand\norm[1]{\left\lVert#1\right\rVert}
 
\newtheorem{general}{Generalization}
\newtheorem{initprob}{Exploration Problem}

\tikzstyle geometryDiagrams=[ultra thick,color=blue!50!black]

%\DefineVerbatimEnvironment{octave}{Verbatim}{numbers=left,frame=lines,label=Octave,labelposition=topline}



\usepackage{mathtools}


\title{Components of Vectors and Matrices} \license{CC BY-NC-SA 4.0}
\begin{document}
\begin{abstract}
\end{abstract}
\maketitle
\section*{Components of Vectors and Matrices}

Sometimes it is useful to extract a single entry from a vector or a matrix, extract a row or a column from a matrix, or combine multiple vectors to form a matrix.  The templates in this section provide code samples for doing all of the above. You can access our Octave code through the link at the bottom of each template.  Feel free to modify the code and experiment to learn more!  Alternatively, go to the \href{https://sagecell.sagemath.org/}{Sage Math Cell Webpage}, copy the code below into the cell, select OCTAVE as the language, and press EVALUATE.  

\begin{template}\label{temp:vectorComp}
    Given vector $\vec{v}=\begin{bmatrix}1\\-2\\-5\end{bmatrix}$, let's find the third component of $\vec{v}$.
\begin{verbatim}
% Define vector v
v=[1;-2;-5];

% Find the third component of v
v_3=v(3)
\end{verbatim}

    \href{https://sagecell.sagemath.org/?z=eJxTVXBJTcvMS1UoS00uyS9SKOPlKrONNrTWNbLWNY215uXi5VJVcMvMS1EoyUgF4syiFIXk_NyC_LzUvBKF_DSwhnhj2zINY00AVTYWYw==&lang=octave&interacts=eJyLjgUAARUAuQ==}{Link to code}.
\end{template}

\begin{template}\label{temp:matrixRowColEntry}
    We can access individual rows, columns, and entries of a matrix.  Let $$A=\begin{bmatrix}2 & -1 & 3 & -1\\1 & 0 & 2 & 1\\1 & -1 & 1 & -2\end{bmatrix}$$
\begin{verbatim}
% Define a matrix 
A=[2 -1 3 -1; 1 0 2 1; 1 -1 1 -2]

% Third row of A
row_3=A(3,:)

% Second column of A
col_2=A(:,2)

% Second row third entry of A
a_23=A(2,3)   
\end{verbatim}
\href{https://sagecell.sagemath.org/?z=eJxVjEEKwjAURPeB3GE2BQsR7M-u0kXAG9SdSAhtigGbQIhYb-9v7cbNMDM8XoWLn0L0cJhdyWGBFKa7EY4NNMcZDU4gbIU_DrpLIUWF6yPkETm9kSYYKbhZ3ZmDVm39I3o_pDhiSM_XHHeKhyWmWkX_1Coqm9LHkj877iytTlK6_gIqSCuZ&lang=octave&interacts=eJyLjgUAARUAuQ==}{Link to code}.
\end{template}

\begin{example}\label{ex:gauss_octave}
Use Octave code to show one step of Gaussian elimination by subtracting half of the first row from the second row of the given matrix $A$.
\begin{verbatim}
% Define matrix A
A=[2 -1 3 -1; 1 0 2 1; 1 -1 1 -2] 

% Replace the second row of A with second row minus half the first row.  
A(2,:)=-1/2*A(1,:)+A(2,:)  
\end{verbatim}

\href{https://sagecell.sagemath.org/?z=eJxNjMsKwjAURPeB_MNsCvVRNXGndHHBL3ArLkK9IYE2kSRSP99YN26Gw5lhGlzY-sCYTEn-DZKC-ptGp3CscYbCARoLVFdD3yGFFA2u_BzNwCiOkXmI4YEUZ0QLwuyL-5eTD68MZ0a7zK1PuXyLHeoZtXp7WvWd2us1tary5qc-770qUw==&lang=octave&interacts=eJyLjgUAARUAuQ==}{Link to code}.
\end{example}

\begin{template}\label{temp:colRow}
    We can create a matrix whose columns (or rows) are given as separate vectors.
    Let $\vec{u}=\begin{bmatrix}1\\2\\3\end{bmatrix}$, $\vec{v}=\begin{bmatrix}4\\5\\6\end{bmatrix}$, $\vec{w}=\begin{bmatrix}7\\8\\9\end{bmatrix}$.  Construct matrices
    $$A=\begin{bmatrix}1 & 4 &7\\2 & 5 & 8\\3 & 6 & 9\end{bmatrix},\quad B=\begin{bmatrix}1 & 2 & 3\\4 & 5 & 6\\7 & 8 & 9\end{bmatrix}$$

    \begin{verbatim}
% Define the vectors u, v, and w
u=[1;2;3];
v=[4;5;6];
w=[7;8;9];

% Define A
A=[u v w]

% Define B
B=transpose(A)
    \end{verbatim}

\href{https://sagecell.sagemath.org/?z=eJxTVXBJTcvMS1UoyUhVKEtNLskvKlYo1VEo01FIzEtRKOflKrWNNrQ2sjaOteblKrONNrE2tTYDsctto82tLawtQWxeLlWYOY68XI620aUKZQrlsSgSTrxcTrYlRYl5xQX5xakajpoAN80hMA==&lang=octave&interacts=eJyLjgUAARUAuQ==}{Link to Code}.
\begin{remark}
    Vectors $\vec{u}$, $\vec{v}$, and $\vec{w}$ can be entered as row vectors.  If this is the case, use the transpose function to convert them to column vectors.
    \begin{verbatim}
% Define the vectors u, v, and w
u=transpose([1 2 3]);
v=transpose([4 5 6]);
w=transpose([7 8 9]);
    \end{verbatim}
\end{remark}
\end{template}

\begin{example}\label{ex:vecsToMatrix}
    Show that the vectors $\vec{u}=\begin{bmatrix}1\\2\\3\end{bmatrix}$, $\vec{v}=\begin{bmatrix}4\\5\\6\end{bmatrix}$, $\vec{w}=\begin{bmatrix}7\\8\\9\end{bmatrix}$ are linearly dependent.
    \begin{explanation}
        Combining $\vec{u}$, $\vec{v}$, and $\vec{w}$ into a matrix will allows us to answer this question easily.  Let $A=\begin{bmatrix}1 & 4 &7\\2 & 5 & 8\\3 & 6 & 9\end{bmatrix}$.  We will consider two methods for showing that the columns of $A$ are linearly dependent.

        \emph{Method 1:} Find the rank of $A$.
        
        \emph{Method 2:} Find the determinant of $A$.

        \begin{verbatim}
% Define the vectors u, v, and w
u=[1; 2; 3];
v=[4; 5; 6];
w=[7; 8; 9];

% Define A
A=[u v w]

% Method 1: find the rank of A
rankA=rank(A)

% Method 2: find the determinant of A
detA=det(A)
        \end{verbatim}
        
\href{https://sagecell.sagemath.org/?z=eJxVjTsKwzAQRHuB7jCNIQE3dv5ZVAjS5gTChYnW2JhI4Mjy9SPlA0kz7OzOmy1w4W5wjNAzIt-Cnx6YS8QSrbNYpJiVqQg1YdOQFFGZLWFH2Ge3KHMgHAmn7KQovm1aCq3MjIileR-uHHpvUZ2RAvb1bmrdCN_lcB61yrrS6z-g_gEsB57ug2td-HBpo1WSRD0Bo444Hw==&lang=octave&interacts=eJyLjgUAARUAuQ==}{Link to code}.  

We find that $A$ is does not have full rank, and that its determinant is $0$.  Each of these results indicates that the columns of $A$ are linearly dependent.
\end{explanation}

How many other methods can you think of to show that $\vec{u}$, $\vec{v}$, and $\vec{w}$ are linearly dependent?  Try implementing them in Octave.
\end{example}

\end{document}