\documentclass{ximera}
%% You can put user macros here
%% However, you cannot make new environments

\listfiles

\graphicspath{{./}{firstExample/}{secondExample/}}

\usepackage{tikz}
\usepackage{tkz-euclide}
\usepackage{tikz-3dplot}
\usepackage{tikz-cd}
\usetikzlibrary{shapes.geometric}
\usetikzlibrary{arrows}
%\usetkzobj{all}
\pgfplotsset{compat=1.13} % prevents compile error.

%\renewcommand{\vec}[1]{\mathbf{#1}}
\renewcommand{\vec}{\mathbf}
\newcommand{\RR}{\mathbb{R}}
\newcommand{\dfn}{\textit}
\newcommand{\dotp}{\cdot}
\newcommand{\id}{\text{id}}
\newcommand\norm[1]{\left\lVert#1\right\rVert}
 
\newtheorem{general}{Generalization}
\newtheorem{initprob}{Exploration Problem}

\tikzstyle geometryDiagrams=[ultra thick,color=blue!50!black]

%\DefineVerbatimEnvironment{octave}{Verbatim}{numbers=left,frame=lines,label=Octave,labelposition=topline}



\usepackage{mathtools}


\title{Homogeneous Linear Systems} \license{CC BY-NC-SA 4.0}

\begin{document}

\begin{abstract}
 
  \end{abstract}
\maketitle


\begin{onlineOnly}
\section*{Homogeneous Linear Systems}
\end{onlineOnly}

\begin{definition}\label{def:homogeneous}
A system of linear equations is called \dfn{homogeneous} if the system can be written in the form
$$\begin{array}{ccccccccc}
      a_{11}x_1 &+ &a_{12}x_2&+&\ldots&+&a_{1n}x_n&= &0 \\
	 a_{21}x_1 &+ &a_{22}x_2&+&\ldots&+&a_{2n}x_n&= &0 \\
     &&&&\vdots&&&& \\
     a_{m1}x_1 &+ &a_{m2}x_2&+&\ldots&+&a_{mn}x_n&= &0
    \end{array}$$
\end{definition}

A homogeneous linear system is always consistent because $x_1=0, x_2=0, \ldots ,x_n=0$ is a solution.  This solution is called the \dfn{trivial solution}.  Geometrically, a homogeneous system can be interpreted as a collection of lines or planes (or hyperplanes) passing through the origin.  Thus, they will always have the origin in common, but may have other points in common as well.

If $A$ is the coefficient matrix for a homogeneous system, then the system can be written as a matrix equation $A\vec{x}=\vec{0}$. The augmented matrix that represents the system looks like this
$$\left[\begin{array}{c|c}  
 A&0
\end{array}\right]$$
As we perform elementary row operations, the entries to the right of the vertical bar remain $0$.  It is customary to omit writing them down and apply elementary row operations to the coefficient matrix only.
\begin{example}\label{ex:homogeneoussys} 
Solve the given homogeneous system and interpret your solution geometrically.
$$\begin{array}{ccccccc}
      4x &+ &5y&-&z&= &0 \\
	 x&- &4y&-&2z&= &0 \\
    3x &- &6y&-&4z&= &0
    \end{array}$$

\begin{explanation}
We start by rewriting this system as a matrix equation
$$\begin{bmatrix}4&5&-1\\1&-4&-2\\3&-6&-4\end{bmatrix}\begin{bmatrix}x\\y\\z\end{bmatrix}=\vec{0}$$
We will proceed to find the reduced row-echelon form of the matrix as usual, but will omit writing the zeros to the right of the vertical bar.
$$\begin{bmatrix}4&5&-1\\1&-4&-2\\3&-6&-4\end{bmatrix}\rightsquigarrow \begin{bmatrix}1&0&-2/3\\0&1&1/3\\0&0&0\end{bmatrix}$$
$x$ and $y$ are leading variables, and $z$ is a free variable.  We let $z=t$ and solve for $x$ and $y$.
\begin{align*}
x&=\frac{2}{3}t\\
y&=-\frac{1}{3}t\\
z&=t
\end{align*}
Each of the equations in the original system represents a plane through the origin in $\RR^3$.  The system has infinitely many solutions.  Geometrically, we can interpret these solutions as points lying on the line shared by the three planes.  The above solution is a parametric representation of this line.  Use the GeoGebra demo below to take a better look at the system. (RIGHT-CLICK and DRAG to rotate the image.)

\pdfOnly{
Access GeoGebra interactives through the online version of this text at 

\href{https://ximera.osu.edu/linearalgebradzv3/LinearAlgebraInteractiveIntro}{https://ximera.osu.edu/linearalgebradzv3/LinearAlgebraInteractiveIntro}.
}

\begin{onlineOnly}
\begin{center}
    \geogebra{gxjnusja}{800}{600} 
  \end{center}
  \end{onlineOnly}

\end{explanation}
\end{example}
\subsection*{General and Particular Solutions}

\begin{definition}\label{def:asshomsys} Given any linear system $A\vec{x}=\vec{b}$, the system $A\vec{x}=\vec{0}$ is called the \dfn{associated homogeneous system}.
\end{definition}

It turns out that there is a relationship between solutions of $A\vec{x}=\vec{b}$ and solutions of the associated homogeneous system.

\begin{exploration}\label{init:generalplusparticular}
Let $$A=\begin{bmatrix}1&2&4\\3&-7&-1\\-1&4&2\end{bmatrix}\quad\text{and}\quad\vec{b}=\begin{bmatrix}-2\\7\\-4\end{bmatrix}$$
Consider the matrix equation $A\vec{x}=\vec{b}$.  Row reduction produces the following.
$$\left[\begin{array}{ccc|c}  
 1&2&4&-2\\3&-7&-1&7\\-1&4&2&-4
 \end{array}\right]\begin{array}{c}
 \\
 \rightsquigarrow\\
 \\
 \end{array}\left[\begin{array}{ccc|c}  
 1&0&2&0\\0&1&1&-1\\0&0&0&0
 \end{array}\right]$$
 We conclude that $\vec{x}=\begin{bmatrix}-2t\\-1-t\\t\end{bmatrix}$.  
 
 Let's take a more careful look at $\vec{x}$.
 $$\vec{x}=\begin{bmatrix}-2t\\-1-t\\t\end{bmatrix}=\begin{bmatrix}0\\-1\\\answer{0}\end{bmatrix}+\begin{bmatrix}\answer{-2}\\\answer{-1}\\\answer{1}\end{bmatrix}t$$
 We now see that the solution vector $\vec{x}$ is made up of two distinct parts: 
 \begin{itemize}
 \item
 one specific vector $\begin{bmatrix}0\\-1\\0\end{bmatrix}$
 \item
 infinitely many scalar multiples of $\begin{bmatrix}-2\\-1\\1\end{bmatrix}$.  
 \end{itemize}
 
 The vector $\begin{bmatrix}0\\-1\\0\end{bmatrix}$ is an example of a \dfn{particular solution}.  This particular ``particular solution" corresponds to $t=0$.  We can find any number of particular solutions by letting $t$ assume different values.  For example, the particular solution that corresponds to $t=1$ is $\begin{bmatrix}-2\\-2\\1\end{bmatrix}$.  Let $\vec{x}_p$ be any particular solution of $A\vec{x}=\vec{b}$.  It turns out that all vectors of the form $$\vec{x}=\vec{x}_p+\begin{bmatrix}-2\\-1\\1\end{bmatrix}t$$ are solutions of $A\vec{x}=\vec{b}$.  We can verify this as follows
 $$A\vec{x}=A\left(\vec{x}_p+\begin{bmatrix}-2\\-1\\1\end{bmatrix}t\right)=A\vec{x}_p+\begin{bmatrix}1&2&4\\3&-7&-1\\-1&4&2\end{bmatrix}\begin{bmatrix}-2\\-1\\1\end{bmatrix}t=A\vec{x}_p+\vec{0}=\vec{b}$$
 This shows that the specific vector $\begin{bmatrix}0\\-1\\0\end{bmatrix}$ is not very special, as any solution of $A\vec{x}=\vec{b}$ can be used in its place.  
 
 The vector $\begin{bmatrix}-2\\-1\\1\end{bmatrix}$, however, is special.  
 Note that
 $$A\begin{bmatrix}-2\\-1\\1\end{bmatrix}=\begin{bmatrix}1&2&4\\3&-7&-1\\-1&4&2\end{bmatrix}\begin{bmatrix}-2\\-1\\1\end{bmatrix}=\vec{0}$$
 So $\begin{bmatrix}-2\\-1\\1\end{bmatrix}$ and all of its scalar multiples are solutions to the associated homogeneous system.  
\end{exploration}

\begin{observation}
In Exploration \ref{init:generalplusparticular} we found that the general solution of the equation $A\vec{x}=\vec{b}$ has the form:
 $$\vec{x}=(\text{Any Particular Solution of}\,A\vec{x}=\vec{b}) + (\text{General Solution of}\,A\vec{x}=\vec{0})$$
 \end{observation}
It turns out that the general solution of any linear system can be written in this format.  Theorem \ref{th:homogeneous} formalizes this result.

\begin{theorem}\label{th:homogeneous}Suppose $\vec{x}_p$ is a particular solution of $A\vec{x}=\vec{b}$.
  \begin{enumerate}
  \item\label{item:homogeneous1} If $\vec{x}_h$ is a solution of the associated homogeneous system, then $\vec{x}_p+\vec{x}_h$ is a solution of $A\vec{x}=\vec{b}$. 
  \item \label{item:homogeneous2}If $\vec{x}_1$ is a solution of $A\vec{x}=\vec{b}$, then there exists a solution of the associated homogeneous system, $\vec{x}_h$, such that  $\vec{x}_1=\vec{x}_p+\vec{x}_h$.
  \end{enumerate}
\end{theorem}
We will prove part \ref{item:homogeneous2}.  The proof of part \ref{item:homogeneous1} is left to the reader.
\begin{proof}[Proof of \ref{item:homogeneous2}]
Let $\vec{x}_h=\vec{x}_1-\vec{x}_p$, then 
$$A\vec{x}_h=A(\vec{x}_1-\vec{x}_p)=A\vec{x}_1-A\vec{x}_p=\vec{b}-\vec{b}=\vec{0}$$
and
$$\vec{x}_1=\vec{x}_p+\vec{x}_h$$
\end{proof}
\begin{example}\label{ex:sumofhomandpart}
Let $$A=\begin{bmatrix}2&-4&-2\\1&-2&-1\end{bmatrix}\quad\text{and}\quad \vec{b}=\begin{bmatrix}8\\4\end{bmatrix}$$
If possible, find a solution of $A\vec{x}=\vec{b}$ and express it as a sum of a particular solution and the general solution of the associated homogeneous system. ($\vec{x}=\vec{x}_p+\vec{x}_h$)
\begin{explanation}
First, we obtain the reduced row-echelon form
$$\left[\begin{array}{ccc|c}  
 2&-4&-2&8\\1&-2&-1&4
 \end{array}\right]\begin{array}{c}
 \\
 \rightsquigarrow\\
 \\
 \end{array}\left[\begin{array}{ccc|c}  
 1&-2&-1&4\\0&0&0&0
 \end{array}\right]$$
 
 So $$\vec{x}=\begin{bmatrix}4+2s+t\\s\\t\end{bmatrix}=\begin{bmatrix}4\\0\\0\end{bmatrix}+\begin{bmatrix}2\\1\\0\end{bmatrix}s+\begin{bmatrix}1\\0\\1\end{bmatrix}t$$
 In this case $$\vec{x}_p=\begin{bmatrix}4\\0\\0\end{bmatrix}$$
 $$\vec{x}_h=\begin{bmatrix}2\\1\\0\end{bmatrix}s+\begin{bmatrix}1\\0\\1\end{bmatrix}t$$
\end{explanation}
\end{example}

\section*{Practice Problems}

\emph{Problems \ref{prob:hompluspart1}-\ref{prob:hompluspart2}}

For each matrix $A$ and vector $\vec{b}$ below, find a solution to $A\vec{x}=\vec{b}$ and  express your solution as a sum of a particular solution and a general solution to the associated homogeneous system.

\begin{problem}\label{prob:hompluspart1}
$$A=\begin{bmatrix}1&1&3&1\\3&4&2&1\end{bmatrix}\quad\text{and}\quad\vec{b}=\begin{bmatrix}6\\1\end{bmatrix}$$
\end{problem}

\begin{problem}\label{prob:hompluspart2}
$$A=\begin{bmatrix}3&2&1\\1&-1&1\\4&1&1\end{bmatrix}\quad\text{and}\quad\vec{b}=\begin{bmatrix}10\\2\\12\end{bmatrix}$$
\end{problem}


\begin{problem}\label{prob:infmanysolutionshom}
Prove that a consistent system has infinitely many solutions if and only if the associated homogeneous system has infinitely many solutions.
\end{problem}

\emph{Problems \ref{prob:homexample1}-\ref{prob:homexample2}}

If possible, construct an example of each of the following.  If not possible, explain why.

  \begin{problem}\label{prob:homexample1}
  An inconsistent system with an associated homogeneous system that has infinitely many solutions.
  \end{problem}
  
  \begin{problem}\label{prob:homexample2}
  An inconsistent system with an associated homogeneous system that has a unique (trivial) solution.
  \end{problem}


\begin{problem}\label{prob:lincombsolutions}
Prove that a linear combination of any number of solutions of a homogeneous equation is a solution of the same equation.
\end{problem}

\begin{problem}\label{prob:homogeneouspart1} Prove Part \ref{item:homogeneous1} of Theorem \ref{th:homogeneous}.
\end{problem}

\end{document} 