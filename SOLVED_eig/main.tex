\documentclass{ximera}
%% You can put user macros here
%% However, you cannot make new environments

\listfiles

\graphicspath{{./}{firstExample/}{secondExample/}}

\usepackage{tikz}
\usepackage{tkz-euclide}
\usepackage{tikz-3dplot}
\usepackage{tikz-cd}
\usetikzlibrary{shapes.geometric}
\usetikzlibrary{arrows}
%\usetkzobj{all}
\pgfplotsset{compat=1.13} % prevents compile error.

%\renewcommand{\vec}[1]{\mathbf{#1}}
\renewcommand{\vec}{\mathbf}
\newcommand{\RR}{\mathbb{R}}
\newcommand{\dfn}{\textit}
\newcommand{\dotp}{\cdot}
\newcommand{\id}{\text{id}}
\newcommand\norm[1]{\left\lVert#1\right\rVert}
 
\newtheorem{general}{Generalization}
\newtheorem{initprob}{Exploration Problem}

\tikzstyle geometryDiagrams=[ultra thick,color=blue!50!black]

%\DefineVerbatimEnvironment{octave}{Verbatim}{numbers=left,frame=lines,label=Octave,labelposition=topline}



\usepackage{mathtools}


\title{Solved Problems} \license{CC BY-NC-SA 4.0}

\begin{document}

\begin{abstract}
\end{abstract}
\maketitle

\section*{Solved Problems for Chapter 8}


\begin{problem}\label{prb:8.1} If $A$ is an invertible $n\times n$ matrix, compare the eigenvalues of
$A$ and $A^{-1}$. More generally, for $m$ an arbitrary integer, compare the
eigenvalues of $A$ and $A^{m}$.

Click the arrow to see answer.
\begin{expandable}
$A^{m}X=\lambda ^{m}X$ for
any integer. In the case of $-1,A^{-1}\lambda X=AA^{-1}X=X$
so $A^{-1}X =\lambda ^{-1}X$. Thus the eigenvalues of $A^{-1}$ are just $\lambda ^{-1}$ where $\lambda $ is an eigenvalue of $A$.
\end{expandable}
\end{problem}

\begin{problem}\label{prb:8.2} If $A$ is an $n\times n$ matrix and $c$ is a nonzero constant, compare
the eigenvalues of $A$ and $cA$. 

Click the arrow to see answer.
\begin{expandable}
Say $AX=\lambda X.$ Then $
cAX=c\lambda X$ and so the eigenvalues of $cA$ are just $
c\lambda $ where $\lambda $ is an eigenvalue of $A$.
\end{expandable}
\end{problem}

\begin{problem}\label{prb:8.3} Let $A,B$ be invertible $n\times n$ matrices which commute. That is, $AB=BA$. Suppose $X$ is an eigenvector of $B$. Show that then
$AX$ must also be an eigenvector for $B$.

Click the arrow to see answer.
\begin{expandable}
 $BAX=ABX
=A\lambda X=\lambda AX$. Here it is assumed that $BX=\lambda X$.
\end{expandable}
\end{problem}

\begin{problem}\label{prb:8.4} Suppose $A$ is an $n\times n$ matrix and it satisfies $A^{m}=A$ for
some $m$ a positive integer larger than 1. Show that if $\lambda $ is an
eigenvalue of $A$ then $\left\vert \lambda \right\vert $ equals either 0 or $
1$. 

Click the arrow to see answer.
\begin{expandable}
Let $X$ be the eigenvector. Then $A^{m}X=\lambda ^{m}
X,A^{m}X=AX=\lambda X$ and so
\[
\lambda ^{m}=\lambda
\]
Hence if $\lambda \neq 0,$ then
\[
\lambda ^{m-1}=1
\]
and so $\left\vert \lambda \right\vert =1.$
\end{expandable}
\end{problem}

\begin{problem}\label{prb:8.5} Show that if $AX=\lambda X$ and $AY=\lambda Y$, then whenever $k,p$ are scalars,
\begin{equation*}
A\left( kX+pY\right) =\lambda \left( kX+pY\right)
\end{equation*}
Does this imply that $kX+pY$ is an eigenvector? Explain.

Click the arrow to see answer.
\begin{expandable}
The formula follows from properties of matrix multiplications. However,
this vector might not be an eigenvector because it might equal $0$
and eigenvectors cannot equal $0$.
\end{expandable}
\end{problem}

\begin{problem}\label{prb:8.12} Find the eigenvalues and eigenvectors of the matrix
\begin{equation*}
\left[
\begin{array}{rrr}
6 & 76 & 16 \\
-2 & -21 & -4 \\
2 & 64 & 17
\end{array}
\right]
\end{equation*}
One eigenvalue is $-2.$

Click the arrow to see answer.
\begin{expandable}
The characteristic polynomial of this matrix is $-\lambda^3+2\lambda^2+5\lambda-6$.  Knowing one of the eigenvalues gives us one factor, $(\lambda+2)$.  Use long division of polynomials to finish factoring $-\lambda^3+2\lambda^2+5\lambda-6=-(\lambda+2)(\lambda-1)(\lambda-3)$.  Now we have the following eigenvalues: $\lambda_1=-2$, $\lambda_2=1$, and $\lambda_3=3$.

The corresponding eigenvectors are: $\vec{v}_1=\begin{bmatrix}7\\-2\\6\end{bmatrix}$, $\vec{v}_2=\begin{bmatrix}8\\-2\\7\end{bmatrix}$, and $\vec{v}_3=\begin{bmatrix}4\\-1\\4\end{bmatrix}$.
\end{expandable}
\end{problem}

\begin{problem}\label{prb:8.13} Find the eigenvalues and eigenvectors of the matrix
\begin{equation*}
\left[
\begin{array}{rrr}
3 & 5 & 2 \\
-8 & -11 & -4 \\
10 & 11 & 3
\end{array}
\right]
\end{equation*}
One eigenvalue is $-3$.

Click the arrow to see answer.
\begin{expandable}
Characteristic polynomial: $-\lambda^3-5\lambda^2-7\lambda-3=-(\lambda+3)(\lambda+1)^2$.
    $$\lambda_1=-3,\quad \vec{v}_1=\begin{bmatrix}1\\-2\\2\end{bmatrix}$$
    $$\lambda_2=-1,\quad \begin{bmatrix}1\\-2\\3\end{bmatrix}$$
\end{expandable}

\end{problem}

\begin{problem}\label{prb:8.14} Is it possible for a nonzero matrix to have only $0$ as an eigenvalue?

Click the arrow to see answer.
\begin{expandable}
Yes. $\left[
\begin{array}{cc}
0 & 1 \\
0 & 0%
\end{array}
\right] $ works.
\end{expandable}
\end{problem}

\begin{problem}\label{prb:8.17} Let $T\,$\ be the linear transformation which reflects vectors about
the $x$ axis. Find a matrix for $T$ and then find its eigenvalues and
eigenvectors.

Click the arrow to see answer.

\begin{expandable}
The matrix of $T$ is $\left[
\begin{array}{rr}
1 & 0 \\
0 & -1
\end{array}
\right]$. 

The eigenvalues and eigenvectors are:
$$\lambda_1=-1,\quad \vec{v}_1=\begin{bmatrix}0\\1\end{bmatrix}$$
$$\lambda_2=1,\quad \vec{v}_2=\begin{bmatrix}1\\0\end{bmatrix}$$
\end{expandable}
\end{problem}

\begin{problem}\label{prb:8.19} Let $T$ be the linear transformation which reflects all vectors in $
\mathbb{R}^{3}$ through the $xy$ plane. Find a matrix for $T$ and then
obtain its eigenvalues and eigenvectors.

Click the arrow to see answer.
\begin{expandable}
The matrix of $T$ is $\left[
\begin{array}{rrr}
1 & 0 & 0 \\
0 & 1 & 0 \\
0 & 0 & -1
\end{array}
\right]$
The eigenvalues are $\lambda_1=-1$, $\lambda_2=1$.  Bases for the corresponding eigenspaces are:
\[
\mathcal{S}_{\lambda_1}=\left\{
\begin{bmatrix}
0 \\
0 \\
1
\end{bmatrix} \right\}, \mathcal{S}_{\lambda_2}=\left\{
\begin{bmatrix}
1 \\
0 \\
0
\end{bmatrix}
 ,
\begin{bmatrix}
0 \\
1 \\
0
\end{bmatrix} \right\}
\]
\end{expandable}
\end{problem}

\begin{problem}\label{prb:8.20} Find the eigenvalues and eigenvectors of the matrix
\begin{equation*}
\left[
\begin{array}{rrr}
5 & -18 & -32 \\
0 & 5 & 4 \\
2 & -5 & -11
\end{array}
\right]
\end{equation*}
One eigenvalue is $1.$ Diagonalize if possible.

Click the arrow to see answer.
\begin{expandable}
The eigenvalues are $-1, -1, 1$. The eigenvectors corresponding to the eigenvalues are:
\[
\left\{ \left[
\begin{array}{c}
10 \\
-2 \\
3
\end{array}
\right] \right\} \leftrightarrow -1,  \left\{ \left[
\begin{array}{c}
7 \\
-2 \\
2
\end{array}
\right] \right\} \leftrightarrow 1
\]
Therefore this matrix is not diagonalizable.
\end{expandable}
\end{problem}

\begin{problem}\label{prb:8.21} Find the eigenvalues and eigenvectors of the matrix
\begin{equation*}
\left[
\begin{array}{rrr}
-13 & -28 & 28 \\
4 & 9 & -8 \\
-4 & -8 & 9
\end{array}
\right]
\end{equation*}
One eigenvalue is $3.$ Diagonalize if possible.

Click the arrow to see answer.
\begin{expandable}
The eigenvectors and eigenvalues are:
\[
\left\{ \left[
\begin{array}{c}
2 \\
0 \\
1
\end{array}
\right] \right\} \leftrightarrow 1, \left\{ \left[
\begin{array}{c}
-2 \\
1 \\
0
\end{array}
\right] \right\} \leftrightarrow 1, \left\{ \left[
\begin{array}{c}
7 \\
-2 \\
2
\end{array}
\right] \right\} \leftrightarrow 3
\]
The matrix $P$ needed to diagonalize the above matrix is
\[
\left[
\begin{array}{rrr}
2 & -2 & 7 \\
0 & 1 & -2 \\
1 & 0 & 2
\end{array}
\right]
\]
and the diagonal matrix $D$ is
\[
\left[
\begin{array}{rrr}
1 & 0 & 0  \\
0 & 1 & 0 \\
0 & 0 & 3
\end{array}
\right]
\]
\end{expandable}
\end{problem}

\begin{problem}\label{prb:8.22} Find the eigenvalues and eigenvectors of the matrix
\begin{equation*}
\left[
\begin{array}{rrr}
89 & 38 & 268 \\
14 & 2 & 40 \\
-30 & -12 & -90
\end{array}
\right]
\end{equation*}
One eigenvalue is $-3.$ Diagonalize if possible.

Click the arrow to see answer.
\begin{expandable}
The eigenvectors and eigenvalues are:
\[
\left\{ \left[
\begin{array}{c}
-6 \\
-1 \\
-2
\end{array}
\right] \right\} \leftrightarrow 6, \left\{ \left[
\begin{array}{c}
-5 \\
-2 \\
2
\end{array}
\right] \right\} \leftrightarrow -3, \left\{ \left[
\begin{array}{c}
-8 \\
-2 \\
3
\end{array}
\right] \right\} \leftrightarrow -2
\]
The matrix $P$ needed to diagonalize the above matrix is
\[
\left[
\begin{array}{rrr}
-6 & -5 & -8 \\
-1 & -2 & -2 \\
2 & 2 & 3
\end{array}
\right]
\]
and the diagonal matrix $D$ is
\[
\left[
\begin{array}{rrr}
6 & 0 & 0  \\
0 & -3 & 0 \\
0 & 0 & -2
\end{array}
\right]
\]
\end{expandable}
\end{problem}

\begin{problem}\label{prob:moresimilarproperties}
If $A \sim B$ and $A$ has any of the following properties, show that $B$ has the same property.

\begin{enumerate}
\item\label{prob:moresimilarproperties_idempotent} A is \dfn{Idempotent}, that is $A^{2} = A$.

\item\label{prob:moresimilarproperties_nilpotent} A is \dfn{Nilpotent}, that is $A^{k} = 0$ for some $k \geq 1$.

Click the arrow to see answer.
\begin{expandable}
If $B = P^{-1}AP$ and $A^{k} = 0$, then $B^{k} = (P^{-1}AP)^{k} = P^{-1}A^{k}P = P^{-1}0P = 0$.
\end{expandable}

\item\label{prob:moresimilarproperties_invertible} A is Invertible.

\end{enumerate}

\end{problem}

\section*{Bibliography}
These problems come from Chapter 7 of Ken Kuttler's \href{https://open.umn.edu/opentextbooks/textbooks/a-first-course-in-linear-algebra-2017}{\it A First Course in Linear Algebra}. (CC-BY)

Ken Kuttler, {\it  A First Course in Linear Algebra}, Lyryx 2017, Open Edition, pp. 359--401. 

\end{document}