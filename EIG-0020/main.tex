\documentclass{ximera}
%% You can put user macros here
%% However, you cannot make new environments

\listfiles

\graphicspath{{./}{firstExample/}{secondExample/}}

\usepackage{tikz}
\usepackage{tkz-euclide}
\usepackage{tikz-3dplot}
\usepackage{tikz-cd}
\usetikzlibrary{shapes.geometric}
\usetikzlibrary{arrows}
%\usetkzobj{all}
\pgfplotsset{compat=1.13} % prevents compile error.

%\renewcommand{\vec}[1]{\mathbf{#1}}
\renewcommand{\vec}{\mathbf}
\newcommand{\RR}{\mathbb{R}}
\newcommand{\dfn}{\textit}
\newcommand{\dotp}{\cdot}
\newcommand{\id}{\text{id}}
\newcommand\norm[1]{\left\lVert#1\right\rVert}
 
\newtheorem{general}{Generalization}
\newtheorem{initprob}{Exploration Problem}

\tikzstyle geometryDiagrams=[ultra thick,color=blue!50!black]

%\DefineVerbatimEnvironment{octave}{Verbatim}{numbers=left,frame=lines,label=Octave,labelposition=topline}



\usepackage{mathtools}


 \title{The Characteristic Equation} \license{CC BY-NC-SA 4.0}


\begin{document}
\begin{abstract}

\end{abstract}
\maketitle

\begin{onlineOnly}
\section*{The Characteristic Equation}
\end{onlineOnly}

Let $A$ be an $n \times n$ matrix.  In \href{https://ximera.osu.edu/linearalgebradzv3/LinearAlgebraInteractiveIntro/EIG-0010/main}{Describing Eigenvalues and Eigenvectors Algebraically and Geometrically} we learned that the eigenvectors and eigenvalues of $A$ are vectors $\vec{x}$ and scalars $\lambda$ that satisfy the equation  
\begin{align}\label{def:eigen} A \vec{x} = \lambda \vec{x}\end{align}
We listed a few reasons why we are interested in finding eigenvalues and eigenvectors, but we did not give any process for finding them.  In this section we will focus on a process which can be used for small matrices.  For larger matrices, the best methods we have are iterative methods, and we will explore some of these in \href{https://ximera.osu.edu/linearalgebradzv3/LinearAlgebraInteractiveIntro/EIG-0070/main}{The Power Method and the Dominant Eigenvalue}.

For an $n \times n$ matrix, we will see that the eigenvalues are the roots of a polynomial called the \dfn{characteristic polynomial}.  So finding eigenvalues is equivalent to solving a polynomial equation of degree $n$.  Finding the corresponding eigenvectors turns out to be a matter of computing the null space of a matrix, as the following exploration demonstrates.

\begin{exploration}\label{exp:slowdown}
If a vector $\vec{x}$ is an eigenvector satisfying Equation (\ref{def:eigen}), then clearly it also satisfies  $A\vec{x}-\lambda \vec{x} =$ \wordChoice{\choice{0} \choice[correct]{$\vec{0}$}}.

It seems natural at this point to try to factor.  We would love to ``factor out'' $\vec{x}$.  Here is the procedure:
\begin{align*}
A\vec{x}-\lambda \vec{x} &= \vec{0} \\
A\vec{x}-\lambda I\vec{x} &= \vec{0} \\
(A-\lambda I)\vec{x} &= \vec{0}
\end{align*}
The middle step was necessary before factoring because \wordChoice{\choice[correct]{we cannot subtract a $1 \times 1$ scalar $\lambda$ from an $n \times n$ matrix $A$} \choice{$\lambda$ is a Greek letter}}.

This shows that any eigenvector $\vec{x}$ of $A$ is in the \wordChoice{\choice{row space}\choice{column space}\choice[correct]{null space}}  of the related matrix, $A-\lambda I$.

Since eigenvectors are non-zero vectors, this means that $A$ will have eigenvectors if and only if the null space of $A-\lambda I$ is nontrivial.  The only way that $\mbox{null}(A-\lambda I)$ can be nontrivial is if $\mbox{rank}(A-\lambda I)$ \wordChoice{\choice{$=$}\choice[correct]{$<$}\choice{$>$}} $n$.

If the rank of an $n \times n$ matrix is less than $n$, then the matrix is singular.  Since $(A-\lambda I)$ must be singular for any eigenvalue $\lambda$, we see that $\lambda$ is an eigenvalue of $A$ if and only if 
\begin{equation}\label{eqn:chareqn}
\mbox{det}(A-\lambda I) = \answer{0}
\end{equation}
\end{exploration}
In theory, Exploration \ref{exp:slowdown} offers us a way to find eigenvalues.  To find the eigenvalues of $A$, one can solve Equation (\ref{eqn:chareqn}) for $\lambda$. 

\subsection*{Eigenvalues}
 
\begin{definition}\label{def:chareqcharpoly}
The equation 
$$\mbox{det}(A-\lambda I) = 0$$ is called the \dfn{characteristic equation} of $A$. 
\end{definition}

\begin{example}\label{ex:2x2eig}
Let $A=\begin{bmatrix} 2& 1\\ 1&2
\end{bmatrix}$.  Compute the eigenvalues of this matrix using the characteristic equation.
\begin{explanation}
\begin{align*}\det(A-\lambda I)=\begin{vmatrix}2-\lambda&1\\1&2-\lambda\end{vmatrix}&=(2-\lambda)^2-1\\
&=\lambda^2-4\lambda+3\\
&=(\lambda-1)(\lambda-3)
\end{align*}
The characteristic equation $(\lambda-1)(\lambda-3)=0$ has solutions $\lambda_1=1$ and $\lambda_2=3$.  These are the eigenvalues of $A$.
\end{explanation}
\end{example}

\begin{example}\label{ex:2x2eig2}
Let $B=\begin{bmatrix} 2& 1\\ 4&2
\end{bmatrix}$.  Compute the eigenvalues of $B$ using the characteristic equation.  (List your answers in an increasing order.)

$\lambda_1=\answer{0}$ and $\lambda_2=\answer{4}$
\end{example}

\begin{example}\label{ex:3x3eig}
Let $C=\begin{bmatrix} 2 & 1 & 1\\ 1 & 2 & 1\\ 1 & 1 & 2\end{bmatrix}$.  Compute the eigenvalues of $C$ using the characteristic equation.
\begin{explanation}
\begin{align*}\det(C-\lambda I)&=\begin{vmatrix}2-\lambda & 1 & 1\\ 1 & 2-\lambda & 1\\ 1 & 1 & 2-\lambda\end{vmatrix}\\
&=(2-\lambda)\begin{vmatrix}2-\lambda&1\\1&2-\lambda\end{vmatrix}-1\begin{vmatrix}1&1\\1&2-\lambda\end{vmatrix}+1\begin{vmatrix}1&2-\lambda\\1&1\end{vmatrix}\\
&=(2-\lambda)^3-(2-\lambda)-((2-\lambda)-1)+1-(2-\lambda)\\
&=8-12\lambda+6\lambda^2-\lambda^3-2+\lambda-2+\lambda+1+1-2+\lambda\\
&=4-9\lambda+6\lambda^2-\lambda^3\\
&=-(\lambda-4)(\lambda-1)^2
\end{align*}
Matrix $C$ has eigenvalues $\lambda_1=1$ and $\lambda_2=4$.
\end{explanation}
\end{example}

In Example \ref{ex:3x3eig}, the factor  $(\lambda-1)$ appears twice.  This repeated factor gives rise to the eigenvalue $\lambda_1=1$.  We say that the eigenvalue $\lambda_1=1$ has \dfn{algebraic multiplicity} $2$. 

The three examples above are a bit contrived.  It is not always possible to completely factor the characteristic polynomial using only real numbers.  However, a fundamental fact from algebra is that every degree $n$ polynomial has $n$ roots (counting multiplicity) provided that we allow complex numbers.  This is why sometimes eigenvalues and their corresponding eigenvectors involve complex numbers.  The next example illustrates this point.

\begin{example}\label{ex:3x3_complex_eig}
Let $D=\begin{bmatrix} 0&0&0\\ 0 &1&1\\ 0 & -1&1\end{bmatrix}$.  Compute the eigenvalues of this matrix.

\begin{explanation}
\begin{align*}\det(D-\lambda I)&=\begin{vmatrix} -\lambda&0&0\\ 0 &1-\lambda&1\\ 0 & -1&1-\lambda\end{vmatrix}\\
&=-\lambda\begin{vmatrix}1-\lambda&1\\-1&1-\lambda\end{vmatrix}\\
&=-\lambda((1-\lambda)^2+1)\\
&=-\lambda(\lambda^2-2\lambda+2)
\end{align*}
So one of the eigenvalues of $D$ is $\lambda=0$.  To get the other eigenvalues we must solve $\lambda^2-2\lambda+2=0$.  Using the quadratic formula, we compute that $\lambda_1=1+i$ and $\lambda_2=1-i$ are also eigenvalues of $D$.
\end{explanation}
\end{example}

\begin{exploration}\label{init:3x3tri}
Let $T=\begin{bmatrix} 1 & 2 & 3\\ 0 & 5 & 6\\ 0 & 0 & 9\end{bmatrix}$.  Compute the eigenvalues of this matrix.  (List your answers in an increasing order.)

$$\lambda_1=\answer{1},\quad\lambda_2=\answer{5},\quad\text{and}\quad\lambda_3=\answer{9}$$

What do you observe about the eigenvalues?
\begin{hint}
The eigenvalues are the diagonal entries of the matrix.
\end{hint}

What property of the matrix makes this ``coincidence" possible?

\begin{hint}
$T$ is a triangular matrix.
\end{hint}
\end{exploration}

The matrix in Exploration Problem \ref{init:3x3tri} is a triangular matrix, and the property you observed holds in general.

\begin{theorem}\label{th:eigtri}
Let $T$ be a triangular matrix.  Then the eigenvalues of $T$ are the entries on the main diagonal.
\end{theorem}

\begin{proof}
See Practice Problem \ref{prob:eigtri}.
\end{proof}

\begin{corollary}\label{th:eigdiag}
Let $D$ be a diagonal matrix.  Then the eigenvalues of $D$ are the entries on the main diagonal.
\end{corollary}

One final note about eigenvalues.  We began this section with the sentence, "In theory, then, to find the eigenvalues of $A$, one can solve Equation (\ref{eqn:chareqn}) for $\lambda$."  In general, one does not attempt to compute eigenvalues by solving the characteristic equation of a matrix, as there is no simple way to solve this polynomial equation for $n>4$.  Instead, one can often approximate the eigenvalues using \dfn{iterative methods}.  We will explore some of these techniques in \href{https://ximera.osu.edu/linearalgebradzv3/LinearAlgebraInteractiveIntro/EIG-0070/main}{The Power Method and the Dominant Eigenvalue}.

\subsection*{Eigenvectors}
Once we have computed an eigenvalue $\lambda$ of an $n \times n$ matrix $A$, the next step is to compute the associated eigenvectors.  In other words, we seek vectors $\vec{x}$ such that $A\vec{x}=\lambda \vec{x}$, or equivalently,
\begin{align}\label{eqn:nullspace}
 (A-\lambda I) \vec{x}=\vec{0}   
\end{align} 
For any given eigenvalue $\lambda$ there are infinitely many eigenvectors associated with it.  In fact, the eigenvectors associated with $\lambda$ form a subspace of $\RR^n$. 

\begin{theorem}\label{th:eigenspace}
    Let $A$ be an $n\times n$ matrix and let $\lambda$ be an eigenvalue of $A$.  Then the set of all eigenvectors associated with $\lambda$ is a subspace of $\RR^n$.
\end{theorem}
\begin{proof}
    See Practice Problems \ref{prob:eigenspace1} and \ref{prob:eigenspace2}.
\end{proof}

This motivates the following definition.

\begin{definition}\label{def:eigspace}
The set of all eigenvectors associated with a given eigenvalue of a matrix is known as the \dfn{eigenspace} associated with that eigenvalue.
\end{definition}

So given an eigenvalue $\lambda$, there is an associated eigenspace $\mathcal{S}$, and our goal is to find a basis of $\mathcal{S}$, for then any eigenvector $\vec{x}$ will be a linear combination of the vectors in that basis.  Moreover, we are trying to find a basis for the set of vectors that satisfy Equation \ref{eqn:nullspace}, which means we seek a basis for $\mbox{null}(A-\lambda I)$.  We have already learned how to compute a basis of a null space - see \href{https://ximera.osu.edu/linearalgebradzv3/LinearAlgebraInteractiveIntro/VSP-0040/main}{Subspaces Associated with Matrices}.

Let's return to the examples we did in the first part of this section.

\begin{example}\label{ex:eigvect2x2eig} (Finding eigenvectors for Example \ref{ex:2x2eig} ) 

Recall that $A=\begin{bmatrix} 2& 1\\ 1&2
\end{bmatrix}$ has eigenvalues $\lambda_1=1$ and $\lambda_2=3$.  Compute a basis for the eigenspace associated with each of these eigenvalues.
\begin{explanation}
Eigenvectors associated with the eigenvalue $\lambda_1=1$ are in the null space of $A-I$.  So we seek a basis for $\mbox{null}(A-I)$.  We compute:
\begin{align*}\mbox{rref}(A-I)=\mbox{rref}\left(\begin{bmatrix}1&1\\1&1\end{bmatrix}\right)&=\begin{bmatrix}1&1\\0&0\end{bmatrix},
\end{align*}
From this we see that the eigenspace $\mathcal{S}_1$ associated with $\lambda_1=1$ consists of vectors of the form $\begin{bmatrix}-1\\1\end{bmatrix}t$.
%for any eigenvector $\begin{bmatrix}x_1\\x_2\end{bmatrix}$, we have $x_1+x_2=0$, so that $x_1=-x_2$  
This means that $\left\{\begin{bmatrix}-1\\1\end{bmatrix}\right\}$ is one possible basis for $\mathcal{S}_1$.

In a similar way, we compute a basis for $\mathcal{S}_3$, the subspace of all eigenvectors associated with the eigenvalue $\lambda_2=3$.  Now we compute:
\begin{align*}\mbox{rref}(A-3I)=\mbox{rref}\left(\begin{bmatrix}-1&1\\1&-1\end{bmatrix}\right)&=\begin{bmatrix}1&-1\\0&0\end{bmatrix},
\end{align*}
Vectors in the null space have the form $\begin{bmatrix}1\\1\end{bmatrix}t$ This means that $\left\{\begin{bmatrix}1\\1\end{bmatrix}\right\}$ is one possible basis for the eigenspace $\mathcal{S}_3$.
\end{explanation}
\end{example}

\begin{example}\label{ex:eigvectors2x2eig2} (Finding eigenvectors for Example \ref{ex:2x2eig2}) 
We know from Example \ref{ex:2x2eig2} that $B=\begin{bmatrix} 2& 1\\ 4&2
\end{bmatrix}$ has eigenvalues $\lambda_1=0$ and $\lambda_2=4$.  Compute a basis for the eigenspace associated with each of these eigenvalues.
\begin{explanation}
Let's begin by finding a basis for the eigenspace $\mathcal{S}_0$, which is the subspace of $\RR^n$ consisting of eigenvectors corresponding to the eigenvalue $\lambda_1=0$.  We need to compute a basis for $\mbox{null}(B-0I) = \mbox{null}(B)$.  We compute:
\begin{align*}\mbox{rref}(B)=\mbox{rref}\left(\begin{bmatrix}2&1\\4&2\end{bmatrix}\right)&=\begin{bmatrix}1&\frac{1}{2}\\0&0\end{bmatrix},
\end{align*}
From this we see that an eigenvector in $\mathcal{S}_0$ has the form $\begin{bmatrix}-1/2\\1\end{bmatrix}t$. %$\begin{bmatrix}x_1\\x_2\end{bmatrix}$ in $\mathcal{S}_0$ satisfies $x_1+\frac{1}{2} x_2=0$, so that $2x_1=-x_2$. 
This means that $\left\{\begin{bmatrix}-1/2\\1\end{bmatrix}\right\}$ is one possible basis for the eigenspace $\mathcal{S}_0$.  By letting $t=-2$, we obtain an arguably nicer-looking basis: $\left\{\begin{bmatrix}1\\-2\end{bmatrix}\right\}$. 

See if you can compute a basis for $\mathcal{S}_4$.  

Click on the arrow if you need help.

\begin{expandable}
To compute a basis for $\mathcal{S}_4$, the subspace of all eigenvectors associated to the eigenvalue $\lambda_2=4$, we compute:
$$\mbox{rref}(B-4I)=\mbox{rref}\left(\begin{bmatrix}-2&1\\4&-2\end{bmatrix}\right)=\begin{bmatrix}1&-\frac{1}{2}\\0&0\end{bmatrix}$$
\end{expandable}

From this we find that $\left\{\begin{bmatrix}1\\\answer{2}\end{bmatrix}\right\}$ is one possible basis for the eigenspace $\mathcal{S}_4$.
\end{explanation}
\end{example}

\begin{example}\label{ex:eigvectors3x3eig} (Finding eigenvectors for Example \ref{ex:3x3eig})
We know from Example \ref{ex:3x3eig} that $C=\begin{bmatrix} 2 & 1 & 1\\ 1 & 2 & 1\\ 1 & 1 & 2\end{bmatrix}$ has eigenvalues $\lambda_1=1$ and $\lambda_2=4$.  Compute a basis for the eigenspace associated to each of these eigenvalues.
\begin{explanation}
We first find a basis for the eigenspace $\mathcal{S}_1$.  We need to compute a basis for $\mbox{null}(C-I)$.  We compute:
\begin{align*}\mbox{rref}(C-I)=\mbox{rref}\left(\begin{bmatrix} 1 & 1 & 1\\ 1 & 1 & 1\\ 1 & 1 & 1\end{bmatrix}\right)&=\begin{bmatrix} 1 & 1 & 1\\ 0 & 0 & 0\\ 0 & 0 & 0\end{bmatrix},
\end{align*}
%From this we see for any eigenvector $\begin{bmatrix}x_1\\x_2\\x_3\end{bmatrix}$ in $\mathcal{S}_1$ satisfies $x_1+x_2+x_3=0$.  
Notice that there are two free variables.  %Let $x_2=s$ and $x_3=t$.
The eigenvectors in $\mathcal{S}_1$ have the form
$$\begin{bmatrix}-s-t\\s\\t\end{bmatrix} = s\begin{bmatrix}-1\\1\\0\end{bmatrix} + t\begin{bmatrix}-1\\0\\1\end{bmatrix}$$

So one possible basis for the eigenspace $\mathcal{S}_1$ is given by $\left\{\begin{bmatrix}-1\\1\\0\end{bmatrix}, \begin{bmatrix}-1\\0\\1\end{bmatrix}\right\}$.

Next we find a basis for the eigenspace $\mathcal{S}_4$.  We need to compute a basis for $\mbox{null}(C-4I)$.  We compute:
\begin{align*}\mbox{rref}(C-4I)=\mbox{rref}\left(\begin{bmatrix} -2 & 1 & 1\\ 1 & -2 & 1\\ 1 & 1 & -2\end{bmatrix}\right)&=\begin{bmatrix} 1 & 0 & -1\\ 0 & 1 & -1\\ 0 & 0 & 0\end{bmatrix}
\end{align*}
This time there is one free variable.  %Setting $x_3=t$, we also get $x_1=t$ and $x_2=t$.  From this we see
The eigenvectors in $\mathcal{S}_4$ have the form $\begin{bmatrix}t\\t\\t\end{bmatrix}$, so a possible basis for the eigenspace $\mathcal{S}_4$ is given by $\left\{\begin{bmatrix}1\\1\\1\end{bmatrix}\right\}$.
\end{explanation}
\end{example}

\begin{example}\label{ex:3x3_complex_ev} (Finding eigenvectors for Example \ref{ex:3x3_complex_eig})
We know from Example \ref{ex:3x3_complex_eig} that $D=\begin{bmatrix} 0&0&0\\ 0 &1&1\\ 0 & -1&1\end{bmatrix}$ has eigenvalues $\lambda=0$, $\lambda_1=1+i$, and $\lambda_2=1-i$.  Compute a basis for the eigenspace associated with each eigenvalue.
\begin{explanation}
We first find a basis for the eigenspace $\mathcal{S}_0$.  We need to compute a basis for $\mbox{null}(D-0I)=\mbox{null}(D)$.  We compute:
\begin{align*}\mbox{rref}(D)=\mbox{rref}\left(\begin{bmatrix} 0&0&0\\ 0 &1&1\\ 0 & -1&1\end{bmatrix}\right)&=\begin{bmatrix} 0 & 1 & 1\\ 0 & 0 & 1\\ 0 & 0 & 0\end{bmatrix},
\end{align*}
From this we see that for any eigenvector $\begin{bmatrix}x_1\\x_2\\x_3\end{bmatrix}$ in $\mathcal{S}_0$ we have $x_2=0$ and $x_3=0$, but $x_1$ is a free variable. 
So one possible basis for the eigenspace $\mathcal{S}_0$ is given by $$\left\{\begin{bmatrix}1\\0\\0\end{bmatrix}\right\}$$
Next we find a basis for the eigenspace $\mathcal{S}_{1+i}$.  We need to compute a basis for $\mbox{null}(D-(1+i)I)$.  We compute:
\begin{align*}\mbox{rref}(D-(1+i)I)&=\mbox{rref}\left(\begin{bmatrix} -(1+i)&0&0\\ 0 &1-(1+i)&1\\ 0 & -1&1-(1+i)\end{bmatrix}\right) \\
&=\begin{bmatrix} 1 & 0 &0\\ 0 & 1 & i\\ 0 & 0 & 0\end{bmatrix}
\end{align*}
There is one free variable.  Setting $x_3=t$, we get $x_1=0$ and $x_2=ti$.  From this we see that eigenvectors in $\mathcal{S}_{1+i}$ have the form $\begin{bmatrix}0\\i\\1\end{bmatrix}t$, so a possible basis for the eigenspace $\mathcal{S}_{1+i}$ is given by $\left\{\begin{bmatrix}0\\i\\1\end{bmatrix}\right\}$.
We ask you in Practice Problem \ref{prob:3x3_complex_ev} to show that $\left\{\begin{bmatrix}0\\-i\\1\end{bmatrix}\right\}$ is a basis for $\mathcal{S}_{1-i}$.
\end{explanation}
\end{example}

We conclude this section by establishing the significance of a matrix having an eigenvalue of zero.

\begin{theorem}\label{th:zero_ew}
A square matrix has an eigenvalue of zero if and only if it is singular.
\end{theorem}

\begin{proof}
A square matrix $A$ is singular if and only if $\det{A}=0$.(see  \ref{th:detofsingularmatrix}).  But $\det{A}=0$ if and only if $\det{A-0I}=0$, which is true if and only if zero is an eigenvalue of $A$.
\end{proof}

\section*{Practice Problems}

\emph{Problems \ref{prob:eigenspace1}-\ref{prob:eigenspace2}}
In this exercise we will prove that the eigenvectors associated with an eigenvalue $\lambda$ of an $n \times n$ matrix $A$ form a subspace of $\RR^n$.

\begin{problem}\label{prob:eigenspace1}
Let $\vec{x}$ and $\vec{y}$ be eigenvectors of $A$ associated with $\lambda$.  Show that $\vec{x}+\vec{y}$ is also an eigenvector of $A$ associated with $\lambda$.  (This shows that the set of eigenvectors of $A$ associated with $\lambda$ is closed under addition).
\end{problem}

\begin{problem}\label{prob:eigenspace2}
Show that the set of eigenvectors of $A$ associated with $\lambda$ is closed under scalar multiplication.
\end{problem}


\emph{Problems \ref{prob:eigenspace3}-\ref{prob:eigenspace4}}
Compute the eigenvalues of the given matrix and find the corresponding eigenspaces.

\begin{problem}\label{prob:eigenspace3}
$$\begin{bmatrix}4&1\\8&-3\end{bmatrix}$$
Answer:
(List the eigenvalues in an increasing order.)
$$\lambda_1=\answer{-4},\quad\lambda_2=\answer{5}$$

A basis for $\mathcal{S}_{\lambda_1}$ is $\left\{\begin{bmatrix}\answer{-1/8}\\1\end{bmatrix}\right\}$.  A basis for $\mathcal{S}_{\lambda_2}$ is $\left\{\begin{bmatrix}\answer{1}\\1\end{bmatrix}\right\}$.
\end{problem}

\begin{problem}\label{prob:eigenspace4}
$$\begin{bmatrix}1&-2\\2&1\end{bmatrix}$$
Answer:

$$\lambda_1=\answer{1}+\answer{2}i,\quad\lambda_2=\answer{1}-\answer{2}i$$

A basis for $\mathcal{S}_{\lambda_1}$ is $\left\{\begin{bmatrix}i\\\answer{1}\end{bmatrix}\right\}$.  A basis for $\mathcal{S}_{\lambda_2}$ is $\left\{\begin{bmatrix}-i\\\answer{1}\end{bmatrix}\right\}$.
\end{problem}


\begin{problem}\label{prob:3x3tri_ev}
Let $T=\begin{bmatrix} 1 & 2 & 3\\ 0 & 5 & 6\\ 0 & 0 & 9\end{bmatrix}$.  Compute a basis for each of the eigenspaces of this matrix, $\mathcal{S}_1$, $\mathcal{S}_5$, and $\mathcal{S}_9$.
\end{problem}

\emph{Problems \ref{prob:3x3fromKuttler1}-\ref{prob:3x3fromKuttler2}}

Let $A=\begin{bmatrix} 9 & 2 & 8\\ 2 & -6 & -2\\ -8 & 2 & -5\end{bmatrix}$.  

\begin{problem}\label{prob:3x3fromKuttler1}
Compute the eigenvalues of this matrix. 
\begin{hint}
One of the eigenvalues of $A$ is -3.
\end{hint}

Answer:

(List your answers in an increasing order.)
$$\lambda_1 = \answer{-3},\quad \lambda_2 = \answer{-1},\quad \lambda_3 = \answer{2}$$
\end{problem}

\begin{problem}\label{prob:3x3fromKuttler2}
Compute a basis for each of the eigenspaces of this matrix,  $\mathcal{S}_{\lambda_1}$, $\mathcal{S}_{\lambda_2}$, and $\mathcal{S}_{\lambda_3}$.

Answer:
A basis for $\mathcal{S}_{\lambda_1}$ is $\left\{\begin{bmatrix}1\\\answer{2}\\\answer{-2}\end{bmatrix}\right\}$, 
a basis for $\mathcal{S}_{\lambda_2}$ is $\left\{\begin{bmatrix}-2\\\answer{-2}\\\answer{3}\end{bmatrix}\right\}$,

and a basis for $\mathcal{S}_{\lambda_3}$ is $\left\{\begin{bmatrix}2\\\answer{1}\\\answer{-2}\end{bmatrix}\right\}$.
\end{problem}


\begin{problem}\label{prob:3x3_complex_ev}
Complete Example \ref{ex:3x3_complex_ev} by showing that a basis for $\mathcal{S}_{1-i}$ is given by $\left\{\begin{bmatrix}0\\-i\\1\end{bmatrix}\right\}$, where  $\mathcal{S}_{1-i}$ is the eigenspace associated with the eigenvalue $\lambda=1-i$ of the matrix $D=\begin{bmatrix} 0&0&0\\ 0 &1&1\\ 0 & -1&1\end{bmatrix}$.
\end{problem}


\begin{problem}\label{prob:eigtri}
Prove Theorem \ref{th:eigtri}.  (HINT:  Proceed by induction on the dimension n.  For the inductive step, compute $\det(A-\lambda I)$ by expanding along the first column (or row) if $T$ is upper (lower) triangular.)
\end{problem}

\emph{Problems \ref{prob:eigvectorstransfr2_1}-\ref{prob:eigvectorstransfr2_3}}
The following set of problems deals with geometric interpretation of eigenvalues and eigenvectors, as well as linear transformations of the plane.  Please use \href{https://ximera.osu.edu/linearalgebradzv3/LinearAlgebraInteractiveIntro/EIG-0010/main}{Describing Eigenvalues and Eigenvectors Algebraically and Geometrically} and \href{https://ximera.osu.edu/linearalgebradzv3/LinearAlgebraInteractiveIntro/LTR-0070/main}{Geometric Transformations of the Plane} for reference.

\begin{problem}\label{prob:eigvectorstransfr2_1}
Recall that a vertical stretch/compression of the plane is a linear transformation whose standard matrix is $$M_v=\begin{bmatrix}1&0\\0&k\end{bmatrix}$$
Find the eigenvalues of $M_v$.  Find a basis for the eigenspace corresponding to each eigenvalue.

Answer:  A basis for $\mathcal{S}_1$ is $\left\{\begin{bmatrix}1\\\answer{0}\end{bmatrix}\right\}$
and a basis for $\mathcal{S}_k$ is $\left\{\begin{bmatrix}\answer{0}\\1\end{bmatrix}\right\}$

Sketch several vectors in each eigenspace and use geometry to explain why the eigenvectors you sketched make sense.
\end{problem}

\begin{problem}\label{prob:eigvectorstransfr2_2}
Recall that a horizontal shear of the plane is a linear transformation whose standard matrix is $$M_{hs}=\begin{bmatrix}1&k\\0&1\end{bmatrix}$$
Find the eigenvalue of $M_{hs}$.  

Answer: $\lambda=\answer{1}$

Find a basis for the eigenspace corresponding to $\lambda$.

Answer:  A basis for $\mathcal{S}$ is $\left\{\begin{bmatrix}1\\\answer{0}\end{bmatrix}\right\}$

Sketch several vectors in the eigenspace and use geometry to explain why the eigenvectors you sketched make sense.
\end{problem}

\begin{problem}\label{prob:rotmatrixrealeig2}
Recall that a counterclockwise rotation of the plane through angle $\theta$ is a linear transformation whose standard matrix is $$M_{\theta}=\begin{bmatrix}\cos\theta&-\sin\theta\\\sin\theta&\cos\theta\end{bmatrix}$$
Verify that the eigenvalues of $M_{\theta}$ are
$$\lambda=\cos\theta\pm\sqrt{\cos^2\theta-1}$$
Explain why $\lambda$ is a real number if and only if $\theta$ is a multiple of $\pi$.  (Compare this to Practice Problem \ref{prob:rotmatrixrealeig1} of \href{https://ximera.osu.edu/linearalgebradzv3/LinearAlgebraInteractiveIntro/EIG-0010/main}{Describing Eigenvalues and Eigenvectors Algebraically and Geometrically}.)

Suppose $\theta$ is a muliple of $\pi$.  Then the eigenspaces corresponding to the two eigenvalues are the same.  Which of the following describes the eigenspace?
\begin{multipleChoice}
    \choice[correct]{All vectors in $\RR^2$.}
    \choice{All vectors along the $x$-axis.}
    \choice{All vectors along the $y$-axis.}
    \choice{All vectors along the line $y=x$.}
\end{multipleChoice}

\end{problem}

\begin{problem}\label{prob:eigvectorstransfr2_3}
Recall that a reflection of the plane about the line $y=mx$ is a linear transformation whose standard matrix is
$$M_{y=mx}=\frac{1}{1+m^2}\begin{bmatrix}
1-m^2 & 2m \\
2m & m^2-1
\end{bmatrix}$$
Verify that the eigenvalues of $M_{y=mx}$ are
$1$ and $-1$.

Find a basis for eigenspaces $\mathcal{S}_{1}$ and $\mathcal{S}_{-1}$.  (For simplicity, assume that $m\neq 0$.)

Answer:  A basis for $\mathcal{S}_{1}$ is $\left\{\begin{bmatrix}1\\\answer{m}\end{bmatrix}\right\}$
and a basis for $\mathcal{S}_{-1}$ is $\left\{\begin{bmatrix}m\\\answer{-1}\end{bmatrix}\right\}$

Choose the best description of $\mathcal{S}_{1}$.
\begin{multipleChoice}
    \choice{All vectors in $\RR^2$.}
    \choice[correct]{All vectors with ``slope" $m$.}
    \choice{All vectors with ``slope" $1/m$.}
    \choice{All vectors with ``slope" $-1/m$.}
\end{multipleChoice}

Choose the best description of $\mathcal{S}_{-1}$.
\begin{multipleChoice}
    \choice{All vectors along the line $y=mx$.}
    \choice{All vectors parallel to the $x$-axis.}
    \choice{All vectors parallel to the $y$-axis.}
    \choice[correct]{All vectors perpendicular to the line $y=mx$.}
\end{multipleChoice}

Use geometry to explain why the eigenspaces you found make sense.

\end{problem}


\section*{Exercise Source}
Practice Problem \ref{prob:3x3fromKuttler1} is adopted from Problem 7.1.11 of Ken Kuttler's \href{https://open.umn.edu/opentextbooks/textbooks/a-first-course-in-linear-algebra-2017}{\it A First Course in Linear Algebra}. (CC-BY)

Ken Kuttler, {\it  A First Course in Linear Algebra}, Lyryx 2017, Open Edition, p. 361.

\end{document}
