\documentclass{ximera}
%% You can put user macros here
%% However, you cannot make new environments

\listfiles

\graphicspath{{./}{firstExample/}{secondExample/}}

\usepackage{tikz}
\usepackage{tkz-euclide}
\usepackage{tikz-3dplot}
\usepackage{tikz-cd}
\usetikzlibrary{shapes.geometric}
\usetikzlibrary{arrows}
%\usetkzobj{all}
\pgfplotsset{compat=1.13} % prevents compile error.

%\renewcommand{\vec}[1]{\mathbf{#1}}
\renewcommand{\vec}{\mathbf}
\newcommand{\RR}{\mathbb{R}}
\newcommand{\dfn}{\textit}
\newcommand{\dotp}{\cdot}
\newcommand{\id}{\text{id}}
\newcommand\norm[1]{\left\lVert#1\right\rVert}
 
\newtheorem{general}{Generalization}
\newtheorem{initprob}{Exploration Problem}

\tikzstyle geometryDiagrams=[ultra thick,color=blue!50!black]

%\DefineVerbatimEnvironment{octave}{Verbatim}{numbers=left,frame=lines,label=Octave,labelposition=topline}



\usepackage{mathtools}


 \title{Introduction to Bases} \license{CC BY-NC-SA 4.0}



\begin{document}
\begin{abstract}

\end{abstract}
\maketitle
\begin{onlineOnly}
\section*{Introduction to Bases}
\end{onlineOnly}

\subsection*{Coordinate Vectors}
When we first introduced vectors we learned to represent them using component notation.  If $\vec{u}=\begin{bmatrix}2\\5\end{bmatrix}$ we know that the head of $\vec{u}$ is located at the point $(2, 5)$.  But there is another way to look at the component form.  Observe that $\vec{u}$ can be expressed as a linear combination of the standard unit vectors $\vec{i}$ and $\vec{j}$:
$$\vec{u}=2\begin{bmatrix}1\\0\end{bmatrix}+5\begin{bmatrix}0\\1\end{bmatrix}=2\vec{i}+5\vec{j}$$
In fact, any vector $\vec{v}=\begin{bmatrix}a\\b\end{bmatrix}$ of $\RR^2$ can be written as a linear combination of $\vec{i}$ and $\vec{j}$:
$$\vec{v}=\begin{bmatrix}a\\b\end{bmatrix}=a\vec{i}+b\vec{j}$$
This gives us an alternative way of interpreting the component notation:
$$\left[\begin{array}{c}  
 a\\b
 \end{array}\right]
 \begin{array}{c}
 \longleftarrow\\
 \longleftarrow
 \end{array}
\begin{array}{c}  
 \mbox{coefficient in front of $\vec{i}$}\\\mbox{coefficient in front of $\vec{j}$}
 \end{array}$$
 
 We say that $a$ and $b$ are \dfn{coordinates} of $\vec{v}$ with respect to $\{\vec{i}, \vec{j}\}$, and $\begin{bmatrix}a\\b\end{bmatrix}$ is said to be the \dfn{coordinate vector} for $\vec{v}$ with respect to $\{\vec{i}, \vec{j}\}$.  Every vector $\vec{v}$ of $\RR^2$ can be thus represented using $\vec{i}$ and $\vec{j}$.  Moreover, such representation in terms of $\vec{i}$ and $\vec{j}$ is unique for each vector, meaning that we will never have two different coordinate vectors representing the same vector. We will refer to $\{\vec{i}, \vec{j}\}$ as a \dfn{basis} of $\RR^2$.  
 
 The order in which the basis elements are written matters.  For example, $\vec{u}$ is represented by the coordinate vector $\begin{bmatrix}2\\5\end{bmatrix}$ with respect to $\{\vec{i}, \vec{j}\}$, but changing the basis to $\{\vec{j}, \vec{i}\}$ would change the coordinate vector to $\begin{bmatrix}5\\2\end{bmatrix}$.
 
 $$\left[\begin{array}{c}  
 5\\2
 \end{array}\right]
 \begin{array}{c}
 \longleftarrow\\
 \longleftarrow
 \end{array}
\begin{array}{c}  
 \mbox{coefficient in front of the first basis element }\\\mbox{coefficient in front of the second basis element}
 \end{array}$$
 
 Clearly, standard unit vectors $\vec{i}$ and $\vec{j}$ are very convenient, but other vectors can also be used in place of $\vec{i}$ and $\vec{j}$ to represent $\vec{u}$.
 
 \begin{exploration}\label{init:altbasis1}
 The diagram below shows $\vec{u}$ together with vectors $\vec{w}_1$ and $\vec{w}_2$.
 
 
 \begin{center}
\begin{tikzpicture}[scale=0.6]
\draw[thin,gray!40] (-6,-2) grid (6,6);
  \draw[<->] (-6,0)--(6,0);
  \draw[<->] (0,-2)--(0,6);
  
  \draw[line width=2pt,red,-stealth](0,0)--(2,5) node[above right]{$\vec{u}$};
  \draw[line width=2pt,blue,-stealth](0,0)--(-1,2) node[below left]{$\vec{w}_1$};
  \draw[line width=2pt,blue,-stealth](0,0)--(4,1) node[below right]{$\vec{w}_2$};

 \end{tikzpicture}
\end{center}
 

It is easy to see that 
$$\vec{u}=2\vec{w}_1+\vec{w}_2$$
as shown below.
\begin{center}
\begin{tikzpicture}[scale=0.6]
\draw[thin,gray!40] (-6,-2) grid (6,6);
  \draw[<->] (-6,0)--(6,0);
  \draw[<->] (0,-2)--(0,6);
  
  \filldraw[blue, opacity=0.3](0,0)--(4,1)--(2,5)--(-2,4)--cycle;
  
  \draw[line width=2pt,red,-stealth](0,0)--(2,5) node[above right]{$\vec{u}$};
  \draw[line width=2pt,blue,-stealth](0,0)--(-1,2) node[below left]{$\vec{w}_1$};
  \draw[line width=2pt,blue,-stealth](0,0)--(4,1) node[below right]{$\vec{w}_2$};
  
   \draw[line width=1pt,blue,dashed](0,0)--(-2,4);

 \end{tikzpicture}
\end{center}

If we declare $\{\vec{w}_1, \vec{w}_2\}$ to be a basis of $\RR^2$, then we can say that the coordinate vector for $\vec{u}$ with respect to $\{\vec{w}_1, \vec{w}_2\}$ is 
$\begin{bmatrix}2\\1\end{bmatrix}$ .

$$\left[\begin{array}{c}  
 2\\1
 \end{array}\right]
 \begin{array}{c}
 \longleftarrow\\
 \longleftarrow
 \end{array}
\begin{array}{c}  
 \mbox{coefficient in front of the first basis element }\\\mbox{coefficient in front of the second basis element}
 \end{array}$$
\end{exploration}

\subsection*{What Constitutes a Basis?}

In the previous section we had used the term \dfn{basis} without defining it.  Now is the time to pause and think about what we {\it want} a basis to do.  Let's focus on $\RR^n$ and subspaces of $\RR^n$.  What we establish here will easily generalize to other vector spaces.  

Based on our previous discussion, given any vector $\vec{v}$ of $\RR^n$ (or a subspace $V$ of $\RR^n$), we want to be able to write a coordinate vector for $\vec{v}$ with respect to the given basis of $\RR^n$ (or $V$).  Based on this condition, we will require that basis vectors span $\RR^n$ (or $V$).

For example, consider $\vec{w}_1$ and $\vec{w}_2$ shown below.  

\begin{center}
\tdplotsetmaincoords{70}{130}
\begin{tikzpicture}
	\draw[->](-2,0,0)--(5,0,0) node[below left]{$y$};
    \draw[->](0,-2,0)--(0,5,0) node[below left]{$z$};
    \draw[->](0,0,-2)--(0,0,6) node[below left]{$x$};
    \filldraw[blue, opacity=0.3] (0,0,0)--(4, 2, 0)--(4,6,5)--(0, 4, 5)--cycle;
    \draw[->, line width=2pt,blue, -stealth](0,0,0)--(0,4,5)node[below left]{$\vec{w}_2$};
    %\draw[line width=1pt,blue, dashed](0,4,0)--(0,4,5);
    %\draw[line width=1pt,blue, dashed](0,0,5)--(0,4,5);
    
    \draw[->, line width=2pt,red, -stealth](0,0,0)--(4,2,0)node[above right]{$\vec{w}_1$};
    
    %\draw[line width=1pt,red, dashed](0,2,0)--(4,2,0);
    %\draw[line width=1pt,red, dashed](4,0,0)--(4,2,0);
    %\draw[->, line width=2pt, -stealth](0,0,0)--(4,6,5)node[above left]{$\begin{bmatrix}5\\4\\6\end{bmatrix}$};
    \end{tikzpicture}
\end{center}
The set $\{\vec{w}_1, \vec{w}_2\}$ cannot be a basis for $\RR^3$ because $\vec{w}_1$ and $\vec{w}_2$ span a plane in $\RR^3$, and any vector not in that plane cannot be written as a linear combination of $\vec{w}_1$ and $\vec{w}_2$.

On the other hand, the plane spanned by $\vec{w}_1$ and $\vec{w}_2$ is a subspace of $\RR^3$.  Because every vector in that plane can be written as a linear combination of $\vec{w}_1$ and $\vec{w}_2$, the set $\{\vec{w}_1, \vec{w}_2\}$ can potentially be a basis for the plane, provided that the set satisfies our second requirement.

Our second requirement is that for a fixed basis of $\RR^n$ (or $V$), the coordinate vector for each $\vec{v}$ in $\RR^n$ (or $V$) should be unique.  Uniqueness of representation in terms of the basis elements will play an important role in our future study of functions that map vector spaces to vector spaces.

The following theorem shows that the uniqueness requirement is equivalent to the requirement that the basis vectors be linearly independent.  %In other words, a basis may not contain redundant vectors.

\begin{theorem}\label{th:linindbasis}
Suppose $\{\vec{w}_1, \vec{w}_2,\ldots,\vec{w}_p\}$ is a spanning set for a subspace $V$ of $\RR^n$.  Then every element $\vec{v}$ of $V$ has a unique representation as a linear combination of  $\vec{w}_1, \vec{w}_2,\ldots,\vec{w}_p$ if and only if the vectors $\vec{w}_1, \vec{w}_2,\ldots,\vec{w}_p$ are linearly independent.
\end{theorem}
\begin{proof}
Suppose that every $\vec{v}$ in $V$ can be expressed as a unique linear combination of $\vec{w}_1, \vec{w}_2,\ldots,\vec{w}_p$.
This means that $\vec{0}$ has a unique representation as a linear combination of $\vec{w}_1, \vec{w}_2,\ldots,\vec{w}_p$.
But 
$$\vec{0}=0\vec{w}_1+0\vec{w}_2+\ldots+0\vec{w}_p$$
is a representation of $\vec{0}$ in terms of $\vec{w}_1, \vec{w}_2,\ldots,\vec{w}_p$.  Since we are assuming that such a representation is unique, we conclude that there is no other.  This means that the vectors $\vec{w}_1, \vec{w}_2,\ldots,\vec{w}_p$ are linearly independent.

Conversely, suppose that vectors $\vec{w}_1, \vec{w}_2,\ldots,\vec{w}_p$ are linearly independent.  An arbitrary element $\vec{v}$ of $V$ can be expressed as a linear combination of $\vec{w}_1, \vec{w}_2,\ldots,\vec{w}_p$:
$$\vec{v}=a_1\vec{w}_1+a_2\vec{w}_2+\ldots+a_p\vec{w}_p$$
Suppose this representation is not unique.  Then there may be another linear combination that is also equal to $\vec{v}$:
$$\vec{v}=b_1\vec{w}_1+b_2\vec{w}_2+\ldots+b_p\vec{w}_p$$
But then
$$a_1\vec{w}_1+a_2\vec{w}_2+\ldots+a_p\vec{w}_p=b_1\vec{w}_1+b_2\vec{w}_2+\ldots+b_p\vec{w}_p$$
This gives us
$$(a_1-b_1)\vec{w}_1+(a_2-b_2)\vec{w}_2+\ldots+(a_p-b_p)\vec{w}_p=\vec{0}$$
Because we assumed that $\vec{w}_1, \vec{w}_2,\ldots,\vec{w}_p$ are linearly independent, we must have
$$a_1-b_1=0,\, a_2-b_2=0,\,\ldots ,\,a_p-b_p=0$$
so that
$$a_1=b_1,\, a_2=b_2,\,\ldots ,\,a_p=b_p$$
This proves the representation of $\vec{v}$ in terms of $\vec{w}_1, \vec{w}_2,\ldots,\vec{w}_p$ is unique.

\end{proof}

\begin{example}\label{ex:lindepandbasis}
Use $V=\mbox{span}(\mathcal{S})$, where $\mathcal{S}=\left\{\begin{bmatrix}5\\2\\4\end{bmatrix},\begin{bmatrix}4\\1\\1\end{bmatrix},\begin{bmatrix}-3\\0\\2\end{bmatrix}\right\}$ to illustrate why a set of linearly dependent vectors cannot be used as a basis for a subspace by showing that linearly dependent vectors fail to ensure uniqueness of coordinate vectors  for vectors in $V$.
\begin{explanation}
We will first show that the elements of $\mathcal{S}$ are linearly dependent.  Let $A$ be a matrix whose columns are the vectors in $\mathcal{S}$.
$$A=\begin{bmatrix}5&4&-3\\2&1&0\\4&1&2\end{bmatrix}$$
We find that 
$$\mbox{rref}(A) = \begin{bmatrix}  
 1&0&1\\0&1&-2\\0&0&0
 \end{bmatrix}$$
Therefore the matrix equation  $A\vec{x}=\vec{0}$ has infinitely many solutions:
$$\vec{x}=\begin{bmatrix}-1\\2\\1\end{bmatrix}t$$
This tells us that there are infinitely many nontrivial linear relations among the elements of $\mathcal{S}$.  Letting $t=1$ gives us one such nontrivial relation.
$$-\begin{bmatrix}5\\2\\4\end{bmatrix}+2\begin{bmatrix}4\\1\\1\end{bmatrix}+\begin{bmatrix}-3\\0\\2\end{bmatrix}=\vec{0}$$

Now let's pick an arbitrary vector $\vec{v}$ in $V$.  Any vector will do, so let 
$$\vec{v}=\begin{bmatrix}5\\2\\4\end{bmatrix}+ (-1)\begin{bmatrix}4\\1\\1\end{bmatrix}+0\begin{bmatrix}-3\\0\\2\end{bmatrix}$$
Based on this representation of $\vec{v}$, the coordinate vector for $\vec{v}$ with respect to $\mathcal{S}$ is 
$$\begin{bmatrix}\answer{1}\\\answer{-1}\\\answer{0}\end{bmatrix}$$
But 
$$\begin{bmatrix}5\\2\\4\end{bmatrix}=2\begin{bmatrix}4\\1\\1\end{bmatrix}+\begin{bmatrix}-3\\0\\2\end{bmatrix}$$
So, by substitution, we have:
$$\vec{v}=\left(2\begin{bmatrix}4\\1\\1\end{bmatrix}+\begin{bmatrix}-3\\0\\2\end{bmatrix}\right)+ (-1)\begin{bmatrix}4\\1\\1\end{bmatrix}+0\begin{bmatrix}-3\\0\\2\end{bmatrix}=0\begin{bmatrix}5\\2\\4\end{bmatrix}+ 1\begin{bmatrix}4\\1\\1\end{bmatrix}+1\begin{bmatrix}-3\\0\\2\end{bmatrix}$$
Based on this representation, the coordinate vector for $\vec{v}$ with respect to $\mathcal{S}$ is
$$\begin{bmatrix}\answer{0}\\\answer{1}\\\answer{1}\end{bmatrix}$$
The set $\mathcal{S}$ is linearly dependent.  As a result, coordinate vectors for elements of $V$ are not unique and we do not want to use $\mathcal{S}$ as a basis for $V$.
\end{explanation}
\end{example}




\subsection*{Definition of a Basis}

\begin{definition}\label{def:basis}
A set $\mathcal{S}$ of vectors is called a \dfn{basis} of $\RR^n$ (or a basis of a subspace $V$ of $\RR^n$) provided that 
\begin{enumerate}
\item \label{item:defbasis1}
$\mbox{span}(\mathcal{S})=\RR^n$ (or $V$)
\item \label{item:defbasis2}
$\mathcal{S}$ is linearly independent.
\end{enumerate}
\end{definition}

\begin{example}\label{ex:standardbasis}
The standard unit vectors $\vec{e}_1, \ldots ,\vec{e}_n$ are linearly independent and span $\RR^n$.  Thus $\{\vec{e}_1, \ldots ,\vec{e}_n\}$ is a basis of $\RR^n$.
\end{example}

\begin{definition}\label{def:standardbasis} The set $\{\vec{e}_1, \ldots ,\vec{e}_n\}$ is called the \dfn{standard basis} of $\RR^n$.
\end{definition}

Bases are not unique.  For example, we know that vectors $\vec{i}$ and $\vec{j}$ form the standard basis of $\RR^2$.  But, as we discussed in Example \ref{ex:spanr2}, vectors 
$$\begin{bmatrix}2\\2\end{bmatrix}, \begin{bmatrix}-1\\0\end{bmatrix}$$
are linearly independent vectors that span $\RR^2$.  Therefore $$\left\{\begin{bmatrix}2\\2\end{bmatrix}, \begin{bmatrix}-1\\0\end{bmatrix}\right\}$$
is also a basis for $\RR^2$.

Any linearly independent spanning set in $\RR^n$ (or a subspace of $\RR^n$) is a basis of $\RR^n$ (or the subspace).  The plural form of the word \dfn{basis} is \dfn{bases}.  It is easy to see that $\RR^n$ and its subspaces each has infinitely many bases.
% \begin{example}
% Let $V=\mbox{span}\left(\begin{bmatrix}-2\\1\\3\end{bmatrix},\begin{bmatrix}2\\-4\\1\end{bmatrix}\right)$.  Prove that $\mathcal{B}=\left\{\begin{bmatrix}-2\\1\\3\end{bmatrix},\begin{bmatrix}2\\-4\\1\end{bmatrix}\right\}$ is a basis for $V$ and find the coordinate vector for $\vec{v}=\begin{bmatrix}2\\-10\\9\end{bmatrix}$ with respect to $\mathcal{B}$.
% \begin{explanation}
% Clearly the two vectors in $\mathcal{B}$ span $V$.  To show that the two vectors are linearly independent, we solve the vector equation:
% $$a_1\begin{bmatrix}-2\\1\\3\end{bmatrix}+a_2\begin{bmatrix}2\\-4\\1\end{bmatrix}=\vec{0}$$
% The augmented matrix corresponding to this equation reduces as follows:
% $$\left[\begin{array}{cc|c}  
%  -2&2&0\\-1&-4&0\\3&1&0
%  \end{array}\right]\rightsquigarrow\left[\begin{array}{cc|c}  
%  1&0&0\\0&1&0\\0&0&0
%  \end{array}\right]$$
%  This gives us $a_1=a_2=0$, and we conclude that the two elements of $\mathcal{B}$ are linearly independent.  By Definition \ref{def:basis}, $\mathcal{B}$ is a basis of $V$.
 
%  To find the coordinate vector for $\begin{bmatrix}2\\-10\\9\end{bmatrix}$ with respect to $\mathcal{B}$, we need to express $\begin{bmatrix}2\\-10\\9\end{bmatrix}$ as a linear combination of the elements of $\mathcal{B}$.  To this end, we need to solve the vector equation:
%  $$a_1\begin{bmatrix}-2\\1\\3\end{bmatrix}+a_2\begin{bmatrix}2\\-4\\1\end{bmatrix}=\begin{bmatrix}2\\-10\\9\end{bmatrix}$$
%  The augmented matrix and the reduced row-echelon form are:
%  $$\left[\begin{array}{cc|c}  
%  -2&2&2\\-1&-4&-10\\3&1&9
%  \end{array}\right]\rightsquigarrow\left[\begin{array}{cc|c}  
%  1&0&2\\0&1&3\\0&0&0
%  \end{array}\right]$$
%  We conclude that $a_1=2$, $a_2=3$.  This gives us
%  $$2\begin{bmatrix}-2\\1\\3\end{bmatrix}+3\begin{bmatrix}2\\-4\\1\end{bmatrix}=\begin{bmatrix}2\\-10\\9\end{bmatrix}$$
%  The coefficient in front of the first basis vector is $2$, the coefficient in front of the second basis vector is $3$.  This means that the coordinate vector for $\begin{bmatrix}2\\-10\\9\end{bmatrix}$ with respect to $\mathcal{B}$ is $\begin{bmatrix}2\\3\end{bmatrix}$.
 
%  It may seem strange to you that the coordinate vector for a vector in $\RR^3$ only has two components.  But remember that subspace $V$ is a plane.  When viewed as a vector in the plane, it  makes sense that the coordinate vector for $\begin{bmatrix}2\\-10\\9\end{bmatrix}$ only requires two components.  This issue is related to the question of dimension, which will be addressed in a subsequent module.
% \end{explanation}
% \end{example}

\begin{example}\label{ex:coordvectorandbasis}
Let $V=\mbox{span}\left(\begin{bmatrix}-2\\1\\3\end{bmatrix},\begin{bmatrix}2\\-4\\1\end{bmatrix}\right)$.  The set $$\mathcal{B}=\left\{\begin{bmatrix}-2\\1\\3\end{bmatrix},\begin{bmatrix}2\\-4\\1\end{bmatrix}\right\}$$ is a basis for $V$ because the two vectors in $\mathcal{B}$ are linearly independent and span $V$.  Find the coordinate vector for $\vec{v}=\begin{bmatrix}2\\-10\\9\end{bmatrix}$ with respect to $\mathcal{B}$.
\begin{explanation}
We need to express $\begin{bmatrix}2\\-10\\9\end{bmatrix}$ as a linear combination of the elements of $\mathcal{B}$.  To this end, we need to solve the vector equation:
 $$a_1\begin{bmatrix}-2\\1\\3\end{bmatrix}+a_2\begin{bmatrix}2\\-4\\1\end{bmatrix}=\begin{bmatrix}2\\-10\\9\end{bmatrix}$$
 The augmented matrix and the reduced row-echelon form are:
 $$\left[\begin{array}{cc|c}  
 -2&2&2\\-1&-4&-10\\3&1&9
 \end{array}\right]\rightsquigarrow\left[\begin{array}{cc|c}  
 1&0&2\\0&1&3\\0&0&0
 \end{array}\right]$$
 We conclude that $a_1=2$, $a_2=3$.  This gives us
 $$2\begin{bmatrix}-2\\1\\3\end{bmatrix}+3\begin{bmatrix}2\\-4\\1\end{bmatrix}=\begin{bmatrix}2\\-10\\9\end{bmatrix}$$
 
 The coefficient in front of the first basis vector is $2$, the coefficient in front of the second basis vector is $3$.  This means that the coordinate vector for $\begin{bmatrix}2\\-10\\9\end{bmatrix}$ with respect to $\mathcal{B}$ is $\begin{bmatrix}2\\3\end{bmatrix}$.
 
 \begin{remark}
 It may seem strange to you that the coordinate vector for a vector in $\RR^3$ only has two components.  But remember that subspace $V$ is a plane.  When viewed as a vector in the plane, it  makes sense that the coordinate vector for $\begin{bmatrix}2\\-10\\9\end{bmatrix}$ only requires two components.  This issue is related to the question of dimension, which will be addressed in \href{https://ximera.osu.edu/linearalgebradzv3/LinearAlgebraInteractiveIntro/VSP-0035/main}{Bases and Dimension}.
 \end{remark}
 
 \begin{remark}
 To construct the coordinate vector for $\begin{bmatrix}2\\-10\\9\end{bmatrix}$ with respect to $\mathcal{B}$, we had to be mindful of the order of the elements in $\mathcal{B}$.  Ordinarily, the order of elements in a set is irrelevant, and the basis $$\left\{\begin{bmatrix}-2\\1\\3\end{bmatrix},\begin{bmatrix}2\\-4\\1\end{bmatrix}\right\}$$ is considered to be the same as $$\left\{\begin{bmatrix}2\\-4\\1\end{bmatrix},\begin{bmatrix}-2\\1\\3\end{bmatrix}\right\}$$ When dealing with coordinate vectors, however, the order of the elements dictates the order of the components of the coordinate vector coefficients.  If we switch the order of the elements in $\mathcal{B}$, the coordinate vector becomes $\begin{bmatrix}3\\2\end{bmatrix}$.  For this reason, when we come back to studying coordinate vectors in more detail, we will use the term \dfn{ordered basis} to avoid confusion.
 \end{remark}
 
\end{explanation}
\end{example}


\section*{Practice Problems}
\emph{Problems \ref{prob:coordvect1}-\ref{prob:coordvect2}}
Let $\mathcal{B}=\left\{\begin{bmatrix}1\\1\end{bmatrix},\begin{bmatrix}-1\\2\end{bmatrix}\right\}$ be a basis for $\RR^2$.  (Do a mental verification that $\mathcal{B}$ is a basis.)  For each $\vec{v}$ given below, find the coordinate vector for $\vec{v}$ with respect to $\mathcal{B}$.

  \begin{problem}\label{prob:coordvect1}
 Vector $\vec{v}$. 
  \begin{center}
\begin{tikzpicture}[scale=0.6]
\draw[thin,gray!40] (-4,-2) grid (4,4);
  \draw[<->] (-4,0)--(4,0);
  \draw[<->] (0,-2)--(0,4);
   \draw[line width=2pt,red,-stealth](0,0)--(-3,0) node[above left]{$\vec{v}$};
   \end{tikzpicture}
\end{center}

Answer:
$$\begin{bmatrix}\answer{-2}\\\answer{1}\end{bmatrix}$$
  \end{problem}
  
  \begin{problem}\label{prob:coordvect2}
  Vector $\vec{v}$.
  \begin{center}
\begin{tikzpicture}[scale=0.6]
\draw[thin,gray!40] (-1,-1) grid (4,8);
  \draw[<->] (-1,0)--(4,0);
  \draw[<->] (0,-1)--(0,8);
   \draw[line width=2pt,red,-stealth](0,0)--(1,7) node[above left]{$\vec{v}$};
   \end{tikzpicture}
\end{center}
Answer:
$$\begin{bmatrix}\answer{3}\\\answer{2}\end{bmatrix}$$
  \end{problem}


\begin{problem}\label{prob:coordvect3}
Let $\mathcal{B}=\left\{\begin{bmatrix}1\\-1\\3\end{bmatrix},\begin{bmatrix}2\\1\\-1\end{bmatrix}\right\}$ be a basis for 
$\mbox{span}\left(\begin{bmatrix}1\\-1\\3\end{bmatrix},\begin{bmatrix}2\\1\\-1\end{bmatrix}\right)$.  Find the coordinate vector for $\begin{bmatrix}-4\\-2\\2\end{bmatrix}$ with respect to $\mathcal{B}$.

Answer:
$$\begin{bmatrix}\answer{0}\\\answer{-2}\end{bmatrix}$$
\end{problem}

\begin{problem}\label{prob:coordvect4}
Suppose $\mathcal{B}=\left\{\begin{bmatrix}1\\1\\1\end{bmatrix},\begin{bmatrix}1\\0\\1\end{bmatrix}, \vec{w}\right\}$ is a basis for $\RR^3$.  Find $\vec{w}$ if the coordinate vector for $\begin{bmatrix}-2\\-7\\4\end{bmatrix}$ is $\begin{bmatrix}-1\\2\\-3\end{bmatrix}$.

Answer:
$$\begin{bmatrix}\answer{1}\\\answer{2}\\\answer{-1}\end{bmatrix}$$
\end{problem}

\begin{problem}\label{prob:basisr2}

Which of the following is a basis for $\RR^2$? 

\begin{selectAll}
  \choice{$\left\{\begin{bmatrix}1\\1\end{bmatrix},\begin{bmatrix}-1\\-1\end{bmatrix}, \begin{bmatrix}1\\2\end{bmatrix}\right\}$}
  \choice[correct]{$\left\{\begin{bmatrix}1\\1\end{bmatrix}, \begin{bmatrix}1\\2\end{bmatrix}\right\}$}
  \choice{$\left\{\begin{bmatrix}3\\-1\end{bmatrix},\begin{bmatrix}1\\2\end{bmatrix}, \begin{bmatrix}-4\\3\end{bmatrix}\right\}$}
   \choice{$\left\{\begin{bmatrix}1\\-3\end{bmatrix}, \begin{bmatrix}-2\\6\end{bmatrix}\right\}$}
  \end{selectAll}
\end{problem}

\begin{problem}\label{prob:basisarbitraryspace}

Which of the following is a basis for $V=\mbox{span}\left(\begin{bmatrix}1\\1\\1\end{bmatrix}, \begin{bmatrix}1\\-2\\1\end{bmatrix}\right)$? 

\begin{selectAll}
  \choice[correct]{$\left\{\begin{bmatrix}2\\-1\\2\end{bmatrix},\begin{bmatrix}1\\-2\\1\end{bmatrix}\right\}$}
  \choice[correct]{$\left\{\begin{bmatrix}0\\3\\0\end{bmatrix}, \begin{bmatrix}3\\-3\\3\end{bmatrix}\right\}$}
  \choice{$\left\{\begin{bmatrix}1\\0\\0\end{bmatrix},\begin{bmatrix}0\\0\\1\end{bmatrix}\right\}$}
  \choice{$\left\{\begin{bmatrix}1\\1\\1\end{bmatrix},\begin{bmatrix}2\\-1\\2\end{bmatrix}, \begin{bmatrix}1\\-2\\1\end{bmatrix}\right\}$}
  \end{selectAll}
\end{problem}
\end{document}
