\documentclass{ximera}
%% You can put user macros here
%% However, you cannot make new environments

\listfiles

\graphicspath{{./}{firstExample/}{secondExample/}}

\usepackage{tikz}
\usepackage{tkz-euclide}
\usepackage{tikz-3dplot}
\usepackage{tikz-cd}
\usetikzlibrary{shapes.geometric}
\usetikzlibrary{arrows}
%\usetkzobj{all}
\pgfplotsset{compat=1.13} % prevents compile error.

%\renewcommand{\vec}[1]{\mathbf{#1}}
\renewcommand{\vec}{\mathbf}
\newcommand{\RR}{\mathbb{R}}
\newcommand{\dfn}{\textit}
\newcommand{\dotp}{\cdot}
\newcommand{\id}{\text{id}}
\newcommand\norm[1]{\left\lVert#1\right\rVert}
 
\newtheorem{general}{Generalization}
\newtheorem{initprob}{Exploration Problem}

\tikzstyle geometryDiagrams=[ultra thick,color=blue!50!black]

%\DefineVerbatimEnvironment{octave}{Verbatim}{numbers=left,frame=lines,label=Octave,labelposition=topline}



\usepackage{mathtools}


\title{Essential Vocabulary} \license{CC BY-NC-SA 4.0}



\begin{document}
\begin{abstract}
\end{abstract}
\maketitle


\begin{onlineOnly}
\section*{Essential Vocabulary}
Here is a  \href{https://quizlet.com/880920377/chapter-4-vocabulary-flash-cards/?i=y06sd&x=1jqt}{link to a list of these terms on Quizlet}
\end{onlineOnly}

\begin{tikzpicture}[scale=1]
   \filldraw[teal, opacity=0.3](0,0)--(20,0)--(20,0.1)--(0,0.1)--cycle;
 \end{tikzpicture}

Column matrix (or column vector)
\begin{expandable}
    A matrix with $m$ rows and only 1 column.
\end{expandable}

\begin{tikzpicture}[scale=1]
   \filldraw[teal, opacity=0.3](0,0)--(20,0)--(20,0.1)--(0,0.1)--cycle;
 \end{tikzpicture}

Diagonal matrix
\begin{expandable}
    A matrix $A=[a_{ij}]$  where $a_{ij}=0$ whenever $i \ne j$
\end{expandable}

\begin{tikzpicture}[scale=1]
   \filldraw[teal, opacity=0.3](0,0)--(20,0)--(20,0.1)--(0,0.1)--cycle;
 \end{tikzpicture}

Dimensions of a matrix
\begin{expandable}
    An $m \times n$ matrix is a matrix with $m$ rows and $n$ columns.
\end{expandable}

\begin{tikzpicture}[scale=1]
   \filldraw[teal, opacity=0.3](0,0)--(20,0)--(20,0.1)--(0,0.1)--cycle;
 \end{tikzpicture}

Elementary matrix
\begin{expandable}
    A matrix formed by applying one elementary row operation to the identity matrix.
\end{expandable}

\begin{tikzpicture}[scale=1]
   \filldraw[teal, opacity=0.3](0,0)--(20,0)--(20,0.1)--(0,0.1)--cycle;
 \end{tikzpicture}

Homogeneous system 
\begin{expandable}
    A system of linear equations is called \dfn{homogeneous} if the system can be written in the form
$$\begin{array}{ccccccccc}
      a_{11}x_1 &+ &a_{12}x_2&+&\ldots&+&a_{1n}x_n&= &0 \\
	 a_{21}x_1 &+ &a_{22}x_2&+&\ldots&+&a_{2n}x_n&= &0 \\
     &&&&\vdots&&&& \\
     a_{m1}x_1 &+ &a_{m2}x_2&+&\ldots&+&a_{mn}x_n&= &0
    \end{array},$$
or as a matrix equation as $A \vec{x} = \vec{0}$.
\end{expandable}

\begin{tikzpicture}[scale=1]
   \filldraw[teal, opacity=0.3](0,0)--(20,0)--(20,0.1)--(0,0.1)--cycle;
 \end{tikzpicture}

Identity matrix
\begin{expandable}
    A square matrix with ones as diagonal entries and zeros for the remaining entries.
\end{expandable}

\begin{tikzpicture}[scale=1]
   \filldraw[teal, opacity=0.3](0,0)--(20,0)--(20,0.1)--(0,0.1)--cycle;
 \end{tikzpicture}

Inverse of a square matrix
\begin{expandable}
    Let $A$ be an $n\times n$ matrix.  An $n\times n$ matrix $B$ is called an \dfn{inverse} of $A$ if 
$$AB=BA=I$$
where $I$ is an $n\times n$ identity matrix.  If such an inverse matrix exists, we say that $A$ is \dfn{invertible}.  If an inverse does not exist, we say that $A$ is not invertible.  The inverse of $A$ is denoted by $A^{-1}$.
\end{expandable}

\begin{tikzpicture}[scale=1]
   \filldraw[teal, opacity=0.3](0,0)--(20,0)--(20,0.1)--(0,0.1)--cycle;
 \end{tikzpicture}

LU factorization
\begin{expandable}
    A factorization $A=LU$ where $L$ is lower triangular and $U$ is upper triangular with ones on the diagonal (called unit upper triangluar).  It is useful for solving $Ax=b$.
\end{expandable}

\begin{tikzpicture}[scale=1]
   \filldraw[teal, opacity=0.3](0,0)--(20,0)--(20,0.1)--(0,0.1)--cycle;
 \end{tikzpicture}

Matrix
\begin{expandable}
    A rectangular array of numbers.  It has $m$ rows and $n$ columns for some positive integers $m$ and $n$.
\end{expandable}

\begin{tikzpicture}[scale=1]
   \filldraw[teal, opacity=0.3](0,0)--(20,0)--(20,0.1)--(0,0.1)--cycle;
 \end{tikzpicture}

Matrix addition
\begin{expandable}
    Let $A=\begin{bmatrix} a_{ij}\end{bmatrix} $ and $B=\begin{bmatrix} b_{ij}\end{bmatrix}$ be two
$m\times n$ matrices. Then the \dfn{sum of matrices} $A$ and $B$, denoted by $A+B$,  is an $m \times n$
matrix  given by 
$$A+B=\begin{bmatrix}a_{ij}+b_{ij}\end{bmatrix}$$
\end{expandable}

\begin{tikzpicture}[scale=1]
   \filldraw[teal, opacity=0.3](0,0)--(20,0)--(20,0.1)--(0,0.1)--cycle;
 \end{tikzpicture}

Matrix factorization
\begin{expandable}
    Representing a matrix as a product of two or more matrices.
\end{expandable}

\begin{tikzpicture}[scale=1]
   \filldraw[teal, opacity=0.3](0,0)--(20,0)--(20,0.1)--(0,0.1)--cycle;
 \end{tikzpicture}

Matrix multiplication
\begin{expandable}
    Let $A$ be an $m\times n$ matrix whose rows are vectors $\vec{r}_1$, $\vec{r}_2,\ldots ,\vec{r}_n$.  Let $B$ be an $n\times p$ matrix with columns $\vec{b}_1, \vec{b}_2, \ldots, \vec{b}_p$.  Then the matrix product $AB$ is an $m \times p$ matrix with entries given by the dot products
$$AB=\begin{bmatrix}-&\vec{r}_1&-\\-&\vec{r}_2&-\\ &\vdots & \\-&\vec{r}_i &-\\ &\vdots& \\-&\vec{r}_m&-\end{bmatrix}\begin{bmatrix}|&|&&|&&|\\\vec{b}_1& \vec{b}_2 &\ldots  & \vec{b}_j&\ldots& \vec{b}_p\\|&|&&|&&|\end{bmatrix}=$$
$$=\begin{bmatrix}\vec{r}_1\dotp \vec{b}_1&\vec{r}_1\dotp \vec{b}_2&\ldots&\vec{r}_1\dotp \vec{b}_j&\ldots &\vec{r}_1\dotp \vec{b}_p\\\vec{r}_2\dotp \vec{b}_1&\vec{r}_2\dotp \vec{b}_2&\ldots&\vec{r}_2\dotp \vec{b}_j&\ldots &\vec{r}_2\dotp \vec{b}_p\\\vdots&\vdots&&\vdots&&\vdots\\\vec{r}_i\dotp \vec{b}_1&\vec{r}_i\dotp \vec{b}_2&\ldots&\vec{r}_i\dotp \vec{b}_j&\ldots &\vec{r}_i\dotp \vec{b}_p\\\vdots&\vdots&&\vdots&&\vdots\\\vec{r}_m\dotp \vec{b}_1&\vec{r}_m\dotp \vec{b}_2&\ldots&\vec{r}_m\dotp \vec{b}_j&\ldots &\vec{r}_m\dotp \vec{b}_p\end{bmatrix}
$$
\end{expandable}

\begin{tikzpicture}[scale=1]
   \filldraw[teal, opacity=0.3](0,0)--(20,0)--(20,0.1)--(0,0.1)--cycle;
 \end{tikzpicture}

Matrix powers
\begin{expandable}
    If $A$ is a square matrix then we can define $A^k$ to be the result of multiplying $A$ by itself $k$ times.
\end{expandable}

\begin{tikzpicture}[scale=1]
   \filldraw[teal, opacity=0.3](0,0)--(20,0)--(20,0.1)--(0,0.1)--cycle;
 \end{tikzpicture}


Negative of a matrix
\begin{expandable}
    The additive inverse of a matrix, formed by multiplying the matrix by the scalar $-1$.
\end{expandable}

\begin{tikzpicture}[scale=1]
   \filldraw[teal, opacity=0.3](0,0)--(20,0)--(20,0.1)--(0,0.1)--cycle;
 \end{tikzpicture}


Nonsingular matrix
\begin{expandable}
    A square matrix $A$ is said to be \dfn{nonsingular} provided that $\mbox{rref}(A)=I$.  Otherwise we say that $A$ is \dfn{singular}.
\end{expandable}

\begin{tikzpicture}[scale=1]
   \filldraw[teal, opacity=0.3](0,0)--(20,0)--(20,0.1)--(0,0.1)--cycle;
 \end{tikzpicture}

Partitioned matrices - block multiplication
\begin{expandable}
    Subdividing a matrix into submatrices using imaginary horizontal and vertical lines - used to multiply matrices more efficiently.
\end{expandable}

\begin{tikzpicture}[scale=1]
   \filldraw[teal, opacity=0.3](0,0)--(20,0)--(20,0.1)--(0,0.1)--cycle;
 \end{tikzpicture}


Permutation matrix
\begin{expandable}
    A matrix formed by permuting the rows of the identity matrix.
\end{expandable}

\begin{tikzpicture}[scale=1]
   \filldraw[teal, opacity=0.3](0,0)--(20,0)--(20,0.1)--(0,0.1)--cycle;
 \end{tikzpicture}


Properties of matrix algebra
\begin{expandable}
    Addition and scalar multiplication of matrices have many nice properties. Matrix multiplication is not commutative, but the associative and distributive laws hold. The inverse and transpose operations of a product follow the "shoes and socks" rule.
\end{expandable}

\begin{tikzpicture}[scale=1]
   \filldraw[teal, opacity=0.3](0,0)--(20,0)--(20,0.1)--(0,0.1)--cycle;
 \end{tikzpicture}

Row matrix
\begin{expandable}
    A matrix with only $1$ row and $n$ columns.
\end{expandable}

\begin{tikzpicture}[scale=1]
   \filldraw[teal, opacity=0.3](0,0)--(20,0)--(20,0.1)--(0,0.1)--cycle;
 \end{tikzpicture}

Scalar multiple of a matrix
\begin{expandable}
    If $A=\begin{bmatrix} a_{ij}\end{bmatrix} $ and $k$ is a scalar,
then $kA=\begin{bmatrix} ka_{ij}\end{bmatrix}$. 
\end{expandable}

\begin{tikzpicture}[scale=1]
   \filldraw[teal, opacity=0.3](0,0)--(20,0)--(20,0.1)--(0,0.1)--cycle;
 \end{tikzpicture}


Singular matrix
\begin{expandable}
    A square matrix $A$ is said to be \dfn{singular} provided that $\mbox{rref}(A)$ is NOT the identity matrix.  If instead $\mbox{rref}(A)=I$, we say that $A$ is \dfn{nonsingular}.
\end{expandable}

\begin{tikzpicture}[scale=1]
   \filldraw[teal, opacity=0.3](0,0)--(20,0)--(20,0.1)--(0,0.1)--cycle;
 \end{tikzpicture}

Square matrix
\begin{expandable}
    A matrix with the same number of rows and columns.
\end{expandable}

\begin{tikzpicture}[scale=1]
   \filldraw[teal, opacity=0.3](0,0)--(20,0)--(20,0.1)--(0,0.1)--cycle;
 \end{tikzpicture}

Symmetric matrix
\begin{expandable}
    An $n\times n$ matrix $A$ is said to be
\dfn{symmetric} if $A=A^{T}.$ It is said to be
\dfn{skew symmetric} if $A=-A^{T}.$
\end{expandable}

\begin{tikzpicture}[scale=1]
   \filldraw[teal, opacity=0.3](0,0)--(20,0)--(20,0.1)--(0,0.1)--cycle;
 \end{tikzpicture}
 

Transpose of a matrix
\begin{expandable}
    Let $A=\begin{bmatrix} a _{ij}\end{bmatrix}$ be an $m\times n$ matrix. Then the \dfn{transpose of $A$}, denoted by $A^{T}$, is the $n\times m$
matrix given by switching the rows and columns:
$$
A^{T} = \begin{bmatrix} a _{ij}\end{bmatrix}^{T}= \begin{bmatrix} a_{ji} \end{bmatrix}
$$
\end{expandable}

\begin{tikzpicture}[scale=1]
   \filldraw[teal, opacity=0.3](0,0)--(20,0)--(20,0.1)--(0,0.1)--cycle;
 \end{tikzpicture}


Zero matrix
\begin{expandable}
    The $m\times n$ \dfn{zero matrix} is the $m\times n$ matrix
having every entry equal to zero. The zero matrix is
denoted by $O$.
\end{expandable}

\begin{tikzpicture}[scale=1]
   \filldraw[teal, opacity=0.3](0,0)--(20,0)--(20,0.1)--(0,0.1)--cycle;
 \end{tikzpicture}
 
\end{document}


