\documentclass{ximera}
%% You can put user macros here
%% However, you cannot make new environments

\listfiles

\graphicspath{{./}{firstExample/}{secondExample/}}

\usepackage{tikz}
\usepackage{tkz-euclide}
\usepackage{tikz-3dplot}
\usepackage{tikz-cd}
\usetikzlibrary{shapes.geometric}
\usetikzlibrary{arrows}
%\usetkzobj{all}
\pgfplotsset{compat=1.13} % prevents compile error.

%\renewcommand{\vec}[1]{\mathbf{#1}}
\renewcommand{\vec}{\mathbf}
\newcommand{\RR}{\mathbb{R}}
\newcommand{\dfn}{\textit}
\newcommand{\dotp}{\cdot}
\newcommand{\id}{\text{id}}
\newcommand\norm[1]{\left\lVert#1\right\rVert}
 
\newtheorem{general}{Generalization}
\newtheorem{initprob}{Exploration Problem}

\tikzstyle geometryDiagrams=[ultra thick,color=blue!50!black]

%\DefineVerbatimEnvironment{octave}{Verbatim}{numbers=left,frame=lines,label=Octave,labelposition=topline}



\usepackage{mathtools}


\title{Octave Problems for Ch 1} \license{CC BY-NC-SA 4.0}

\begin{document}

\begin{abstract}
\end{abstract}
\maketitle

\section*{Octave Problems for Chapter 1}

To access Octave cells online, go to the \href{https://sagecell.sagemath.org/}{Sage Math Cell Webpage}, select OCTAVE as the language, enter your code, and press EVALUATE.  
To ''save" or share your code, click on the \emph{Share} button, select \emph{Permalink}, then copy the address directly from the browser window.  You can store this link to access your work later or share this link with others.  You will need to get a new Permalink every time you modify the code.

Refer to \href{https://ximera.osu.edu/linearalgebrav3/XOctaveTutorial/octave/systems}{Octave Tutorial for Solving Systems of Equations}.

\begin{problem}\label{prob_oct_0}
    Given the code below, write out the system of equations and the corresponding solution(s).

    \begin{verbatim}
% Coefficient matrix A
A=[2 -4 1 1 3;
0 2 -1 4 -1;
2 -2 1 1 4];

% Column vector b
b=[-2; 1; 4];

% Augmented matrix A_b
A_b=[A b];

% Reduced row-echelon form
rref_A_b=rref(A_b)
    \end{verbatim}

\href{https://sagecell.sagemath.org/?z=eJxFjcEKwjAQRO-B_MNcCnoI2NpbySH4B16LFJtuNNA0sKTq57sVVBaGGeaxU-GUKYToIy0F6Vo4vuC0crZvYFrUcsdOqwMk1mhFJIlvPlV7kaRVJV_mNS14kC-ZMWo12t40Heruz7j1lmSFpt_OIKCI7R3GL3WmafXCcH4a8nea84KQOWnFTGHY8M3sxOzf0oYzHQ==&lang=octave&interacts=eJyLjgUAARUAuQ==}{Link to code}.    
\end{problem}

\begin{problem}\label{prob_oct_1}
    Solve each of the following systems of equations using the backslash ($\setminus$) operator, and the \emph{rref} command.  Reconcile the seeming differences.

    \begin{enumerate}
        \item 
        \begin{equation}
\begin{array}{ccccccccc}
      3x &- &y&+&4z&= &2 \\
	 &&y&+&z&=&1\\
     -2x&&&+&3z&=&1
    \end{array}
    \end{equation}
  
    
    What is true about this system?
    
    \begin{multipleChoice}
    \choice[correct]{The system is consistent.  Using the backslash operator produces the same result as using the \emph{rref} function.}
    \choice{The system is inconsistent.  The backslash operator gives the least squares approximation.}
    \choice{The system is consistent but has infinitely many solutions.  The backslash operator gives one particular solution.}
    \end{multipleChoice}

       \item 
    \begin{equation}
\begin{array}{ccccccccc}
      3x &- &y&+&4z&= &2 \\
	 x&-&y&+&2z&=&4\\
     -2x&+&y&-&3z&=&-3
    \end{array}
    \end{equation}

    What is true about this system?
    \begin{multipleChoice}
    \choice{The system is consistent.  Using the backslash operator produces the same result as using the \emph{rref} function.}
    \choice{The system is inconsistent.  The backslash operator gives the least squares approximation.}
    \choice[correct]{The system is consistent but has infinitely many solutions.  The backslash operator gives one particular solution.}
    \end{multipleChoice}

    \item 
    \begin{equation}
\begin{array}{ccccccccc}
      3x &- &y&+&4z&= &2 \\
	 x&-&y&+&2z&=&1\\
     -2x&+&y&-&3z&=&1
    \end{array}
    \end{equation}

    What is true about this system?
    \begin{multipleChoice}
    \choice{The system is consistent.  Using the backslash operator produces the same result as using the \emph{rref} function.}
    \choice[correct]{The system is inconsistent.  The backslash operator gives the least squares approximation.}
    \choice{The system is consistent but has infinitely many solutions.  The backslash operator gives one particular solution.}
    \end{multipleChoice}

         \end{enumerate}
\end{problem}

\begin{problem}\label{prob_oct_2}
Generate a random $20\times 3$ matrix $A$ whose entries are integers $-1$, $0$, and $1$.  Generate a random $20\times 1$ vector $\vec{b}$ whose components are integers $-1$, $0$, and $1$.  (See Templates \ref{temp:randMat} and \ref{temp:rref} for coding reference.)

Find the rank and the reduced row-echelon form of $[A | \vec{b}]$.  Experiment with several choices of $A$ and $\vec{b}$.  Are you getting the answers you were expecting?  Answer the following questions based on your theoretical knowledge.

What are the possibilities for the rank of $[A | \vec{b}]$?  

\begin{multipleChoice}
    \choice{The rank can be any positive integer less than or equal to $20$.}
    \choice{The rank can be any positive integer less than or equal to $3$.}
    \choice{The rank is always equal to $4$.}
    \choice[correct]{The rank can be any non-negative integer less than or equal to $4$.}
\end{multipleChoice}

What are the possibilities for the number of solutions of $[A | \vec{b}]$?

\begin{multipleChoice}
    \choice{This system could have a unique solution.}
    \choice[correct]{This system has either no solutions or infinitely many solutions.}
    \choice{This system will always have infinitely many solutions.}
    \choice{This system will always be infeasible.}
\end{multipleChoice}

There is a good chance that your experiments might have led you to believe that $\text{rank}[A | \vec{b}]=4$.  Can you manually construct $[A | \vec{b}]$, satisfying the condition of the problem, with a smaller rank?  How many solutions does your system have?  Why do you think it is so rare to randomly generate a system $[A | \vec{b}]$ of rank less than $4$?

Here is a fun piece of code.  Try to figure out what it does.

\begin{verbatim}
for i=1:200
% Define the coefficient matrix A
A = randi([0 1],5,3);

% Define vector b
b = randi([0 1],5,1);

% Create an augmented matrix
A_b=[A b];

% Find the rank
rank_A_b=rank(A_b)
end
\end{verbatim}

\href{https://sagecell.sagemath.org/?z=eJxljssKwjAQRfeB-YfZCC10kShulC6C4k-UUvKY6CBNIUTx802kguBmuItz75mwJOReHbZSgtjgmQJHwnwjdAuFwI4pZpxNTvxCDUJjj8lEz80gUY3dvtu1RxA_3Se5XEYtCPvHqi97SmQyoYloHte5KMivkqKYbD9otOPKXjj6z0dl6w6i3qkyNTQltCAo-jehYTnK&lang=octave&interacts=eJyLjgUAARUAuQ==}{Link to code}.

Why did we have better luck generating $[A | \vec{b}]$ of rank less than $4$?
    
\end{problem}

\end{document}