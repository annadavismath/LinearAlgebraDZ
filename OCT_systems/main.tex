\documentclass{ximera}
%% You can put user macros here
%% However, you cannot make new environments

\listfiles

\graphicspath{{./}{firstExample/}{secondExample/}}

\usepackage{tikz}
\usepackage{tkz-euclide}
\usepackage{tikz-3dplot}
\usepackage{tikz-cd}
\usetikzlibrary{shapes.geometric}
\usetikzlibrary{arrows}
%\usetkzobj{all}
\pgfplotsset{compat=1.13} % prevents compile error.

%\renewcommand{\vec}[1]{\mathbf{#1}}
\renewcommand{\vec}{\mathbf}
\newcommand{\RR}{\mathbb{R}}
\newcommand{\dfn}{\textit}
\newcommand{\dotp}{\cdot}
\newcommand{\id}{\text{id}}
\newcommand\norm[1]{\left\lVert#1\right\rVert}
 
\newtheorem{general}{Generalization}
\newtheorem{initprob}{Exploration Problem}

\tikzstyle geometryDiagrams=[ultra thick,color=blue!50!black]

%\DefineVerbatimEnvironment{octave}{Verbatim}{numbers=left,frame=lines,label=Octave,labelposition=topline}



\usepackage{mathtools}


\title{Octave for Ch 2} \license{CC BY-NC-SA 4.0}

\begin{document}

\begin{abstract}
\end{abstract}
\maketitle

\section*{Octave for Chapter 2}

To access Octave cells online, go to the \href{https://sagecell.sagemath.org/}{Sage Math Cell Webpage}, select OCTAVE as the language, enter your code, and press EVALUATE.  

The templates in this section provide sample Octave code for solving systems of equations. You can access our code through the link at the bottom of each template.  Feel free to modify the code and experiment to learn more!  

To ''save" or share your code, click on the \emph{Share} button, select \emph{Permalink}, then copy the address directly from the browser window.  You can store this link to access your work later or share this link with others.  You will need to get a new Permalink every time you modify the code.

%Refer to \href{https://ximera.osu.edu/linearalgebrav3/XOctaveTutorial/octave/systems}{Octave Tutorial for Solving Systems of Equations}.

\subsection*{Octave Tutorial}

When solving systems by hand, you utilized Gauss-Jordan elimination to find rref.  Having the reduced row echelon form available made it easy to write out the solution(s) or determine that the system is inconsistent.  We can find rref and the rank of a matrix using Octave as follows.

\begin{template}\label{temp:rref}
Consider the system
\begin{equation}
\begin{array}{ccccccccc}
      x &- &y&&&&&= &0 \\
	 2x& -&2y&+&z&+&2w&=&4\\
     & &y&&&+&w&=&0\\
     & &&&2z&+&w&=&5
    \end{array}
    \end{equation}

 (You may remember this system from \href{https://ximera.osu.edu/linearalgebrav3/LinearAlgebraInteractiveIntro/SYS-0020/main}{Augmented Matrix Notation and Elementary Row Operations}.)  We will solve this system by finding the reduced row echelon form of the corresponding augmented matrix in Octave.  
    \begin{verbatim}
% Define the augmented matrix 
% corresponding to the coefficient matrix A and vector b
A_b = [1 -1 0 0 0;
     2 -2 1 2 4;
     0 1 0 1 0;
     0 0 2 1 5];
     
% Find rref of A_b
rref(A_b)

% Find the rank of A_b
rank(A_b)
    \end{verbatim}

\href{https://sagecell.sagemath.org/?z=eJxFjsEKwjAMhu-FvsN_Gehh0A49iYfB8CVEpOvSWWSt1Co-vumkmBCS_-cjSYOBnA-EfCOY17xQyDRhMTn5D2xMiZ6PGCYfZuS4UjaSc956JivXw4QJb7I5JoxS9NcRR5w1Wg1V8iAFSnRoO2huu-oo6F_9DYXC7C_VkaLByfMBfsYhOvB6KYrY8LSVgoUJ91V8ASwdNw8=&lang=octave&interacts=eJyLjgUAARUAuQ==}{Link to code}.    

\begin{remark}
    If you start with a matrix $A$ and a separate vector $\vec{b}$, you can create an augmented matrix without re-typing as follows:
    \begin{verbatim}
% Define the coefficient matrix A
A = [1 -1 0 0;
     2 -2 1 2;
     0 1 0 1;
     0 0 2 1];

% Define vector b
b = [0;4;0;5];

% Create an augmented matrix
A_b=[A b]
    \end{verbatim}

\href{https://sagecell.sagemath.org/?z=eJxFjrEKAyEQRHvBf5jmyoNVkkqukOQvjiOoWROL80BMyOdHIZLtHjvMvAlXjikz6pMRDo4xhcS5Yne1pA-sFBYLVoVZgUBGCvTTmDUU9GBCf6s_UouorbEU09h4c6hHgZfC904yJ0PmPEKXwq4yXIZ7PfbmwPefRXO4-WW18NsXtMsrmQ==&lang=octave&interacts=eJyLjgUAARUAuQ==}{Link to code}.
    \end{remark}
\end{template}

\subsection*{Octave Problems}

\begin{problem}\label{prob_oct_0}
    Given the code below, write out the system of equations and the corresponding solution(s).

    \begin{verbatim}
% Coefficient matrix A
A=[2 -4 1 1 3;
0 2 -1 4 -1;
2 -2 1 1 4];

% Column vector b
b=[-2; 1; 4];

% Augmented matrix A_b
A_b=[A b];

% Reduced row-echelon form
rref_A_b=rref(A_b)
    \end{verbatim}

\href{https://sagecell.sagemath.org/?z=eJxFjcEKwjAQRO-B_MNcCnoI2NpbySH4B16LFJtuNNA0sKTq57sVVBaGGeaxU-GUKYToIy0F6Vo4vuC0crZvYFrUcsdOqwMk1mhFJIlvPlV7kaRVJV_mNS14kC-ZMWo12t40Heruz7j1lmSFpt_OIKCI7R3GL3WmafXCcH4a8nea84KQOWnFTGHY8M3sxOzf0oYzHQ==&lang=octave&interacts=eJyLjgUAARUAuQ==}{Link to code}.    
\end{problem}

\begin{problem}\label{prob_oct_2}
Generate a random $20\times 3$ matrix $A$ whose entries are integers $-1$, $0$, and $1$.  Generate a random $20\times 1$ vector $\vec{b}$ whose components are integers $-1$, $0$, and $1$.  (See Templates \ref{temp:randMat} and \ref{temp:rref} for coding reference.)

Find the rank and the reduced row-echelon form of $[A | \vec{b}]$.  Experiment with several choices of $A$ and $\vec{b}$.  Are you getting the answers you were expecting?  Answer the following questions based on your theoretical knowledge.

What are the possibilities for the rank of $[A | \vec{b}]$?  

\begin{multipleChoice}
    \choice{The rank can be any positive integer less than or equal to $20$.}
    \choice{The rank can be any positive integer less than or equal to $3$.}
    \choice{The rank is always equal to $4$.}
    \choice[correct]{The rank can be any non-negative integer less than or equal to $4$.}
\end{multipleChoice}

What are the possibilities for the number of solutions of $[A | \vec{b}]$?

\begin{multipleChoice}
    \choice{This system could have a unique solution.}
    \choice[correct]{This system has either no solutions or infinitely many solutions.}
    \choice{This system will always have infinitely many solutions.}
    \choice{This system will always be infeasible.}
\end{multipleChoice}

There is a good chance that your experiments might have led you to believe that $\text{rank}[A | \vec{b}]=4$.  Can you manually construct $[A | \vec{b}]$, satisfying the condition of the problem, with a smaller rank?  How many solutions does your system have?  Why do you think it is so rare to randomly generate a system $[A | \vec{b}]$ of rank less than $4$?

% Here is a fun piece of code.  Try to figure out what it does.

% \begin{verbatim}
% for i=1:200
% % Define the coefficient matrix A
% A = randi([0 1],5,3);

% % Define vector b
% b = randi([0 1],5,1);

% % Create an augmented matrix
% A_b=[A b];

% % Find the rank
% rank_A_b=rank(A_b)
% end
% \end{verbatim}

% \href{https://sagecell.sagemath.org/?z=eJxljssKwjAQRfeB-YfZCC10kShulC6C4k-UUvKY6CBNIUTx802kguBmuItz75mwJOReHbZSgtjgmQJHwnwjdAuFwI4pZpxNTvxCDUJjj8lEz80gUY3dvtu1RxA_3Se5XEYtCPvHqi97SmQyoYloHte5KMivkqKYbD9otOPKXjj6z0dl6w6i3qkyNTQltCAo-jehYTnK&lang=octave&interacts=eJyLjgUAARUAuQ==}{Link to code}.

% Why did we have better luck generating $[A | \vec{b}]$ of rank less than $4$?
    
\end{problem}

\begin{warning}
    The \emph{rref} function utilizes a modified version of the Gauss-Jordan elimination algorithm (\href{https://www.mathworks.com/help/matlab/ref/rref.html}{Reference Link}).  For some matrices, implementing this algorithm directly leads to round-off errors and other problems that are outside of the scope of this text.  %While the backslash operator ($\setminus$) is a relatively robust operator, it too, can lead to computational problems.  
    We illustrate what can happen in the next example.   
\end{warning}

\begin{example}\label{ex:rrefFail}
    We will use a $12\times 12$ Hilbert matrix to illustrate how the \emph{rref} function can fail.  You can learn more about Hilbert matrices \href{https://en.wikipedia.org/wiki/Hilbert_matrix}{here}.

\begin{verbatim}
% Use a 12 by 12 Hilbert matrix
A=hilb(12);

% Theoretical rank of A is 12, but the following computation shows a rank of 11
rank(A)

% rref of A should be the identity matrix, but the following result contains a row of zeros
rref(A)
\end{verbatim}

\href{https://sagecell.sagemath.org/?z=eJxtjjEOwjAMRfdKuYOXSkViSVfE0I0DwAGS1iUWaYwSR6WcnrSFjcWyrP_fcw23hGBAt2CXdV7IW4wCk5FIL1V1Z1cujW4PJ1WpqoarQ44o1BsP0YQH8AgdUCrlI9gsIA5hZO95pnCHnqdnFiPEAZLjORXZr6a1qta96Q47O0Ycd16JZj-AxQ1HAwYhWb5f_fNETNlL0QUxFDYLzyvrjZFT8RR08XwAMi9N1g==&lang=octave&interacts=eJyLjgUAARUAuQ==}{Link to code}.

\end{example}

\subsection*{Acknowledgements}

The inspiration for Example \ref{ex:rrefFail} came from \href{https://stackoverflow.com/questions/42893111/matlab-rref-function-precision-error-after-12th-column-of-hilbert-matrices}{this} \emph{Stack Overflow} comment.



\end{document}