\documentclass{ximera}
%% You can put user macros here
%% However, you cannot make new environments

\listfiles

\graphicspath{{./}{firstExample/}{secondExample/}}

\usepackage{tikz}
\usepackage{tkz-euclide}
\usepackage{tikz-3dplot}
\usepackage{tikz-cd}
\usetikzlibrary{shapes.geometric}
\usetikzlibrary{arrows}
%\usetkzobj{all}
\pgfplotsset{compat=1.13} % prevents compile error.

%\renewcommand{\vec}[1]{\mathbf{#1}}
\renewcommand{\vec}{\mathbf}
\newcommand{\RR}{\mathbb{R}}
\newcommand{\dfn}{\textit}
\newcommand{\dotp}{\cdot}
\newcommand{\id}{\text{id}}
\newcommand\norm[1]{\left\lVert#1\right\rVert}
 
\newtheorem{general}{Generalization}
\newtheorem{initprob}{Exploration Problem}

\tikzstyle geometryDiagrams=[ultra thick,color=blue!50!black]

%\DefineVerbatimEnvironment{octave}{Verbatim}{numbers=left,frame=lines,label=Octave,labelposition=topline}



\usepackage{mathtools}


 \title{Orthogonality and Projections} \license{CC BY-NC-SA 4.0}

\begin{document}
\begin{abstract}
\end{abstract}
\maketitle

\begin{onlineOnly}
\section*{Orthogonality and Projections}
\end{onlineOnly}

\subsection*{Orthogonal and Orthonormal Sets}
In this section, we examine what it means for vectors (and sets of
vectors) to be orthogonal and orthonormal. Recall that two non-zero vectors are orthogonal if their dot product is zero.  A collection of non-zero vectors in $\RR^n$ is called \dfn{orthogonal} if the vectors are pair-wise orthogonal.  The diagram below shows two orthogonal vectors in $\RR^2$ and three orthogonal vectors in $\RR^3$.
\begin{center}
\begin{tikzpicture}[scale=0.5]
\draw[line width=0.5pt, dashed](-2,-0.5)--(2,-0.5)--(2,3.5)--(-2,3.5)--cycle;
\draw[line width=2pt,red,-stealth](0,0)--(1.5,0.5);  
\draw[line width=2pt,blue,-stealth](0,0)--(-1,3);
\node[] at (-0.5, -1.5)  (p2)    {Orthogonal vectors in $\RR^2$};
 \end{tikzpicture}
 \quad\quad\quad
\begin{tikzpicture}[scale=0.5]
	\draw[line width=2pt,red,-stealth](0,0,0)--(4,1,0);
    \draw[line width=2pt,blue,-stealth](0,0,0)--(-0.5,2,0);
    \draw[line width=2pt,black,-stealth](0,0,0)--(0,0,5);
    \node[] at (3, -3,1.5)  (p2)    {Orthogonal vectors in $\RR^3$};
        \draw[-,line width=0.2mm, dashed](6.5,3.5,5.5)--(6.5,-1,5.5) ;
    \draw[-,line width=0.2mm, dashed](6.5,3.5,5.5)--(-1,3.5,5.5) ;
    \draw[-,line width=0.2mm, dashed](6.5,3.5,5.5)--(6.5,3.5,-1) ;
    \draw[-,line width=0.2mm, dashed](-1,-1,5.5)--(6.5,-1,5.5) ;
    \draw[-,line width=0.2mm, dashed](-1,-1,5.5)--(-1,3.5,5.5) ;
    \draw[-,line width=0.2mm, dashed](6.5,-1,5.5)--(6.5,-1,-1) ;
    \draw[-,line width=0.2mm, dashed](6.5,3.5,-1)--(6.5,-1,-1) ;
    \draw[-,line width=0.2mm, dashed](6.5,3.5,-1)--(-1,3.5,-1) ;
    \draw[-,line width=0.2mm, dashed](-1,3.5,5.5)--(-1,3.5,-1) ;
        \end{tikzpicture}
\end{center}

If every vector in an orthogonal set of vectors is also a unit vector, then we say that the given set of vectors is \dfn{orthonormal}.

\begin{center}
\begin{tikzpicture}[scale=0.5]
\draw[line width=2pt,red,-stealth](0,0)--(3,4);  
\draw[line width=2pt,blue,-stealth](0,0)--(-4,3);
\node[] at (-1, -1)  (p2)    {An orthonormal set of two vectors};
\node[] at (-1.8, 2)  (p2)    {1};
\node[] at (1.3, 2.5)  (p2)    {1};
 \end{tikzpicture}
 \end{center}

Formally, we can define orthogonal and orthonormal vectors as follows.

\begin{definition}\label{orthset}
Let $\{ \vec{v}_1, \vec{v}_2, \cdots, \vec{v}_k \}$ be a set of nonzero
vectors in $\RR^n$. Then this set is called an
\dfn{orthogonal set} if 
$\vec{v}_i \dotp \vec{v}_j = 0$ for all $i \neq j$.
Moreover, if $\norm{\vec{v}_i}=1$ for $i=1,\ldots,m$ (i.e. each vector in the set is a unit vector), we say the set of vectors is an \dfn{orthonormal set}.
\end{definition}

An orthogonal set of vectors may not be orthonormal.  To convert an orthogonal set to an orthonormal set, we need to divide each vector by its own length.

\begin{definition}\label{normalizing}
\dfn{Normalizing} an orthogonal set is the process of turning an orthogonal set into an orthonormal set.
If $\{ \vec{v}_1, \vec{v}_2, \ldots, \vec{v}_k\}$
is an orthogonal subset of $\RR^n$,
then
\[ \left\{
\frac{1}{\norm{\vec{v}_1}}\vec{v}_1,
\frac{1}{\norm{\vec{v}_2}}\vec{v}_2, \ldots,
\frac{1}{\norm{\vec{v}_k}}\vec{v}_k \right\}
\]
is an orthonormal set.
\end{definition}

We illustrate this concept in the following example.

\begin{example}\label{ex:orthonormalset}
Consider the vectors
\[
\vec{v}_1=\begin{bmatrix}
1 \\
1
\end{bmatrix},\quad \vec{v}_2  =
\begin{bmatrix}
-1 \\
1
\end{bmatrix}
\]
Show that $\{\vec{v}_1,\vec{v}_2\}$ is an orthogonal set of vectors  but not an orthonormal one. Find the corresponding orthonormal set.

\begin{explanation}
One easily verifies that $\vec{v}_1 \dotp \vec{v}_2 = 0$ and
$\left\{ \vec{v}_1, \vec{v}_2 \right\}$ is an orthogonal set of
vectors. On the other hand one can compute that ${\norm{\vec{v}_1}}= {\norm{\vec{v}_2}} =
\sqrt{2} \neq 1$ and so the set is not orthonormal.

To find a corresponding orthonormal set, we need to
normalize each vector. 
\begin{eqnarray*}
\vec{q}_1 &=& \frac{1}{\norm{\vec{v}_1}}\vec{v}_1\\
&=& \frac{1}{\sqrt{2}} \begin{bmatrix}
1 \\
1
\end{bmatrix} \\
&=&
\begin{bmatrix}
\frac{1}{\sqrt{2}} \\
\frac{1}{\sqrt{2}}
\end{bmatrix}
\end{eqnarray*}

Similarly,
\begin{eqnarray*}
\vec{q}_2 &=& \frac{1}{\norm{\vec{v}_2}}\vec{v}_2\\
&=& \frac{1}{\sqrt{2}} \begin{bmatrix}
-1 \\
1
\end{bmatrix} \\
&=&
\begin{bmatrix}
-\frac{1}{\sqrt{2}} \\
\frac{1}{\sqrt{2}}
\end{bmatrix}
\end{eqnarray*}

Therefore the corresponding orthonormal set is
\[
\left\{ \vec{q}_1, \vec{q}_2 \right\} =
\left\{
\begin{bmatrix}
\frac{1}{\sqrt{2}} \\
\frac{1}{\sqrt{2}}
\end{bmatrix},
\begin{bmatrix}
-\frac{1}{\sqrt{2}} \\
\frac{1}{\sqrt{2}}
\end{bmatrix}
\right\}
\]

You can verify that this set is orthonormal.
\end{explanation}
\end{example}

\subsection*{Orthogonal and Orthonormal Bases}
Recall that every basis of $\RR^n$ (or a subspace $W$ of $\RR^n$) imposes a coordinate system on $\RR^n$ (or $W$) that can be used to express any vector of $\RR^n$ (or $W$) as a linear combination of the elements of the basis.  For example, vectors $\vec{v}_1$ and $\vec{v}_2$ impose a coordinate system onto the plane, as shown in the figure below.  We readily see that $\vec{x}$, contained in the plane, can be written as $\vec{x}=\vec{v}_1+2\vec{v}_2$.

\begin{center}
\begin{tikzpicture}[scale=1]
\draw[line width=0.5pt, gray](-2,1)--(0.5,6);  
\draw[line width=0.5pt, gray](-1,-2)--(3,6);
\draw[line width=0.5pt, gray](1.5,-2)--(5.5,6);
\draw[line width=0.5pt, gray](4,-2)--(8,6);
\draw[line width=0.5pt, gray](6.5,-2)--(8,1);
\draw[line width=0.5pt, gray](-2, 4.33)--(3,6);
\draw[line width=0.5pt, gray](-2, 2.66)--(8,6);
\draw[line width=0.5pt, gray](-2, 1)--(8,4.33);
\draw[line width=0.5pt, gray](-2, -0.66)--(8,2.66);
\draw[line width=0.5pt, gray](-1, -2)--(8,1);
\draw[line width=0.5pt, gray](4,-2)--(8,-0.66);
 \draw[line width=2pt,blue,-stealth](0,0)--(1,2);
\draw[line width=2pt,red,-stealth](0,0)--(3,1);
\draw[line width=2pt,-stealth](0,0)--(7,4);
\node[blue] at (0.5, 1.5)  (p2)    {$\vec{v}_1$};
\node[red] at (1.6, 0.2)  (p2)    {$\vec{v}_2$};
\node[] at (3.5, 2.3)  (p2)    {$\vec{x}$};
%\node[] at (-0.2, 0.2)  (p2)    {$\vec{O}$};
 \end{tikzpicture}
 \end{center}
Vector $\vec{x}$ is visually easy to work with.  In general, one way to express an arbitrary vector as a linear combination of the basis vectors is to solve a system of linear equations, which can be costly.  One reason we like $\{\vec{i},\vec{j}\}$ as a basis of $\RR^2$ is because any vector $\vec{x}$ of $\RR^2$ can be easily expressed as the sum of the orthogonal projections of $\vec{x}$ onto the basis vectors $\vec{i}$ and $\vec{j}$, as shown below.
\begin{center}
\begin{tikzpicture}[scale=1.4]
 \draw[<->] (-1,0)--(3.5,0);
  \draw[<->] (0,-1)--(0,3.5);
  \draw[line width=6pt,-stealth, black!20!white](0,0)--(2,0);
  \draw[line width=6pt,-stealth, black!20!white](0,0)--(0,3);
\draw[line width=2pt,red,-stealth](0,0)--(1,0);  
\draw[line width=2pt,blue,-stealth](0,0)--(0,1);
\draw[line width=2pt,-stealth](0,0)--(2,3);  
\draw[line width=0.5pt,dashed](2,3)--(2,0);  
\draw[line width=0.5pt,dashed](2,3)--(0,3); 
\node[] at (1.7, -0.4)  (p2)    {$\mbox{proj}_{\vec{i}}\vec{x}$};
\node[] at (-0.7, 2.5)  (p2)    {$\mbox{proj}_{\vec{j}}\vec{x}$};
\node[] at (3, 3.1)  (p2)    {$\vec{x}=\mbox{proj}_{\vec{i}}\vec{x}+\mbox{proj}_{\vec{j}}\vec{x}$};
\node[red] at (0.5, -0.3)  (p2)    {$\vec{i}$};
\node[blue] at (-0.3, 0.5)  (p2)    {$\vec{j}$};
 \end{tikzpicture}
\end{center}

We can see why an ``upright" coordinate system with basis $\{\vec{i},\vec{j}\}$ works well.  What if we tilt this coordinate system while preserving the orthogonal relationship between the basis vectors?  The following exploration allows you to investigate the consequences.

\begin{exploration}\label{exp:orth1a}
    In the following GeoGebra interactive, vectors $\vec{v}_1$ and $\vec{v}_2$ are orthogonal (slopes of the lines containing them are negative reciprocals of each other).  These vectors are clearly linearly independent and span $\RR^2$.  Therefore $\{\vec{v}_1,\vec{v}_2\}$ is a basis of $\RR^2$.  
    
    Let $\vec{x}$ be an arbitrary vector.  Orthogonal projections of $\vec{x}$ onto $\vec{v}_1$ and $\vec{v}_2$ are depicted in light grey.
    \begin{itemize}
           \item Use the tip of vector $\vec{x}$ to manipulate the vector and convince yourself that $\vec{x}$ is always the diagonal of the parallelogram (a rectangle!) determined by the projections.
        \item Use the tips of $\vec{v}_1$ and $\vec{v}_2$ to change the basis vectors.  What happens when $\vec{v}_1$ and $\vec{v}_2$ are no longer orthogonal?
        \item Pick another pair of orthogonal vectors $\vec{v}_1$ and $\vec{v}_2$.  Verify that $\vec{x}$ is the sum of its projections.
    \end{itemize}

\pdfOnly{
Access GeoGebra interactives through the online version of this text at 

\href{https://ximera.osu.edu/linearalgebradzv3/LinearAlgebraInteractiveIntro}{https://ximera.osu.edu/linearalgebradzv3/LinearAlgebraInteractiveIntro}.
}

\begin{onlineOnly}
    \begin{center}
\geogebra{nsqzhsxv}{800}{600}
\end{center}
\end{onlineOnly}
\end{exploration}


As you have just discovered in Exploration \ref{exp:orth1a}, we can express an arbitrary vector of $\RR^2$ as the sum of its projections onto the basis vectors, provided that the basis is orthogonal. It turns out that this result holds for any subspace of $\RR^n$, making a basis consisting of orthogonal vectors especially useful. 

If an orthogonal set is a basis, we call it an
\dfn{orthogonal basis}. Similarly, if an orthonormal set is a basis, we call it an \dfn{orthonormal basis}.


The following theorem generalizes our observation in Exploration \ref{exp:orth1a}.  As you read the statement of the theorem, it will be helpful to recall that the orthogonal projection of vector $\vec{x}$ onto a non-zero vector $\vec{d}$ is given by
\begin{equation}\label{eq:orthProj}
\mbox{proj}_{\vec{d}}\vec{x}=\left(\frac{\vec{x}\cdot\vec{d}}{\norm{\vec{d}}^2}\right)\vec{d}
\end{equation}
(See \href{https://ximera.osu.edu/linearalgebradzv3/LinearAlgebraInteractiveIntro/VEC-0070/main}{Orthogonal Projections}.)

\begin{theorem}\label{th:fourierexpansion}
Let $W$ be a subspace of $\RR^n$ and suppose $\{ \vec{f}_1, \vec{f}_2, \ldots, \vec{f}_m \}$
is an orthogonal basis of $W$.
Then for every $\vec{x}$ in $W$,
\begin{equation}\label{FourierEqn}
\vec{x} =
\left(\frac{\vec{x}\dotp \vec{f}_1}{\norm{\vec{f}_1}^2}\right) \vec{f}_1 +
\left(\frac{\vec{x}\dotp \vec{f}_2}{\norm{\vec{f}_2}^2}\right) \vec{f}_2 +
\cdots +
\left(\frac{\vec{x}\dotp \vec{f}_m}{\norm{\vec{f}_m}^2}\right) \vec{f}_m.
\end{equation}
\end{theorem}

\begin{proof}
We may express $\vec{x}$ as a linear combination of the basis elements:
\[ \vec{x} =
c_1 \vec{f}_1 +
c_2 \vec{f}_2 +
\cdots +
c_m \vec{f}_m.
\]
We claim that $c_i = \frac{\vec{x}\dotp \vec{f}_i}{\norm{\vec{f}_i}^2}$ for $i=1,\ldots,m$. To see this, we take the dot product of
each side with the vector $\vec{f}_i$ and obtain the following.

\begin{equation*}
  \vec{x} \dotp \vec{f}_i =  \left(c_1\vec{f}_1 +
c_2\vec{f}_2 +
\cdots +
c_m\vec{f}_m\right) \dotp \vec{f}_i 
\end{equation*}
Our basis is orthogonal, so $\vec{f}_j \dotp \vec{f}_i = 0$ for all $j \neq i$, which means after we distribute the dot product, only one term will remain on the right-hand side.  We have 
\begin{equation*}
  \vec{x} \dotp \vec{f}_i =  c_i\vec{f}_i \dotp \vec{f}_i 
\end{equation*}

We now divide both sides by $\vec{f}_i \dotp \vec{f}_i = \norm{\vec{f}_i}^2$, and since our claim holds for $i=1,\ldots,m$, the proof is complete.
\end{proof}

Theorem~\ref{th:fourierexpansion} shows one important benefit of a basis being orthogonal.  With an orthogonal basis it is easy to represent any vector in terms of the basis vectors.  

\begin{example}\label{fourier}
Let
$\vec{f}_1= \begin{bmatrix}
1 \\ -1 \\ 2
\end{bmatrix},
\vec{f}_2= \begin{bmatrix}
0 \\ 2 \\ 1 
\end{bmatrix},
\vec{f}_3 =\begin{bmatrix}
5 \\ 1 \\ -2
\end{bmatrix}$,
and let
$\vec{x} =\begin{bmatrix}
1 \\ 1 \\ 1
\end{bmatrix}$.  

Notice that $\mathcal{B}=\{ \vec{f}_1, \vec{f}_2, \vec{f}_3\}$
is an orthogonal set of vectors, and $\mathcal{B}$ spans $\RR^3$.  Use this fact to write $\vec{x}$ as  a linear combination of the vectors of $\mathcal{B}$.

\begin{explanation}
We first observe that $\mathcal{B}$ is a linearly independent set of vectors, and so $\mathcal{B}$ is a basis for $\RR^3$. Next we apply Theorem~\ref{th:fourierexpansion} to express $\vec{x}$ as  a linear combination of the vectors of $\mathcal{B}$.  We wish to write:

\[
\vec{x}   =
\left(\frac{\vec{x}\dotp \vec{f}_1}{\norm{\vec{f}_1}^2}\right) \vec{f}_1 +
\left(\frac{\vec{x}\dotp \vec{f}_2}{\norm{\vec{f}_2}^2}\right) \vec{f}_2 +
\left(\frac{\vec{x}\dotp \vec{f}_3}{\norm{\vec{f}_3}^2}\right) \vec{f}_3.
\]

We readily compute:

\[
\frac{\vec{x}\dotp\vec{f}_1}{\norm{\vec{f}_1}^2} = \frac{2}{6}, \;
\frac{\vec{x}\dotp\vec{f}_2}{\norm{\vec{f}_2}^2} = \frac{3}{5},
\mbox{ and }
\frac{\vec{x}\dotp\vec{f}_3}{\norm{\vec{f}_3}^2} = \frac{4}{30}.\]

Therefore,
\[ \begin{bmatrix}
1 \\ 1 \\ 1
\end{bmatrix}
= \frac{1}{3}\begin{bmatrix}
1 \\ -1 \\ 2
\end{bmatrix}
+\frac{3}{5}\begin{bmatrix}
0 \\ 2 \\ 1
\end{bmatrix}
+\frac{2}{15}\begin{bmatrix}
5 \\ 1 \\ -2
\end{bmatrix}.\]
\end{explanation} 
\end{example}

The formula from Theorem~\ref{th:fourierexpansion} is easy to use, and it becomes even easier when our basis is \emph{orthonormal}.

\begin{corollary}\label{cor:orthonormal}
Let $W$ be a subspace of $\RR^n$ and suppose $\{ \vec{q}_1, \vec{q}_2, \ldots, \vec{q}_m \}$
is an orthonormal basis of $W$.
Then for any $\vec{x}$ in $W$,
\[ \vec{x} =
\left(\vec{x}\dotp \vec{q}_1\right) \vec{q}_1 +
\left(\vec{x}\dotp \vec{q}_2\right) \vec{q}_2 +
\cdots +
\left(\vec{x}\dotp \vec{q}_m\right)  \vec{q}_m.
\]
\end{corollary}
\begin{proof}
This is a special case of Theorem \ref{th:fourierexpansion}.  Because $\norm{\vec{u_i}} = 1$ for $i=1,\ldots,m$, %where we can compute the coefficients of $x$ with respect to the basis by simply taking the dot product with each basis vector, for in this case 
the terms are given by 
$$\left(\frac{\vec{x}\cdot \vec{q}_i}{\norm{\vec{q}_i}^2}\right)\vec{q}_i=\left(\vec{x}\dotp \vec{q}_i\right) \vec{q}_i.$$

\end{proof}

\subsection*{Orthogonal Projection onto a Subspace}
In the previous section we found that given a subspace $W$ of $\RR^n$ with an orthogonal basis $\mathcal{B}$, every vector $\vec{x}$ in $W$ can be expressed as the sum of the orthogonal projections of $\vec{x}$ onto the elements of $\mathcal{B}$.  Note that our premise was that $\vec{x}$ is in $W$.  In this section, we look into the meaning of the sum of orthogonal projections of $\vec{x}$ onto the elements of an orthogonal basis of $W$ for those vectors $\vec{x}$ of $\RR^n$ that are \emph{not} in $W$.

\begin{exploration}\label{exp:orthProjSub}
In the GeoGebra interactive below, $W$ is a plane spanned by $\vec{v}_1$ and $\vec{v}_2$, in $\RR^3$.  $W$ is subspace of $\RR^3$.  In the initial set up, $\vec{v}_1$ and $\vec{v}_2$ are orthogonal.  Vector $\vec{x}$ is not in $W$.  

Use check-boxes to construct the sum of orthogonal projections of $\vec{x}$ onto $\vec{v}_1$ and $\vec{v}_2$.  RIGHT-CLICK and DRAG to rotate the image.   

\pdfOnly{
Access GeoGebra interactives through the online version of this text at 

\href{https://ximera.osu.edu/linearalgebradzv3/LinearAlgebraInteractiveIntro}{https://ximera.osu.edu/linearalgebradzv3/LinearAlgebraInteractiveIntro}.
}

\begin{onlineOnly}
\begin{center}
\geogebra{hehqyayz}{950}{800}
\end{center}
\end{onlineOnly}

\begin{question}
If moved, return the basis vectors $\vec{v}_1$ and $\vec{v}_2$ to their default position (set $s_1=s_2=0$) to ensure that they are orthogonal.  

\begin{itemize}
\item Rotate the image to convince yourself that the perpendiculars dropped from the tip of $\vec{x}$ to $\vec{v}_1$ and $\vec{v}_2$ are indeed perpendicular to $\vec{v}_1$ and $\vec{v}_2$ in the diagram. (You'll have to look at it just right to convince yourself of this.)  Are both of these perpendiculars also necessarily perpendicular to the plane? \wordChoice{\choice{Yes}, \choice[correct]{No}}

\item Use sliders $x_1, x_2$ and $x_3$ to manipulate $\vec{x}$.  Rotate the figure for a better view.  What is true about about vector $\vec{p}$?
    
    \begin{multipleChoice}
 \choice{$\vec{p}=\vec{x}-(\mbox{proj}_{\vec{v}_1}\vec{x}+\mbox{proj}_{\vec{v}_2}\vec{x})$.}
 \choice{Vector $\vec{p}$ is orthogonal to $W$.}
 \choice[correct]{All of the above.}
 \end{multipleChoice}
 
  \item Rotate the figure so that you're looking directly down at the plane.  If you're looking at it correctly, you will notice that (1) the parallelogram determined by the projections of $\vec{x}$ onto $\vec{v}_1$ and $\vec{v}_2$ is a rectangle; (2) the sum of projections, $\mbox{proj}_{\vec{v}_1}\vec{x}+\mbox{proj}_{\vec{v}_2}\vec{x}$, is located directly underneath $\vec{x}$, like a shadow at midday.
 \end{itemize} 
 \end{question}

 \begin{question}
 Use sliders $s_1$ and $s_2$ to manipulate the basis vectors $\vec{v}_1$ and $\vec{v}_2$ so that they are no longer orthogonal.  

\begin{itemize}
 \item 
 Rotate the figure for a better view.  Which of the following is true?
 \begin{multipleChoice}
 \choice[correct]{$\vec{p}=\vec{x}-(\mbox{proj}_{\vec{v}_1}\vec{x}+\mbox{proj}_{\vec{v}_2}\vec{x})$.}
 \choice{Vector $\vec{p}$ is orthogonal to $W$.}
  \choice{All of the above.}
 \end{multipleChoice}
 \item
 Rotate your figure so that you're looking directly down at the plane. Which of the following is true?
 \begin{multipleChoice}
 \choice{Parallelogram determined by $\vec{v}_1$ and $\vec{v}_2$ is a rectangle.}
 \choice{$\mbox{proj}_{\vec{v}_1}\vec{x}+\mbox{proj}_{\vec{v}_2}\vec{x}$ is located directly underneath $\vec{x}$.}
  \choice[correct]{None of the above.}
 \end{multipleChoice}
\end{itemize}
\end{question}
\end{exploration}

In Exploration \ref{exp:orthProjSub}, you discovered that given a plane, spanned by orthogonal vectors $\vec{v}_1,\vec{v}_2$, in $\RR^3$, and a vector $\vec{x}$, not in the plane, we can interpret the sum of orthogonal projections of $\vec{x}$ onto $\vec{v}_1$ and $\vec{v}_2$ as a ``shadow" of $\vec{x}$ that lies in the plane directly underneath the vector $\vec{x}$. We say that this ``shadow" is an \dfn{orthogonal projection of $\vec{x}$ onto $W$}. You have also found that if $\vec{v}_1,\vec{v}_2$ are not orthogonal, the parallelogram representing the sum of the orthogonal projections of $\vec{x}$ onto $\vec{v}_1$ and $\vec{v}_2$ will not be a rectangle.  In this case, $\vec{x}$ minus this sum will NOT be orthogonal to the plane.  It is essential that $\vec{v}_1,\vec{v}_2$ are orthogonal for $\mbox{proj}_{\vec{v}_1}\vec{x}+\mbox{proj}_{\vec{v}_2}\vec{x}$ to be considered an orthogonal projection.  

In general, we can define an orthogonal projection of $\vec{x}$ in $\RR^n$ onto a subspace $W$ of $\RR^n$ as the sum of the orthogonal projections of $\vec{x}$ onto the elements of an orthogonal basis of $W$.  %We denote such a projection by $\mbox{proj}_V(\vec{x})$. An important aspect of this definition is that it allows us to express $\vec{x}$ as the sum of its orthogonal projection, $\vec{w}$, onto $W$ and a vector orthogonal to $\vec{w}$, called $\vec{w}^\perp$.  
Definition \ref{def:projOntoSubspace} and the subsequent diagram summarize this discussion.


\begin{definition}[Projection onto a Subspace of $\RR^n$]\label{def:projOntoSubspace}
Let $W$ be a subspace of $\RR^n$ with orthogonal basis $\{\vec{f}_{1}, \vec{f}_{2}, \dots, \vec{f}_{m}\}$. If $\vec{x}$ is in $\RR^n$, the vector
\begin{equation}\label{def:projectontoWeasy}
\vec{w}=\mbox{proj}_W(\vec{x}) = \mbox{proj}_{\vec{f}_1}\vec{x} + \mbox{proj}_{\vec{f}_2}\vec{x} + \dots + \mbox{proj}_{\vec{f}_m}\vec{x}
\end{equation}
is called the \dfn{orthogonal projection} of $\vec{x}$ onto $W$.  %The vector $\vec{x}-\vec{w}$ is called the \dfn{component of $\vec{x}$ orthogonal to $\vec{w}$}.  We will write $\vec{w}^\perp = \vec{x}-\vec{w}$.
\end{definition}

An illustration of Definition \ref{def:projOntoSubspace} for a two-dimensional subspace $W$ with orthogonal basis $\{\vec{f}_1,\vec{f}_2\}$ is shown below.
\begin{center}
\tdplotsetmaincoords{70}{130}
	\begin{tikzpicture}[scale=0.8]
\filldraw[blue, opacity=0.2] (0,0,0)--(5,0,0)--(5,0,5)--(0,0,5)--cycle;
\draw[->,line width=0.4mm, -stealth, blue](0,0,0)--(6,0,0) ;
\draw[->,line width=0.4mm, -stealth, blue](0,0,0)--(0,0,6) ;
\node[label={above:$\vec{f}_1$}] at (6,0,0) {};
\node[label={above:$W$}] at (5.2,0,5) {};
\node[label={left:$\vec{f}_2$}] at (0,0,6) {};
\node[label={above:$\vec{x}$}] at (2,1.5,2) {};
\node[label={above:$\vec{w}$}] at (2.2,0,3.5) {};
\draw[-,line width=0.2mm, dashed](4,3,4)--(4,0,4) ;
    \draw[-,line width=0.2mm, dashed](0,0,4)--(4,0,4) ;
    \draw[-,line width=0.2mm, dashed](4,0,4)--(4,0,0) ;
    \draw[->,line width=1.5mm, -stealth, black,opacity=0.4](0,0,0)--(0,0,4) ;
    \draw[->,line width=1.5mm, -stealth, black,opacity=0.4](0,0,0)--(4,0,0) ;
   \draw[->,line width=0.8mm, -stealth, red](0,0,0)--(4,3,4) ;
    \draw[->,line width=1.5mm, -stealth, red,opacity=0.2](0,0,0)--(4,0,4) ;
    \node[label={below:$\vec{w}=\mbox{proj}_W\vec{x}=\mbox{proj}_{\vec{f}_1}\vec{x}+\mbox{proj}_{\vec{f}_2}\vec{x}$}] at (3,-1,3) {};
      \end{tikzpicture}
\end{center}

Using equation (\ref{eq:orthProj}) multiple times, we can also express $\vec{w}$ in Definition \ref{def:projOntoSubspace} using the following formula.

\begin{formula}\label{form:orthProjOntoW}
\begin{equation}\label{def:projectontoW}
\vec{w} = \mbox{proj}_W(\vec{x}) =\frac{\vec{x} \dotp \vec{f}_{1}}{\norm{\vec{f}_{1}}^2}\vec{f}_{1} + \frac{\vec{x} \dotp \vec{f}_{2}}{\norm{\vec{f}_{2}}^2}\vec{f}_{2}+ \dots +\frac{\vec{x} \dotp \vec{f}_{m}}{\norm{\vec{f}_{m}}^2}\vec{f}_{m}
\end{equation}
\end{formula}

\subsection*{Orthogonal Decomposition of $\vec{x}$}
Definition \ref{def:projOntoSubspace} allows us to express $\vec{x}$ as the sum of its orthogonal projection, $\vec{w}=\mbox{proj}_W\vec{x}$, located in $W$, and a vector we will call $\vec{w}^\perp$ (pronounced ``W-perp"), given by $\vec{w}^\perp=\vec{x}-\vec{w}$. This decomposition of $\vec{x}$ is shown in the diagram below.  
\begin{center}
\tdplotsetmaincoords{70}{130}
	\begin{tikzpicture}[scale=0.8]
\filldraw[blue, opacity=0.2] (0,0,0)--(5,0,0)--(5,0,5)--(0,0,5)--cycle;
\node[label={above:$\vec{x}$}] at (2,1.5,2) {};
\node[label={above:$\vec{w}$}] at (2.2,0,3.5) {};
\node[label={above:$\vec{w}^{\perp}$}] at (5.3,2.2,6) {};
\draw[->,line width=0.8mm, -stealth, red](0,0,0)--(4,3,4) ;
    \draw[->,line width=1.5mm, -stealth, red,opacity=0.2](0,0,0)--(4,0,4) ;
\draw[->,line width=1.5mm, -stealth, red,opacity=0.2](4,0,4)--(4,3,4) ;
    \node[label={below:$\vec{x}=\vec{w}+\vec{w}^{\perp}$}] at (3,-1,3) {};
    \node[label={above:$W$}] at (5,0,1.9) {};
      \end{tikzpicture}
          
     \end{center}
You have already met $\vec{w}^\perp$, under the name of $\vec{p}$ in Exploration \ref{exp:orthProjSub}, and observed that this vector is orthogonal to $W$. We will now prove that $\vec{w}^\perp$ is orthogonal to every vector in $W$.  This will be accomplished in two steps.  First, in Theorem \ref{th:orthDecompX} we will prove that $\vec{w}^\perp$ is orthogonal to all of the basis elements of $W$. Next, you will use this result to demonstrate that $\vec{w}^\perp$ is orthogonal to every vector in $W$.

\begin{theorem}\label{th:orthDecompX}
Let $W$ be a subspace of $\RR^n$ with orthogonal basis $\{\vec{f}_{1}, \vec{f}_{2}, \dots, \vec{f}_{m}\}$. Let $\vec{x}$ be in $\RR^n$, and define $\vec{w}^\perp$ as
\begin{equation*}
\vec{w}^\perp=\vec{x}-\mbox{proj}_W\vec{x} = \vec{x}-(\mbox{proj}_{\vec{f}_1}\vec{x} + \mbox{proj}_{\vec{f}_2}\vec{x} + \dots + \mbox{proj}_{\vec{f}_m}\vec{x})
\end{equation*}
Then $\vec{w}^\perp$ is orthogonal to $\vec{f}_i$ for $1\leq i\leq m$.
\end{theorem}
\begin{proof}
We will use Formula \ref{form:orthProjOntoW} to show that $\vec{w}^\perp\cdot \vec{f}_i$=0.  Recall that $\{\vec{f}_{1}, \vec{f}_{2}, \dots, \vec{f}_{m}\}$ is an orthogonal basis.  Therefore $\vec{f}_j\dotp\vec{f}_i=0$ for $i\neq j$.  This observation enables us to compute as follows.
\begin{eqnarray*}
\vec{w}^\perp\cdot \vec{f}_i&=&\left[\vec{x}-\left(\frac{\vec{x} \dotp \vec{f}_{1}}{\norm{\vec{f}_{1}}^2}\vec{f}_{1} +\dots + \frac{\vec{x} \dotp \vec{f}_{i}}{\norm{\vec{f}_{i}}^2}\vec{f}_{i}+ \dots +\frac{\vec{x} \dotp \vec{f}_{m}}{\norm{\vec{f}_{m}}^2}\vec{f}_{m}\right)\right]\cdot \vec{f}_i\\
& =& \vec{x}\dotp \vec{f}_i- \frac{\vec{x} \dotp \vec{f}_{i}}{\norm{\vec{f}_{i}}^2}(\vec{f}_{i}\dotp\vec{f}_i)\\
&=& \vec{x}\dotp \vec{f}_i- \frac{\vec{x} \dotp \vec{f}_{i}}{\norm{\vec{f}_{i}}^2}\norm{\vec{f}_{i}}^2=\vec{x}\dotp \vec{f}_i-\vec{x}\dotp \vec{f}_i=0
\end{eqnarray*}
\end{proof}

We leave the proof of the following Corollary as Practice Problem \ref{prob:proofCor}
\begin{corollary}\label{cor:orthProjOntoW}
Let $W$ be a subspace of $\RR^n$ with orthogonal basis $\{\vec{f}_{1}, \vec{f}_{2}, \dots, \vec{f}_{m}\}$. Let $\vec{x}$ be in $\RR^n$, and define $\vec{w}^\perp$ as
\begin{equation*}
\vec{w}^\perp=\vec{x}-\mbox{proj}_W\vec{x} = \vec{x}-(\mbox{proj}_{\vec{f}_1}\vec{x} + \mbox{proj}_{\vec{f}_2}\vec{x} + \dots + \mbox{proj}_{\vec{f}_m}\vec{x})
\end{equation*}
Then $\vec{w}^\perp$ is orthogonal to every vector in $W$.
\end{corollary}



The fact that the decomposition of $\vec{x}$ into the sum of $\vec{w}$ and $\vec{w}^\perp$ is unique is the subject of the Orthogonal Decomposition Theorem which we will prove in \href{https://ximera.osu.edu/linearalgebradzv3/LinearAlgebraInteractiveIntro/RTH-0020/main}{Orthogonal Complements and Decompositions}.

Throughout this section we have worked with orthogonal bases of subspaces.  Does every subspace of $\RR^n$ have an orthogonal basis?  If so, how do we find one?  These questions will be addressed in \href{https://ximera.osu.edu/linearalgebradzv3/LinearAlgebraInteractiveIntro/RTH-0015/main}{Gram-Schmidt Orthogonalization}.

\section*{Practice Problems}
\begin{problem}\label{prob:rref_way}
Retry Example~\ref{fourier} using Gaussian elimination.  Which method seems easier to you?
\end{problem}

\begin{problem}\label{prob:vec_eq_0}
    Let $\vec{x}_1, \vec{x}_2, \ldots, \vec{x}_k\in\RR^n$ and
suppose $\mbox{span}\{\vec{x}_1, \vec{x}_2, \ldots, \vec{x}_k\}=\RR^n$.
Furthermore, suppose that there exists a vector $\vec{v}\in\RR^n$ for which $\vec{v}\dotp \vec{x}_j=0$ for all $j$, $1\leq j\leq k$.
Show that $\vec{v}=\vec{0}$.
\end{problem}

\emph{Problems \ref{OrthoProj1.1}-\ref{OrthoProj1.3}}

Let $\vec{x} = \begin{bmatrix}1\\ -2\\ 1\\ 6\end{bmatrix}$ in $\RR^4$, and let $W = \mbox{span}\left(\begin{bmatrix}2\\ 1\\ 3\\ -4\end{bmatrix}, \begin{bmatrix}1\\ 2\\ 0\\ 1\end{bmatrix}\right)$.

\begin{problem}\label{OrthoProj1.1}
Compute $\mbox{proj}_W(\vec{x})$.

Answer:  $$\frac{1}{10}\begin{bmatrix}\answer{-9}\\\answer{3}\\\answer{-21}\\\answer{33}\end{bmatrix}$$
\end{problem}

\begin{problem}\label{OrthoProj1.2}
Show that $\left\{\begin{bmatrix}1\\ 0\\ 2\\ -3\end{bmatrix}, \begin{bmatrix}4\\ 7\\ 1\\ 2\end{bmatrix}\right\}$ is another orthogonal basis of $W$.
\end{problem}

\begin{problem}\label{OrthoProj1.3}
Use the basis in Problem \ref{OrthoProj1.2} to compute $\mbox{proj}_W(\vec{x})$.

Answer:  $$\frac{1}{70}\begin{bmatrix}\answer{-63}\\\answer{21}\\\answer{-147}\\\answer{231}\end{bmatrix}$$
\end{problem}


\begin{problem}\label{prob:proofCor}
Prove Corollary \ref{cor:orthProjOntoW}
\end{problem}
  
\section*{Text Source}
A portion of the text in this section is an adaptation of Section 4.11.1 of Ken Kuttler's \href{https://open.umn.edu/opentextbooks/textbooks/a-first-course-in-linear-algebra-2017}{\it A First Course in Linear Algebra}. (CC-BY)

Ken Kuttler, {\it  A First Course in Linear Algebra}, Lyryx 2017, Open Edition, p. 233-238.  

\end{document}
