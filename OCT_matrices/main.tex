\documentclass{ximera}
%% You can put user macros here
%% However, you cannot make new environments

\listfiles

\graphicspath{{./}{firstExample/}{secondExample/}}

\usepackage{tikz}
\usepackage{tkz-euclide}
\usepackage{tikz-3dplot}
\usepackage{tikz-cd}
\usetikzlibrary{shapes.geometric}
\usetikzlibrary{arrows}
%\usetkzobj{all}
\pgfplotsset{compat=1.13} % prevents compile error.

%\renewcommand{\vec}[1]{\mathbf{#1}}
\renewcommand{\vec}{\mathbf}
\newcommand{\RR}{\mathbb{R}}
\newcommand{\dfn}{\textit}
\newcommand{\dotp}{\cdot}
\newcommand{\id}{\text{id}}
\newcommand\norm[1]{\left\lVert#1\right\rVert}
 
\newtheorem{general}{Generalization}
\newtheorem{initprob}{Exploration Problem}

\tikzstyle geometryDiagrams=[ultra thick,color=blue!50!black]

%\DefineVerbatimEnvironment{octave}{Verbatim}{numbers=left,frame=lines,label=Octave,labelposition=topline}



\usepackage{mathtools}


\title{Octave for Chapter 4} \license{CC BY-NC-SA 4.0}

\begin{document}

\begin{abstract}
\end{abstract}
\maketitle

\section*{Octave for Chapter 4}
The templates in this section provide sample Octave code for answering questions about linear combinations, span, and linear independence. You can access our code through the link at the bottom of each template.  Feel free to modify the code and experiment to learn more!  

You can write your own code using Octave software or online Octave cells.  To access Octave cells online, go to the \href{https://sagecell.sagemath.org/}{Sage Math Cell Webpage}, select OCTAVE as the language, enter your code, and press EVALUATE.  

To ''save" or share your online code, click on the \emph{Share} button, select \emph{Permalink}, then copy the address directly from the browser window.  You can store this link to access your work later or share this link with others.  You will need to get a new Permalink every time you modify the code.

\subsection*{Octave Tutorial}

\begin{template}\label{temp:matrixOps}

    \begin{verbatim}
        % Define matrices A, B, and C
        A = [1 -1 0 0;
            2 -2 1 2;
            0 1 0 1];
            
        B = [2 3 -1 4;
            1 -1 2 -2;
            1 -2 0 3];   
            
        C = [1 -2 1 4;
            2 -2 0 1;
            -2 1 1 0;
            1 3 -2 -1];     
            
        % Find A+B
        sum=A+B
        
        % Find the transpose of A
        % Copying and pasting A'  may not work because the prime comes through
        % in a different font.
        A_trans1=A'
        % we can also use the transpose function
        A_trans2=transpose(A)
        
        % Find AC
        product=A*C
        
        % Find the inverse of C
        inv_C=inv(C)
    \end{verbatim}
    
    \href{https://sagecell.sagemath.org/?z=eJxdUMGOwiAUvDfpP8zFqLuaFPRmOFCMP2GMIZVmSVYwhe7-vg-0tcqBMDNv3rzHDHvTWmdw1bGzjQmQK9QraHeBKgsJgSPDmqFCtSsLpMOx5mDgA66QZHYacFnUycaxScbtQOc2yTshOBk3p102Pb3qGZkitm-RlDHgrLLXSCxlURHLvSbdZjhYWkV-12UR-qvIj5GOPwax0y7cfDDwLSStfM4ME3Ke6v4NGu2gf4NHT0XvlrZ3TbTejTYuRnEhl5MoSb956_ylb6KQX-pjCOv-TPcYgSRCZyXoXqjlHXo6X7Q=&lang=octave&interacts=eJyLjgUAARUAuQ==}{Link to code}.
    \end{template}
    
    \begin{template}\label{temp:id_matrix}
    Sometimes it is useful to generate a commonly used matrix or a vector (e.g. identity matrix, zero vector). 
    
    \begin{verbatim}
        % 5 by 5 identity matrix
        I_5=eye(5)
        
        % 5 by 5 zero matrix
        O_5=zeros(5)
        
        % The zero vector in R^5
        O_col=zeros(5,1)
        
        % Row of zeros
        O_row=zeros(1,5)
        
        % Non-square zero matrix
        O_block=zeros(2,3)
        
        % Matrix filled with 1s
        One_block=ones(2,3)
    \end{verbatim}
    
    \href{https://sagecell.sagemath.org/?z=eJxdzT0OwjAMBeA9Uu7gpRJIYSgoIwdgAKSKGdQfV40IsUgDoZyepk06sHiwv_ecgYRqGIdq0DjlBniUzqoPZ4eb3OOAK7nmjLMsuS9aWsx5NGHRL-rS4UzeWDuyoAwUVxlkTTpZkUddkAdqp0AfjCUfTS5S44nMpn--Sov_vytN9T36rdhFf5zu0CqtsQGvXAd5KDcYA2Rw9j9PMkr_&lang=octave&interacts=eJyLjgUAARUAuQ==}{Link to code}.
    \end{template}
    
    \begin{template}\label{temp:randMat}
        When doing computational experiments you might need to create large matrices.  If the type of matrix is irrelevant, you may want to use random matrices.
    
        \begin{verbatim}
            % Generate a random 3 by 3 matrix A with real-number values between 0 and 1
            A=rand(3,3)
            
            % Generate a random 4 by 5 matrix B with integer values between -10 and 25
            B=randi([-10,25],4,5)
        \end{verbatim}
    
    \href{https://sagecell.sagemath.org/?z=eJxtzLEKwjAUheE9kHc4S6GFFGzTjA7t0ocQhxu8aKCJEFNb397U4qTLGX44X4GRA0dKDEKkcLl7aNhXHk8puhU9FpduiExTHWZvOeJJ08wPWE4Lc8AB-YdGiv64CaVWupJCiuKP3W22-drDbruQ-Prr1s0ut0aK4UO78pSjas1ZdcpUb3XOOqE=&lang=octave&interacts=eJyLjgUAARUAuQ==}{Link to code}.    
    \end{template}

\subsection*{Octave Exercises}
\begin{problem}\label{prob_oct_mat5}
    Let $$A=\begin{bmatrix}2 & 6 & -4 & 1 & 1\\
-1 & 0 & 2 & 2 & 4\\
-2 & 1 & 3 & -2 & 1\\
1 & 1 & 0 & 0 & 6\\
-3 & 4 & 1 & -2 & 0\end{bmatrix}, \quad \vec{b}=\begin{bmatrix}-12\\
 6\\
 -2\\
 -7\\
 -8\end{bmatrix}$$
Solve $A\vec{x}=\vec{b}$ in two different ways.  First, use the reduced row-echelon form of $[A|\vec{b}]$, second multiply both sides of the equation by $A^{-1}$.  Verify that both methods produce the same answer.
\end{problem}

\begin{problem}\label{prob_oct_mat1}
Let $A=\begin{bmatrix} 0 & 1\\5 & 3\end{bmatrix}$.  Find $A^{-1}$ in two different ways.  First, use the \emph{inv} function.  Second, find $\text{rref}[A|I]$.
    \begin{hint}
        To allow for an $n\times n$ input matrix, you can use \emph{length} to automatically detect matrix size.
        
        \begin{verbatim}
            M = [A eye(length(A))]  
        \end{verbatim}
    \end{hint}
\end{problem}   

\begin{problem}{prob_oct_mat4}
    Generate a $12\times 14$ matrix filled with $3$'s.
\end{problem}

\begin{problem}\label{prob_oct_mat2}
Write a routine that multiplies an $n\times m$ matrix by a vector by computing the linear combination of the columns of the matrix. (See Example \ref{exp:matrixvectormultdef} for reference.)  
\end{problem}

\begin{problem}\label{prob_oct_mat3}
Write a routine that multiplies an $n\times m$ matrix by a vector using the dot product definition of matrix-vector multiplication. (See Definition \ref{def:matrixvectormult}.)
    
\end{problem}

\begin{problem}\label{prob_oct_6}
    Write a routine that multiplies an $n\times m$ matrix by an $m\times k$ matrix.  

    \begin{hint}
        You can use \href{https://en.wikipedia.org/wiki/Matrix_multiplication_algorithm}{this Wikipedia page} for reference.
    \end{hint}    
\end{problem}

\end{document}