\documentclass{ximera}
%% You can put user macros here
%% However, you cannot make new environments

\listfiles

\graphicspath{{./}{firstExample/}{secondExample/}}

\usepackage{tikz}
\usepackage{tkz-euclide}
\usepackage{tikz-3dplot}
\usepackage{tikz-cd}
\usetikzlibrary{shapes.geometric}
\usetikzlibrary{arrows}
%\usetkzobj{all}
\pgfplotsset{compat=1.13} % prevents compile error.

%\renewcommand{\vec}[1]{\mathbf{#1}}
\renewcommand{\vec}{\mathbf}
\newcommand{\RR}{\mathbb{R}}
\newcommand{\dfn}{\textit}
\newcommand{\dotp}{\cdot}
\newcommand{\id}{\text{id}}
\newcommand\norm[1]{\left\lVert#1\right\rVert}
 
\newtheorem{general}{Generalization}
\newtheorem{initprob}{Exploration Problem}

\tikzstyle geometryDiagrams=[ultra thick,color=blue!50!black]

%\DefineVerbatimEnvironment{octave}{Verbatim}{numbers=left,frame=lines,label=Octave,labelposition=topline}



\usepackage{mathtools}


\title{Octave Problems for Ch 3} \license{CC BY-NC-SA 4.0}

\begin{document}

\begin{abstract}
\end{abstract}
\maketitle

\section*{Octave Problems for Chapter 3}

To access Octave cells online, go to the \href{https://sagecell.sagemath.org/}{Sage Math Cell Webpage}, select OCTAVE as the language, enter your code, and press EVALUATE.  
To ''save" or share your code, click on the \emph{Share} button, select \emph{Permalink}, then copy the address directly from the browser window.  You can store this link to access your work later or share this link with others.  You will need to get a new Permalink every time you modify the code.

%Refer to \href{https://ximera.osu.edu/linearalgebrav3/XOctaveTutorial/octave/systems}{Octave Tutorial for Solving Systems of Equations}.

\begin{problem}\label{prob_oct_7}
    Let $$A=\begin{bmatrix}2 & 6 & -4 & 1 & 1\\
-1 & 0 & 2 & 2 & 4\\
-2 & 1 & 3 & -2 & 1\\
1 & 1 & 0 & 0 & 6\\
-3 & 4 & 1 & -2 & 0\end{bmatrix}, \quad \vec{b}=\begin{bmatrix}-12\\
 6\\
 -2\\
 -7\\
 -8\end{bmatrix}$$
Solve $A\vec{x}=\vec{b}$ in two different ways.  First, use the reduced row-echelon form of $[A|\vec{b}]$, second multiply both sides of the equation by $A^{-1}$.  Verify that both methods produce the same answer.
\end{problem}

\begin{problem}\label{prob_oct_mat1}
Let $A=\begin{bmatrix} 0 & 1\\5 & 3\end{bmatrix}$.  Find $A^{-1}$ in two different ways.  First, use the \emph{inv} function.  Second, find $\text{rref}[A|I]$.
    \begin{hint}
        To allow for an $n\times n$ input matrix, you can use \emph{length} to automatically detect matrix size.
        
        \begin{verbatim}
            M = [A eye(length(A))]  
        \end{verbatim}
    \end{hint}
\end{problem}   

\begin{problem}\label{prob_oct_mat2}
Write a routine that multiplies an $n\times m$ matrix by a vector by computing the linear combination of the columns of the matrix. (See Example \ref{exp:matrixvectormultdef} for reference.)  
\end{problem}

\begin{problem}\label{prob_oct_mat3}
Write a routine that multiplies an $n\times m$ matrix by a vector using the dot product definition of matrix-vector multiplication. (See Definition \ref{def:matrixvectormult}.)
    
\end{problem}

\begin{problem}\label{prob_oct_6}
    Write a routine that multiplies an $n\times m$ matrix by an $m\times k$ matrix.  

    \begin{hint}
        You can use \href{https://en.wikipedia.org/wiki/Matrix_multiplication_algorithm}{this Wikipedia page} for reference.
    \end{hint}    
\end{problem}

\end{document}