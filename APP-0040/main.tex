\documentclass{ximera}
%% You can put user macros here
%% However, you cannot make new environments

\listfiles

\graphicspath{{./}{firstExample/}{secondExample/}}

\usepackage{tikz}
\usepackage{tkz-euclide}
\usepackage{tikz-3dplot}
\usepackage{tikz-cd}
\usetikzlibrary{shapes.geometric}
\usetikzlibrary{arrows}
%\usetkzobj{all}
\pgfplotsset{compat=1.13} % prevents compile error.

%\renewcommand{\vec}[1]{\mathbf{#1}}
\renewcommand{\vec}{\mathbf}
\newcommand{\RR}{\mathbb{R}}
\newcommand{\dfn}{\textit}
\newcommand{\dotp}{\cdot}
\newcommand{\id}{\text{id}}
\newcommand\norm[1]{\left\lVert#1\right\rVert}
 
\newtheorem{general}{Generalization}
\newtheorem{initprob}{Exploration Problem}

\tikzstyle geometryDiagrams=[ultra thick,color=blue!50!black]

%\DefineVerbatimEnvironment{octave}{Verbatim}{numbers=left,frame=lines,label=Octave,labelposition=topline}



\usepackage{mathtools}


\title{Application to Computer Graphics} \license{CC BY-NC-SA 4.0}

\begin{document}

\begin{abstract}
\end{abstract}
\maketitle

\begin{onlineOnly}
\section*{Application to Computer Graphics}
\end{onlineOnly}

Computer graphics deals with images
displayed on a computer screen, and so arises in a variety of
applications, ranging from word processors, to \textit{Star Wars}
animations, to video games, to wire-frame images of an airplane.  These
images consist of a number of points on the screen, together with
instructions on how to fill in areas bounded by lines and curves. Often
curves are approximated by a set of short straight-line segments, so
that the curve is specified by a series of points on the screen at the
end of these segments. Matrix transformations are important here because
 matrix images of straight line segments are again line segments.
 % \begin{remark}
 %     If $\vec{v}_{0}$ and $\vec{v}_{1}$ are vectors, the vector from $\vec{v}_{0}$ to $\vec{v}_{1}$ is $\vec{d} = \vec{v}_{1} - \vec{v}_{0}$. So a vector $\vec{v}$ lies on the line segment between $\vec{v}_{0}$ and $\vec{v}_{1}$ if and only if $\vec{v} = \vec{v}_{0} + t\vec{d}$ for some number $t$ in the range $0 \leq t \leq 1$. Thus the image of this segment is the set of vectors $A\vec{v} = A\vec{v}_{0} + tA \vec{d}$ with $0 \leq t \leq 1$, that is the image is the segment between $A\vec{v}_{0}$ and $A\vec{v}_{1}$.
 % \end{remark}
 
 Note that a color image requires that three images are sent, one to
each of the red, green, and blue phosphorus dots on the screen, in
varying intensities.\index{computer graphics}\index{linear transformations!in computer graphics}\index{vector geometry!computer graphics}

Consider displaying the letter $A$. In reality, it is depicted on the screen, as in the figure below, by specifying the coordinates of the 11 corners and filling in the interior.

  \begin{center}
\begin{tikzpicture}[scale=0.4]
\draw[thin,gray!40] (0,0) grid (9,13);
  \filldraw[white, opacity=1](1,1)--(2.36,6)--(6.64,6)--(8,1)--(7,1)--(5.91,5)--(3.09,5)--(2,1)--cycle;
  \filldraw[white, opacity=1](2.36,6)--(4,12)--(5,12)--(6.64,6)--(5.64,6)--(4.5,10.17)--(3.36,6)--cycle;
  \draw[line width=2pt,blue](1,1)--(4,12);
  \draw[line width=2pt,blue](5,12)--(4,12);
  \draw[line width=2pt,blue](5,12)--(8,1);
  \draw[line width=2pt,blue](8,1)--(7,1);
  \draw[line width=2pt,blue](7,1)--(5.91,5);
  \draw[line width=2pt,blue](5.91,5)--(3.09,5);
  \draw[line width=2pt,blue](3.09,5)--(2,1);
  \draw[line width=2pt,blue](1,1)--(2,1);
  \draw[line width=2pt,blue](4.5,10.17)--(3.36,6);
  \draw[line width=2pt,blue](4.5,10.17)--(5.64,6);
  \draw[line width=2pt,blue](3.36,6)--(5.64,6);
  \fill[] (1,1) circle (0.2cm);
  \fill[] (4,12) circle (0.2cm);
  \fill[] (5,12) circle (0.2cm);
  \fill[] (8,1) circle (0.2cm);
  \fill[] (7,1) circle (0.2cm);
  \fill[] (5.91,5) circle (0.2cm);
  \fill[] (3.09,5) circle (0.2cm);
  \fill[] (2,1) circle (0.2cm);
  \fill[] (4.5,10.17) circle (0.2cm);
  \fill[] (3.36,6) circle (0.2cm);
  \fill[] (5.64,6) circle (0.2cm);
 \end{tikzpicture}
\end{center}  

 For simplicity, we will disregard the thickness of the letter,
 so we require only five coordinates as in the figure below.

   \begin{center}
\begin{tikzpicture}[scale=0.4]
\draw[thin,gray!40] (-1,-1) grid (7,10);
  \draw[line width=2pt,red](0,0)--(3,9);
  \draw[line width=2pt,red](6,0)--(3,9);
  \draw[line width=2pt,red](5,3)--(1,3);
  \fill[] (0,0)node[above=2mm, left=1mm]{$(0,0)$} circle (0.2cm);
  \fill[] (6,0)node[above=2mm, right=1mm]{$(6,0)$} circle (0.2cm);
  \fill[] (5,3)node[above=2mm, right=1mm]{$(5,3)$} circle (0.2cm);
  \fill[] (1,3)node[above=2mm, left=1mm]{$(1,3)$} circle (0.2cm);
  \fill[] (3,9)node[above=2mm, right=1mm]{$(3,9)$} circle (0.2cm);
 \end{tikzpicture}
\end{center}  

This simplified letter can then be stored as a data matrix

$$D =\begin{bmatrix}
0 & 6 & 5 & 1 & 3\\
0 & 0 & 3 & 3 & 9
\end{bmatrix}$$

where the columns are the coordinates of the vertices. Then if we want to transform the letter by a $2 \times 2$ matrix $A$, we left-multiply this data matrix by $A$ (the effect is to multiply each column by $A$ and so transform each vertex).

For example, we can slant the letter to the right by multiplying by a horizontal shear matrix $A = \left[
\begin{array}{ll}
1 & 0.2\\
0 & 1\\
\end{array}
\right]$
 (See Formula \ref{form:shears}.) The result is the letter with data matrix
\begin{equation*}
A = \left[
\begin{array}{ll}
1 & 0.2\\
0 & 1
\end{array}
\right] \left[
\begin{array}{rrrrr}
0 & 6 & 5 & 1 & 3\\
0 & 0 & 3 & 3 & 9
\end{array}
\right] =
 \left[
\begin{array}{lllll}
0 & 6 & 5.6 & 1.6 & 4.8\\
0 & 0 & 3 & 3 & 9
\end{array}
\right]
\end{equation*}
which is shown in the figure below.

 \begin{center}
\begin{tikzpicture}[scale=0.4]
\draw[thin,gray!40] (-1,-1) grid (7,10);
  \draw[line width=2pt,red](0,0)--(4.8,9);
  \draw[line width=2pt,red](6,0)--(4.8,9);
  \draw[line width=2pt,red](5.6,3)--(1.6,3);
  \fill[] (0,0)node[above=2mm, left=1mm]{$(0,0)$} circle (0.2cm);
  \fill[] (6,0)node[above=2mm, right=1mm]{$(6,0)$} circle (0.2cm);
  \fill[] (5.6,3)node[above=2mm, right=1mm]{$(5.6,3)$} circle (0.2cm);
  \fill[] (1.6,3)node[above=2mm, left=1mm]{$(1.6,3)$} circle (0.2cm);
  \fill[] (4.8,9)node[above=2mm, right=1mm]{$(4.8,9)$} circle (0.2cm);
 \end{tikzpicture}
\end{center}  


 If we want to make this slanted matrix narrower, we can now apply a scaling matrix $B = \left[
\begin{array}{ll}
	0.8 & 0\\
	0 & 1
\end{array}
\right]$ that shrinks the $x$-coordinate by $0.8$. (See Formula \ref{form:horvertscaling}.) The result is the composite transformation
\begin{align*}
BAD &= \left[
\begin{array}{ll}
0.8 & 0\\
0 & 1
\end{array}
\right] \left[
\begin{array}{ll}
1 & 0.2\\
0 & 1
\end{array}
\right] \left[
\begin{array}{rrrrr}
0 & 6 & 5 & 1 & 3\\
0 & 0 & 3 & 3 & 9
\end{array}
\right] \\ &=
\left[
\begin{array}{lllll}
0 & 4.8 & 4.48 & 1.28 & 3.84\\
0 & 0 & 3 & 3 & 9
\end{array}
\right]
\end{align*}
which is drawn in the figure below.

 \begin{center}
\begin{tikzpicture}[scale=0.4]
\draw[thin,gray!40] (-1,-1) grid (7,10);
  \draw[line width=2pt,red](0,0)--(3.84,9);
  \draw[line width=2pt,red](4.8,0)--(3.84,9);
  \draw[line width=2pt,red](4.48,3)--(1.28,3);
  \fill[] (0,0)node[above=2mm, left=1mm]{$(0,0)$} circle (0.2cm);
  \fill[] (4.8,0)node[above=2mm, right=1mm]{$(4.8,0)$} circle (0.2cm);
  \fill[] (4.48,3)node[above=2mm, right=1mm]{$(4.48,3)$} circle (0.2cm);
  \fill[] (1.28,3)node[above=2mm, left=1mm]{$(1.28,3)$} circle (0.2cm);
  \fill[] (3.84,9)node[above=2mm, right=1mm]{$(3.84,9)$} circle (0.2cm);
 \end{tikzpicture}
\end{center}  

On the other hand, we can rotate the letter about the origin through $\frac{\pi}{6}$ (or $30^\circ$) by multiplying by the rotation matrix  $$R_{\frac{\pi}{2}} = \left[
\def\arraystretch{1.5}\begin{array}{rr}
\cos(\frac{\pi}{6}) & -\sin(\frac{\pi}{6})\\
\sin(\frac{\pi}{6}) & \cos(\frac{\pi}{6})
\end{array}
\right] = \left[
\begin{array}{ll}
0.866 & -0.5\\
0.5 & 0.866
\end{array}
\right]$$
(See Formula \ref{form:rotation}.)  This gives
\begin{align*}
R_{\frac{\pi}{2}} &= \left[
\begin{array}{ll}
0.866 & -0.5\\
0.5 & 0.866
\end{array}
\right] \left[
\begin{array}{rrrrr}
0 & 6 & 5 & 1 & 3\\
0 & 0 & 3 & 3 & 9
\end{array}
\right] \\ &=
\left[
\begin{array}{lllrr}
0 & 5.196 & 2.83 & -0.634 & -1.902\\
0 & 3 & 5.098 & 3.098 & 9.294
\end{array}
\right]
\end{align*}
and is plotted in the figure below.

 \begin{center}
\begin{tikzpicture}[scale=0.4]
\draw[thin,gray!40] (-1,-1) grid (7,10);
  \draw[line width=2pt,red](0,0)--(-1.902,9.294);
  \draw[line width=2pt,red](-0.634,3.098)--(2.83,5.098);
  \draw[line width=2pt,red](5.196,3)--(-1.902,9.294);
  \fill[] (0,0)node[above=2mm, left=1mm]{$(0,0)$} circle (0.2cm);
  \fill[] (2.83,5.098)node[above=2mm, right=1mm]{$(2.83,5.098)$} circle (0.2cm);
  \fill[] (5.196,3)node[above=2mm, right=1mm]{$(5.196,3)$} circle (0.2cm);
  \fill[] (-0.634,3.098)node[above=2mm, left=1mm]{$(-0.634,3.098)$} circle (0.2cm);
  \fill[] (-1.902,9.294)node[above=2mm, right=1mm]{$(-1.902,9.294)$} circle (0.2cm);
 \end{tikzpicture}
\end{center}  

This poses a problem: How do we rotate
about a point other than the origin? It turns out that we can do this when
we have solved another more basic problem. It is clearly important to be
 able to translate a screen image by a fixed vector $\vec{w}$, that is apply the transformation $T_{w} : \RR^2 \to \RR^2$ given by $T_{w}(\vec{v}) = \vec{v} + \vec{w}$ for all $\vec{v}$ in $\RR^2$. The problem is that these translations are not matrix transformations $\RR^2 \to \RR^2$ because they do not carry $\vec{0}$ to $\vec{0}$ (unless $\vec{w} = \vec{0}$). (See Practice Problem \ref{prob:translation_does_not_work}.) However, there is a clever way around this.

The idea is to represent a point $\vec{v} = \left[
\begin{array}{r}
x\\
y
\end{array}
\right]$
 as a $3 \times 1$ column $\left[
 \begin{array}{r}
 x\\
 y\\
 1
 \end{array}
 \right]$, called the \dfn{homogeneous coordinates}\index{homogeneous coordinates} of $\vec{v}$. Then translation by $\vec{w} = \left[
 \begin{array}{r}
 p\\
 q
 \end{array}
 \right]$ can be achieved by multiplying by a $3 \times 3$ matrix:
\begin{equation*}
\left[
\begin{array}{rrr}
1 & 0 & p\\
0 & 1 & q\\
0 & 0 & 1
\end{array}
\right] \left[
\begin{array}{r}
x\\
y\\
1
\end{array}
\right]
=
\left[
\begin{array}{c}
x + p\\
y + q\\
1
\end{array}
\right]
=
\left[
\begin{array}{c}
T_{\vec{w}}(\vec{v})\\
1
\end{array}
\right]
\end{equation*}
Thus, by using homogeneous coordinates we can implement the translation $T_{w}$ in the top two coordinates. At the same time, matrix transformations we performed earlier can still be accomplished using homogeneous coordinates.  For instance, the matrix transformation induced by $A = \left[
\begin{array}{rr}
a & b\\
c & d
\end{array}
\right]$ is given by a $3 \times 3$ matrix:
\begin{equation*}
\left[
\begin{array}{rrr}
a & b & 0\\
c & d & 0\\
0 & 0 & 1
\end{array}
\right] \left[
\begin{array}{r}
x\\
y\\
1
\end{array}
\right]
=
\left[
\begin{array}{c}
ax + by\\
cx + dy\\
1
\end{array}
\right]
=
\left[
\begin{array}{c}
A\vec{v}\\
1
\end{array}
\right]
\end{equation*}
So everything can be accomplished at the expense of using $3 \times 3$ matrices and homogeneous coordinates.

It is interesting to examine the geometric implications of homogeneous coordinates.  All points of the form $\begin{bmatrix}x\\y\end{bmatrix}$, originally located in the $xy$-plane in $\RR^2$, now have the form $\begin{bmatrix}x\\y\\1\end{bmatrix}$ and are located in a horizontal plane $z=1$ in $\RR^3$.  What we perceive as a translation in the horizontal plane $z=1$, is actually a shear in $\RR^3$.  This is similar to how horizontal shears in $\RR^2$ affect points along the line $y=1$: all such points move a fixed distance to the left or to the right. 


\begin{example}\label{exa:013322}
Rotate the letter $A$ through $\frac{\pi}{6}$ about the point $\left[
\begin{array}{c}
4\\
5
\end{array}
\right]$.

\begin{center}
\begin{tikzpicture}[scale=0.4]
\draw[thin,gray!40] (-1,-1) grid (7,10);
  \draw[line width=2pt,red](0,0)--(3,9);
  \draw[line width=2pt,red](6,0)--(3,9);
  \draw[line width=2pt,red](5,3)--(1,3);
  \fill[] (0,0)node[above=2mm, left=1mm]{$(0,0)$} circle (0.2cm);
  \fill[] (6,0)node[above=2mm, right=1mm]{$(6,0)$} circle (0.2cm);
  \fill[] (5,3)node[above=2mm, right=1mm]{$(5,3)$} circle (0.2cm);
  \fill[] (1,3)node[above=2mm, left=1mm]{$(1,3)$} circle (0.2cm);
  \fill[] (3,9)node[above=2mm, right=1mm]{$(3,9)$} circle (0.2cm);
 \end{tikzpicture}
\end{center} 

\begin{explanation}
Using homogenous coordinates for the vertices of the letter results in a data matrix with three rows:
\begin{equation*}
K_{d} =  \left[
\begin{array}{rrrrr}
0 & 6 & 5 & 1 & 3\\
0 & 0 & 3 & 3 & 9 \\
1 & 1 & 1 & 1 & 1
\end{array}
\right]
\end{equation*}

%\begin{wrapfigure}[6]{l}{5cm}
%\vspace*{-5em}
%\centering
%\input{4-vector-geometry/figures/5-an-application-to-computer-graphics/figure4.5.6}
%\caption{\label{fig:013335}}
%\end{wrapfigure}


If we write $\vec{w} = \left[
\begin{array}{c}
4\\
5
\end{array}
\right]$, the idea is to use a composite of transformations: First translate the letter by $-\vec{w}$ so that the point $\vec{w}$ moves to the origin, then rotate this translated letter, and then translate it by $\vec{w}$ back to its original position. The following GeoGebra interactive illustrates this sequence.

\pdfOnly{
Access GeoGebra interactives through the online version of this text at 

\href{https://ximera.osu.edu/oerlinalg}{https://ximera.osu.edu/oerlinalg}.
}

\begin{onlineOnly}
\begin{center}
\geogebra{yd5agbdg}{950}{650}
\end{center}
\end{onlineOnly}

The matrix arithmetic is as follows (remember the order of composition!):
\begin{align*}
& \left[
\begin{array}{rrr}
1 & 0 & 4\\
0 & 1 & 5\\
0 & 0 & 1
\end{array}
\right]
\left[
\begin{array}{lrr}
0.866 & -0.5 & 0\\
0.5 & 0.866 & 0\\
0 & 0 & 1
\end{array}
\right]
\left[
\begin{array}{rrr}
1 & 0 & -4\\
0 & 1 & -5\\
0 & 0 & 1
\end{array}
\right]
\left[
\begin{array}{rrrrr}
0 & 6 & 5 & 1 & 3\\
0 & 0 & 3 & 3 & 9 \\
1 & 1 & 1 & 1 & 1
\end{array}
\right] \\
&= \left[
\begin{array}{lllll}
3.036 & 8.232 & 5.866 & 2.402 & 1.134\\
-1.33 & 1.67 & 3.768 & 1.768 & 7.964 \\
1 & 1 & 1 & 1 & 1
\end{array}
\right]
\end{align*}

\end{explanation}
\end{example}

This discussion merely touches the
surface of computer graphics, and the reader is referred to specialized
books on the subject. Realistic graphic rendering requires an enormous
number of matrix calculations. In fact, matrix multiplication algorithms
 are now embedded in microchip circuits, and can perform over 100
million matrix multiplications per second. This is particularly
important in the field of three-dimensional graphics where the
homogeneous coordinates have four components and $4 \times 4$ matrices are
required.


\section*{Practice Problems}

\begin{problem}\label{prob:app0040_1}
Consider the letter $A$ below. 
\begin{center}
\begin{tikzpicture}[scale=0.4]
\draw[thin,gray!40] (-1,-1) grid (7,10);
  \draw[line width=2pt,red](0,0)--(3,9);
  \draw[line width=2pt,red](6,0)--(3,9);
  \draw[line width=2pt,red](5,3)--(1,3);
  \fill[] (0,0)node[above=2mm, left=1mm]{$(0,0)$} circle (0.2cm);
  \fill[] (6,0)node[above=2mm, right=1mm]{$(6,0)$} circle (0.2cm);
  \fill[] (5,3)node[above=2mm, right=1mm]{$(5,3)$} circle (0.2cm);
  \fill[] (1,3)node[above=2mm, left=1mm]{$(1,3)$} circle (0.2cm);
  \fill[] (3,9)node[above=2mm, right=1mm]{$(3,9)$} circle (0.2cm);
 \end{tikzpicture}
\end{center}  

Find the data matrix for the letter obtained by:


\begin{enumerate} 
\item Rotating the letter through $\frac{\pi}{4}$ about the origin.

\item Rotating the letter through $\frac{\pi}{4}$
 about the point $\left[
 \begin{array}{c}
 1\\
 2
 \end{array}
 \right]$.

Click the arrow to see the answer.
\begin{expandable}
${\frac{1}{2}\left[
\begin{array}{ccccc}
\sqrt{2} + 2 & 7\sqrt{2} + 2 & 3\sqrt{2} + 2 & -\sqrt{2} + 2 & -5\sqrt{2} + 2\\
-3\sqrt{2} + 4 & 3\sqrt{2} + 4 & 5\sqrt{2} + 4 & \sqrt{2} + 4 & 9\sqrt{2} + 4 \\
2 & 2 & 2 & 2 & 2
\end{array}
\right]}$
\end{expandable}
\end{enumerate}
\end{problem}

% \begin{problem}\label{prob:app0040_2}
% Find the matrix for turning the letter $A$ in Figure~\ref{fig:013290} upside-down in place.
% \end{problem}

\begin{problem}\label{prob:app0040_3}
Find the $3 \times 3$ matrix for reflecting in the line $y = mx + b$. Use $\left[
\begin{array}{c}
1\\
m
\end{array}
\right]$ as direction vector for the line.
\end{problem}

\begin{problem}\label{prob:app0040_4}
Find the $3 \times 3$ matrix for rotating through the angle $\theta$ about the point $P(a, b)$.
\end{problem}

\begin{problem}\label{prob:app0040_5}
Find the reflection of the point $P$ in the line $y = 1 + 2x$ in $\RR^2$ if:


\begin{enumerate} 
\item $P = P(1, 1)$

\item $P = P(1, 4)$

Click the arrow to see the answer.
\begin{expandable}
$P(\frac{9}{5}, \frac{18}{5})$
\end{expandable}

\item What about $P = P(1, 3)$? Explain.
\begin{hint}
    See Example~\ref{exa:013322}.
\end{hint}
\end{enumerate}
\end{problem}

\end{document}