\documentclass{ximera}
%% You can put user macros here
%% However, you cannot make new environments

\listfiles

\graphicspath{{./}{firstExample/}{secondExample/}}

\usepackage{tikz}
\usepackage{tkz-euclide}
\usepackage{tikz-3dplot}
\usepackage{tikz-cd}
\usetikzlibrary{shapes.geometric}
\usetikzlibrary{arrows}
%\usetkzobj{all}
\pgfplotsset{compat=1.13} % prevents compile error.

%\renewcommand{\vec}[1]{\mathbf{#1}}
\renewcommand{\vec}{\mathbf}
\newcommand{\RR}{\mathbb{R}}
\newcommand{\dfn}{\textit}
\newcommand{\dotp}{\cdot}
\newcommand{\id}{\text{id}}
\newcommand\norm[1]{\left\lVert#1\right\rVert}
 
\newtheorem{general}{Generalization}
\newtheorem{initprob}{Exploration Problem}

\tikzstyle geometryDiagrams=[ultra thick,color=blue!50!black]

%\DefineVerbatimEnvironment{octave}{Verbatim}{numbers=left,frame=lines,label=Octave,labelposition=topline}



\usepackage{mathtools}

\author{}
\license{Creative Commons 4.0 By-NC-SA}
%\outcome{Compute an antiderivative using basic formulas}
\begin{document}
\begin{exercise}

True or False?  If False, you should come up with a counterexample.  If True, can you give a proof?

 \begin{enumerate}
     \item If $V$ is a subspace of $\RR^n$ and $\vec{x}+\vec{y}$ is in $V$, then $\vec{x}$ is in $V$ or $\vec{y}$ is in $V$.

 \begin{multipleChoice}
 \choice{True}
 \choice[correct]{False}
 \end{multipleChoice}

\item If $V$ is a set in $\RR^n$ such that $c_1{\vec{v}_1}+c_2{\vec{v}_2}$ is in $V$ whenever ${\vec{v}_1}$ and ${\vec{v}_2}$ are in $V$ for any scalars $c_1$, $c_2$, then $V$ is a subspace.

 \begin{multipleChoice}
 \choice[correct]{True}
 \choice{False}
 \end{multipleChoice}

 \item Every set of four non-zero vectors in $\RR^4$ is a basis.

 \begin{multipleChoice}
 \choice{True}
 \choice[correct]{False}
 \end{multipleChoice}

 \item $\RR^3$ has a basis of the form $\left\{\vec{x},\vec{x}+\vec{y},\vec{y}\right\}$.

 \begin{multipleChoice}
 \choice{True}
 \choice[correct]{False}
 \end{multipleChoice}
 \end{enumerate}

 
\end{exercise}

\subsection*{Source}
[Nicholson] W. Keith Nicholson, {\it Linear Algebra with Applications}, Lyryx 2021, Open Edition, Exercise 5.1.
\end{document}