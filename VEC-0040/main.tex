\documentclass{ximera}
%% You can put user macros here
%% However, you cannot make new environments

\listfiles

\graphicspath{{./}{firstExample/}{secondExample/}}

\usepackage{tikz}
\usepackage{tkz-euclide}
\usepackage{tikz-3dplot}
\usepackage{tikz-cd}
\usetikzlibrary{shapes.geometric}
\usetikzlibrary{arrows}
%\usetkzobj{all}
\pgfplotsset{compat=1.13} % prevents compile error.

%\renewcommand{\vec}[1]{\mathbf{#1}}
\renewcommand{\vec}{\mathbf}
\newcommand{\RR}{\mathbb{R}}
\newcommand{\dfn}{\textit}
\newcommand{\dotp}{\cdot}
\newcommand{\id}{\text{id}}
\newcommand\norm[1]{\left\lVert#1\right\rVert}
 
\newtheorem{general}{Generalization}
\newtheorem{initprob}{Exploration Problem}

\tikzstyle geometryDiagrams=[ultra thick,color=blue!50!black]

%\DefineVerbatimEnvironment{octave}{Verbatim}{numbers=left,frame=lines,label=Octave,labelposition=topline}



\usepackage{mathtools}


\title{Linear Combinations of Vectors} \license{CC BY-NC-SA 4.0}

\begin{document}
\begin{abstract}
 \end{abstract}
\maketitle

\begin{onlineOnly}
\section*{Linear Combinations of Vectors}
\end{onlineOnly}

When studying vectors, the two main operations we have learned about are vector addition and scalar multiplication.  Both are involved in the important concept of a \dfn{linear combination} of vectors.

\begin{definition}\label{def:lincomb}
A vector $\vec{v}$ is said to be a \dfn{linear combination} of vectors $\vec{v}_1, \vec{v}_2,\ldots, \vec{v}_n$ if 
$$\vec{v}=a_1\vec{v}_1+ a_2\vec{v}_2+\ldots + a_n\vec{v}_n$$
for some scalars $a_1, a_2, \ldots ,a_n$.
\end{definition}
For example, $\begin{bmatrix} -4\\9\\-10\\-1\end{bmatrix}$ is a linear combination of $\begin{bmatrix} -1\\3\\-3\\0\end{bmatrix}$,
$\begin{bmatrix} 2\\0\\1\\4\end{bmatrix}$ and $\begin{bmatrix} 0\\1\\-1\\1\end{bmatrix}$ because 
$$\begin{bmatrix} -4\\9\\-10\\-1\end{bmatrix}=2\begin{bmatrix} -1\\3\\-3\\0\end{bmatrix}+(-1)\begin{bmatrix} 2\\0\\1\\4\end{bmatrix}+3\begin{bmatrix} 0\\1\\-1\\1\end{bmatrix}$$

In this section we will focus on vectors in $\RR^2$ and $\RR^3$.
\subsection*{Visualizing Linear Combinations in $\RR^2$ and $\RR^3$}

Let's start by visualizing linear combinations of two vectors in $\RR^2$.

\begin{exploration}\label{exp:linCombR2}
Answer the questions below using the GeoGebra interactive.  To use the interactive, you can
\begin{enumerate}
    \item Change vectors $\vec{v}$ and $\vec{w}$ by dragging the tips of these vectors.
    \item Change the coefficients $k_1$ and $k_2$ of the linear combination by using sliders.
\end{enumerate}

\pdfOnly{
Access GeoGebra interactives through the online version of this text at 

\href{https://ximera.osu.edu/oerlinalg}{https://ximera.osu.edu/oerlinalg}.
}

\begin{onlineOnly}
\begin{center} 
\geogebra{md3srgdg}{800}{600} 
\end{center}
\end{onlineOnly}

\begin{enumerate}
\item Let $\vec{w}=\begin{bmatrix}1\\2\end{bmatrix}$ and $\vec{v}=\begin{bmatrix}1\\-1\end{bmatrix}$.  Find $k_1$ and $k_2$ such that $k_1\vec{w}+k_2\vec{v}=\begin{bmatrix}4\\-1\end{bmatrix}$
$$k_1=\answer{1},\quad k_2=\answer{3}$$

\item Let $\vec{w}=\begin{bmatrix}1\\2\end{bmatrix}$ and $\vec{v}=\begin{bmatrix}-2\\0\end{bmatrix}$.  Find $k_1$ and $k_2$ such that $k_1\vec{w}+k_2\vec{v}=\begin{bmatrix}3\\-2\end{bmatrix}$
$$k_1=\answer{-1},\quad k_2=\answer{-2}$$

\item
Use the same vectors $\vec{w}$ and $\vec{v}$ as in the previous part.  Do you think it is possible to express any vector in $\RR^2$ as a linear combination of $\vec{w}$ and $\vec{v}$?  \wordChoice{\choice[correct]{Yes}, \choice{No}}

\item Let $\vec{w}=\begin{bmatrix}4\\2\end{bmatrix}$ and $\vec{v}=\begin{bmatrix}-2\\-1\end{bmatrix}$.  Do you think it is possible to express any vector in $\RR^2$ as a linear combination of $\vec{w}$ and $\vec{v}$?  \wordChoice{\choice{Yes}, \choice[correct]{No}}
\end{enumerate}
    
\end{exploration}

Visualizing linear combinations of vectors in $\RR^3$ is more difficult than doing so in $\RR^2$.  The following Exploration will help you do this.

\begin{exploration}\label{exp:linCombR3}
  we will start by visualizing linear combinations of two vectors $\vec{u}$ and $\vec{v}$ in $\RR^3$.
  To use the interactive, define vectors $\vec{u}$, and $\vec{v}$.  Use sliders to change the coefficients $k_1$ and $k_2$ of the linear combination.  You will see the linear combination $k_1\vec{u}+k_2\vec{v}$ as the pink vector along the diagonal of the parallelogram determined by $\vec{u}$ and $\vec{v}$. 

  RIGHT-CLICK and DRAG the left panel to rotate the image.

\pdfOnly{
Access GeoGebra interactives through the online version of this text at 

\href{https://ximera.osu.edu/oerlinalg}{https://ximera.osu.edu/oerlinalg}.
}

\begin{onlineOnly}
\begin{center} 
\geogebra{tdkfuuc9}{900}{800} 
\end{center}

We will now consider three vectors. Define vectors $\vec{u}$, $\vec{v}$ and $\vec{w}$.  Use sliders to change the coefficients $k_1, k_2$ and $k_3$ of the linear combination.  The linear combination $k_1\vec{u}+k_2\vec{v}+k_3\vec{w}$ is shown as the pink vector along the diagonal of the parallelepiped. 

RIGHT-CLICK and DRAG the left panel to rotate the image.

\pdfOnly{
Access GeoGebra interactives through the online version of this text at 

\href{https://ximera.osu.edu/oerlinalg}{https://ximera.osu.edu/oerlinalg}.
}

\begin{onlineOnly}
\begin{center} 
\geogebra{zvzxujp2}{900}{800} 
\end{center}
\end{onlineOnly}
\end{exploration}

\subsection*{Geometry of Linear Combinations}

\begin{example}\label{ex:lincombparallelogrammethod}
Use geometry to express $\begin{bmatrix}3\\-4\end{bmatrix}$ as a linear combination of $\begin{bmatrix}-1\\3\end{bmatrix}$ and $\begin{bmatrix}-4\\2\end{bmatrix}$.

%\youtube{MMQbX6winjg}

\begin{explanation}
We are looking for $\begin{bmatrix}3\\-4\end{bmatrix}$ to be the diagonal of a parallelogram determined by scalar multiples of $\begin{bmatrix}-1\\3\end{bmatrix}$ and $\begin{bmatrix}-4\\2\end{bmatrix}$.  

\begin{center}
\begin{tikzpicture}[scale=0.8]
\draw[thin,gray!40] (-4,-4) grid (4,4);
  \draw[<->] (-4,0)--(4,0);
  \draw[<->] (0,-4)--(0,4);
  
\draw[line width=2pt,red,-stealth](0,0)--(-1,3) node[left]{$\begin{bmatrix}-1\\3\end{bmatrix}$};

 \draw[line width=2pt,blue,-stealth](0,0)--(-4,2) node[above right]{$\begin{bmatrix}-4\\2\end{bmatrix}$};

 \draw[line width=2pt,-stealth](0,0)--(3,-4)node[right]{$\begin{bmatrix}3\\-4\end{bmatrix}$}; 

\end{tikzpicture}
\end{center}

Because a scalar multiple of a vector can point in the same direction as the vector or in the opposite direction, we will start by drawing straight lines determined by the two vectors.

\begin{center}
\begin{tikzpicture}[scale=0.8]
\draw[thin,gray!40] (-4,-4) grid (4,4);
  \draw[<->] (-4,0)--(4,0);
  \draw[<->] (0,-4)--(0,4);
  

\draw[line width=2pt,red,-stealth](0,0)--(-1,3) node[left]{$\begin{bmatrix}-1\\3\end{bmatrix}$};
\draw[line width=1pt,red, dashed](4/3,-4)--(-4/3,4) ;


 \draw[line width=2pt,blue,-stealth](0,0)--(-4,2) node[above right]{$\begin{bmatrix}-4\\2\end{bmatrix}$};
\draw[line width=1pt,blue, dashed](-4,2)--(4,-2) ;
 
 \draw[line width=2pt,-stealth](0,0)--(3,-4)node[right]{$\begin{bmatrix}3\\-4\end{bmatrix}$}; 
 
\end{tikzpicture}
\end{center}

The two lines that we drew will contain the sides of the parallelogram we are looking for.  To find the other two sides we will draw lines parallel to $\begin{bmatrix}-1\\3\end{bmatrix}$ and $\begin{bmatrix}-4\\2\end{bmatrix}$ through the head of vector $\begin{bmatrix}3\\-4\end{bmatrix}$.

\begin{center}
\begin{tikzpicture}[scale=0.8]
\draw[thin,gray!40] (-4,-4) grid (4,4);
  \draw[<->] (-4,0)--(4,0);
  \draw[<->] (0,-4)--(0,4);
  
\draw[line width=2pt,red,-stealth](0,0)--(-1,3) node[left]{$\begin{bmatrix}-1\\3\end{bmatrix}$};
\draw[line width=1pt,red, dashed](4/3,-4)--(-4/3,4) ;
\draw[line width=1pt,red, dashed](3,-4)--(1/3,4) ;

 \draw[line width=2pt,blue,-stealth](0,0)--(-4,2) node[above right]{$\begin{bmatrix}-4\\2\end{bmatrix}$};
\draw[line width=1pt,blue, dashed](-4,2)--(4,-2) ;
  \draw[line width=1pt,blue, dashed](-4,-0.5)--(3,-4) ;
 \draw[line width=2pt,-stealth](0,0)--(3,-4)node[right]{$\begin{bmatrix}3\\-4\end{bmatrix}$}; 
 
\end{tikzpicture}
\end{center}

Now the parallelogram is clearly visible.

\begin{center}
\begin{tikzpicture}[scale=0.8]
\draw[thin,gray!40] (-4,-4) grid (4,4);
  \draw[<->] (-4,0)--(4,0);
  \draw[<->] (0,-4)--(0,4);
  
 \filldraw[blue, opacity=0.3](0,0)--(1,-3)--(3,-4)--(2,-1)--cycle;
\draw[line width=2pt,red,-stealth](0,0)--(-1,3) node[left]{$\begin{bmatrix}-1\\3\end{bmatrix}$};
\draw[line width=1pt,red, dashed](4/3,-4)--(-4/3,4) ;
\draw[line width=1pt,red, dashed](3,-4)--(1/3,4) ;

 \draw[line width=2pt,blue,-stealth](0,0)--(-4,2) node[above right]{$\begin{bmatrix}-4\\2\end{bmatrix}$};
\draw[line width=1pt,blue, dashed](-4,2)--(4,-2) ;
  \draw[line width=1pt,blue, dashed](-4,-0.5)--(3,-4) ;
 \draw[line width=2pt,-stealth](0,0)--(3,-4)node[right]{$\begin{bmatrix}3\\-4\end{bmatrix}$}; 
 
\end{tikzpicture}
\end{center}

The last remaining task is to identify the sides of the parallelogram as scalar multiples of $\begin{bmatrix}-1\\3\end{bmatrix}$ and $\begin{bmatrix}-4\\2\end{bmatrix}$.  We do this by identifying vectors $\begin{bmatrix}1\\-3\end{bmatrix}$ and $\begin{bmatrix}2\\-1\end{bmatrix}$ as the vectors that determine the parallelogram.

\begin{center}
\begin{tikzpicture}[scale=0.8]
\draw[thin,gray!40] (-4,-4) grid (4,4);
  \draw[<->] (-4,0)--(4,0);
  \draw[<->] (0,-4)--(0,4);
  
 \filldraw[blue, opacity=0.3](0,0)--(1,-3)--(3,-4)--(2,-1)--cycle;
\draw[line width=2pt,red,-stealth](0,0)--(-1,3) node[left]{$\begin{bmatrix}-1\\3\end{bmatrix}$};
\draw[line width=1pt,red, dashed](4/3,-4)--(-4/3,4) ;
\draw[line width=1pt,red, dashed](3,-4)--(1/3,4) ;

 \draw[line width=2pt,blue,-stealth](0,0)--(-4,2) node[above right]{$\begin{bmatrix}-4\\2\end{bmatrix}$};
\draw[line width=1pt,blue, dashed](-4,2)--(4,-2) ;
  \draw[line width=1pt,blue, dashed](-4,-0.5)--(3,-4) ;
 \draw[line width=2pt,-stealth](0,0)--(3,-4)node[right]{$\begin{bmatrix}3\\-4\end{bmatrix}$}; 
 
  \fill[red] (1,-3) node[below left]{$(1,-3)$} circle (0.1cm);
  \fill[blue] (2,-1) node[above right]{$(2,-1)$} circle (0.1cm);
\end{tikzpicture}
\end{center}

Observe that vector $\begin{bmatrix}2\\-1\end{bmatrix}$ is half the length of $\begin{bmatrix}-4\\2\end{bmatrix}$ and points in the opposite direction, while the vector $\begin{bmatrix}1\\-3\end{bmatrix}$ is the same length as $\begin{bmatrix}-1\\3\end{bmatrix}$ and also points in the opposite direction.

Now we can write $\begin{bmatrix}3\\-4\end{bmatrix}$ as a linear combination of $\begin{bmatrix}-1\\3\end{bmatrix}$ and $\begin{bmatrix}-4\\2\end{bmatrix}$ as follows
 $$\begin{bmatrix}3\\-4\end{bmatrix}=\begin{bmatrix}1\\-3\end{bmatrix}+\begin{bmatrix}2\\-1\end{bmatrix}=(-1)\begin{bmatrix}-1\\3\end{bmatrix}+\left(-\frac{1}{2}\right)\begin{bmatrix}-4\\2\end{bmatrix}$$
 
\end{explanation}
\end{example}

The method we used in Example \ref{ex:lincombparallelogrammethod} to express the given vector as a linear combination of two other vectors is sufficiently useful that we summarize the steps.

\begin{procedure}\label{pro:lincombgeo} Given two non-collinear vectors $\vec{u}$ and $\vec{v}$ in $\RR^2$, and a vector $\vec{w}$, we can express $\vec{w}$ as a linear combination of $\vec{u}$ and $\vec{v}$ as follows:
\begin{enumerate}
\item Draw lines $L_{\vec{u}}$ and $L_{\vec{v}}$ determined by $\vec{u}$ and $\vec{v}$, respectively.
\item Through the head of vector $\vec{w}$, draw lines $P_{\vec{u}}$ and $P_{\vec{v}}$, parallel to $L_{\vec{u}}$ and $L_{\vec{v}}$, respectively.
\item Let $A$ be the point of intersection of $P_{\vec{u}}$ and $L_{\vec{v}}$.
\item Let $B$ be the point of intersection of $P_{\vec{v}}$ and $L_{\vec{u}}$.
\item Let $O$ denote the origin.  Then $\overrightarrow{OA}=k_1\vec{v}$ and $\overrightarrow{OB}=k_2\vec{u}$ for some scalars $k_1$ and $k_2$.  
\item We have $\vec{w}=k_2\vec{u}+k_1\vec{v}$.
\end{enumerate}

\begin{center}
\begin{tikzpicture}[scale=0.8]

  \draw[<->] (-4,0)--(4,0);
  \draw[<->] (0,-4)--(0,4);
  
 \filldraw[blue, opacity=0.3](0,0)--(1,-3)--(3,-4)--(2,-1)--cycle;
\draw[line width=2pt,red,-stealth](0,0)--(-1,3);
\node[red] at (-1.2, 2.5)   (b) {$\vec{u}$};
\node[red] at (-0.7, 3.5)   (b) {$L_{\vec{u}}$};

\draw[line width=1pt,red, dashed](4/3,-4)--(-4/3,4) ;
\draw[line width=1pt,red, dashed](3,-4)--(1/3,4);
\node[red] at (0.9, 3.5)   (b) {$P_{\vec{u}}$};

 \draw[line width=2pt,blue,-stealth](0,0)--(4,-2) ;
 \node[blue] at (3.5, -1.5)   (b) {$\vec{v}$};
 
\draw[line width=1pt,blue, dashed](-4,2)node[above right]{$L_{\vec{v}}$}--(4,-2) ;
  \draw[line width=1pt,blue, dashed](-4,-0.5)--(3,-4) ;
  \node[blue] at (-3.5, -1.2)   (b) {$P_{\vec{v}}$};
  
 \draw[line width=2pt,-stealth](0,0)node[below left]{$O$}--(3,-4)node[right]{$\vec{w}$}; 
 
  \fill[red] (1,-3) node[below left]{$B$} circle (0.1cm);
  \fill[blue] (2,-1) node[above right]{$A$} circle (0.1cm);
\end{tikzpicture}
\end{center}

\end{procedure}

\begin{exploration}\label{exp:proc4}
This GeoGebra interactive will allow you to go through the steps given in Procedure \ref{pro:lincombgeo} for a combination of vectors of your choice.  To use the interactive
\begin{enumerate}
    \item Enter components of vectors $\vec{u}$ and $\vec{v}$.
    \item Enter components of vector $\vec{w}$ that you want to express as a linear combination of $\vec{u}$ and $\vec{v}$.
    \item Use the navigation bar to go through the steps of Procedure \ref{pro:lincombgeo}
\end{enumerate}

\pdfOnly{
Access GeoGebra interactives through the online version of this text at 

\href{https://ximera.osu.edu/oerlinalg}{https://ximera.osu.edu/oerlinalg}.
}

\begin{onlineOnly}
\begin{center} 
\geogebra{u77b52k8}{800}{600} 
\end{center}
\end{onlineOnly}

\end{exploration}

\subsection*{From Geometry to Algebra of Linear Combinations}
One of the stipulations in Procedure \ref{pro:lincombgeo} is that vectors $\vec{u}$ and $\vec{v}$ should be non-collinear.  You can use the interactive in Exploration \ref{exp:proc4} to investigate what happens when $\vec{u}$ and $\vec{v}$ are collinear.  The following example examines what happens from a geometric as well as an algebraic standpoint.

\begin{example}\label{ex:lincombgeometry2}
Can the vector $\begin{bmatrix}3\\2\end{bmatrix}$ be written as a linear combination of vectors $\begin{bmatrix}-3\\1\end{bmatrix}$ and $\begin{bmatrix}6\\-2\end{bmatrix}$?
\begin{explanation}
We will start with a geometric approach.
\begin{center}
\begin{tikzpicture}[scale=0.8]
\draw[thin,gray!40] (-4,-5) grid (7,4);
  \draw[<->] (-4,0)--(7,0);
  \draw[<->] (0,-5)--(0,4);
  \draw[line width=2pt,red,-stealth](0,0)--(-3,1) node[above right]{$\begin{bmatrix}-3\\1\end{bmatrix}$};

 \draw[line width=2pt,blue,-stealth](0,0)--(6,-2) node[below left]{$\begin{bmatrix}6\\-2\end{bmatrix}$};

 \draw[line width=2pt,-stealth](0,0)--(3,2)node[right]{$\begin{bmatrix}3\\2\end{bmatrix}$}; 
 
\end{tikzpicture}
\end{center}

Observe that $\begin{bmatrix}-3\\1\end{bmatrix}$ and $\begin{bmatrix}6\\-2\end{bmatrix}$ are scalar multiples of each other and lie on the same line.  

A linear combination of $\begin{bmatrix}-3\\1\end{bmatrix}$ and $\begin{bmatrix}6\\-2\end{bmatrix}$ has the form:

$$a\begin{bmatrix}6\\-2\end{bmatrix}+b\begin{bmatrix}-3\\1\end{bmatrix}=a(-2)\begin{bmatrix}-3\\1\end{bmatrix}+b\begin{bmatrix}-3\\1\end{bmatrix}=(-2a+b)\begin{bmatrix}-3\\1\end{bmatrix}$$

This shows that all linear combinations of $\begin{bmatrix}-3\\1\end{bmatrix}$ and $\begin{bmatrix}6\\-2\end{bmatrix}$ will be scalar multiples of $\begin{bmatrix}-3\\1\end{bmatrix}$, and therefore lie on the same line as $\begin{bmatrix}-3\\1\end{bmatrix}$.
Since $\begin{bmatrix}3\\2\end{bmatrix}$ does not lie on the line determined by $\begin{bmatrix}-3\\1\end{bmatrix}$ it cannot be expressed as a linear combination of $\begin{bmatrix}-3\\1\end{bmatrix}$ and $\begin{bmatrix}6\\-2\end{bmatrix}$.

We can also address this question algebraically.  To express $\begin{bmatrix}3\\2\end{bmatrix}$ as a linear combination of $\begin{bmatrix}-3\\1\end{bmatrix}$ and $\begin{bmatrix}6\\-2\end{bmatrix}$, we need to solve the equation.
$$a\begin{bmatrix}-3\\1\end{bmatrix}+b\begin{bmatrix}6\\-2\end{bmatrix}=\begin{bmatrix}3\\2\end{bmatrix}$$
This gives us a system of equations
\begin{align*}
-3a+6b&=3\\
a-2b&=2
\end{align*}

When you try to solve this system, you will find that the system is inconsistent.  Thus, $\begin{bmatrix}3\\2\end{bmatrix}$ cannot be written as a linear combination of $\begin{bmatrix}-3\\1\end{bmatrix}$ and $\begin{bmatrix}6\\-2\end{bmatrix}$.

We know that there is no way to express $\begin{bmatrix}3\\2\end{bmatrix}$ as a linear combination of vectors $\begin{bmatrix}-3\\1\end{bmatrix}$ and $\begin{bmatrix}6\\-2\end{bmatrix}$.  What would happen if we tried to apply Procedure \ref{pro:lincombgeo} to these vectors?  You can use the GeoGebra interactive in Exploration \ref{exp:proc4} to find out.

\end{explanation}
\end{example}

\begin{example}\label{ex:lincombgeometry1}
Express $\begin{bmatrix}2\\4\end{bmatrix}$ as a linear combination of $\begin{bmatrix}2\\1\end{bmatrix}$ and $\begin{bmatrix}2\\-2\end{bmatrix}$.  Interpret your results geometrically.
\begin{explanation}
We need to find scalars $a$ and $b$ such that 
$$\begin{bmatrix}2\\4\end{bmatrix}=a\begin{bmatrix}2\\1\end{bmatrix}+b\begin{bmatrix}2\\-2\end{bmatrix}$$
This amounts to solving a system of linear equations
\begin{align*}
2a+2b&=2\\
a-2b&=4
\end{align*}
Use your favorite method to solve this system. (Hint: adding the second equation to the first will work well for this system.) You will find that $a=2$ and $b=-1$.  Now we can write $\begin{bmatrix}2\\4\end{bmatrix}$ as a linear combination of $\begin{bmatrix}2\\1\end{bmatrix}$ and $\begin{bmatrix}2\\-2\end{bmatrix}$ as follows:

$$\begin{bmatrix}2\\4\end{bmatrix}=2\begin{bmatrix}2\\1\end{bmatrix}+(-1)\begin{bmatrix}2\\-2\end{bmatrix}$$

Geometrically speaking, this means that the vector $\begin{bmatrix}2\\4\end{bmatrix}$ is the diagonal of the parallelogram determined by $2\begin{bmatrix}2\\1\end{bmatrix}$ and $(-1)\begin{bmatrix}2\\-2\end{bmatrix}$.  The original vectors $\begin{bmatrix}2\\1\end{bmatrix}$ and $\begin{bmatrix}2\\-2\end{bmatrix}$ are shown below together with the parallelogram and its diagonal. 

\begin{center}
\begin{tikzpicture}[scale=1]
\draw[thin,gray!40] (-4,-4) grid (4,4);
  \draw[<->] (-4,0)--(4,0);
  \draw[<->] (0,-4)--(0,4);
  
  \filldraw[blue, opacity=0.3](0,0)--(-2,2)--(2,4)--(4,2)--cycle;
\draw[line width=2pt,red,-stealth](0,0)--(2,1) node[below right]{$\begin{bmatrix}2\\1\end{bmatrix}$};
\draw[line width=2pt,red,-stealth, dashed](0,0)--(4,2) node[below right]{$2\begin{bmatrix}2\\1\end{bmatrix}$};
 % \fill[blue!40!white] (10,0) circle (0.2cm);
 \draw[line width=2pt,blue,-stealth](0,0)--(2,-2) node[right]{$\begin{bmatrix}2\\-2\end{bmatrix}$};
 \draw[line width=2pt,blue,-stealth, dashed](0,0)--(-2,2) node[above left]{$(-1)\begin{bmatrix}2\\-2\end{bmatrix}$};
 \draw[line width=2pt,-stealth](0,0)--(2,4); 
\end{tikzpicture}
\end{center}

\end{explanation}
\end{example}

\begin{example}\label{ex:lincombalgebra1}
If possible, express $\begin{bmatrix}7\\4\\-5\end{bmatrix}$ as a linear combination of $\begin{bmatrix}1\\-2\\1\end{bmatrix}$ and $\begin{bmatrix}3\\0\\-1\end{bmatrix}$.
\begin{explanation}
We are looking for coefficients $a$ and $b$ such that 
$$a\begin{bmatrix}1\\-2\\1\end{bmatrix}+b\begin{bmatrix}3\\0\\-1\end{bmatrix}=\begin{bmatrix}7\\4\\-5\end{bmatrix}$$
This translates into a system of equations
\begin{align*}
a+3b&=7\\
-2a&=4\\
a-b&=-5
\end{align*}
Solve this system for $a$ and $b$, and enter your answers below:
$$a=\answer{-2}\quad\text{and}\quad b=\answer{3}$$
We conclude that $\begin{bmatrix}7\\4\\-5\end{bmatrix}$ is a linear combination of $\begin{bmatrix}1\\-2\\1\end{bmatrix}$ and $\begin{bmatrix}3\\0\\-1\end{bmatrix}$, and write:
$$\begin{bmatrix}7\\4\\-5\end{bmatrix}=\answer{-2}\begin{bmatrix}1\\-2\\1\end{bmatrix}+\answer{3}\begin{bmatrix}3\\0\\-1\end{bmatrix}$$
\end{explanation}
\end{example}

\begin{example}\label{ex:lincombalgebra2}
Set up a system of equations that can be used to express $\begin{bmatrix}2\\-1\\3\\0\end{bmatrix}$ as a linear combination of $\begin{bmatrix}1\\0\\4\\-2\end{bmatrix}$, $\begin{bmatrix}-2\\-1\\1\\-1\end{bmatrix}$, $\begin{bmatrix}0\\4\\-3\\1\end{bmatrix}$ and $\begin{bmatrix}1\\1\\-1\\4\end{bmatrix}$, or to determine that such a combination does not exist.  Do not solve the system.
\begin{explanation}
We are looking for $x_1$, $x_2$, $x_3$ and $x_4$ such that 
$$x_1\begin{bmatrix}1\\0\\4\\-2\end{bmatrix}+x_2\begin{bmatrix}-2\\-1\\1\\-1\end{bmatrix}+x_3\begin{bmatrix}0\\4\\-3\\1\end{bmatrix}+x_4\begin{bmatrix}1\\1\\-1\\4\end{bmatrix}=\begin{bmatrix}2\\-1\\3\\0\end{bmatrix}$$
This translates into the following system of equations:

$$\begin{array}{ccccccccc}
      x_1 & -&2x_2 & & &+ & x_4&= &2 \\
        & &-x_2  &+ &4x_3 & +& x_4&= &-1 \\
      4x_1 &+ &x_2 & -&3x_3 &-&x_4&= &3\\
	 -2x_1& -&x_2 & +&x_3&+&4x_4&=&0
    \end{array}$$
    
\end{explanation}
\end{example}




\section*{Practice Problems}
\begin{problem}\label{prob:lincombtwovectors1}
Solve a system of linear equations to express $\begin{bmatrix}-1\\7\end{bmatrix}$ as a linear combination of $\begin{bmatrix}1\\2\end{bmatrix}$ and $\begin{bmatrix}-1\\1\end{bmatrix}$.

System of linear equations:
$$\begin{array}{ccccc}
      \answer{1}a & +&\answer{-1}b&= &\answer{-1} \\
	 \answer{2}a& +&\answer{1}b&=&\answer{7}
    \end{array}$$
    
    Values of $a$ and $b$:
    $$a=\answer{2}\quad\text{and}\quad b=\answer{3}$$
    
    Linear Combination:
    $$\begin{bmatrix}-1\\7\end{bmatrix}=\answer{2}\begin{bmatrix}1\\2\end{bmatrix}+\answer{3}\begin{bmatrix}-1\\1\end{bmatrix}$$

\end{problem}


\begin{problem}\label{prob:lincombtwovectors2}
Use Procedure \ref{pro:lincombgeo} to express $\begin{bmatrix}-3\\0\end{bmatrix}$ as a linear combination of $\begin{bmatrix}2\\4\end{bmatrix}$ and $\begin{bmatrix}-1\\1\end{bmatrix}$.

Linear Combination:
$$\begin{bmatrix}-3\\0\end{bmatrix}=\answer{-0.5}\begin{bmatrix}2\\4\end{bmatrix}+\answer{2}\begin{bmatrix}-1\\1\end{bmatrix}$$
\end{problem}

\begin{problem}\label{prob:lincombtwovectors3}
Use two different approaches (algebraic and geometric) to explain why the vector $\begin{bmatrix}5\\1\end{bmatrix}$ cannot be expressed as a linear combination of vectors $\begin{bmatrix}2\\-1\end{bmatrix}$ and $\begin{bmatrix}-4\\2\end{bmatrix}$.
\end{problem}

\begin{problem}%\label{prob:lincombtwovectors4}
We have seen Procedure \ref{pro:lincombgeo} applied to vectors in $\RR^2$.  The same process can, in certain cases be applied to vectors in $\RR^3$.  In both parts of this problem you will be asked to follow the steps in Procedure \ref{pro:lincombgeo} to express one vector as a linear combination of two given vectors.  Then you will be asked to identify the condition which makes it possible to do so.

\begin{problem}\label{prob:lincombtwovectors4a}
The following GeoGebra interactive shows vectors ${\bf u}=\begin{bmatrix}3\\0\\-1\end{bmatrix}$, ${\bf v}=\begin{bmatrix}1\\-2\\1\end{bmatrix}$ and ${\bf w}=\begin{bmatrix}7\\4\\-5\end{bmatrix}$.
RIGHT-CLICK and DRAG to rotate the image.

\pdfOnly{
Access GeoGebra interactives through the online version of this text at 

\href{https://ximera.osu.edu/oerlinalg}{https://ximera.osu.edu/oerlinalg}.
}

\begin{onlineOnly}
\begin{center} 
\geogebra{kk5hwjka}{800}{600} 
\end{center}
\end{onlineOnly}

\begin{enumerate}
 \item
 Can $\vec{w}$ be expressed as a linear combination of $\vec{u}$ and $\vec{v}$?
 \begin{multipleChoice}
 \choice{No, because $\vec{w}$ is not between $\vec{u}$ and $\vec{v}$.}
 \choice{Yes, because all three vectors are in the same plane.}
 \choice[correct]{Yes, because all three vectors are in the same plane, AND $\vec{u}$ and $\vec{v}$ are not collinear.}
 \end{multipleChoice}
 \item 
 Use the navigation bar at the bottom of the interactive window to view construction steps of Procedure \ref{pro:lincombgeo} applied to vectors $\vec{u}$, $\vec{v}$ and $\vec{w}$.  (Right-click and drag to rotate the image.)  Use the final image to 
 express ${\bf w}$ as a linear combination of ${\bf v}$ (blue) and ${\bf u}$ (red).
 $${\bf w}=\answer{-2}{\bf v}+\answer{3}{\bf u}$$
 \end{enumerate}
\end{problem}

\begin{problem}\label{prob:lincombtwovectors4b}
The following GeoGebra interactive shows vectors ${\bf v}=\begin{bmatrix}1\\-2\\1\end{bmatrix}$, ${\bf u}=\begin{bmatrix}3\\0\\-1\end{bmatrix}$, and ${\bf w}=\begin{bmatrix}7\\4\\0\end{bmatrix}$. 
RIGHT-CLICK and DRAG to rotate the image.  Use geometry to explain why ${\bf w}$ cannot be expressed as a linear combination of ${\bf v}$ and ${\bf u}$.

\pdfOnly{
Access GeoGebra interactives through the online version of this text at 

\href{https://ximera.osu.edu/oerlinalg}{https://ximera.osu.edu/oerlinalg}.
}

\begin{onlineOnly}
\begin{center} 
\geogebra{x9f6qnyr}{800}{600} 
\end{center}
\end{onlineOnly}

We can also show that ${\bf w}$ is not a linear combination of ${\bf v}$ and ${\bf u}$ algebraically by attempting to solve a system of equations corresponding to 
$$x_1{\bf v}+x_2{\bf u}={\bf w}$$
Set up the system of equations
$$\begin{array}{ccccccc}
      \answer{1}x_1 & +&\answer{3}x_2&= &\answer{7} \\
	 \answer{-2}x_1& +&\answer{0}x_2&=&\answer{4}\\
     \answer{1}x_1& +&\answer{-1}x_2&=&\answer{0}
    \end{array}$$

Find the reduced row echelon form.

$$\left[\begin{array}{cc|c} 
 \answer{1}&\answer{0}&\answer{0}\\\answer{0}&\answer{1}&\answer{0}\\\answer{0}&\answer{0}&\answer{1}
 \end{array}\right]$$

\end{problem}

\end{problem}

\end{document}
