\documentclass{ximera}
%% You can put user macros here
%% However, you cannot make new environments

\listfiles

\graphicspath{{./}{firstExample/}{secondExample/}}

\usepackage{tikz}
\usepackage{tkz-euclide}
\usepackage{tikz-3dplot}
\usepackage{tikz-cd}
\usetikzlibrary{shapes.geometric}
\usetikzlibrary{arrows}
%\usetkzobj{all}
\pgfplotsset{compat=1.13} % prevents compile error.

%\renewcommand{\vec}[1]{\mathbf{#1}}
\renewcommand{\vec}{\mathbf}
\newcommand{\RR}{\mathbb{R}}
\newcommand{\dfn}{\textit}
\newcommand{\dotp}{\cdot}
\newcommand{\id}{\text{id}}
\newcommand\norm[1]{\left\lVert#1\right\rVert}
 
\newtheorem{general}{Generalization}
\newtheorem{initprob}{Exploration Problem}

\tikzstyle geometryDiagrams=[ultra thick,color=blue!50!black]

%\DefineVerbatimEnvironment{octave}{Verbatim}{numbers=left,frame=lines,label=Octave,labelposition=topline}



\usepackage{mathtools}


\title{Diagonalizable Matrices and Multiplicity} \license{CC BY-NC-SA 4.0}

\begin{document}  
\begin{abstract}
\end{abstract}
\maketitle

\begin{onlineOnly}
\section*{Diagonalizable Matrices and Multiplicity}
\end{onlineOnly}

%When a matrix is similar to a diagonal matrix, the matrix is said to be \dfn{diagonalizable}. 
Recall that a \dfn{diagonal matrix} $D$ is a matrix containing a zero in every entry except those on the main diagonal.  More precisely, if $d_{ij}$ is the $ij^{th}$ entry of a diagonal matrix $D$, then
$d_{ij}=0$ unless $i=j$. Such
matrices look like the following.
\begin{equation*}
D = 
\begin{bmatrix}
* &  & 0 \\
& \ddots &  \\
0 &  & *
\end{bmatrix}
\end{equation*}
where $*$ is a number which might not be zero.


Diagonal matrices have some nice properties, as we demonstrate below.

\begin{exploration}\label{init:multiplydiag}
Let
$M =\begin{bmatrix}1 & 2 & 3\\ 4&5&6\\7&8&9\end{bmatrix}$ and let $D =\begin{bmatrix}2 & 0 & 0\\ 0&-5&0\\0&0&10\end{bmatrix}$.  Compute $MD$ and $DM$ using Octave.

%To use Octave, go to the \href{https://sagecell.sagemath.org/}{Sage Math Cell Webpage}, copy the code below into the cell, select OCTAVE as the language, and press EVALUATE.

\begin{verbatim}
% Define M
M=[1 2 3; 4 5 6; 7 8 9]

% The function diag(vector) creates a square matrix with diagonal 
%vector and the remaining entries zero.
D=diag([2 -5 10]) 

% The function diag() can also be used to return the diagonal of a matrix.  
diag_M=diag(M)

% Understanding multiplication by a diagonal matrix
product_MD=M*D
product_DM=D*M

% Raising a diagonal matrix to a power
D5=D^5
\end{verbatim}

\href{https://sagecell.sagemath.org/?z=eJx1zztPwzAUBeA9Uv7DWSqRSlS0EB6qsnn1gmCqSnWb3LSWErv4QYFfz00ilQExWvb5zvEMiltjGTrPdLVZYoXbNe5Q4n6NBzziaZtneTbDy5HRJltH4ywaQ4erD66j8wVqzxQ5gBDeE3lGT9GbT5xNPI4vnaUOYkwBkG0QRfPck7HGHsBWAiJ8s3eLPFPV6G9WuC6xvNkW-G-ClJMFdcFhz0iBRXYCx-Tt2HGpd63sm4YtIN5wsdNTkS4m_9U27EOUfcOoPnXRnDpT09i3_xLgwk1Snp28a1Idd1pVeq5-z0pXaq4n9plMGMA_8WEr4eTO7OXTZaXeyh9wS35E&lang=octave&interacts=eJyLjgUAARUAuQ==}{Link to code}.

%$$MD = \begin{bmatrix} \answer{2} & \answer{-10} & \answer{30}\\ \answer{8}&\answer{-25}&\answer{60}\\\answer{14}&\answer{-40}&\answer{90}\end{bmatrix} \quad DM= \begin{bmatrix} \answer{2} & \answer{4} & \answer{6}\\ \answer{-20}&\answer{-25}&\answer{-30}\\\answer{70}&\answer{80}&\answer{90}\end{bmatrix}$$

Notice the patterns present in the product matrices.  
\begin{itemize}
    \item Each row of $DM$ is the same as its corresponding row of $M$ multiplied by the scalar which is the corresponding diagonal element of $D$.
    \item In the product $MD$, it is the columns of $M$ that have been multiplied by the diagonal elements.
    \item Raising $D$ to the $5^{th}$ power amounts to raising each diagonal element to the $5^{th}$ power.
\end{itemize}   
These patterns hold in general for any diagonal matrix, and they are fundamental to understanding \emph{diagonalization} and its uses.  We discuss diagonalization next.
\end{exploration}

\begin{definition}\label{def:diagonalizable}
Let $A$ be an $n\times n$ matrix. Then $A$ is said to be \dfn{diagonalizable} if there exists an invertible matrix $P$ such that
\begin{equation*}
P^{-1}AP=D
\end{equation*}
where $D$ is a diagonal matrix.  In other words, a matrix $A$ is diagonalizable if it is similar to a diagonal matrix, $A \sim D$.
\end{definition}

% Notice that the above equation can be rearranged as $A=PDP^{-1}$. Suppose we wanted to compute $A^{100}$. By diagonalizing $A$ first it suffices to then compute $\left(PDP^{-1}\right)^{100}$, which reduces to $PD^{100}P^{-1}$. This last computation is much simpler than $A^{100}$. While this process is described in detail later, it provides motivation for diagonalization. 


If we are given a matrix $A$ that is diagonalizable, then we can write $P^{-1}AP=D$ for some matrix $P$, or, equivalently,
\begin{equation}\label{eq:understand_diag}
AP=PD   
\end{equation}
If we pause to examine Equation (\ref{eq:understand_diag}), the work that we did in Exploration  \ref{init:multiplydiag} can help us to understand how to find $P$ that will diagonalize $A$. The product $PD$ is formed by multiplying each column of $P$ by a scalar which is the corresponding element on the diagonal of $D$.  To restate this, if $\vec{x}_i$ is column $i$ in our matrix $P$, then Equation (\ref{eq:understand_diag}) tells us that 
\begin{equation}\label{eq:ev_ew_diag}
A \vec{x}_i = \lambda_i \vec{x}_i,  
\end{equation}
where $\lambda_i$ is the $i$th diagonal element of $D$.  

Of course, Equation (\ref{eq:ev_ew_diag}) is very familiar!  We see that if we are able to diagonalize a matrix $A$, the columns of matrix $P$ will be the eigenvectors of $A$, and the corresponding diagonal entries of $D$ will be the corresponding eigenvalues of $A$.  This is summed up in the following theorem.

\begin{theorem}\label{th:eigenvectorsanddiagonalizable}
An $n\times n$ matrix $A$ is diagonalizable if and only if there is an
invertible matrix $P$ given by
$$P=\begin{bmatrix}
| & | &   & | \\
\vec{x}_1 & \vec{x}_2  & \cdots & \vec{x}_n \\
| & | &   & |
\end{bmatrix},$$
where the columns $\vec{x}_i$ are eigenvectors of $A$.

Moreover, if $A$ is diagonalizable, the corresponding eigenvalues of $A$ are the
diagonal entries of the diagonal matrix $D$.
\end{theorem}

\begin{proof}
Suppose $P$ is given as above as an invertible matrix whose columns are eigenvectors of $A$. To show that $A$ is diagonalizable, we will show 
$$AP=PD,$$
which is equivalent to $P^{-1}AP=D$.  We have
$$AP=\begin{bmatrix}
| & | &   & | \\
A\vec{x}_1 & A\vec{x}_2  & \cdots & A\vec{x}_n \\
| & | &   & |
\end{bmatrix},$$
while
\begin{equation*}
PD=\begin{bmatrix}
| & | &   & | \\
\vec{x}_1 & \vec{x}_2  & \cdots & \vec{x}_n \\
| & | &   & |
\end{bmatrix} 
\begin{bmatrix}
\lambda _{1} &  & 0 \\
& \ddots &  \\
0 &  & \lambda _{n}
\end{bmatrix}=\begin{bmatrix}
| & | &   & | \\
\lambda _{1}\vec{x}_1 & \lambda _{2}\vec{x}_2 & \cdots & \lambda_{n}\vec{x}_n \\
| & | &   & | 
\end{bmatrix}
\end{equation*}
We can complete this half of the proof by comparing columns, and noting that 
\begin{equation}
A \vec{x}_i = \lambda_i \vec{x}_i 
\end{equation}
for $i=1,\ldots,n$ since the $ \vec{x}_i$ are eigenvectors of $A$ and the $\lambda_i$ are corresponding eigenvalues of $A$.

Conversely, suppose $A$ is diagonalizable so that $P^{-1}AP=D.$ Let
\begin{equation*}
P=\begin{bmatrix}
| & | &   & | \\
\vec{x}_1 & \vec{x}_2  & \cdots & \vec{x}_n \\
| & | &   & |
\end{bmatrix}
\end{equation*}
 where the columns are the vectors $\vec{x}_i$ and
\begin{equation*}
D=\begin{bmatrix}
\lambda _{1} &  & 0 \\
& \ddots &  \\
0 &  & \lambda _{n}
\end{bmatrix}
\end{equation*}
Then
\begin{equation*}
AP=PD=\begin{bmatrix}
| & | &   & | \\
\vec{x}_1 & \vec{x}_2  & \cdots & \vec{x}_n \\
| & | &   & |
\end{bmatrix} \begin{bmatrix}
\lambda _{1} &  & 0 \\
& \ddots &  \\
0 &  & \lambda _{n}
\end{bmatrix}
\end{equation*}
and so
\begin{equation*}
\begin{bmatrix}
| & | &   & | \\
A\vec{x}_1 & A\vec{x}_2  & \cdots & A\vec{x}_n \\
| & | &   & |
\end{bmatrix} =\begin{bmatrix}
| & | &   & | \\
\lambda _{1}\vec{x}_1 & \lambda _{2}\vec{x}_2 & \cdots & \lambda_{n}\vec{x}_n \\
| & | &   & | 
\end{bmatrix}
\end{equation*}
showing the $\vec{x}_i$ are eigenvectors of $A$ and the $\lambda _{i}$
are eigenvalues.
%% Now the $x_{k}$ form a basis for $\mathbb{R}^{n}$
%%because the matrix $S$ having these vectors as columns is given to be
%%invertible.
\end{proof}

Notice that because the matrix $P$ defined above is invertible it follows that the set of eigenvectors of $A$, $\left\{ \vec{x}_1 , \vec{x}_2 , \ldots, , \vec{x}_n  \right\}$, is a basis of $\mathbb{R}^n$.

We demonstrate the concept given in the above theorem in the next example. Note that not only
are the columns of the matrix $P$ formed by eigenvectors, but $P$ must
be invertible, and therefore must consist of a linearly independent set of eigenvectors. 
%We achieve this by using basic eigenvectors for the columns of $P$.

\begin{example}\label{ex:diagonalizematrix}
Let
\begin{equation*}
A=\begin{bmatrix}
2 & 0 & 0 \\
1 & 4 & -1 \\
-2 & -4 & 4
\end{bmatrix}
\end{equation*}
 Find an invertible matrix $P$ and a diagonal matrix $D$ such that $P^{-1}AP=D$.

\begin{explanation}
We will use eigenvectors of $A$ as the columns of $P$, and
the corresponding eigenvalues of $A$ as the diagonal entries of $D$. The eigenvalues of $A$ are $\lambda_1 =2,\lambda_2 = 2$, and $\lambda_3 = 6$.   We leave these computations as exercises, as well as the computations to find a basis for each eigenspace.  

One possible basis for $\mathcal{S}_2$, the eigenspace corresponding to $2$, is 
$\left\{
\begin{bmatrix}
-2 \\
1 \\
0
\end{bmatrix},
\begin{bmatrix}
1 \\
0 \\
1
\end{bmatrix}
\right\}$, 
while a basis for $\mathcal{S}_6$ is given by 
$\left\{\begin{bmatrix}
0 \\
1 \\
-2
\end{bmatrix}\right\}$.

We construct the matrix $P$ by using these basis elements as columns.
\begin{equation*}
P=\begin{bmatrix}
-2 & 1 & 0 \\
1 & 0 & 1 \\
0 & 1 & -2
\end{bmatrix}
\end{equation*}
You can verify (See Practice Problem ) that
\begin{equation*}
P^{-1}=\begin{bmatrix}
-1/4 & 1/2 & 1/4 \\
1/2 & 1 & 1/2 \\
1/4 & 1/2 & -1/4
\end{bmatrix}
\end{equation*}
Thus,
\begin{eqnarray*}
P^{-1}AP &=&\begin{bmatrix}
-1/4 & 1/2 & 1/4 \\
1/2 & 1 & 1/2 \\
1/4 & 1/2 & -1/4
\end{bmatrix} \begin{bmatrix}
2 & 0 & 0 \\
1 & 4 & -1 \\
-2 & -4 & 4
\end{bmatrix} \begin{bmatrix}
-2 & 1 & 0 \\
1 & 0 & 1 \\
0 & 1 & -2
\end{bmatrix} \\
&=&\begin{bmatrix}
2 & 0 & 0 \\
0 & 2 & 0 \\
0 & 0 & 6
\end{bmatrix}
\end{eqnarray*}
You can see that the result here is a diagonal matrix where the entries on the main diagonal are the eigenvalues of $A$. Notice that eigenvalues on the main diagonal {\it must} be in the same order as the corresponding eigenvectors in $P$.
\end{explanation}
\end{example}

%It is often easier to work with matrices that are diagonalizable, as the next Exploration demonstrates.  

\begin{exploration}\label{exp:motivate_diagonalization}
We will continue to use matrices from Example \ref{ex:diagonalizematrix} to illustrate why diagonalizable matrices are easier to work with.  Let $A=\begin{bmatrix}
2 & 0 & 0 \\
1 & 4 & -1 \\
-2 & -4 & 4
\end{bmatrix}$ and let $D=\begin{bmatrix}
2 & 0 & 0 \\
0 & 2 & 0 \\
0 & 0 & 6
\end{bmatrix}$.  
Would it be easier to compute $A^5$ or $D^5$ if you had to do so by hand, without a computer?  Certainly $D^5$ is easier, due to the number of zero entries!  %Let's use Octave to compute both.

%To use Octave, go to the \href{https://sagecell.sagemath.org/}{Sage Math Cell Webpage}, copy the code below into the cell, select OCTAVE as the language, and press EVALUATE.

% \begin{verbatim}
% % Define matrices A and D 
% A=[2 0 0; 1 4 -1; -2 -4 4]  
% D=diag([2 2 6])  

% % Compare computing times for raising A and D to 
% % the fifth power.
% tic
% HARD = A^5  
% toc

% tic
% EZ = D^5
% toc
% \end{verbatim}

%\href{https://sagecell.sagemath.org/?z=eJw1jbsKgzAYhfdA3uEsQjtYVLSLOISm0Llji0KIif6DRmJKX7-J0OnAd24ZpLG0GiwqeNJmh4BaR0hwJrp3hQJFixI18rJFXiGvUfeIruxGUtMpRipc-3NCnGW4uWVT3kBH_QRaJwRa4qx1Hl7Rnsj_IjikSpgNLNkwY3Nf4y-cBdKcPcRTooMYmrQdnE4Hh3N_RS6H5qA_pDczlw==&lang=octave&interacts=eJyLjgUAARUAuQ==}{Link to code}.

Raising a diagonal matrix to a power amounts to simply raising each entry to that same power, whereas computing $A^5$ requires many more calculations.  However, we learned in Example~\ref{ex:diagonalizematrix} that $A$ is similar to $D$.  We will use this to make our computation easier as follows
\begin{align*}
    A^5&=\left(PDP^{-1}\right)^5 \\
       &=(PDP^{-1})(PDP^{-1})(PDP^{-1})(PDP^{-1})(PDP^{-1}) \\
       &=PD(P^{-1}P)D(P^{-1}P)D(P^{-1}P)D(P^{-1}P)DP^{-1} \\
       &=PD(I)D(I)D(I)D(I)DP^{-1} \\
       &=PD^5P^{-1}
\end{align*}
With this in mind, it is not as daunting to calculate $A^5$ by hand.  

%We can compute the product $PD^5$ quite easily since $D^5$ is diagonal, as we learned in Exploration~\ref{init:multiplydiag}.  That leaves just one product of $3 \times 3$ matrices to compute by hand to find $A^5$.  And t

$$A^5=\begin{bmatrix}
-2 & 1 & 0 \\
1 & 0 & 1 \\
0 & 1 & -2
\end{bmatrix}\begin{bmatrix}32 & 0 & 0\\0 &32 & 0\\0 & 0 & 7776\end{bmatrix}\begin{bmatrix}
-1/4 & 1/2 & 1/4 \\
1/2 & 1 & 1/2 \\
1/4 & 1/2 & -1/4
\end{bmatrix}=\begin{bmatrix}32 & 0 & 0\\
 1936 & 3904 & -1936\\
 -3872 & -7744 & 3904\end{bmatrix}$$
The savings in work would certainly be more pronounced for larger matrices or for powers larger that 5.

%To use Octave, go to the \href{https://sagecell.sagemath.org/}{Sage Math Cell Webpage}, copy the code below into the cell, select OCTAVE as the language, and press EVALUATE.

% \begin{verbatim}
% A=[2 0 0; 1 4 -1; -2 -4 4]
% D=diag([2 2 6])
% P=[-2 1 0; 1 0 1; 0 1 -2]
% EZ = D^5
% P*EZ*P^{-1}
% HARD = A^5
% \end{verbatim}
\end{exploration}  

In Exploration~\ref{exp:motivate_diagonalization}, because matrix $A$ was diagonalizable, we were able to cut down on computations.  When we chose to work with $D$ and $P$ instead of $A$ we worked with the eigenvalues and eigenvectors of $A$.  Each column of $P$ is an eigenvector of $A$, and so we repeatedly made use of the following theorem (with $m=5$).

\begin{theorem}\label{th:eigpowers}
Let $A$ be an $n \times n$ matrix and suppose $A\vec{x}=\lambda \vec{x}$.  Then $A^m \vec{x} = \lambda^m \vec{x}$
\end{theorem}

\begin{proof}
We prove this theorem by induction on $m$.  Clearly $A^m \vec{x} = \lambda^m \vec{x}$ holds when $m=1$, as that was given.  For the inductive step, suppose that we know $A^{m-1} \vec{x} = \lambda^{m-1} \vec{x}$.  Then
\begin{align*}
    A^m \vec{x} &= (A A^{m-1}) \vec{x} \\
          &= A (A^{m-1} \vec{x}) \\
          &= A (\lambda^{m-1} \vec{x}) \\
          &= \lambda^{m-1} A\vec{x} \\
          &= \lambda^{m-1} \lambda \vec{x} \\
          &= \lambda^m \vec{x}
\end{align*}
as desired.
\end{proof}

Matrix $A$ from the Example \ref{ex:diagonalizematrix} and Exploration \ref{exp:motivate_diagonalization} had a repeated eigenvalue of 2.  The next theorem and corollary show that matrices which have \emph{distinct} eigenvalues (where none are repeated) have desirable properties.

\begin{theorem}\label{th:linindepeigenvectors}
Let $A$ be an $n\times n$ matrix, and suppose that $A$
has distinct eigenvalues $\lambda_1, \lambda_2, \ldots, \lambda_m$.
For each $i$, let $\vec{x}_i$ be a $\lambda_i$-eigenvector of $A$.
Then $\{ \vec{x}_1, \vec{x}_2, \ldots, \vec{x}_m\}$ is
linearly independent.
\end{theorem}

\begin{proof}
We prove this by induction on $m$, the number of vectors in the set. If $m = 1$, then $\{\vec{x}_{1}\}$ is a linearly independent set because $\vec{x}_{1} \neq \vec{0}$. In general, suppose we have established that the theorem is true for some $m \geq 1$. Given eigenvectors $\{\vec{x}_{1}, \vec{x}_{2}, \dots, \vec{x}_{m+1}\}$, suppose
\begin{equation}
\label{eq:thm_proof_5_5_4_1}
c_1\vec{x}_1 + c_2\vec{x}_2 + \dots + c_{m+1}\vec{x}_{m+1} = \vec{0}.
\end{equation}
We must show that each $c_{i} = 0$. Multiply both sides of (\ref{eq:thm_proof_5_5_4_1}) on the left by $A$ and use the fact that $A\vec{x}_{i} = \lambda_{i}\vec{x}_{i}$ to get
\begin{equation}
\label{eq:thm_proof_5_5_4_2}
c_1\lambda_1\vec{x}_1 + c_2\lambda_2\vec{x}_2 + \dots + c_{m+1}\lambda_{m+1}\vec{x}_{m+1} = \vec{0}.
\end{equation}
If we multiply (\ref{eq:thm_proof_5_5_4_1}) by $\lambda_{1}$ and subtract the result from (\ref{eq:thm_proof_5_5_4_2}), the first terms cancel and we obtain
\begin{equation*}
c_2(\lambda_2 - \lambda_1)\vec{x}_2 + c_3(\lambda_3 - \lambda_1)\vec{x}_3 + \dots + c_{m+1}(\lambda_{m+1} - \lambda_1)\vec{x}_{m+1} = \vec{0}.
\end{equation*}
Since $\vec{x}_{2}, \vec{x}_{3}, \dots, \vec{x}_{m+1}$
correspond to distinct eigenvalues $\lambda_{2}, \lambda_{3}, \dots, \lambda_{m+1}$, the set $\{\vec{x}_{2}, \vec{x}_{3}, \dots, \vec{x}_{m+1}\}$ is linearly independent by the induction hypothesis. Hence,
\begin{equation*}
c_2(\lambda_2 - \lambda_1) = 0, \quad c_3(\lambda_3 - \lambda_1) = 0, \quad \dots, \quad c_{m+1}(\lambda_{m+1} - \lambda_1) = 0
\end{equation*}
and so $c_{2} = c_{3} = \dots = c_{m+1} = 0$ because the $\lambda_{i}$ are distinct. It follows that (\ref{eq:thm_proof_5_5_4_1}) becomes $c_{1}\vec{x}_{1} = \vec{0}$, which implies that $c_{1} = 0$ because $\vec{x}_{1} \neq \vec{0}$, and the proof is complete. 
\end{proof}

The corollary that follows from this theorem gives a useful tool in determining if $A$ is diagonalizable.

\begin{corollary}\label{th:distincteigenvalues}
Let $A$ be an $n \times n$ matrix and suppose it has $n$ distinct eigenvalues. Then it follows that $A$ is diagonalizable.
\end{corollary}

%The corollary tells us that many matrices can be diagonalized in this way.
\begin{remark}
    Note that Corollary \ref{th:distincteigenvalues} is NOT an ``if and only if statement".  This means that if $A$ has repeated eigenvalues it is still sometimes possible to diagonalize $A$, as seen in Example \ref{ex:diagonalizematrix}.
\end{remark}

\begin{definition}\label{def:eigdecomposition}
If we are able to diagonalize $A$, say $A=PDP^{-1}$, we say that $PDP^{-1}$ is an \dfn{eigenvalue decomposition} of $A$.
\end{definition}
Not every matrix has an eigenvalue decomposition. Sometimes we cannot find an invertible matrix $P$ such that $P^{-1}AP=D$.  Consider the following example.

\begin{example}\label{ex:impossiblediagonalize}
Let
\begin{equation*}
A =
\begin{bmatrix}
1 & 1 \\
0 & 1
\end{bmatrix}
\end{equation*}
If possible, find an invertible matrix $P$ and a diagonal matrix $D$ so that $P^{-1}AP=D$.


\begin{explanation}
We see immediately (how?) that the eigenvalues of $A$ are $\lambda_1 =1$ and  $\lambda_2=1$.
To find $P$, the next step would be to find a basis for the corresponding eigenspace $\mathcal{S}_1$.  We solve the equation $\left( A - \lambda I \right) \vec{x} = \vec{0}$.
Writing this equation as an augmented matrix, we already have a matrix in row echelon form:
\begin{equation*}
\left[\begin{array}{cc|c}  
0 & -1 & 0 \\
0 & 0 & 0
 \end{array}\right]
\end{equation*}
We see that the eigenvectors in $\mathcal{S}_1$ are of the form
$$
\begin{bmatrix}
t \\
0
\end{bmatrix}
=t\begin{bmatrix}
1 \\
0
\end{bmatrix},
$$
so a basis for the eigenspace $\mathcal{S}_1$ is given by 
$\begin{bmatrix}
1 \\
0
\end{bmatrix}$.
It is easy to see that we cannot form an invertible matrix $P$, because any two eigenvectors will be of the form 
$\begin{bmatrix}
t \\
0
\end{bmatrix}$,
and so the second row of $P$ would have a row of zeros, and $P$ could not be invertible.  Hence $A$ cannot be diagonalized.
\end{explanation}
\end{example}
We saw earlier in Corollary \ref{th:distincteigenvalues} that an $n \times n$ matrix with $n$ distinct eigenvalues is diagonalizable.  It turns out that there are other useful diagonalizability tests.

Recall that the \dfn{algebraic multiplicity} of an eigenvalue $\lambda$ is the number of times that it occurs as a root of the characteristic polynomial.

\begin{definition}\label{def:geommulteig}
The \dfn{geometric multiplicity} of an eigenvalue $\lambda$ is the dimension of the corresponding eigenspace $\mathcal{S}_\lambda$.
\end{definition}

Consider now the following lemma.

\begin{lemma}\label{lemma:dimeigenspace}
Let $A$ be an $n\times n$ matrix, and let $\mathcal{S}_{\lambda_1}$ be the eigenspace corresponding to the eigenvalue $\lambda_1$ which has algebraic multiplicity $m$.  Then
$$\mbox{dim}(\mathcal{S}_{\lambda_1})\leq m$$
\end{lemma}

In other words, the geometric multiplicity of an eigenvalue is less than or equal to the algebraic multiplicity of that same eigenvalue.

\begin{proof}
Let $k$ be the geometric multiplicity of $\lambda_1$, i.e., $k=\mbox{dim}(\mathcal{S}_{\lambda_1})$.  Suppose $\left\{\vec{x}_1, \vec{x}_2, \ldots ,\vec{x}_k\right\}$ is a basis for the eigenspace $\mathcal{S}_{\lambda_1}$.  Let $P$ be any invertible matrix having $\vec{x}_1,  \vec{x}_2, \ldots ,\vec{x}_k$ as its first $k$ columns, say 
$$P=\begin{bmatrix}
| & | &  & | & | & & | \\
\vec{x}_1 & \vec{x}_2 & \cdots & \vec{x}_k & \vec{x}_{k+1} & \cdots & \vec{x}_n \\
| & | &  & | & | & & |
\end{bmatrix}.$$  
In block form we may write 
$$P=\begin{bmatrix}
B&C
\end{bmatrix} \quad \text{and} \quad P^{-1}=\begin{bmatrix}
D \\
E
\end{bmatrix}$$
where $B$ is $n \times k$, $C$ is  $n \times (n-k)$, $D$ is $k \times n$, and $E$ is $(n-k) \times n$.  We observe
$I_n = P^{-1}P = \left[\begin{array}{c|c}  
DB & DC \\ \hline
EB & EC
 \end{array}\right] $. 
This implies
$$DB = I_k,\quad DC=O_{k\,\,n-k},\quad EB = O_{n-k\,\,k} \quad\text{ and }\quad EC = I_{n-k}$$
Therefore,
\begin{equation*}
P^{-1}AP=\begin{bmatrix}
D \\
E
\end{bmatrix} 
A
\begin{bmatrix}
B&C
\end{bmatrix} =
\left[\begin{array}{c|c}  
DAB & DAC \\ \hline
EAB & EAC
 \end{array}\right]
\end{equation*}
\begin{equation*}
=  \left[\begin{array}{c|c}  
\lambda_1 DB & DAC \\ \hline
\lambda_1 EB & EAC
 \end{array}\right]  
=  \left[\begin{array}{c|c}  
\lambda_1 I_k & DAC \\ \hline
O & EAC
 \end{array}\right]  
\end{equation*}
We finish the proof by comparing the characteristic polynomials on both sides of this equation, and making use of the fact that similar matrices have the same characteristic polynomials.  
$$\det(A-\lambda I) = \det(P^{-1}AP-\lambda I)=(\lambda_1 - \lambda)^k \det(EAC)$$
We see that the characteristic polynomial of $A$ has $(\lambda_1 - \lambda)^k$ as a factor.  This tells us that algebraic multiplicity of $\lambda_1$ is at least $k$, proving the desired inequality. 
\end{proof}

This result tells us that if $\lambda$ is an eigenvalue of $A$, then
the number of linearly independent $\lambda$-eigenvectors
is never more than the multiplicity of $\lambda$. We now use this fact to provide a useful diagonalizability condition.

\begin{theorem}\label{th:diagonalizability}
Let $A$ be an $n \times n$ matrix $A$. Then $A$ is diagonalizable if and only if for each eigenvalue $\lambda$ of $A$, the algebraic multiplicity of $\lambda$ is equal to the geometric multiplicity of $\lambda$.
\end{theorem}

\begin{proof}
Suppose $A$ is diagonalizable and let $\lambda_1, \ldots, \lambda_t$ be the distinct eigenvalues of $A$, with algebraic multiplicities $m_1, \ldots, m_t$, respectively and geometric multiplicities $k_1, \ldots, k_t$, respectively.  Since $A$ is diagonalizable, Theorem \ref{th:eigenvectorsanddiagonalizable} implies that $ k_1+\cdots+k_t=n$.  By applying Lemma \ref{lemma:dimeigenspace} $t$ times, we have
$$n = k_1+\cdots+k_t \le m_1+\cdots+m_t = n,$$
which is only possible if $k_i=m_i$ for $i=1,\ldots,t$.

Conversely, if the geometric multiplicity equals the algebraic multiplicity of each eigenvalue, then obtaining a basis for each eigenspace yields $n$ eigenvectors.  Applying Theorem \ref{th:linindepeigenvectors}, we know that these $n$ eigenvectors are linearly independent, so Theorem \ref{th:eigenvectorsanddiagonalizable} implies that $A$ is diagonalizable.
\end{proof}

\section*{Practice Problems}
\emph{Problems \ref{prob:ex:diagonalizematrix1}-\ref{prob:ex:diagonalizematrix4}}
In this exercise you will "fill in the details" of Example \ref{ex:diagonalizematrix}.

\begin{problem}\label{prob:ex:diagonalizematrix1}
Find the eigenvalues of matrix 
\begin{equation*}
A=\begin{bmatrix}
2 & 0 & 0 \\
1 & 4 & -1 \\
-2 & -4 & 4
\end{bmatrix}
\end{equation*}
\end{problem} 

\begin{problem}\label{prob:ex:diagonalizematrix2}
Find a basis for each eigenspace of the matrix $A$.
\end{problem}

\begin{problem}\label{prob:ex:diagonalizematrix3}
Compute the inverse of \begin{equation*}
P=
\begin{bmatrix}
-2 & 1 & 0 \\
1 & 0 & 1 \\
0 & 1 & -2
\end{bmatrix}
\end{equation*}
\end{problem}

\begin{problem}\label{prob:ex:diagonalizematrix5}
Compute $D=P^{-1}AP$
\end{problem}

\begin{problem}\label{prob:ex:diagonalizematrix4}
Show that computing the inverse of $P$ is not really necessary by comparing the products  $PD$ and $AP$.
\end{problem}
 
\begin{problem}\label{prb:diagonalizable}
In each case, decide whether the matrix $A$ is diagonalizable. If so, find $P$ such that $P^{-1}AP$ is diagonal.

\begin{enumerate}
\item $\begin{bmatrix}
1 & 0 & 0 \\
1 & 2 & 1 \\
0 & 0 & 1
\end{bmatrix}$
\wordChoice{[correct]\choice{Diagonalizable}\choice{Not Diagonalizable}}
\item $\begin{bmatrix}
3 &  0 & 6 \\
0 & -3 & 0 \\
5 &  0 & 2
\end{bmatrix}$
\wordChoice{[correct]\choice{Diagonalizable}\choice{Not Diagonalizable}}
\item $\begin{bmatrix}
 3 &  1 &  6 \\
 2 &  1 &  0 \\
-1 &  0 & -3
\end{bmatrix}$
\wordChoice{[correct]\choice{Diagonalizable}\choice{Not Diagonalizable}}
\item $\begin{bmatrix}
4 & 0 & 0 \\
0 & 2 & 2 \\
2 & 3 & 1
\end{bmatrix}$
\wordChoice{\choice{Diagonalizable}\choice[correct]{Not Diagonalizable}}
\end{enumerate}
\end{problem}

%\setcounter{enumi}{1}
%problem  Yes, $ P =
%\leftB \begin{array}{rrr}
%-1 & 0 & 6 \\
% 0 & 1 & 0 \\
% 1 & 0 & 5
%\end{array} \rightB$, $P^{-1}AP =
%\leftB \begin{array}{rrr}
%-3 &  0 & 0 \\
% 0 & -3 & 0 \\
% 0 &  0 & 8
%\end{array} \rightB$

%\setcounter{enumi}{3}
%problem  No, $c_{A}(x) = (x + 1)(x - 4)^{2}$ so $\lambda = 4$ has multiplicity 2. But $\func{dim}(E_{4}) = 1$ so Theorem~\ref{thm:016250} applies.


\begin{problem}\label{prb:upper_triangular_case}
Let $A$ denote an $n \times n$ upper triangular matrix.
\begin{enumerate}
\item If all the main diagonal entries of $A$ are distinct, show that $A$ is diagonalizable.

\item If all the main diagonal entries of $A$ are equal, show that $A$ is diagonalizable only if it is \textit{already} diagonal.
Click the arrow to see the answer.
\begin{expandable}
The eigenvalues of $A$ are all equal (they are the diagonal elements), so if $P^{-1}AP = D$ is diagonal, then $D = \lambda I$. Hence $A = P^{-1}(\lambda I)P = \lambda I$.
\end{expandable}

\item Show that $\begin{bmatrix}
1 & 0 & 1 \\
0 & 1 & 0 \\
0 & 0 & 2
\end{bmatrix}$ is diagonalizable but that $\begin{bmatrix}
 1 & 1 & 0 \\
 0 & 1 & 0 \\
 0 & 0 & 2
 \end{bmatrix}$ is not diagonalizable.
 
\end{enumerate}

\end{problem}

\begin{problem}\label{prb:det_and_tr_diagonalizable}
Let $A$ be a diagonalizable $n \times n$ matrix with eigenvalues $\lambda_{1}, \lambda_{2}, \dots, \lambda_{n}$ (including multiplicities). Show that:

\begin{enumerate}
\item $\det A = \lambda_{1}\lambda_{2}\cdots \lambda_{n}$
\item $\mbox{tr} A = \lambda_{1} + \lambda_{2} + \cdots + \lambda_{n}$
\end{enumerate}
\end{problem}

\begin{problem}\label{prb:A_sim_A^T_diagonalizable}
\begin{enumerate}

\item Show that two diagonalizable matrices are similar if and only if they have the same eigenvalues with the same multiplicities.

\item If $A$ is diagonalizable, show that $A \sim A^{T}$.

\item Show that $A \sim A^{T}$ if
 $A = \begin{bmatrix}
 1 & 1 \\
 0 & 1
 \end{bmatrix}$

\end{enumerate}
\end{problem}

\section*{Text Source}
The text in this section is a compilation of material from Section 7.2.1 of Ken Kuttler's \href{https://open.umn.edu/opentextbooks/textbooks/a-first-course-in-linear-algebra-2017}{\it A First Course in Linear Algebra} (CC-BY) and Section 5.5 of Keith Nicholson's \href{https://open.umn.edu/opentextbooks/textbooks/linear-algebra-with-applications}{\it Linear Algebra with Applications} (CC-BY-NC-SA).

Ken Kuttler, {\it  A First Course in Linear Algebra}, Lyryx 2017, Open Edition, p. 362-364.

Many of the Practice Problems are Exercises from 
W. Keith Nicholson, {\it Linear Algebra with Applications}, Lyryx 2018, Open Edition, pp. 298-310.

\end{document}
