\documentclass{ximera}
%% You can put user macros here
%% However, you cannot make new environments

\listfiles

\graphicspath{{./}{firstExample/}{secondExample/}}

\usepackage{tikz}
\usepackage{tkz-euclide}
\usepackage{tikz-3dplot}
\usepackage{tikz-cd}
\usetikzlibrary{shapes.geometric}
\usetikzlibrary{arrows}
%\usetkzobj{all}
\pgfplotsset{compat=1.13} % prevents compile error.

%\renewcommand{\vec}[1]{\mathbf{#1}}
\renewcommand{\vec}{\mathbf}
\newcommand{\RR}{\mathbb{R}}
\newcommand{\dfn}{\textit}
\newcommand{\dotp}{\cdot}
\newcommand{\id}{\text{id}}
\newcommand\norm[1]{\left\lVert#1\right\rVert}
 
\newtheorem{general}{Generalization}
\newtheorem{initprob}{Exploration Problem}

\tikzstyle geometryDiagrams=[ultra thick,color=blue!50!black]

%\DefineVerbatimEnvironment{octave}{Verbatim}{numbers=left,frame=lines,label=Octave,labelposition=topline}



\usepackage{mathtools}


\title{Octave Problems for Ch 4} \license{CC BY-NC-SA 4.0}

\begin{document}

\begin{abstract}
\end{abstract}
\maketitle

\section*{Octave Problems for Chapter 4}

To access Octave cells online, go to the \href{https://sagecell.sagemath.org/}{Sage Math Cell Webpage}, select OCTAVE as the language, enter your code, and press EVALUATE.  
To ''save" or share your code, click on the \emph{Share} button, select \emph{Permalink}, then copy the address directly from the browser window.  You can store this link to access your work later or share this link with others.  You will need to get a new Permalink every time you modify the code.

%Refer to \href{https://ximera.osu.edu/linearalgebrav3/XOctaveTutorial/octave/systems}{Octave Tutorial for Solving Systems of Equations}.

\begin{problem}\label{prb:5.1} Let $H = \mbox{span}\left\{ \left[
\begin{array}{r}
2 \\
1 \\
1 \\
1
\end{array}
\right] ,\left[
\begin{array}{r}
-1 \\
0 \\
-1 \\
-1
\end{array}
\right] ,\left[
\begin{array}{r}
5 \\
2 \\
3 \\
3
\end{array}
\right] ,\left[
\begin{array}{r}
-1 \\
1 \\
-2 \\
-2
\end{array}
\right] \right\} .$ Find the dimension of $H$ and determine a basis.

\begin{hint}
    \begin{verbatim}
% Find a basis for column vectors
v1=transpose([2 1 1 1]);
v2=transpose([-1 0 -1 -1]);
v3=transpose([5 2 3 3]);
v4=transpose([-1 1 -2 -2]);
A=[v1 v2 v3 v4];
% We can answer all of these questions by performing Gauss-Jordan elimination.
rref(A)
\end{verbatim}
\end{hint}
\end{problem}

\begin{problem}\label{prb:5.3} Let $H$ denote $\mbox{span}\left\{ \left[
\begin{array}{r}
-2 \\
1 \\
1 \\
-3
\end{array}
\right] ,\left[
\begin{array}{r}
-9 \\
4 \\
3 \\
-9
\end{array}
\right] ,\left[
\begin{array}{r}
-33 \\
15 \\
12 \\
-36
\end{array}
\right] ,\left[
\begin{array}{r}
-22 \\
10 \\
8 \\
-24
\end{array}
\right] \right\} .$ Find the dimension of $H$ and determine a basis.

\end{problem}

\begin{problem}\label{prb:5.5} Let $H$ denote $\mbox{span}\left\{ \left[
\begin{array}{r}
2 \\
3 \\
2 \\
1
\end{array}
\right] ,\left[
\begin{array}{r}
8 \\
15 \\
6 \\
3
\end{array}
\right] ,\left[
\begin{array}{r}
3 \\
6 \\
2 \\
1
\end{array}
\right] ,\left[
\begin{array}{r}
4 \\
6 \\
6 \\
3
\end{array}
\right] ,\left[
\begin{array}{r}
8 \\
15 \\
6 \\
3
\end{array}
\right] \right\} .$ Find the dimension of $H$ and determine a basis.

\end{problem}

\begin{problem}\label{prb:5.32} Find the rank of the following matrix. Also find a basis for the row
and column spaces.

\begin{equation*}
\left[
\begin{array}{rrrrrr}
1 & 3 & 0 & -2 & 0 & 3 \\
3 & 9 & 1 & -7 & 0 & 8 \\
1 & 3 & 1 & -3 & 1 & -1 \\
1 & 3 & -1 & -1 & -2 & 10
\end{array}
\right]
\end{equation*}

\begin{hint}
    \begin{verbatim}
% Find a basis for row(A) and column(A)
A=[1 3 0 -2 0 3; 
3 9 1 -7 0 8;
1 3 1 -3 1 -1;
1 3 -1 -1 -2 10]
% We can answer all of these questions by finding rref(A).
rref(A)
\end{verbatim}
\end{hint}
\end{problem}

\begin{problem}\label{prb:5.33} Find the rank of the following matrix. Also find a basis for the row
and column spaces.
\begin{equation*}
\left[
\begin{array}{rrrrrr}
1 & 3 & 0 & -2 & 7 & 3 \\
3 & 9 & 1 & -7 & 23 & 8 \\
1 & 3 & 1 & -3 & 9 & 2 \\
1 & 3 & -1 & -1 & 5 & 4
\end{array}
\right]
\end{equation*}
%Click the arrow to see the answer.  \begin{expandable}
%\end{expandable}
\end{problem}

\begin{problem}\label{prb:5.34} Find the rank of the following matrix. Also find a basis for the row
and column spaces.
\begin{equation*}
\left[
\begin{array}{rrrrrr}
1 & 0 & 3 & 0 & 7 & 0 \\
3 & 1 & 10 & 0 & 23 & 0 \\
1 & 1 & 4 & 1 & 7 & 0 \\
1 & -1 & 2 & -2 & 9 & 1
\end{array}
\right]
\end{equation*}

\end{problem}

\begin{problem}\label{prb:5.35} Find the rank of the following matrix. Also find a basis for the row and column spaces.
\begin{equation*}
\left[
\begin{array}{rrr}
1 & 0 & 3 \\
3 & 1 & 10 \\
1 & 1 & 4 \\
1 & -1 & 2
\end{array}
\right]
\end{equation*}

\end{problem}

\begin{problem}\label{prob_oct_Rn1}
    In Template \ref{temp:colSpace} we used the \emph{null} command to find the null of matrix $$A=\begin{bmatrix}2&-1&1&-4&1\\1&0&3&3&0\\-2&1&-1&5&2\\4&-1&7&2&1\end{bmatrix}$$ 

    \begin{enumerate}
        \item We followed up by verifying that the basis for the null space that Octave found are orthonormal.  Verify that the product of $A$ with of each element of the basis is $\vec{0}$.
        \item Use $\text{rref}(A)$ to find another basis for the null space.  
        \item The basis we found using the \emph{null} command and the basis you found in part (b) look very different.  Verify that they span the same subspace.
    \end{enumerate}
\end{problem}

\begin{problem}\label{prob_oct_Rn2}
    Use some of the earlier matrices, or create your own examples to explore the relationship between the vectors of the null space of a matrix and the rows of the matrix.  Formulate a conjecture about the relationship and prove it.
\end{problem}

\section*{Bibliography}
The Review Exercises come from the end of Chapter 4 of Ken Kuttler's \href{https://open.umn.edu/opentextbooks/textbooks/a-first-course-in-linear-algebra-2017}{\it A First Course in Linear Algebra}. (CC-BY)

Ken Kuttler, {\it  A First Course in Linear Algebra}, Lyryx 2017, Open Edition, pp. 227--232.  


\end{document}