\documentclass{ximera}
%% You can put user macros here
%% However, you cannot make new environments

\listfiles

\graphicspath{{./}{firstExample/}{secondExample/}}

\usepackage{tikz}
\usepackage{tkz-euclide}
\usepackage{tikz-3dplot}
\usepackage{tikz-cd}
\usetikzlibrary{shapes.geometric}
\usetikzlibrary{arrows}
%\usetkzobj{all}
\pgfplotsset{compat=1.13} % prevents compile error.

%\renewcommand{\vec}[1]{\mathbf{#1}}
\renewcommand{\vec}{\mathbf}
\newcommand{\RR}{\mathbb{R}}
\newcommand{\dfn}{\textit}
\newcommand{\dotp}{\cdot}
\newcommand{\id}{\text{id}}
\newcommand\norm[1]{\left\lVert#1\right\rVert}
 
\newtheorem{general}{Generalization}
\newtheorem{initprob}{Exploration Problem}

\tikzstyle geometryDiagrams=[ultra thick,color=blue!50!black]

%\DefineVerbatimEnvironment{octave}{Verbatim}{numbers=left,frame=lines,label=Octave,labelposition=topline}



\usepackage{mathtools}


\title{Octave for Chapters 5-6} \license{CC BY-NC-SA 4.0}

\begin{document}

\begin{abstract}
\end{abstract}
\maketitle

\section*{Octave for Chapters 5-6}

The templates in this section provide sample Octave code for finding subspaces associated with matrices as well as the kernel and image of a linear transformation. You can access our code through the link at the bottom of each template.  Feel free to modify the code and experiment to learn more!  

You can write your own code using Octave software or online Octave cells.  To access Octave cells online, go to the \href{https://sagecell.sagemath.org/}{Sage Math Cell Webpage}, select OCTAVE as the language, enter your code, and press EVALUATE.  

To ''save" or share your online code, click on the \emph{Share} button, select \emph{Permalink}, then copy the address directly from the browser window.  You can store this link to access your work later or share this link with others.  You will need to get a new Permalink every time you modify the code.

\subsection*{Octave Tutorial}
\subsubsection*{Column Space of a Matrix (Image of a Linear Transformation)}
Recall that the column space of a matrix is the span of the columns of the matrix.  We have the following procedure for finding the column space.

\begin{procedure}[\ref{proc:colspace}]
Given a matrix $B$, a basis for $\mbox{col}(B)$ can be found as follows:
\begin{enumerate}
\item Find $\mbox{rref}(B)$ (or any row-echelon form $B'$ of $B$.)
\item Identify the pivot columns of $\mbox{rref}(B)$ (or $B'$).
\item The columns of $B$ corresponding to the pivot columns of $\mbox{rref}(B)$ (or $B'$) form a basis for $\mbox{col}(B)$.
\end{enumerate}
\end{procedure}

We will implement this procedure in Octave.

\begin{template}\label{temp:colSpace}
We will use the matrix from Example \ref{ex:basiscolspace}.  Let $$A=\begin{bmatrix}2&-1&1&-4&1\\1&0&3&3&0\\-2&1&-1&5&2\\4&-1&7&2&1\end{bmatrix}$$
We can use Octave to find the reduced row-echelon form of $A$ (\emph{rref} command), and identify the pivot columns visually.  We can also take advantage of the expanded version of the \emph{rref} command to identify the pivot columns automatically, as shown below.

\begin{verbatim}
% Define A
A = [2 -1 1 -4 1;
    1 0 3 3 0;
    -2 1 -1 5 2;
    4 -1 7 2 1];

% Compute rref of A
[rref_A, pivot_cols] = rref(A);

% the pivot column indices are...
pivot_cols

% Extract the pivot columns from A, using pivot column indices
colSpace_basis = A(:, pivot_cols)
\end{verbatim}

\href{https://sagecell.sagemath.org/?z=eJxtTkEKwjAQvAfyh7kULNhiqiIoHoL6Ao9SSq2JBmxTklR8vhssKOjuZWeGmdkEe6VNpyA5k9jiVCATEMgWEBvOQCMww5x2NuKsiLrAEsXILCJcgfiSGM4S7GzbD0HBOaVhdUw_xbuSU_TmYUPV2LsvqTCyE5mOvnBTbx2kD20H011Mozxqp_I85-xjfhsOz-DqJvwYPbSzLahu8Ka7_g3ljPCxrxtVnWtvPH0jJ-vvB9MX3FlRVA==&lang=octave&interacts=eJyLjgUAARUAuQ==}{Link to code}.

Compare these findings to our answer in Example \ref{ex:basiscolspace}.
\end{template}  

When you study linear transformations, you will learn that the column space of a matrix is the image of the linear transformation induced by the matrix.  So, Template \ref{temp:colSpace} can be used to find the image of a linear transformation.

\subsubsection*{Null Space of a Matrix (Kernel of a Linear Transformation)}

Recall the following definition of the null space of a matrix.
\begin{definition}[\ref{def:nullspace}] Let $A$ be an $m\times n$ matrix.  The \dfn{null space} of $A$, denoted by $\mbox{null}(A)$, is the set of all vectors $\vec{x}$ in $\RR^n$ such that $A\vec{x}=\vec{0}$.
\end{definition}

We can use the reduced row-echelon form (\emph{rref} command) to find the null space, as shown in Example \ref{ex:dimnull}.  We will omit the details here.

As you work with Octave, you may come across the \emph{null} command.  This command finds an \emph{orthonormal} basis for the null space.  An orthonormal set of vectors consists of unit vectors that are pairwise orthogonal.  We demonstrate the \emph{null} function below.

\begin{template}\label{temp:nullSpace}
    We will use the matrix from Example \ref{ex:dimnull}.  Let 
$$A=\begin{bmatrix}2&-1&1&-4&1\\1&0&3&3&0\\-2&1&-1&5&2\\4&-1&7&2&1\end{bmatrix}$$
Find $\mbox{null}(A)$.

\begin{verbatim}
% Define A
A = [2 -1 1 -4 1;
    1 0 3 3 0;
    -2 1 -1 5 2;
    4 -1 7 2 1];

% Basis for null(A)
% Octave constructs an orthonormal basis, meaning that the basis elements are pairwise orthogonal unit vectors.
B=null(A)

% To find the dimension of the null space, find the number of columns in B
[rows, cols]=size(B)
\end{verbatim}

\href{https://sagecell.sagemath.org/?z=eJxFj0FrwzAMhe-B_Id3KbTQjCbr2GHkkLD7LruVHlxXaQ2JVCynhf36yW3ZLDDo83t68gKfNAQmdGXRocWuQVWjRrVF_VEWsFNjg1erzbOvmvxe4w3Nk2xz-w7jeyNlsUDvNCgGieB5HJfdKsMvn9yV4IU1xdknhWNITGdhiZMbcciuNSZyHPiEdHbJLnpw0EgTcXZFwsWFeAtKD_9J2Owzh4Qr-SRRX8qib_-yc_q3wD56vA88BpukQSx-uIOshF6cp_W_iufpQDFLvIzzxIrA6MtiF-VmaxrUfavhh5b96hc072Hf&lang=octave&interacts=eJyLjgUAARUAuQ==}{Link to code}.
\end{template}

In Template \ref{temp:nullSpace} we took it for granted that the \emph{null} command returned an orthonormal basis for the null space.  In the next example we will verify that $B$ is indeed an orthonormal collection.  In Problem \ref{prob_oct_Rn1} you will verify that $B$ spans the null space of $A$.

\begin{example}\label{ex_oct_orth}
    Verify that $B$ in Template \ref{temp:nullSpace} is an orthonormal set.
    \begin{explanation}
    We will replicate the above code, then check that elements of $B$ are unit vectors and that they are pairwise orthogonal.
        \begin{verbatim}
            % Define A
            A = [2 -1 1 -4 1;
                1 0 3 3 0;
                -2 1 -1 5 2;
                4 -1 7 2 1];

            % Basis for null(A)
            % Octave constructs an orthonormal basis, meaning that the basis elements are pairwise orthogonal unit vectors.
            B=null(A)

            % To find the dimension of the null space, find the number of columns in B
            [rows, cols]=size(B)

            % Verify that the columns of B are unit vectors
            for i=1:cols
                NormOfCol=norm(B(:,i))
            end    

            % Verify that the columns of B are pairwise orthogonal (if more than one column)
            if cols==1
                disp('The dimension of the null space is 1.');
            else    
                for i=1:cols
                    for j=i+1:cols
                            DotProduct=dot(B(:,i),B(:,j))
                    end
                end
            end
        \end{verbatim}

    \href{https://sagecell.sagemath.org/?z=eJyNUk1vwjAMvVfqf_AFUbSCCNs0iSkHOs5jB7QL4lBaF4zaBCUpaPv1c_oxpm2a5kpp8-JnP790AEssSCEswmABEjYzGAsQML4D8RgGwCFgCrf8TLv9eObPBdzDrEPu_PYBGN8yEgYDSFJLFgptQNVlGS1GHlxlLj0jZFpZZ-rMWUgVaOMOWmlTpSXsPCuGClNFag_ukDpesMUBS6xQeZZBOKVkLmSx5e-1YnqtyMEZM6eNnYRBIj97--5rDTxo3hTMiStZ0ty-aACfCfaUZhhfs1Rd7dD4lEyXdaUskIIkDDZGX1gmg3YrLb1jlHQ9XtFQ8XYV3vO4RNLI_ioxDLw_JMXcV2qdfGYjVsWTLqW3JEqieUwjLo4sieOfXX4zJ6ICKs2HTOOxVU_j4tQMaKUUrYic7Ckarv_2CfhGxGQ44gvHkls16vz6c6gePUq6-Y73sdTuxeic_wqZa9cNHvvX0c_fp7EP7ab54OUDMZzNzg==&lang=octave&interacts=eJyLjgUAARUAuQ==}{Link to code}.
    \end{explanation}
\end{example}

If you are familiar with linear transformations, you may recall that the \emph{kernel} of a linear transformation induced by a matrix is the null space of the matrix.  The \emph{null} function gives us a way to compute the kernel.

\subsection*{Octave Exercises}
For Problems \ref{prb:5.1} and \ref{prb:5.3} use Octave to find $\text{col}(A)$, $\text{ker}(A)$, and their respective dimensions.  Verify that the Rank-Nullity Theorem (\ref{th:matrixranknullity}) holds for $A$.

\begin{problem}\label{prb:5.1} Let 
    $$A = 
\begin{bmatrix}
2 & -1 & 5 & -1\\
1 & 0 &2 & 1 \\
1 & -1 & 3 & -2\\
1 & -1 & 3 & -2
\end{bmatrix}$$ 
\end{problem}

\begin{problem}\label{prb:5.3} Let $$A=
\begin{bmatrix}
-2 & -9 & -33 & -22\\
0 & 4 & 15 & 10\\
1 & 3 & 12 & 10\\
-3 & -9 &-36 & -24
\end{bmatrix}$$ 
\end{problem}

\begin{problem}\label{prb:5.5} Let $$H=\mbox{span}\left\{ \left[
\begin{array}{r}
2 \\
3 \\
2 \\
1
\end{array}
\right] ,\left[
\begin{array}{r}
8 \\
15 \\
6 \\
3
\end{array}
\right] ,\left[
\begin{array}{r}
3 \\
6 \\
2 \\
1
\end{array}
\right] ,\left[
\begin{array}{r}
4 \\
6 \\
6 \\
3
\end{array}
\right] ,\left[
\begin{array}{r}
8 \\
15 \\
6 \\
3
\end{array}
\right] \right\} $$ 
Find the dimension of $H$ and determine a basis.

\end{problem}

\begin{problem}\label{prb:5.32} Find the rank of the following matrix. Also find a basis for the row
and column spaces.


$$\begin{bmatrix}
1 & 3 & 0 & -2 & 0 & 3 \\
3 & 9 & 1 & -7 & 0 & 8 \\
1 & 3 & 1 & -3 & 1 & -1 \\
1 & 3 & -1 & -1 & -2 & 10
\end{bmatrix}$$

\begin{hint}
    \begin{verbatim}
% Find a basis for row(A) and column(A)
A=[1 3 0 -2 0 3; 
3 9 1 -7 0 8;
1 3 1 -3 1 -1;
1 3 -1 -1 -2 10]
% We can answer all of these questions by finding rref(A).
rref(A)
\end{verbatim}
\end{hint}
\end{problem}

\begin{problem}\label{prb:5.33} Find the rank of the following matrix. Also find a basis for the row
and column spaces.

$$\begin{bmatrix}
1 & 3 & 0 & -2 & 7 & 3 \\
3 & 9 & 1 & -7 & 23 & 8 \\
1 & 3 & 1 & -3 & 9 & 2 \\
1 & 3 & -1 & -1 & 5 & 4
\end{bmatrix}$$
\end{problem}

\begin{problem}\label{prb:5.34} Find the rank of the following matrix. Also find a basis for the row
and column spaces.

$$\begin{bmatrix}
1 & 0 & 3 & 0 & 7 & 0 \\
3 & 1 & 10 & 0 & 23 & 0 \\
1 & 1 & 4 & 1 & 7 & 0 \\
1 & -1 & 2 & -2 & 9 & 1
\end{bmatrix}$$
\end{problem}

% \begin{problem}\label{prb:5.35} Find the rank of the following matrix. Also find a basis for the row and column spaces.
% \begin{equation*}
% \left[
% \begin{array}{rrr}
% 1 & 0 & 3 \\
% 3 & 1 & 10 \\
% 1 & 1 & 4 \\
% 1 & -1 & 2
% \end{array}
% \right]
% \end{equation*}

% \end{problem}

\begin{problem}\label{prob_oct_Rn1}
    In Template \ref{temp:nullSpace} we used the \emph{null} command to find the null of matrix $$A=\begin{bmatrix}2&-1&1&-4&1\\1&0&3&3&0\\-2&1&-1&5&2\\4&-1&7&2&1\end{bmatrix}$$ 

    \begin{enumerate}
        \item We followed up by verifying that the basis for the null space found using the \emph{null} function are orthonormal.  Verify that the product of $A$ with of each element of the basis is $\vec{0}$.
        \item Use $\text{rref}(A)$ to find another basis for the null space.  
        \item The basis we found using the \emph{null} command and the basis you found in part (b) look very different.  Verify that they span the same subspace.
    \end{enumerate}
\end{problem}

\begin{problem}\label{prob_oct_Rn2}
    Use some of our earlier matrices, together with their null spaces, for the following experiment.
    \begin{itemize}
        \item Given a matrix, find the product of the matrix with each of the vectors in its null space.  What do you observe?  why does this make sense?
        \item Formulate a conjecture about the relationship between the rows of a matrix and the elements of its null space.
        \item Prove your conjecture.
    \end{itemize}
\end{problem}

\section*{Bibliography}
Matrices used in Problems \ref{prb:5.1}-\ref{prb:5.34} come from the end of Chapter 4 of Ken Kuttler's \href{https://open.umn.edu/opentextbooks/textbooks/a-first-course-in-linear-algebra-2017}{\it A First Course in Linear Algebra}. (CC-BY)

Ken Kuttler, {\it  A First Course in Linear Algebra}, Lyryx 2017, Open Edition, pp. 227--232.  


\end{document}