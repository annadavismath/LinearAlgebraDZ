\documentclass{ximera}
%% You can put user macros here
%% However, you cannot make new environments

\listfiles

\graphicspath{{./}{firstExample/}{secondExample/}}

\usepackage{tikz}
\usepackage{tkz-euclide}
\usepackage{tikz-3dplot}
\usepackage{tikz-cd}
\usetikzlibrary{shapes.geometric}
\usetikzlibrary{arrows}
%\usetkzobj{all}
\pgfplotsset{compat=1.13} % prevents compile error.

%\renewcommand{\vec}[1]{\mathbf{#1}}
\renewcommand{\vec}{\mathbf}
\newcommand{\RR}{\mathbb{R}}
\newcommand{\dfn}{\textit}
\newcommand{\dotp}{\cdot}
\newcommand{\id}{\text{id}}
\newcommand\norm[1]{\left\lVert#1\right\rVert}
 
\newtheorem{general}{Generalization}
\newtheorem{initprob}{Exploration Problem}

\tikzstyle geometryDiagrams=[ultra thick,color=blue!50!black]

%\DefineVerbatimEnvironment{octave}{Verbatim}{numbers=left,frame=lines,label=Octave,labelposition=topline}



\usepackage{mathtools}


\title{Challenge Problems} \license{CC BY-NC-SA 4.0}

\begin{document}

\begin{abstract}
\end{abstract}
\maketitle

\section*{Challenge Problems for Chapter 4}

\begin{problem}\label{prob:4.75}
Solve for the matrix $X$ if:
\begin{enumerate}
\item $PXQ = R$;
\item $XP = S$;
\end{enumerate}
where
$
P = \left[ \begin{array}{rr}
1 & 0 \\
2 & -1 \\
0 & 3
\end{array} \right]$, $
Q = \left[ \begin{array}{rrr}
1 & 1 & -1 \\
2 & 0 & 3
\end{array} \right]$, \\ $
R = \left[ \begin{array}{rrr}
-1 & 1 & -4 \\
-4 & 0 & -6 \\
6 & 6 & -6
\end{array} \right]$, $
S = \left[ \begin{array}{rr}
1 & 6\\
3 & 1
\end{array} \right]$
\end{problem}

\begin{problem}\label{prob:4.76}
Consider \begin{equation*}
p(X) = X^{3} - 5X^{2} + 11X - 4I.
\end{equation*}

\begin{enumerate}
\item If $p(U) = \left[ \begin{array}{rr}
1 & 3 \\
-1 & 0
\end{array} \right]$
 compute $p(U^{T})$.

\item If $p(U) = 0$ where $U$ is $n \times n$, find $U^{-1}$ in terms of $U$.
\end{enumerate}

Click on the arrow to see answer.
\begin{expandable}
 $U^{-1} = \frac{1}{4}(U^{2} - 5U + 11I)$.
\end{expandable}
\end{problem}

\begin{problem}\label{prob:4.78}
Assume that a system $A\vec{x} = \vec{b}$ of linear equations has at least two distinct solutions $\vec{y}$ and $\vec{z}$.

\begin{enumerate}
\item Show that $\vec{x}_{k} = \vec{y} + k(\vec{y} - \vec{z})$ is a solution for every $k$.

\item Show that $\vec{x}_{k} = \vec{x}_{m}$ implies $k = m$. 

\item Deduce that $A\vec{x} = \vec{b}$ has infinitely many solutions.

\end{enumerate}

Click on the arrow to see answer.
\begin{expandable}
\begin{enumerate}
\item  If $\vec{x}_{k} = \vec{x}_{m}$, then $\vec{y} + k(\vec{y} - \vec{z}) = \vec{y} + m(\vec{y} - \vec{z})$. So $(k - m)(\vec{y} - \vec{z}) = \vec{0}$. But $\vec{y} - \vec{z}$ is not zero (because $\vec{y}$ and $\vec{z}$ are distinct), so $k - m = 0$.
\end{enumerate}
\end{expandable}
\end{problem}

\begin{problem}\label{prob:4.81}
Let $A = \left[ \begin{array}{rr}
0 & 1 \\
1 & 0
\end{array} \right]$
\begin{enumerate}
\item 
 show that $A^{2} = I$.
\item What is wrong with the following argument? If $A^{2} = I$, then $A^{2} - I = 0$, so $(A - I)(A + I) = 0$, whence $A = I$ or $A = -I$.
\end{enumerate}
\end{problem}

\begin{problem}\label{prob:4.82}
Let $E$ and $F$ be elementary matrices obtained from the identity matrix by adding multiples of row $k$ to rows $p$ and $q$. If $k \neq p$ and $k \neq q$, show that $EF = FE$.

\end{problem}



\section*{Bibliography}
These exercises come from the end of Chapter 2 of Keith Nicholson's \href{https://open.umn.edu/opentextbooks/textbooks/linear-algebra-with-applications}{\it Linear Algebra with Applications}. (CC-BY-NC-SA)

W. Keith Nicholson, {\it Linear Algebra with Applications}, Lyryx 2018, Open Edition, pp. 143--144. 

\end{document}