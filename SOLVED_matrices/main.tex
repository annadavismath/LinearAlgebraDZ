\documentclass{ximera}
%% You can put user macros here
%% However, you cannot make new environments

\listfiles

\graphicspath{{./}{firstExample/}{secondExample/}}

\usepackage{tikz}
\usepackage{tkz-euclide}
\usepackage{tikz-3dplot}
\usepackage{tikz-cd}
\usetikzlibrary{shapes.geometric}
\usetikzlibrary{arrows}
%\usetkzobj{all}
\pgfplotsset{compat=1.13} % prevents compile error.

%\renewcommand{\vec}[1]{\mathbf{#1}}
\renewcommand{\vec}{\mathbf}
\newcommand{\RR}{\mathbb{R}}
\newcommand{\dfn}{\textit}
\newcommand{\dotp}{\cdot}
\newcommand{\id}{\text{id}}
\newcommand\norm[1]{\left\lVert#1\right\rVert}
 
\newtheorem{general}{Generalization}
\newtheorem{initprob}{Exploration Problem}

\tikzstyle geometryDiagrams=[ultra thick,color=blue!50!black]

%\DefineVerbatimEnvironment{octave}{Verbatim}{numbers=left,frame=lines,label=Octave,labelposition=topline}



\usepackage{mathtools}


\title{Solved Problems for Ch 3} \license{CC BY-NC-SA 4.0}

\begin{document}

\begin{abstract}
\end{abstract}
\maketitle

\section*{Solved Problems for Chapter 3}

\begin{problem}\label{prb:4.5} Using only the properties given in Theorem~\ref{th:propertiesofaddition}
 and Theorem~\ref{th:propertiesscalarmult},
show that the additive inverse of $A$, $-A$, is unique.

Click on the arrow to see answer.
\begin{expandable}
 Suppose $B$ is also an additive inverse of $A$. Then
\[
-A=-A+\left( A+B\right) =\left( -A+A\right) +B=0+B=B
\]
\end{expandable}
\end{problem}

\begin{problem}\label{prb:4.6} Using only the properties given in Theorem~\ref{th:propertiesofaddition}
 and Theorem~\ref{th:propertiesscalarmult},
show that the $n\times m$ zero matrix, $O$, is unique.

Click on the arrow to see answer.
\begin{expandable}
Suppose $O^{\prime }$ is also an $n\times m$ additive identity. Then $O^{\prime }=O^{\prime }+O=O.$
\end{expandable}
\end{problem}

\begin{problem}\label{prb:4.7} Using only the properties given in Theorem~\ref{th:propertiesofaddition}
 and Theorem~\ref{th:propertiesscalarmult}, show that $0A=O.$ 

Click on the arrow to see answer.
\begin{expandable}
$0A=\left( 0+0\right) A=0A+0A.$ Now add $-\left(
0A\right) $ to both sides. Then $O=0A$.
\end{expandable}
\end{problem}

\begin{problem}\label{prb:4.8} Using only the properties given in Theorem~\ref{th:propertiesofaddition}
 and Theorem~\ref{th:propertiesscalarmult}, as well as previous
problems, show $\left( -1\right) A=-A.$

Click on the arrow to see answer.
\begin{expandable}
$A+\left( -1\right) A=\left( 1+\left(
-1\right) \right) A=0A=O.$ Therefore, from the uniqueness of the additive
inverse proved in the above Problem \ref{addinvrstunique}, it follows that $
-A=\left( -1\right) A$.
\end{expandable}
\end{problem}

\begin{problem}\label{prb:4.9} Consider the matrices $
A =\left[
\begin{array}{rrr}
1 & 2 & 3 \\
2 & 1 & 7
\end{array}
\right],  B=\left[
\begin{array}{rrr}
3 & -1 & 2 \\
-3 & 2 & 1
\end{array}
\right],
C =\left[
\begin{array}{rr}
1 & 2 \\
3 & 1
\end{array}
\right], \\ D=\left[
\begin{array}{rr}
-1 & 2 \\
2 & -3
\end{array}
\right],  E=\left[
\begin{array}{r}
2 \\
3
\end{array}
\right]$.

Find the following if possible. If it is not possible explain why.
\begin{enumerate}
\item $-3A$

Click on the arrow to see answer.
\begin{expandable}
$\left[
\begin{array}{rrr}
-3 & -6 & -9 \\
-6 & -3 & -21
\end{array}
\right]$
\end{expandable}

\item $3B-A$

Click on the arrow to see answer.
\begin{expandable}
$\left[
\begin{array}{rrr}
8 & -5 & 3 \\
-11 & 5 & -4
\end{array}
\right]$
\end{expandable}

\item $AC$

Click on the arrow to see answer.
\begin{expandable}
Not possible
\end{expandable}

\item $CB$

Click on the arrow to see answer.
\begin{expandable}
$\left[
\begin{array}{rrr}
-3 & 3 & 4 \\
6 & -1 & 7
\end{array}
\right]$
\end{expandable}

\item $AE$

Click on the arrow to see answer.
\begin{expandable}
Not possible
\end{expandable}

\item $EA$

Click on the arrow to see answer.
\begin{expandable}
Not possible
\end{expandable}

\end{enumerate}
\end{problem}


\begin{problem}\label{prb:4.12} Let $A=\left[
\begin{array}{rr}
-1 & -1 \\
3 & 3
\end{array}
\right] $. Find all $2\times 2$ matrices, $B$
such that $AB=O.$

Click on the arrow to see answer.
\begin{expandable}
$$
\left[
\begin{array}{rr}
-1 & -1 \\
3 & 3
\end{array}
\right] \left[
\begin{array}{cc}
x & y \\
z & w
\end{array}
\right]  =\left[
\begin{array}{cc}
-x-z & -w-y \\
3x+3z & 3w+3y
\end{array}
\right] 
=\left[
\begin{array}{cc}
0 & 0 \\
0 & 0
\end{array}
\right]
$$
Solution is: $ w=-y,x=-z $ so the
matrices are of the form $\left[
\begin{array}{rr}
x & y \\
-x & -y
\end{array}
\right].$
\end{expandable}
\end{problem}

\begin{problem}\label{prb:4.14} Let $A=\left[
\begin{array}{rr}
1 & 2 \\
3 & 4
\end{array}
\right] ,B=\left[
\begin{array}{rr}
1 & 2 \\
3 & k
\end{array}
\right] .$ Is it possible to choose $k$ such that $AB=BA?$ If so, what
should $k$ equal?

Click on the arrow to see answer.
\begin{expandable}
$$
\left[
\begin{array}{cc}
1 & 2 \\
3 & 4
\end{array}
\right] \left[
\begin{array}{cc}
1 & 2 \\
3 & k
\end{array}
\right] = \left[
\begin{array}{cc}
7 & 2k+2 \\
15 & 4k+6
\end{array}
\right] $$
$$ \left[
\begin{array}{cc}
1 & 2 \\
3 & k
\end{array}
\right] \left[
\begin{array}{cc}
1 & 2 \\
3 & 4
\end{array}
\right] = \left[
\begin{array}{cc}
7 & 10 \\
3k+3 & 4k+6
\end{array}
\right]
$$
 Thus you must have $
\begin{array}{c}
3k+3=15 \\
2k+2=10
\end{array}
$.  Therefore $k=4$.
\end{expandable}
\end{problem}

\begin{problem}\label{prb:4.18} Find $2 \times 2$ matrices $A$ and $B$ such that $A \neq O$ and $B \neq O$ but $AB = O$.

Click on the arrow to see answer.
\begin{expandable}
Let $A = \left[
\begin{array}{rr}
1 & -1 \\
-1 & 1
\end{array}
\right], B = \left[
\begin{array}{cc}
1 & 1 \\
1 & 1
\end{array}
\right].$
\[
\left[
\begin{array}{rr}
1 & -1 \\
-1 & 1
\end{array}
\right] \left[
\begin{array}{cc}
1 & 1 \\
1 & 1
\end{array}
\right] = \left[
\begin{array}{cc}
0 & 0 \\
0 & 0
\end{array}
\right]
\]
\end{expandable}
\end{problem}

\begin{problem}\label{prb:4.32} Suppose $AB=AC$ and $A$ is an invertible $n\times n$ matrix. Does it
follow that $B=C?$ Explain why or why not.

Click on the arrow to see answer.
\begin{expandable}
Yes $B=C$. Multiply $AB = AC$ on the left by $A^{-1}$.
\end{expandable}
\end{problem}

\begin{problem}\label{prb:4.40}Let
\begin{equation*}
A=\left[
\begin{array}{rrr}
1 & 2 & 3 \\
2 & 1 & 4 \\
1 & 0 & 2
\end{array}
\right]
\end{equation*}
Find $A^{-1}$ if possible. If $A^{-1}$ does not exist, explain why.

Click on the arrow to see answer.
\begin{expandable}
$\left[
\begin{array}{ccc}
1 & 2 & 3 \\
2 & 1 & 4 \\
1 & 0 & 2
\end{array}
\right]^{-1}= \left[
\begin{array}{rrr}
-2 & 4 & -5 \\
0 & 1 & -2 \\
1 & -2 & 3
\end{array}
\right]$
\end{expandable}
\end{problem}

\begin{problem}\label{prb:4.42}Let
\begin{equation*}
A=\left[
\begin{array}{rrr}
1 & 2 & 3 \\
2 & 1 & 4 \\
4 & 5 & 10
\end{array}
\right]
\end{equation*}
Find $A^{-1}$ if possible. If $A^{-1}$ does not exist, explain why.

Click on the arrow to see answer.
\begin{expandable}
The reduced row echelon form is
$\left[
\begin{array}{ccc}
1 & 0 &  \frac{5}{3} \\
0 & 1 &  \frac{2}{3} \\
0 & 0 & 0
\end{array}
\right]$. There is no inverse.
\end{expandable}
\end{problem}

\begin{problem}\label{prb:4.48}
Show that if $A^{-1}$ exists for an $n\times n$
matrix, then it is unique. That is, if $BA=I$ and $AB=I,$ then $B=A^{-1}.$

Click on the arrow to see answer.
\begin{expandable}
 $A^{-1}=A^{-1}I=A^{-1}\left( AB\right) =\left( A^{-1}A\right) B=IB=B.$
\end{expandable}
\end{problem}

\begin{problem}\label{prb:4.49}Show that if $A$ is an invertible $n\times n$ matrix, then so is
$A^{T} $ and $\left( A^{T}\right) ^{-1}=\left( A^{-1}\right) ^{T}.$

Click the arrow to see answer.

\begin{expandable}
 You need to show that $\left( A^{-1}\right) ^{T}$ acts like the inverse of $A^{T}
$ because from uniqueness in the above problem, this will imply it is the
inverse. From properties of the transpose,
$$
A^{T}\left( A^{-1}\right) ^{T} =\left( A^{-1}A\right) ^{T}=I^{T}=I $$
$$\left( A^{-1}\right) ^{T}A^{T} =\left( AA^{-1}\right) ^{T}=I^{T}=I
$$
Hence $\left( A^{-1}\right) ^{T}=\left( A^{T}\right) ^{-1}$ and this last
matrix exists.
\end{expandable}
\end{problem}

\begin{problem}\label{prb:4.50}Show $\left( AB\right) ^{-1}=B^{-1}A^{-1}$ by verifying that
\begin{equation*}
AB\left(
B^{-1}A^{-1}\right) =I
\end{equation*} and
\begin{equation*}
B^{-1}A^{-1}\left( AB\right) =I
\end{equation*}

Click on the arrow to see answer.
\begin{expandable}
$\left( AB\right)
B^{-1}A^{-1}=A\left( BB^{-1}\right) A^{-1}=AA^{-1}=I$ $B^{-1}A^{-1}\left(
AB\right) =B^{-1}\left( A^{-1}A\right) B=B^{-1}IB=B^{-1}B=I$
\end{expandable}
\end{problem}

\begin{problem}\label{prb:4.70} Find an $LU$ factorization of the coefficient matrix and use it to solve the system of equations.
\begin{equation*}
\begin{array}{c}
x+2y=5 \\
2x+3y=6
\end{array}
\end{equation*}

Click on the arrow to see answer.
\begin{expandable}
An $LU$ factorization of the coefficient matrix is
\[
\left[
\begin{array}{cc}
1 & 2 \\
2 & 3
\end{array}
\right] =  \left[
\begin{array}{cc}
1 & 0 \\
2 & 1
\end{array}
\right] \left[
\begin{array}{cc}
1 & 2 \\
0 & -1
\end{array}
\right]
\]
First solve
\[
\left[
\begin{array}{cc}
1 & 0 \\
2 & 1
\end{array}
\right] \left[
\begin{array}{c}
u \\
v
\end{array}
\right] =\left[
\begin{array}{c}
5 \\
6
\end{array}
\right]
\]
which gives $\left[
\begin{array}{c}
u \\
v
\end{array}
\right] =$ $\left[
\begin{array}{r}
5 \\
-4
\end{array}
\right] .$ Then solve
\[
\left[
\begin{array}{rr}
1 & 2 \\
0 & -1
\end{array}
\right] \left[
\begin{array}{c}
x \\
y
\end{array}
\right] =\left[
\begin{array}{r}
5 \\
-4
\end{array}
\right]
\]
which says that $y=4$ and $x=-3.$
\end{expandable}
\end{problem}

\begin{problem}\label{prb:4.71} Find an $LU$ factorization of the coefficient matrix and use it to solve the system of equations.
\begin{equation*}
\begin{array}{c}
x+2y+z=1 \\
y+3z=2 \\
2x+3y=6
\end{array}
\end{equation*}

Click on the arrow to see answer.
\begin{expandable}
An $LU$ factorization of the coefficient matrix is
\[
\left[
\begin{array}{rrr}
1 & 2 & 1 \\
0 & 1 & 3 \\
2 & 3 & 0
\end{array}
\right] = \left[
\begin{array}{rrr}
1 & 0 & 0 \\
0 & 1 & 0 \\
2 & -1 & 1
\end{array}
\right] \left[
\begin{array}{rrr}
1 & 2 & 1 \\
0 & 1 & 3 \\
0 & 0 & 1
\end{array}
\right]
\]
First solve
\[
 \left[
\begin{array}{rrr}
1 & 0 & 0 \\
0 & 1 & 0 \\
2 & -1 & 1
\end{array}
\right] \left[
\begin{array}{c}
u \\
v \\
w
\end{array}
\right] =\left[
\begin{array}{c}
1 \\
2 \\
6
\end{array}
\right]
\]
which yields $u=1,v=2,w=6$. Next solve
\[
\left[
\begin{array}{rrr}
1 & 2 & 1 \\
0 & 1 & 3 \\
0 & 0 & 1
\end{array}
\right] \left[
\begin{array}{c}
x \\
y \\
z
\end{array}
\right] =\left[
\begin{array}{c}
1 \\
2 \\
6
\end{array}
\right]
\]
This yields $z=6,y=-16,x=27.$
\end{expandable}
\end{problem}



\section*{Bibliography}
These problems come from the end of Chapter 2 of Ken Kuttler's \href{https://open.umn.edu/opentextbooks/textbooks/a-first-course-in-linear-algebra-2017}{\it A First Course in Linear Algebra}. (CC-BY)

Ken Kuttler, {\it  A First Course in Linear Algebra}, Lyryx 2017, Open Edition, pp. 90--98, 104--106. 

\end{document}