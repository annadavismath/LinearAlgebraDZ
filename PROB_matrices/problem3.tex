\documentclass{ximera}
%% You can put user macros here
%% However, you cannot make new environments

\listfiles

\graphicspath{{./}{firstExample/}{secondExample/}}

\usepackage{tikz}
\usepackage{tkz-euclide}
\usepackage{tikz-3dplot}
\usepackage{tikz-cd}
\usetikzlibrary{shapes.geometric}
\usetikzlibrary{arrows}
%\usetkzobj{all}
\pgfplotsset{compat=1.13} % prevents compile error.

%\renewcommand{\vec}[1]{\mathbf{#1}}
\renewcommand{\vec}{\mathbf}
\newcommand{\RR}{\mathbb{R}}
\newcommand{\dfn}{\textit}
\newcommand{\dotp}{\cdot}
\newcommand{\id}{\text{id}}
\newcommand\norm[1]{\left\lVert#1\right\rVert}
 
\newtheorem{general}{Generalization}
\newtheorem{initprob}{Exploration Problem}

\tikzstyle geometryDiagrams=[ultra thick,color=blue!50!black]

%\DefineVerbatimEnvironment{octave}{Verbatim}{numbers=left,frame=lines,label=Octave,labelposition=topline}



\usepackage{mathtools}

\author{}
\license{Creative Commons 4.0 By-NC-SA}
%\outcome{Compute an antiderivative using basic formulas}
\begin{document}
\begin{exercise}
Let $A=\begin{bmatrix}|&|&|\\\vec{c}_1& \vec{c}_2 & \vec{c}_3\\|&|&|\end{bmatrix}$ be a $3\times 3$ matrix with columns $\vec{c}_1$, $\vec{c}_2$ and $\vec{c}_3$.

Suppose $A\begin{bmatrix}-1\\-1\\2\end{bmatrix}=\begin{bmatrix}0\\1\\1\end{bmatrix}$

Which of the following can we conclude from the given information?  Select ALL that apply.
\begin{selectAll}
 \choice{$\vec{c}_2+\vec{c}_3=\begin{bmatrix}-1\\-1\\2\end{bmatrix}$}
 \choice[correct]{$-\vec{c}_1-\vec{c}_2+2\vec{c}_3=\begin{bmatrix}0\\1\\1\end{bmatrix}$}
 \choice{$A$ is non-singular.}
 \choice[correct]{$\begin{bmatrix}0\\1\\1\end{bmatrix}$ is in $\text{span}(\vec{c}_1, \vec{c}_2, \vec{c}_3)$.}
 \choice[correct]{Vector $\begin{bmatrix}0\\1\\1\end{bmatrix}$  can be written as a linear combination of the columns of $A$.}
  \end{selectAll}
\end{exercise}

% Let A=\begin{bmatrix}1 & 1 & 1\\1 & 2& 2\\2&3&3\end{bmatrix} for counterexample to part (c).

\end{document}