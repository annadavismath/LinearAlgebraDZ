\documentclass{ximera}
%% You can put user macros here
%% However, you cannot make new environments

\listfiles

\graphicspath{{./}{firstExample/}{secondExample/}}

\usepackage{tikz}
\usepackage{tkz-euclide}
\usepackage{tikz-3dplot}
\usepackage{tikz-cd}
\usetikzlibrary{shapes.geometric}
\usetikzlibrary{arrows}
%\usetkzobj{all}
\pgfplotsset{compat=1.13} % prevents compile error.

%\renewcommand{\vec}[1]{\mathbf{#1}}
\renewcommand{\vec}{\mathbf}
\newcommand{\RR}{\mathbb{R}}
\newcommand{\dfn}{\textit}
\newcommand{\dotp}{\cdot}
\newcommand{\id}{\text{id}}
\newcommand\norm[1]{\left\lVert#1\right\rVert}
 
\newtheorem{general}{Generalization}
\newtheorem{initprob}{Exploration Problem}

\tikzstyle geometryDiagrams=[ultra thick,color=blue!50!black]

%\DefineVerbatimEnvironment{octave}{Verbatim}{numbers=left,frame=lines,label=Octave,labelposition=topline}



\usepackage{mathtools}


\title{Solved Problems} \license{CC BY-NC-SA 4.0}

\begin{document}

\begin{abstract}
\end{abstract}
\maketitle

\section*{Solved Problems for Chapter 2} 


\begin{problem}\label{prb:2.3} You have a system of $k$ equations in two variables, $k\geq 2$.
Explain the geometric significance of

\begin{enumerate}
\item No solution.

\begin{expandable}
    The $k$ lines do not have a point common to all of them.
\end{expandable}
\item A unique solution.

\begin{expandable}
    All $k$ lines intersect at a single point.
\end{expandable}
\item An infinite number of solutions.

\begin{expandable}
    The $k$ lines coincide.
\end{expandable}
\end{enumerate}
Click the arrow to see answer.

\end{problem}

\begin{problem}\label{prb:2.7}
Consider the following augmented matrix in which $\ast $ denotes an arbitrary
number and $\blacksquare $ denotes a nonzero number. Determine whether the
given augmented matrix corresponds to a consistent system. If consistent, is the solution unique?
\begin{equation*}
\left[
\begin{array}{ccccc|c}
\blacksquare & \ast & \ast & \ast & \ast & \ast \\
0 & \blacksquare & \ast & \ast & 0 & \ast \\
0 & 0 & 0 & 0 & \blacksquare & 0 \\
0 & 0 & 0 & 0 & \blacksquare & \blacksquare
\end{array}
\right]
\end{equation*}
Click the arrow to see answer. 
\begin{expandable}
The third equation implies that $x_5 = 0$.  The fourth equation implies that $x_5 \ne 0$.  We conclude that the system is inconsistent.
\end{expandable}
\end{problem}

\begin{problem}\label{prb:2.8}
Suppose a system of equations has fewer equations than variables. Will such a system necessarily be consistent? If so, explain why and if not, give an example which is not consistent.

Click the arrow to see answer.
\begin{expandable}
No. Consider $x+y+z=2$ and $x+y+z=1.$
\end{expandable}
\end{problem}

\begin{problem}\label{prb:2.9}
If a system of equations has more equations than variables, can it
have a solution? If so, give an example and if not, explain why not.

Click the arrow to see answer. 
\begin{expandable}
These can
have a solution. For example, $x+y=1,2x+2y=2,3x+3y=3$ even has an infinite
set of solutions.
\end{expandable}
\end{problem}

\begin{problem}\label{prb:2.10}
Find $h$ such that
\begin{equation*}
\left[
\begin{array}{rr|r}
2 & h & 4 \\
3 & 6 & 7
\end{array}
\right]
\end{equation*}
is the augmented matrix of an \textit{inconsistent} system.

\begin{expandable}
 $\left[
\begin{array}{cc|c}
2 & h & 4 \\
3 & 6 & 7
\end{array}
\right]\rightsquigarrow \left[
\begin{array}{cc|c}
3 & (3/2)h & 6 \\
3 & 6 & 7
\end{array}
\right]$.  The system is inconsistent for $h=4$.
\end{expandable}

\end{problem}

\begin{problem}\label{prb:2.11}
Find $h$ such that
\begin{equation*}
\left[
\begin{array}{rr|r}
1 & h & 3 \\
2 & 4 & 6
\end{array}
\right]
\end{equation*}
is the augmented matrix of a \textit{consistent} system.

Click the arrow to see answer. 
\begin{expandable}
 The reduced row-echelon form will never have a row of the form $[0\, 0|1]$. The system is consistent for all $h$.
\end{expandable}
\end{problem}

\begin{problem}\label{prb:2.14}
Choose $h$ and $k$ such that the augmented matrix shown has each of the following:
\begin{enumerate}
\item one solution
\item no solution
\item infinitely many solutions
\end{enumerate}
\begin{equation*}
\left[
\begin{array}{rr|r}
1 & 2 & 2 \\
2 & h & k
\end{array}
\right]
\end{equation*}
Click the arrow to see answer. 
\begin{expandable}
If $h\neq 4,$ then there is exactly one solution. If $h=4$ and $k\neq 4,$
then there are no solutions. If $h=4$ and $k=4,$ then there are infinitely
many solutions.
\end{expandable}
\end{problem}

\begin{problem}\label{prb:2.15}
Determine if the system is consistent. If so, is the solution unique?
$$\begin{array}{ccccccccc}
      x & +&2y&+&z&-&w&= &2 \\
	 x& -&y&+&z&+&w&=&1\\
     2x& +&y&-&z&&&=&1\\
     4x&+&2y&+&z&&&=&5
    \end{array}$$

Click the arrow to see answer. 
\begin{expandable}
There is no solution because the rref of the augmented matrix contains a row of the form $\begin{bmatrix}0 & 0 & 0 & 0 & | & 1\end{bmatrix}$ $$\mbox{rref}\left(\left[
\begin{array}{rrrr|r}
1 & 2 & 1 & -1 & 2 \\
1 & -1 & 1 & 1 & 1 \\
2 & 1 & -1 & 0 & 1 \\
4 & 2 & 1 & 0 & 5
\end{array}
\right]\right) = \left[
\begin{array}{rrrr|r}
1 & 0 & 0 &  \frac{1}{3} & 0 \\
0 & 1 & 0 & - \frac{2}{3} & 0 \\
0 & 0 & 1 & 0 & 0 \\
0 & 0 & 0 & 0 & 1
\end{array}
\right] .$$
\end{expandable}
\end{problem}

\begin{problem}\label{prb:2.16}
Determine if the system is consistent. If so, is the solution unique?

$$\begin{array}{ccccccccc}
      x & +&2y&+&z&-&w&= &2 \\
	 x& -&y&+&z&+&w&=&0\\
     2x& +&y&-&z&&&=&1\\
     4x&+&2y&+&z&&&=&3
    \end{array}$$

Click the arrow to see answer. 
\begin{expandable}
$$\text{rref}\left(\left[
\begin{array}{rrrr|r}
1 & 2 & 1 & -1 & 2 \\
1 & -1 & 1 & 1 & 0 \\
2 & 1 & -1 & 0 & 1 \\
4 & 2 & 1 & 0 & 3
\end{array}
\right]\right)=\left[
\begin{array}{rrrr|r}
1 & 0 & 0 & 1/3 & 1/3 \\
0 & 1 & 0 & -2/3 & 2/3 \\
0 & 0 & 1 & 0 & 1/3 \\
0 & 0 & 0 & 0 & 0
\end{array}
\right] $$
There are infinitely many solutions: $w=t$, $z=\frac{1}{3}$, $y=\frac{2}{3}+\frac{2}{3}t$, $x=\frac{1}{3}-\frac{1}{3}t$.
\end{expandable}
\end{problem}

\begin{problem}\label{prb:2.25} Find the solution of the system whose augmented matrix is
\begin{equation*}
\left[
\begin{array}{rrr|r}
1 & 2 & 0 & 2 \\
1 & 3 & 4 & 2 \\
1 & 0 & 2 & 1
\end{array}
\right]
\end{equation*}

Click the arrow to see answer. 
\begin{expandable}

$$\text{rref}\left(\left[
\begin{array}{rrr|r}
1 & 2 & 0 & 2 \\
1 & 3 & 4 & 2 \\
1 & 0 & 2 & 1
\end{array}
\right]\right)=\left[
\begin{array}{rrr|r}
1 & 0 & 0 & 6/5 \\
0 & 1 & 0 & 2/5 \\
0 & 0 & 1 & -1/10
\end{array}
\right]$$
$x=\frac{6}{5}, y=\frac{2}{5}, z=-\frac{1}{10}$
\end{expandable}
\end{problem}

\begin{problem}\label{prb:2.27} Solve the system whose augmented matrix is
\begin{equation*}
\left[
\begin{array}{rrr|r}
1 & 1 & 0 & 1 \\
1 & 0 & 4 & 2
\end{array}
\right]
\end{equation*}
Click the arrow to see answer. 
\begin{expandable}
The reduced row echelon form is $\left[
\begin{array}{rrr|r}
1 & 0 & 4 & 2 \\
0 & 1 & -4 & -1
\end{array}
\right] $ and so the solution is $z=t,y=4t,x=2-4t.$
\end{expandable}
\end{problem}

\begin{problem}\label{prb:2.28} Solve the system if the rref of its augmented matrix is
$$\left[
\begin{array}{rrrrr|r}
1 & 0 & 0 & 0 & 9 & 3 \\
0 & 1 & 0 & 0 & -4 & 0 \\
0 & 0 & 1 & 0 & -7 & -1 \\
0 & 0 & 0 & 1 & 6 & 1
\end{array}
\right] $$
Click the arrow to see answer. 
\begin{expandable}
$x_{5}=t,x_{4}=1-6t,x_{3}=-1+7t,x_{2}=4t,x_{1}=3-9t$.
\end{expandable}
\end{problem}

\begin{problem}\label{prb:2.29} Solve the system if the rref of its augmented matrix is
$$\left[
\begin{array}{rrrrr|r}
1 & 0 & 2 & 0 & -1/2 &  5/2 \\
0 & 1 & 0 & 0 &  1/2 &  3/2 \\
0 & 0 & 0 & 1 &  3/2 & -1/2 \\
0 & 0 & 0 & 0 & 0 & 0
\end{array}
\right] $$
Click the arrow to see answer. 
\begin{expandable}
The free variables are $x_{5}=t,x_{3}=s$. The other variables are
given by $x_{4}=-\frac{1}{2}-\frac{3}{2}t$, $x_{2}=\frac{3}{2}-\frac{1}{2}t$, $x_{1}=\frac{5}{2}+\frac{1}{2}t-2s$.
\end{expandable}
\end{problem}

\begin{problem}\label{prb:2.39} Suppose a system of equations has fewer equations than variables and
you have found a solution to this system of equations. Is it possible that
your solution is the only one? Explain.

Click the arrow to see answer. 
\begin{expandable}
No. The rank of the coefficient matrix in this case is smaller than the number of columns (variables).  So, there has to be a free variable.  The parameter ($t$ is a typical choice) assigned to the free variable will guarantee infinitely many solutions.
\end{expandable}
\end{problem}

\begin{problem}\label{prb:2.55} Suppose $A$ is an $m\times n$ matrix. Explain why the rank of $A$ is
always no larger than $\min \left( m,n\right).$

Click the arrow to see answer. 
\begin{expandable}
It is because you cannot
have more leading 1's than columns and you cannot have more leading 1's than rows.
\end{expandable}
\end{problem}

\begin{problem}\label{prb:2.53} Find the rank of the following matrix.
$$\begin{bmatrix}
4 & 4 & 20 & -1 & 17 \\
1 & 1 & 5 & 0 & 5 \\
1 & 1 & 5 & -1 & 2 \\
3 & 3 & 15 & -3 & 6
\end{bmatrix}$$

Click the arrow to see answer. 
\begin{expandable}
$$\text{rref}\left(\begin{bmatrix}4 & 4 & 20 & -1 & 17 \\
1 & 1 & 5 & 0 & 5 \\
1 & 1 & 5 & -1 & 2 \\
3 & 3 & 15 & -3 & 6\end{bmatrix}\right)=\begin{bmatrix}1 &1 &5 &0 &5\\
 0 & 0 &0 &1 &3\\
 0 &0 &0 &0 &0\\
 0& 0& 0& 0& 0\end{bmatrix}$$

 The rank is the number of leading $1$'s.  The rank of the given matrix is $2$.
\end{expandable}
\end{problem}

\section*{Bibliography}
These problems come from the end of Chapter 1 of Ken Kuttler's \href{https://open.umn.edu/opentextbooks/textbooks/a-first-course-in-linear-algebra-2017}{\it A First Course in Linear Algebra}. (CC-BY)

Ken Kuttler, {\it  A First Course in Linear Algebra}, Lyryx 2017, Open Edition, pp. 42--49. 

\end{document}