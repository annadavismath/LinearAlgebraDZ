\documentclass{ximera}
%% You can put user macros here
%% However, you cannot make new environments

\listfiles

\graphicspath{{./}{firstExample/}{secondExample/}}

\usepackage{tikz}
\usepackage{tkz-euclide}
\usepackage{tikz-3dplot}
\usepackage{tikz-cd}
\usetikzlibrary{shapes.geometric}
\usetikzlibrary{arrows}
%\usetkzobj{all}
\pgfplotsset{compat=1.13} % prevents compile error.

%\renewcommand{\vec}[1]{\mathbf{#1}}
\renewcommand{\vec}{\mathbf}
\newcommand{\RR}{\mathbb{R}}
\newcommand{\dfn}{\textit}
\newcommand{\dotp}{\cdot}
\newcommand{\id}{\text{id}}
\newcommand\norm[1]{\left\lVert#1\right\rVert}
 
\newtheorem{general}{Generalization}
\newtheorem{initprob}{Exploration Problem}

\tikzstyle geometryDiagrams=[ultra thick,color=blue!50!black]

%\DefineVerbatimEnvironment{octave}{Verbatim}{numbers=left,frame=lines,label=Octave,labelposition=topline}



\usepackage{mathtools}


\title{Solved Problems for Ch 2} \license{CC BY-NC-SA 4.0}

\begin{document}

\begin{abstract}
\end{abstract}
\maketitle

\section*{Solved Problems for Chapter 2}

\begin{problem}\label{prb:3.3}
If possible, express the vector
\begin{equation*}
\vec{v}= 
\begin{bmatrix}
4 \\
4 \\
-3
\end{bmatrix}
\end{equation*}
as a linear combination of the vectors

\begin{equation*}
\vec{u}_1 = 
\begin{bmatrix}
3 \\
1 \\
-1
\end{bmatrix}
\mbox{ and  }
\vec{u}_2 =
\begin{bmatrix}
2 \\
-2\\
1
\end{bmatrix}.
\end{equation*}

Click the arrow to see answer.
\begin{expandable}
    We start with a linear combination equation $a\begin{bmatrix}3\\1\\-1\end{bmatrix}+b\begin{bmatrix}2\\-2\\1\end{bmatrix}=\begin{bmatrix}4\\4\\-3\end{bmatrix}$.  This gives us the following:
    $$\left[
\begin{array}{rr|r}
3 & 2 & 4 \\
1 & -2 & 4\\
-1 & 1 & -3
\end{array}
\right] \rightsquigarrow \left[
\begin{array}{rr|r}1& 0& 2\\
 0& 1 &-1\\
 0& 0& 0\end{array}
\right]$$
This gives us $\vec{v}=2\vec{u}_1 -\vec{u}_2$.
\end{expandable}
\end{problem}

\begin{problem}\label{prb:3.4}
Decide whether
$$
\vec{v}= 
\begin{bmatrix}
4 \\
4 \\
4
\end{bmatrix}
$$
is a linear combination of the vectors
$$
\vec{u}_1 = 
\begin{bmatrix}
3 \\
1 \\
-1
\end{bmatrix}
\mbox{ and  }
\vec{u}_2 =
\begin{bmatrix}
2 \\
-2\\
1
\end{bmatrix}.
$$

Click the arrow to see answer.

\begin{expandable}
$$\left[
\begin{array}{rr|r}
3 & 2 & 4 \\
1 & -2 & 4\\
-1 & 1 & 4
\end{array}
\right] \rightsquigarrow \left[
\begin{array}{rr|r}1& 0& 0\\
 0& 1 & 0\\
 0& 0& 1\end{array}
\right]$$
The last row indicates that there is no solution to the linear combination equation $a\vec{u}_1+b\vec{u}_2=\vec{v}$.  Therefore, $\vec{v}$ is not a linear combination of $\vec{u}_1$ and $\vec{u}_2$.
\end{expandable}
\end{problem}

\begin{problem}\label{prb:sovedCh2_1}
Determine whether the following set of vectors is linearly independent.
$$\left\{\begin{bmatrix}
    2\\ -3\\0
\end{bmatrix}, \begin{bmatrix}
    1\\0\\1
\end{bmatrix}, \begin{bmatrix}
    1\\1\\4
\end{bmatrix}\right\}$$

Click the arrow for answer.

\begin{expandable}
    The linear combination equation $a\begin{bmatrix}
    2\\ -3\\0
\end{bmatrix}+ b\begin{bmatrix}
    1\\0\\1
\end{bmatrix}+ c\begin{bmatrix}
    1\\1\\4
\end{bmatrix}=\vec{0}$ has only the trivial solution.  We can see this as follows.
$$\text{rref}\left(\begin{bmatrix}2 & 1 & 1\\-3 & 0 & 1\\0 & 1 & 4\end{bmatrix}\right)=\begin{bmatrix}
    1 & 0 & 0\\0 & 1 &0\\ 0 & 0 &1
\end{bmatrix}$$
Therefore the given set of vectors is linearly independent.
\end{expandable}
\end{problem}

\begin{problem}\label{prb:solvedCh2_2}
Determine whether the following set of vectors is linearly independent.
$$\left\{\begin{bmatrix}
    2\\ -3\\0
\end{bmatrix}, \begin{bmatrix}
    4\\-1\\8
\end{bmatrix}, \begin{bmatrix}
    1\\1\\4
\end{bmatrix}\right\}$$

Click the arrow for answer.

\begin{expandable}
    The linear combination equation $a\begin{bmatrix}
    2\\ -3\\0
\end{bmatrix}+ b\begin{bmatrix}
    4\\-1\\8
\end{bmatrix}+ c\begin{bmatrix}
    1\\1\\4
\end{bmatrix}=\vec{0}$ has a non-trivial solution.  We can see this as follows.
$$\text{rref}\left(\begin{bmatrix}2 & 4 & 1\\-3 & -1 & 1\\0 & 8 & 4\end{bmatrix}\right)=\begin{bmatrix}
    1 & 0 & -1/2\\0 & 1 &1/2\\ 0 & 0 &0
\end{bmatrix}$$
Therefore the given set of vectors is linearly dependent.
\end{expandable}
\end{problem}

\begin{problem}\label{prb:3.14} Suppose $\left\{ \vec{x}_{1},\ldots ,\vec{x}_{k}\right\} $ is a
set of vectors from $\RR^{n}.$ Show that $\vec{0}$ is in $\mbox{
span}\left\{ \vec{x}_{1},\ldots ,\vec{x}_{k}\right\} .$

Click the arrow to see answer.
\begin{expandable}
Let all of the coefficients in the linear combination be zero: 
 $\sum_{i=1}^{k}0\vec{x}_{k}=\vec{0}$
\end{expandable}
\end{problem}

\begin{problem}\label{prob:solvedCh2_3}
    True or false?

    Click the arrow to see each answer.

    \begin{enumerate}
        \item The span of any one vector in $\RR^n$ is a line.

        \begin{expandable}
            This is technically false because the span of the zero vector is a point.  For all non-zero vectors this statement is true.
        \end{expandable}

        \item The span of two non-zero vectors in $\RR^n$ is a plane.

        \begin{expandable}
            This is false.  If two vectors are scalar multiples of each other (i.e. they are linearly dependent), then their span is a line, not a plane.
        \end{expandable}

        \item Given two linearly independent vectors, $\vec{v}$ and $\vec{w}$, let $\vec{n}$ be a normal vector to the plane spanned by $\vec{v}$ and $\vec{w}$.  The set $\left\{\vec{v}, \vec{w}, \vec{n}\right\}$ is linearly independent.

        \begin{expandable}
            This is true.  A normal to the plane is a vector perpendicular to the plane.  Only vectors that lie in the plane spanned by $\vec{v}$ and $\vec{w}$ can be expressed as a linear combination of $\vec{v}$ and $\vec{w}$.  Since $\vec{n}$ is not in the plane, it cannot be expressed as a linear combination of $\vec{v}$ and $\vec{w}$.  Therefore the set $\left\{\vec{v}, \vec{w}, \vec{n}\right\}$ is linearly independent.
        \end{expandable}
    \end{enumerate}
\end{problem}

\section*{Bibliography}
Some of the Review Exercises come from the end of Chapter 4 of Ken Kuttler's \href{https://open.umn.edu/opentextbooks/textbooks/a-first-course-in-linear-algebra-2017}{\it A First Course in Linear Algebra}. (CC-BY)

Ken Kuttler, {\it  A First Course in Linear Algebra}, Lyryx 2017, Open Edition, pp. 151--152.

\end{document}