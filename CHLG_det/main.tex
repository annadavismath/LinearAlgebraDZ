\documentclass{ximera}
%% You can put user macros here
%% However, you cannot make new environments

\listfiles

\graphicspath{{./}{firstExample/}{secondExample/}}

\usepackage{tikz}
\usepackage{tkz-euclide}
\usepackage{tikz-3dplot}
\usepackage{tikz-cd}
\usetikzlibrary{shapes.geometric}
\usetikzlibrary{arrows}
%\usetkzobj{all}
\pgfplotsset{compat=1.13} % prevents compile error.

%\renewcommand{\vec}[1]{\mathbf{#1}}
\renewcommand{\vec}{\mathbf}
\newcommand{\RR}{\mathbb{R}}
\newcommand{\dfn}{\textit}
\newcommand{\dotp}{\cdot}
\newcommand{\id}{\text{id}}
\newcommand\norm[1]{\left\lVert#1\right\rVert}
 
\newtheorem{general}{Generalization}
\newtheorem{initprob}{Exploration Problem}

\tikzstyle geometryDiagrams=[ultra thick,color=blue!50!black]

%\DefineVerbatimEnvironment{octave}{Verbatim}{numbers=left,frame=lines,label=Octave,labelposition=topline}



\usepackage{mathtools}


\title{Challenge Problems} \license{CC BY-NC-SA 4.0}

\begin{document}

\begin{abstract}
\end{abstract}
\maketitle

\section*{Challenge Problems for Chapter 7}

\begin{problem}\label{prb:7.32} Consider the matrix
\begin{equation*}
A =
\left[
\begin{array}{ccc}
1 & 0 & 0 \\
0 & \cos t & -\sin t \\
0 & \sin t & \cos t
\end{array}
\right]
\end{equation*}
Does there exist a value of $t$ for which this matrix fails to have an
inverse? Explain.

Click the arrow to see answer.
\begin{expandable}
 No. It has a nonzero determinant for all $t$.
\end{expandable}
\end{problem}

\begin{problem}\label{prb:7.41} If $A,B,$ and $C$ are each $n\times n$ matrices and $ABC$ is
invertible, show why each of $A,B,$ and $C$ are invertible.

Click the arrow to see answer.

\begin{expandable}
This follows
because $\det \left( ABC\right) =\det \left( A\right) \det \left( B\right)
\det \left( C\right) $ and if this product is nonzero, then each determinant
in the product is nonzero and so each of these matrices is invertible.
\end{expandable}
\end{problem}

\begin{problem}\label{prb:7.37} Suppose $A,B$ are $n\times n$ matrices and that $AB=I.$ Show that then
$BA=I.$ 
\begin{hint}
First explain why
$\det \left( A\right) ,\det \left( B\right) $ are both nonzero. Then $\left(
AB\right) A=A$ and then show $BA\left( BA-I\right) =0.$ Now use what
is given to conclude $A\left( BA-I\right) =0.$ 
\end{hint}
\end{problem}

\begin{problem}\label{prb:7.40} Suppose $A$ is an upper triangular matrix. Show that $A^{-1}$ exists
if and only if all elements of the main diagonal are non zero. Is it true
that $A^{-1}$ will also be upper triangular? Explain. Could the same be concluded for lower triangular matrices?
\begin{hint}
The given condition is what it takes for the
determinant to be non zero. Recall that the determinant of an upper
triangular matrix is just the product of the entries on the main diagonal.
\end{hint}
\end{problem}

\begin{problem}\label{prob:nich3.2.3}
Let $A$, $B$, and $C$ denote $n\times n$ matrices.  Assume that $\det A=-1$, $\det B=2$, and $\det C=3$.  Evaluate
\begin{enumerate}
    \item $\det (A^3BC^TB^{-1})$
    \item $\det (B^2C^{-1}AB^{-1}C^T)$
    \item $\det (A^{-1}B^{-1}AB)$
\end{enumerate}
    
\end{problem}

\begin{problem}\label{prob:nich3.2.12}
If $A$ and $B$ are $n\times n$ matrices such that $AB =-BA$, and if $n$ is odd, show that either $A$ or $B$ has no inverse.
\end{problem}

\begin{problem}\label{prob:nich3.2.16}
    Show that no $3\times 3$ matrix $A$ exists such
that $A^2+I = O$. Find a $2\times 2$ matrix $A$ with this property.
\end{problem}

\begin{problem}\label{prob:nich3.2.17}
    Show that $\det (A+B^T ) = \det (A^T +B)$
for any $n\times n$ matrices $A$ and $B$.
\end{problem}



\section*{Bibliography}

These problems come from Chapter 3 of Ken Kuttler's \href{https://open.umn.edu/opentextbooks/textbooks/a-first-course-in-linear-algebra-2017}{\it A First Course in Linear Algebra}. (CC-BY)

Ken Kuttler, {\it  A First Course in Linear Algebra}, Lyryx 2017, Open Edition, pp. 272--315.  

Several problems came from the end of Chapter 3 of Keith Nicholson's \href{https://open.umn.edu/opentextbooks/textbooks/linear-algebra-with-applications}{\it Linear Algebra with Applications}. (CC-BY-NC-SA)

W. Keith Nicholson, {\it Linear Algebra with Applications}, Lyryx 2018, Open Edition, pp. 167. 

\end{document}