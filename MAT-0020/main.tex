\documentclass{ximera}
%% You can put user macros here
%% However, you cannot make new environments

\listfiles

\graphicspath{{./}{firstExample/}{secondExample/}}

\usepackage{tikz}
\usepackage{tkz-euclide}
\usepackage{tikz-3dplot}
\usepackage{tikz-cd}
\usetikzlibrary{shapes.geometric}
\usetikzlibrary{arrows}
%\usetkzobj{all}
\pgfplotsset{compat=1.13} % prevents compile error.

%\renewcommand{\vec}[1]{\mathbf{#1}}
\renewcommand{\vec}{\mathbf}
\newcommand{\RR}{\mathbb{R}}
\newcommand{\dfn}{\textit}
\newcommand{\dotp}{\cdot}
\newcommand{\id}{\text{id}}
\newcommand\norm[1]{\left\lVert#1\right\rVert}
 
\newtheorem{general}{Generalization}
\newtheorem{initprob}{Exploration Problem}

\tikzstyle geometryDiagrams=[ultra thick,color=blue!50!black]

%\DefineVerbatimEnvironment{octave}{Verbatim}{numbers=left,frame=lines,label=Octave,labelposition=topline}



\usepackage{mathtools}


\title{Matrix Multiplication} \license{CC BY-NC-SA 4.0}

\begin{document}

\begin{abstract}

\end{abstract}
\maketitle

\begin{onlineOnly}
\section*{Matrix Multiplication}
\end{onlineOnly}

We will introduce matrix multiplication by first considering the special case of a matrix-vector product.  In other words, the first matrix is $m \times n$ and the second matrix is $n \times 1$ for some positive integers $m,n$. % We will employ Octave in the following exploration.

\subsection*{Matrix-Vector Multiplication}

%\begin{exploration}\label{exp:matrixvectormultdef} 
\begin{example}\label{exp:matrixvectormultdef}
Let $$A=\begin{bmatrix}2&-1&3&2\\0&3&-2&1\\-2&4&1&0\end{bmatrix}\quad\text{and}\quad \vec{x}=\begin{bmatrix}3\\-1\\4\\1\end{bmatrix}$$ %Find $A\vec{x}$ using Octave.

%Notice in the Octave window below, the semicolon is used to indicate a new line of a matrix, and the asterisk indicates matrix multiplication.

%\begin{octave}
%A=[2 -1 3 2; 0 3 -2 1; -2 4 1 0]
%x=[3;-1;4;1]
%A*x
%\end{octave}

%The purpose of this exploration is to understand the matrix-vector product $A\vec{x}$ we computed above.  Note that we can select an entry of a matrix in Octave by naming its row and column.  So A(2,3) = -2.  In the case of a column matrix like $\vec{x}$, we only need to indicate the row to select an entry, so x(3)=4.  We can also select an entire column of matrix $A$ by placing the "dummy colon" as the row.  So A(:,2) gives us the second column of $A$.

%\begin{octave}
%A=[2 -1 3 2; 0 3 -2 1; -2 4 1 0]
%x=[3;-1;4;1]
%A(2,3)
%x(3)
%A(:,2)
%\end{octave}  

One way to understand the matrix-vector product $A\vec{x}$ is by thinking of it as a linear combination of the columns of $A$, using the entries in $\vec{x}$ as our coefficients.  %In the next Octave window we demonstrate this.  

%\begin{octave}
%A=[2 -1 3 2; 0 3 -2 1; -2 4 1 0];
%x=[3;-1;4;1]; %semicolons at the end of a line suppress the output
%x(1)*A(:,1)+x(2)*A(:,2)+x(3)*A(:,3)+x(4)*A(:,4)
%A*x
%\end{octave} 

%We have computed 

$$A\vec{x}=\begin{bmatrix}\color{red}2&\color{blue}-1&\color{brown}3&2\\ \color{red}0&\color{blue}3&\color{brown}-2&1\\ \color{red}-2&\color{blue}4&\color{brown}1&0\end{bmatrix}\begin{bmatrix}\color{red}3\\ \color{blue}-1\\ \color{brown}4\\1\end{bmatrix}=
\color{red}3\begin{bmatrix}\color{red}2\\ \color{red}0\\ \color{red}-2\end{bmatrix}\color{black}+
\color{blue}(-1)\begin{bmatrix}\color{blue}-1\\ \color{blue}3\\ \color{blue}4\end{bmatrix}
\color{black}+\color{brown}4\begin{bmatrix}\color{brown}3\\ \color{brown}-2\\ \color{brown}1\end{bmatrix}
\color{black}+\begin{bmatrix}2\\1\\0\end{bmatrix}
=\begin{bmatrix}21\\-10 \\ -6\end{bmatrix}
$$

We can also compute the product one entry at a time.  First, let's focus on the first row of $A$.
$$\begin{bmatrix}{\color{red}2}&{\color{blue}-1}&{\color{brown}3}&2\\0&3&-2&1\\-2&4&1&0\end{bmatrix}\begin{bmatrix}{\color{red}3}\\{\color{blue}-1}\\{\color{brown}4}\\1\end{bmatrix}=\begin{bmatrix}{\color{red}(2)( 3)}+{\color{blue}(-1)(-1)}+{\color{brown}(3)(4)}+(2)(1)\\ \\ \\\end{bmatrix}=\begin{bmatrix}21\\ \\ \\\end{bmatrix}
$$
Next, let's look a the second row of $A$.
$$\begin{bmatrix}2&-1&3&2\\{\color{red}0}&{\color{blue}3}&{\color{brown}-2}&1\\-2&4&1&0\end{bmatrix}\begin{bmatrix}{\color{red}3}\\{\color{blue}-1}\\{\color{brown}4}\\1\end{bmatrix}=\begin{bmatrix}21\\{\color{red}(0)( 3)}+{\color{blue}(3)(-1)}+{\color{brown}(-2)(4)}+(1)(1)\\ \\ \end{bmatrix}=\begin{bmatrix}21\\-10 \\ \\\end{bmatrix}
$$
Finally, let's do the third row of $A$.
$$\begin{bmatrix}2&-1&3&2\\0&3&-2&1\\{\color{red}-2}&{\color{blue}4}&{\color{brown}1}&0\end{bmatrix}\begin{bmatrix}{\color{red}3}\\{\color{blue}-1}\\{\color{brown}4}\\1\end{bmatrix}=\begin{bmatrix}21\\-10\\{\color{red}(-2)( 3)}+{\color{blue}(4)(-1)}+{\color{brown}(1)(4)}+(0)(1) \end{bmatrix}=\begin{bmatrix}21\\-10 \\ -6\end{bmatrix}
$$
\end{example}
%\end{exploration}
\begin{definition}\label{def:matrixvectormult}
Let $A$ be an $m\times n$ matrix, and let $\vec{x}$ be an $n\times 1$ vector.  The product $A\vec{x}$ is the $m\times 1$ vector given by:
$$A\vec{x}=\begin{bmatrix}
           a_{11} & a_{12}&\dots&a_{1n}\\
           a_{21}&a_{22} &\dots &a_{2n}\\
		\vdots & \vdots&\ddots &\vdots\\
		a_{m1}&\dots &\dots &a_{mn}
         \end{bmatrix}\begin{bmatrix}x_1\\x_2\\\vdots\\x_n\end{bmatrix}=
         x_1\begin{bmatrix}a_{11}\\a_{21}\\ \vdots \\a_{m1}\end{bmatrix}+
         x_2\begin{bmatrix}a_{12}\\a_{22}\\ \vdots \\a_{m2}\end{bmatrix}+\dots+
         x_n\begin{bmatrix}a_{1n}\\a_{2n}\\ \vdots \\a_{mn}\end{bmatrix}$$
or, equivalently,
$$A\vec{x}=\begin{bmatrix}a_{11}x_1+a_{12}x_2+\ldots +a_{1n}x_n\\a_{21}x_1+a_{22}x_2+\ldots +a_{2n}x_n\\\vdots\\a_{m1}x_1+a_{m2}x_2+\ldots +a_{mn}x_n\end{bmatrix}$$
\end{definition}
We can now make a couple of observations about the matrix-vector product.  The first observation is part of the definition, but it is still worth pointing out.
\begin{observation}\label{obs:mat0020_obs1}
In order for the product $A\vec{x}$ to exist, $A$ and $\vec{x}$ must have compatible dimensions.  In particular, vector $\vec{x}$ must have as many components as the number of columns of $A$.  (Otherwise, we would not be have a well-defined linear combination of the columns.)  So, if $A$ is an $m\times n$ matrix, $\vec{x}$ must be an $n\times 1$ vector.  If we write these dimensions next to each other, we will notice that the inner dimensions ($n$) must match, while the outer dimensions, $m$ and $1$, give us the dimensions of the product.
$$\begin{array}{ccccc}
A& & \vec{x} &=& A\vec{x}\\
(m\times n) & &(n\times 1) & &(m\times 1)
\end{array}$$
\end{observation}

\begin{observation}\label{obs:mat0020_obs2}
    If you are familiar with the \dfn{dot product} (see \href{https://ximera.osu.edu/linearalgebradzv3/LinearAlgebraInteractiveIntro/VEC-0050/main}{Dot Product and its Properties}), you may have noticed that each individual entry in the product matrix $A\vec{x}$ is the dot product of a row of $A$ with $\vec{x}$.  Thus, if the rows of $A$ are vectors $\vec{r}_1$, $\vec{r}_2,\ldots ,\vec{r}_n$ we can restate Definition \ref{def:matrixvectormult} as follows:
$$A\vec{x}=\begin{bmatrix}-&\vec{r}_1&-\\-&\vec{r}_2&-\\ &\vdots & \\-&\vec{r}_n &-\end{bmatrix}\vec{x}=\begin{bmatrix}\vec{r}_1\dotp\vec{x}\\\vec{r}_2\dotp\vec{x}\\\vdots\\\vec{r}_n\dotp\vec{x}\end{bmatrix}$$
\end{observation}

Let's find another matrix-vector product.

\begin{example}\label{ex:matrixvectormultpractice}
Let $$A=\begin{bmatrix}1&-1\\2&3\\-2&1\\4&0\end{bmatrix}\quad\text{and}\quad\vec{x}=\begin{bmatrix}-3\\5\end{bmatrix}
$$
Find $A\vec{x}$.
\begin{explanation}
Fill in the blanks below.  
$$A\vec{x}=\begin{bmatrix}1&-1\\2&3\\-2&1\\4&0\end{bmatrix}\begin{bmatrix}-3\\5\end{bmatrix}=\begin{bmatrix}\answer{-8}\\\answer{9}\\\answer{11}\\\answer{-12}\end{bmatrix}
$$
\end{explanation}
\end{example}

\subsection*{Matrix-Matrix Multiplication}

Matrix-matrix multiplication is simply an extension of the idea of matrix-vector multiplication.  In order for the product definition to work, matrix dimensions must be compatible.  Let $A$ be an $m\times n$ matrix, and let $B$ be an $n\times p$ matrix, then the product $AB$ will be an $m\times p$ matrix.
$$\begin{array}{ccccc}
A& & B &=& AB\\
(m\times n) & &(n\times p) & &(m\times p)
\end{array}$$
Just like with vector products, the inner dimensions must be the same, while the outer dimensions, $m$ and $p$, give us the dimensions of the product.

\begin{definition}\label{def:matmatproduct} Let $A$ be an $m\times n$ matrix whose rows are vectors $\vec{r}_1$, $\vec{r}_2,\ldots ,\vec{r}_n$.  Let $B$ be an $n\times p$ matrix with columns $\vec{b}_1, \vec{b}_2, \ldots, \vec{b}_p$.  Then the entries of the matrix product $AB$ are given by the dot products
\begin{align*}AB&=\begin{bmatrix}-&\vec{r}_1&-\\-&\vec{r}_2&-\\ &\vdots & \\-&\vec{r}_i &-\\ &\vdots& \\-&\vec{r}_m&-\end{bmatrix}\begin{bmatrix}|&|&&|&&|\\\vec{b}_1& \vec{b}_2 &\ldots  & \vec{b}_j&\ldots& \vec{b}_p\\|&|&&|&&|\end{bmatrix}\\ &=\begin{bmatrix}\vec{r}_1\dotp \vec{b}_1&\vec{r}_1\dotp \vec{b}_2&\ldots&\vec{r}_1\dotp \vec{b}_j&\ldots &\vec{r}_1\dotp \vec{b}_p\\\vec{r}_2\dotp \vec{b}_1&\vec{r}_2\dotp \vec{b}_2&\ldots&\vec{r}_2\dotp \vec{b}_j&\ldots &\vec{r}_2\dotp \vec{b}_p\\\vdots&\vdots&&\vdots&&\vdots\\\vec{r}_i\dotp \vec{b}_1&\vec{r}_i\dotp \vec{b}_2&\ldots&\vec{r}_i\dotp \vec{b}_j&\ldots &\vec{r}_i\dotp \vec{b}_p\\\vdots&\vdots&&\vdots&&\vdots\\\vec{r}_m\dotp \vec{b}_1&\vec{r}_m\dotp \vec{b}_2&\ldots&\vec{r}_m\dotp \vec{b}_j&\ldots &\vec{r}_m\dotp \vec{b}_p\end{bmatrix}
\end{align*}
\end{definition}

So the $(i,j)$-entry of $AB$ is the dot product of the $i^{th}$ row of $A$ and the $j^{th}$ column of $B$.  

In terms of components, if the $i^{th}$ row of $A$ is $$\begin{bmatrix}a_{i1}& a_{i2} &\ldots &a_{in}\end{bmatrix}$$ and the $j^{th}$ column of $B$ is 
$$\begin{bmatrix}b_{1j}\\b_{2j}\\\vdots\\b_{nj}\end{bmatrix}$$
then the $(i,j)$-entry of $AB$ is given by
\begin{equation}\label{eq:ijentrymatrixproduct}a_{i1}b_{1j}+a_{i2}b_{2j}+\dots +a_{in}b_{nj}=\sum_{k=1}^na_{ik}b_{kj}
\end{equation}

\begin{example}\label{ex:matmatproduct}
Let $$A=\begin{bmatrix}3 & 2 & -1 & 1\\0 & 3 & 1 & 1\\1 & 4 & -1 & 0\end{bmatrix}\quad\text{and}\quad B=\begin{bmatrix}1 & -2 \\-1 & 4 \\2 & 3 \\2 & 0\end{bmatrix}$$
Find $AB$.

\begin{explanation}
Observe that $A$ is a $3\times 4$ matrix and $B$ is a $4\times 2$ matrix.  So, we expect the product $AB$ to be a $3\times 2$ matrix.
\begin{align*}
AB&=\begin{bmatrix}3 & 2 & -1 & 1\\0 & 3 & 1 & 1\\1 & 4 & -1 & 0\end{bmatrix}\begin{bmatrix}1 & -2 \\-1 & 4 \\2 & 3 \\2 & 0\end{bmatrix}\\
&=\begin{bmatrix}(3)(1)+(2)(-1)+(-1)(2)+(1)(2) & (3)(-2)+(2)(4)+(-1)(3)+(1)(0)\\(0)(1)+(3)(-1)+(1)(2)+(1)(2) & (0)(-2)+(3)(4)+(1)(3)+(1)(0)\\(1)(1)+(4)(-1)+(-1)(2)+(0)(2) & (1)(-2)+(4)(4)+(-1)(3)+(0)(0) \end{bmatrix}\\
&=\begin{bmatrix}1 & -1 \\ 1 & 15\\ -5 & 11\end{bmatrix}
\end{align*}
\end{explanation}
\end{example}

\begin{remark}
It is possible to use linear combinations rather than dot products to compute a matrix-matrix product.  If the columns of matrix $B$ are given by ${\vec{b}_1, \vec{b}_2, ..., \vec{b}_p}$, then the matrix product consists of $p$ columns, each of which is a matrix-vector product:

$$AB = \begin{bmatrix} | & | & \dots & |\\
A\vec{b}_1 & A\vec{b}_2 & \dots & A\vec{b}_p\\
| & | & \dots & |
\end{bmatrix}.
$$

If we return to Example \ref{ex:matmatproduct} above, we may use this approach to compute
$$A\vec{b}_1=\begin{bmatrix}3\\0\\1\end{bmatrix}
-\begin{bmatrix}2\\3\\4\end{bmatrix}
+2\begin{bmatrix}-1\\1\\-1\end{bmatrix}
+2\begin{bmatrix}1\\1\\0\end{bmatrix} 
= \begin{bmatrix}1\\1\\-5\end{bmatrix}$$
and 
$$A\vec{b}_2=-2\begin{bmatrix}3\\0\\1\end{bmatrix}
+4\begin{bmatrix}2\\3\\4\end{bmatrix}
+3\begin{bmatrix}-1\\1\\-1\end{bmatrix}+0 = \begin{bmatrix}-1\\15\\11\end{bmatrix}.
$$

\end{remark}

\begin{example}\label{ex:matmatproduct2} Let
$$A=\begin{bmatrix}-2 & 1\\3 & 0\\1 & -3\end{bmatrix}\quad\text{and}\quad B=\begin{bmatrix}4 & -2 & 3 & -2 &2\\1 & -3 & 1 & 1 &0\end{bmatrix}$$
Find $AB$.
\begin{explanation}
Fill in the blanks below.  
$$AB=\begin{bmatrix}-2 & 1\\3 & 0\\1 & -3\end{bmatrix}\begin{bmatrix}4 & -2 & 3 & -2 &2\\1 & -3 & 1 & 1 &0\end{bmatrix}=\begin{bmatrix}\answer{-7} &\answer{1} & \answer{-5} & \answer{5} & \answer{-4}\\\answer{12} & \answer{-6} & \answer{9} & \answer{-6} & \answer{6}\\\answer{1} & \answer{7} & \answer{0} & \answer{-5} & \answer{2}\end{bmatrix}$$

\end{explanation}
\end{example}

\begin{example}\label{ex:linearcombofcols1}
Suppose that we know that $\vec{x}=\begin{bmatrix} -2\\5\end{bmatrix}$ satisfies
$$\begin{bmatrix}
-2&3\\
4&-1
\end{bmatrix}\vec{x}=\begin{bmatrix} 19\\-13\end{bmatrix}$$
Use this information to express $\begin{bmatrix} 19\\-13\end{bmatrix}$ as a linear combination of the columns of $\begin{bmatrix}
-2&3\\
4&-1
\end{bmatrix}$.
\begin{explanation}
$$-2\begin{bmatrix} -2\\4\end{bmatrix}+5\begin{bmatrix} 3\\-1\end{bmatrix}=\begin{bmatrix} 19\\-13\end{bmatrix}$$
\end{explanation}

\end{example}






\subsection*{Properties of Matrix Multiplication}
%{\color{red} This section was adopted from Kuttler matricesMatrixArithmeticMultiplicationProperties.tex hyperlink ref needed}
\begin{exploration}\label{init:matmultnotcomm}
Let $$A=\begin{bmatrix}1&2\\3&4\end{bmatrix}\quad\text{and}\quad B=\begin{bmatrix}5&6\\7&8\end{bmatrix}$$
Observe that both $AB$ and $BA$ are defined, and both products are $2\times 2$ matrices.  Let's compute the two products
$$AB=\begin{bmatrix}19&22\\43&50\end{bmatrix}\quad\text{and}\quad BA=\begin{bmatrix}23&34\\31&46\end{bmatrix}$$
Clearly $AB\neq BA$. We say that $A$ and $B$ \dfn{do not commute}.
\end{exploration}
While it is possible to find specific matrices that commute, matrix multiplication is not commutative in general.
\begin{warning}Matrix multiplication is not commutative.
\end{warning}

One example of an $n\times n$ square matrix that commutes with all $n\times n$ matrices is the matrix $I_n$ defined by
$$I_n=\begin{bmatrix}1&0&\ldots &0\\0&1&\ldots &0\\\vdots &\vdots &\ddots &\vdots \\0 &0 &\ldots & 1\end{bmatrix}$$
$I_n$ has 1's along the main diagonal and 0's everywhere else.  It is often useful to think of $I_n$ as a matrix whose $j^{th}$ column (and $j^{th}$ row) is $\vec{e}_j$, the $j^{th}$ standard unit vector of $\RR^n$.  When the dimensions of $I_n$ are clear from the context, or irrelevant, we will omit the subscript $n$ and simply refer to this matrix as $I$.

You can easily convince yourself that $I$ commutes with all square matrices of appropriate dimensions.  Let $$A=\begin{bmatrix}a&b&c\\d&e&f\\g&h&i\end{bmatrix}\quad\text{and}\quad I=\begin{bmatrix}1&0&0\\0&1&0\\0&0&1\end{bmatrix}$$
Verify that $AI=A$ and $IA=A$.  

Because $I$ acts like the multiplicative identity $1$ in regular multiplication, $I$ (or $I_n$) is called the \dfn{identity matrix}.

Next we list several important properties of matrix multiplication. These properties hold only when matrix sizes are such that the products are defined. 

\begin{theorem}[Properties of Matrix Multiplication]\label{th:propertiesofmatrixmultiplication}
The following hold for matrices $A,B,$ and $C$ and for scalar $c$,
\begin{enumerate}
\item \label{item:matrixproperties1} Left Distributive Property
$$A\left( B+C\right) =AB +AC$$
\item\label{item:matrixproperties2} Right Distributive Property
$$\left( B+C\right) A=BA+CA$$
\item \label{item:matrixproperties3} Associativity
$$A\left( BC\right) =\left( AB\right) C$$
\item\label{item:matrixproperties4}
$$c(AB)=(cA)B=A(cB)$$
\item \label{item:identitymatrix} Multiplicative Identity
$$AI=IA=A$$
\end{enumerate}
\end{theorem}

\begin{proof}
We prove \ref{item:matrixproperties1} using the expression in (\ref{eq:ijentrymatrixproduct}) for the $(i,j)$-entry of a matrix product. (The proof of \ref{item:matrixproperties2} is similar.) The $(i,j)$-entry of $A(B+C)$ is given by

$$
\sum_{k}a_{ik}( b_{kj}+c_{kj})=\sum_{k}a_{ik}b_{kj}+\sum_{k}a_{ik}c_{kj}
$$
We recognize the right hand side as the $(i,j)$-entry of $AB+AC$.
Thus $A(B+C) =AB+AC$.

For \ref{item:matrixproperties4},
$$c\left(\sum_{k}a_{ik}b_{kj}\right) = \sum_{k}\left(c a_{ik}\right)b_{kj} = \sum_{k}a_{ik} \left(b_{kj} c\right).$$
For \ref{item:identitymatrix}, the $(i,j)$-entry of the product $AI$ is given by the dot product of the $i^{th}$ row of $A$ with the standard unit vector $\vec{e}_j$.  Clearly, this dot product is $a_{ij}$.  Because the $(i,j)$-entry of the product $AI$ is equal to the $(i,j)$-entry $A$, we conclude that $AI=A$.  The proof that $IA=A$ is similar.

Note that we skipped the proof of \ref{item:matrixproperties3}, which is quite cumbersome using sigma notation.  We will easily tackle this proof later in the course when we cover \href{https://ximera.osu.edu/linearalgebradzv3/LinearAlgebraInteractiveIntro/LTR-0030/main}{Composition of Linear Transformations}.
\end{proof}






\section*{Practice Problems}

\begin{problem}\label{prob:matprodundefined} Explain why the following product is not defined.
$$\begin{bmatrix}a&b&c\\d&e&f\\g&h&i\end{bmatrix}\begin{bmatrix}1\\2\\3\\4\end{bmatrix}$$
\end{problem}

\begin{problem}\label{prob:lincombcols}
Express the given product as a linear combination of the columns of the matrix.
$$\begin{bmatrix}1&2&3&4\\5&6&7&8\end{bmatrix}\begin{bmatrix}-2\\3\\-5\\7\end{bmatrix}=\answer{-2}\begin{bmatrix}1\\5\end{bmatrix}+\answer{3}\begin{bmatrix}2\\6\end{bmatrix}+\answer{-5}\begin{bmatrix}3\\7\end{bmatrix}+\answer{7}\begin{bmatrix}4\\8\end{bmatrix}$$
\end{problem}

\emph{Problems \ref{prob:matproddim1}-\ref{prob:matproddim2}}
Predict the dimensions of each product.

\begin{problem}\label{prob:matproddim1}
$$\begin{bmatrix}1&2&3\end{bmatrix}\begin{bmatrix}4\\5\\6\end{bmatrix}$$
Dimensions of product: $\answer{1}\times \answer{1}$
\end{problem}

\begin{problem}\label{prob:matproddim2}
$$\begin{bmatrix}1&2\\3&4\\5&6\end{bmatrix}\begin{bmatrix}1&2&3&4\\5&6&7&8\end{bmatrix}$$
Dimensions of product: $\answer{3}\times \answer{4}$
\end{problem}


\emph{Problems \ref{prob:matprod1}-\ref{prob:matprod3}}
Find each product.

\begin{problem}\label{prob:matprod1}
$$\begin{bmatrix}1&3&-2&1\\-2&1&0&4\end{bmatrix}\begin{bmatrix}4\\-2\\1\\1\end{bmatrix}=\begin{bmatrix}\answer{-3}\\\answer{-6}\end{bmatrix}$$
\end{problem}

\begin{problem}\label{prob:matprod2}
$$\begin{bmatrix}1&2&3\end{bmatrix}\begin{bmatrix}-1&2&-3&1\\1&1&-1&-2\\0&1&2&1\end{bmatrix}=\begin{bmatrix}\answer{1}&\answer{7}&\answer{1}&\answer{0}\end{bmatrix}$$
\end{problem}

\begin{problem}\label{prob:matprod3}
$$\begin{bmatrix}1&2\\-1&0\\2&3\\-4&-1\end{bmatrix}\begin{bmatrix}-1&1\\2&-3\end{bmatrix}=\begin{bmatrix}\answer{3}&\answer{-5}\\\answer{1}&\answer{-1}\\\answer{4}&\answer{-7}\\\answer{2}&\answer{-1}\end{bmatrix}
$$
\end{problem}



\section*{Text Source}
The section on Properties of Matrix Multiplication is an adaptation of Section 2.1 of Ken Kuttler's \href{https://open.umn.edu/opentextbooks/textbooks/a-first-course-in-linear-algebra-2017}{\it A First Course in Linear Algebra}. (CC-BY)

Ken Kuttler, {\it  A First Course in Linear Algebra}, Lyryx 2017, Open Edition, p. 67-68.

\end{document} 