\documentclass{ximera}
%% You can put user macros here
%% However, you cannot make new environments

\listfiles

\graphicspath{{./}{firstExample/}{secondExample/}}

\usepackage{tikz}
\usepackage{tkz-euclide}
\usepackage{tikz-3dplot}
\usepackage{tikz-cd}
\usetikzlibrary{shapes.geometric}
\usetikzlibrary{arrows}
%\usetkzobj{all}
\pgfplotsset{compat=1.13} % prevents compile error.

%\renewcommand{\vec}[1]{\mathbf{#1}}
\renewcommand{\vec}{\mathbf}
\newcommand{\RR}{\mathbb{R}}
\newcommand{\dfn}{\textit}
\newcommand{\dotp}{\cdot}
\newcommand{\id}{\text{id}}
\newcommand\norm[1]{\left\lVert#1\right\rVert}
 
\newtheorem{general}{Generalization}
\newtheorem{initprob}{Exploration Problem}

\tikzstyle geometryDiagrams=[ultra thick,color=blue!50!black]

%\DefineVerbatimEnvironment{octave}{Verbatim}{numbers=left,frame=lines,label=Octave,labelposition=topline}



\usepackage{mathtools}

\author{}
\license{Creative Commons 4.0 By-NC-SA}
%\outcome{Compute an antiderivative using basic formulas}
\begin{document}
\begin{exercise}

True or False?  If False, you should come up with a counterexample.  If True, can you give a proof?

 \begin{enumerate}
     \item If $A$ is symmetric, then $A$ has orthonormal columns.

 \begin{multipleChoice}
 \choice{True}
 \choice[correct]{False}
 \end{multipleChoice}

 \item The product of two orthogonal matrices is an orthogonal matrix.

 \begin{multipleChoice}
 \choice[correct]{True}
 \choice{False}
 \end{multipleChoice}

 \item The orthogonal complement of the row space of a matrix is the null space of that matrix.

 \begin{multipleChoice}
 \choice[correct]{True}
 \choice{False}
 \end{multipleChoice}

 \item For any proper subspace of $\RR^n$, it is possible to produce an orthogonal basis.

 \begin{multipleChoice}
 \choice[correct]{True}
 \choice{False}
 \end{multipleChoice}


 \end{enumerate}

 
\end{exercise}


\end{document}